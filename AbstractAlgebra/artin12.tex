\documentclass[11pt]{article}
\usepackage[T1]{fontenc}
\usepackage{geometry, changepage, hyperref}
\usepackage{amsmath, amssymb, amsthm, bm}
\usepackage{physics, esint}

\hypersetup{colorlinks=true, linkcolor=blue, urlcolor=cyan}
\setlength{\parindent}{0pt}
\setlength{\parskip}{5pt}

\newtheorem{theorem}{Theorem}
\newtheorem{lemma}{Lemma}
\newtheorem{proposition}{Proposition}
\newtheorem{corollary}{Corollary}
\newtheorem{claim}{Claim}

\newcommand{\conjugate}[1]{\overline{#1}}
\newcommand{\s}{$\\ \text{ }$}

\title{Artin: Factoring}
\author{James Pagan}
\date{February 2024}

% --------------------------------------------- %

\begin{document}

\maketitle
\tableofcontents
\newpage

% --------------------------------------------- %

\section{Unique Factorization Domains}

% --------------------------------------------- %

\subsection{Terminology}

% --------------------------------------------- %

Let $R$ be an integral domain. Before we introduce unique factorization domains, we must define several terms for $a, b \in R$:

\begin{enumerate}
  \item $a$ \textbf{divides} $b$ if $(b) \subseteq (a)$.
  \item $a$ is a \textbf{proper divisor} if $b$ if $(b) \subset (a) \subset R$.
  \item $a$ and $b$ are \textbf{associates} if $(a) = (b)$.
  \item $a$ is \textbf{irreducible} if $(a) \subset R$ and there is no principal ideal $(c)$ such that $(a) \subset (c) \subset R$.
  \item $p$ is a \textbf{prime element} if $p \ne 0$ and $(p)$ is prime.
\end{enumerate}

These may be equivalently expressed ideal-free (AbstractAlgebra/homework3.tex):
\begin{enumerate}
  \item $a$ \textbf{divides} $b$ if $b = aq$ for some $q \in R$.
  \item $a$ is a \textbf{proper divisor} of $b$ if $b = aq$ and neither $a$ nor $q$ is a unit.
  \item $a$ and $b$ are \textbf{associates} if each divides the other --- that is, $b = ua$ for some unit $u$.
  \item $a$ is \textbf{irreducible} if it has no proper divisors --- its only divisors are units and associates.
  \item $p$ is a \textbf{prime element} if $p \ne 0$ and $p$ divides $ab$ implies $p$ divides $a$ or $p$ divides $b$.
\end{enumerate}

A \textbf{size function} is a mapping $\sigma : R \setminus \{ 0 \} \to \mathbb{Z}_{\ge 0}$.

\begin{adjustwidth}{0.5cm}{}
  \begin{theorem}
    Let $R$ be an integral domain. Then all prime elements of $R$ are irreducible.
  \end{theorem}
  \begin{proof}
    Suppose that $p$ is prime and that $(p) \subseteq (c) \subset R$. Hence there exists $x$ such that $p = cx$, so $cx \in (p)$. We have two possibilities: $c \in (p)$ or $x \in (p)$.

    Suppose for contradiction that $x \in (p)$. Then $x = py$ for some $y$ --- substituting into the above equality yields
    \[
      p = c(py) \implies p(1 - cy) = 0.
    \]
    Since $p \ne 0$, we have $1 = cy$ --- hence $c$ is a unit and $(c) = R$, a contradiction. We must have $c \in (p)$, so $(c) = (p)$. We conclude that $(p)$ is irreducible.
  \end{proof}
\end{adjustwidth}



% --------------------------------------------- %

\subsection{Definition}

A \textbf{unique factorization domain} $R$ is an integral domain if for every nonzero $x \in R$, there exists a unit $u$ and irreducible elements $p_{1}, \ldots, p_{n}$ such that
\[
  x \, = \, u p_{1} \cdots p_{n},
\]
and this factorization is unique in the following sense: if there exists a second factorization
\[
  x \, = \, w q_{1} \cdots q_{m},
\]
then $n = m$ and there exists a bijection such that $(p_{i}) = (q_{j})$ for each paired $i, j$ (that is, $p_{i}$ and $q_{j}$ associate).

\begin{adjustwidth}{0.5cm}{}
  \begin{theorem}
    Every irreducible element in a unique factorization domain is prime.
  \end{theorem}
  \begin{proof}
    Suppose that $(p)$ is not prime --- then there exist $a, b \notin (p)$ such that $ab \in (p)$. Thus we have $(p) \subset (a)$. Since $a$ is a nonunit, $(a) \subset R$, so
    \[
      (p) \subset (a) \subset R.
    \]
    Hence $(p)$ is not irreducible. Taking the contrapositive yields the desired result.
  \end{proof}
\end{adjustwidth}

Hence, we could equivalently define unique factorization as decomposition to \textit{prime} elements. In this sense, factoriation in $R$ ``terminates'' if and only if $R$ satisfies the ascending chain condition for principal ideals; namely, the chain
\[
  x \quad \subseteq \quad \bigcap\limits_{i = 1}^{\infty} \, (p_{i}) \quad \subseteq \quad \bigcap\limits_{i = 2}^{\infty} \, (p_{i}) \quad \subseteq \quad \bigcap\limits_{i = 3}^{\infty} \, (p_{i}) \quad \subseteq \quad \cdots
\]
is stationary. This is the terminology favored by Artin.

% --------------------------------------------- %

\section{Principal Ideal Domains}

% --------------------------------------------- %

\subsection{Definition}

A \textbf{principal ideal domain} is an integral domain in which all ideals are principal. It is clear that all such domains are Noetherian.

\newpage

\begin{adjustwidth}{0.5cm}{}
	\begin{theorem}
		Let $R$ be a principal ideal domain. Then all nonzero prime ideals of $R$ are maximal.
	\end{theorem}
	\begin{proof}
    Let $(p)$ be a prime ideal contained in the maximal ideal $(m)$. Supposing for contradiction that
    \[
      (p) \subset (m) \subset R,
    \]
    we obtain that $(p)$ is not irreducible, which contradicts Theorem 1. Hence $(p) = (m)$, so $(p)$ is maximal.
	\end{proof}
\end{adjustwidth}
Four helpful facts about principal ideal domains are as follows:
\begin{enumerate}
  \item If $\mathfrak{a}_{1} = (a_{1})$ and $\mathfrak{a}_{2} = (a_{2})$ are principal ideals, then $\mathfrak{a}_{1} \mathfrak{a}_{2} = (a_{1}a_{2})$. This holds in any commutative ring.
  \item Prime ideals cannot contain other prime ideals: if $(p_{1}) \subset (p_{2})$ are prime, then the fact
  \[
    (p_{1}) \, \subset \, (p_{2}) \, \subset \, R
  \]
  implies that $(p_{1})$ is not irreducible --- a contradiction.
  \item All prime ideals are relatively prime. This is because if $(p_{1})$ and $(p_{2})$ are prime, we have
  \[
    (p_{1}) \, \subseteq \, (p_{1}) + (p_{2}) \, \subseteq \, R
  \]
  We cannot have $(p_{1}) = (p_{1}) + (p_{2})$ by Fact 2; thus since $(p_{1})$ to be irreducible, we conclude that $(p_{1}) + (p_{2}) = R$.
  \item If $(p_{1}), \ldots, (p_{n})$ are prime ideals, then
  \[
    (p_{1}) \cap \cdots \cap (p_{n}) \, = \, (p_{1}) \times \cdots \times (p_{n}) \, = \, (p_{1} \cdots p_{n}).
  \]
\end{enumerate}

% --------------------------------------------- %

\subsection{Relation with Unique Factorization Domains}

\begin{adjustwidth}{0.5cm}{}
  \begin{theorem}
    All principal ideal domains are unique factorization domains.
  \end{theorem}
  \begin{proof}
    Let $R$ be a principal ideal domain and select $x \in R$. Then since $R$ is Noetherian, factoring terminates: each ascending chain of principal ideals is stationary.

    Let $(p_{1}), \ldots, (p_{n})$ be the prime ideals which contain $x$. By Fact 4, we deduce that $x \in (p_{1}p_{2} \cdots p_{n})$. Thus we can write $x$ in the form
    \[
      x \, = \, u_{1} p_{1} \cdots p_{n}.
    \]
    If $u_{1}$ is contained in prime ideals, then they must be among $(p_{1}), \ldots, (p_{n})$. Hence we can express $u_{1}$ as a product of some $p_{1}, \ldots, p_{n}$ times $u_{2}$. Repeating at nauseum, we obtain a sequence $u_{1}, u_{2}, \ldots$ which yields the stationary chain
    \[
      (x) \subseteq (u_{1}) \subseteq (u_{2}) \subseteq \cdots.
    \]
    Hence there must exist $n \in \mathbb{Z}_{> 0}$ such that $(u_{n}) = (u_{n + 1}) = \cdots$. Thus we have $u_{n} = u \cdot u_{n + 1}$ for some unit $u$. Recursive substitution into our expression for $x$ yields
    \[
      x = u p_{1}^{e_{1}} \cdots p_{n}^{e_{n}},
    \]
    which completes the existence portion of the proof. As per uniqueness, suppose that
    \[
      u p_{1} \cdots p_{n} \, = \, x \, = \, w q_{1} \cdots q_{m}
    \]
    A quick induction on $\max \{ m, n \}$ yields that since two primes on either side must be adjoints, we can divide and yield a number which factors uniquely. This completes the proof.
  \end{proof}
\end{adjustwidth}

% --------------------------------------------- %

\subsection{Greatest Common Divisor}

Let $R$ be an integral domain, and select $a, b \in R$. A \textbf{greatest common divisor} of $a$ and $b$ is an element $d \in R$ such that:
\begin{enumerate}
  \item $d \mid a$ and $d \mid b$.
  \item $c \mid a$ and $c \mid b$ implies $c \mid d$.
\end{enumerate}
It is clear that GCDs are unique up to association by Condition 2 --- thus we can speak of \textit{the} GCD. If the only greatest common divisors of $a$ and $b$ are units, we set $\gcd(a, b) = 1$ and call $a, b$ \textbf{relatively prime}.

\begin{adjustwidth}{0.5cm}{}
  \begin{theorem}
    Suppose $R$ is a principal ideal domain. Then the generator of the ideal $(a, b)$ is the greatest common divisor of $a, b$.
  \end{theorem}
  \begin{proof}
    It is clear that $a, b \in (d)$ implies $d \mid a$ and $d \mid b$. We need only demonstrate the second condition.  Thus, suppose $c \mid a$ and $c \mid b$ --- hence $(a) \subseteq (c)$ and $(b) \subseteq (c)$. Thus
    \[
      (d) \, = \, (a) + (b) \subseteq (c),
    \]
    so $c \mid d$. We conclude that $\gcd(a, b) = d$.
  \end{proof}
\end{adjustwidth}

It is now easy to demonstrate that $\gcd(a_{1}, a_{2}, \ldots, a_{n}) = \gcd(a_{1}, \gcd(a_{2}, \ldots, a_{n}))$. This yields the following lemma:

\begin{adjustwidth}{0.5cm}{}
  \begin{lemma}[Bezout's Identity]
    If $R$ is a principal ideal domain and $\gcd(a_{1}, \ldots, a_{n}) = d$, there exist integers $b_{1}, \ldots, b_{n}$ such that $d \, = \, a_{1}b_{1} + \cdots + a_{n}b_{n}$. \s
  \end{lemma}
\end{adjustwidth}

Much simpler than the proof in your 2nd Conest Math Notebook, right?

% --------------------------------------------- %

\section{Euclidean Domain}

% --------------------------------------------- %

\subsection{Definition}

An integral domain $R$ is a  \textbf{Euclidean domain} if there exists a size function $\sigma$ such that $a \in R$ and \textit{nonzero} $b \in R$ implies the existence of $q, r \in R$ such that $a = bq + r$, where $\sigma(r) < \sigma(b)$. It is clear that $\mathbb{Z}$ is a Euclidean domain.

\begin{adjustwidth}{0.5cm}{}
  \begin{theorem}
    All fields are Euclidean domains.
  \end{theorem}
  \begin{proof}
    Let $R$ be a field, and select $a, b \in F$. Then
    \[
      a = b \left( \tfrac{a}{b} \right) + 0.
    \]
    If $\sigma$ is an arbitrary size function on $R$, then the caveat of remainder zero ensures that the above equations dictate a valid Euclidean division.
  \end{proof}
\end{adjustwidth}

For a field $F$, the ring $F[x]$ is a Euclidean domain. I proved this in my contest algebra notes.

% --------------------------------------------- %

\subsection{Relation with Principal Ideal Domains}

\begin{adjustwidth}{0.5cm}{}
  \begin{theorem}
    All Euclidean domains are principal ideal domains.
  \end{theorem}
  \begin{proof}
    Let $R$ be a Euclidean domain with size function $\sigma$ and let $\mathfrak{a} \subseteq R$ be an ideal. If $\mathfrak{a} = 0$, then $\mathfrak{a}$ is principal; otherwise, the Well-Ordering Theorem guarantees that there exists a nonzero element $a \in \mathfrak{a}$ of minimal size.

    Let $b \in \mathfrak{a}$. Then there exist $q, r \in R$ such that
    \[
      b = aq + r,
    \]
    where $\sigma(r) < \sigma(a)$. Since $a$ is minimal, we must have $r = 0$, in which case $b \in (a)$. We conclude that $\mathfrak{a} = (a)$, so all ideals of $R$ are principal.
  \end{proof}
\end{adjustwidth}

We have thus attained a sequence of types of rings:
\[
  \text{rings} \, \subseteq \, \text{commutative rings} \, \subseteq \, \text{integral domains} \, \subseteq \, \text{UFDs} \, \subseteq \, \text{PIDs} \, \subseteq \, \text{GDs} \, \subseteq \, \text{fields}.
\]

% --------------------------------------------- %

\section{The Polynomial Ring \texorpdfstring{$\mathbb{Z}[x]$}{Z[x]}}

We have proved the following facts about polynomial rings: for any field $F$,
\begin{enumerate}
  \item $F[x]$ is a Euclidean domain.
  \item $F[x_{1}, \ldots, x_{n}]$ is a unique factorization domain and Noetherian.
\end{enumerate}
Polynomial rings over arbitrary commutative rings obey significantly fewer restrictions. This section characterizes the polynomial ring $\mathbb{Z}[x]$. There are two main tools in its study: first is the embedding
\[
  \mathbb{Z}[x] \subset \mathbb{Z}[x],
\]
and second is reduction modulo some prime $p$: the mappings $\psi : \mathbb{Z}[x] \to \mathbb{F}_{p}[x]$.

% --------------------------------------------- %

\subsection{Primative Polynomials}

The following lemma is quite obvious:

\begin{adjustwidth}{0.5cm}{}
  \begin{lemma}
    Let $f(x) = a_{n}x^{n} + \cdots + a_{0}$ have integer coefficents. Then the following are equivalent:
    \begin{enumerate}
      \item $p$ divides each $a_{i}$.
      \item $p$ divides $f$ in $\mathbb{Z}[x]$
      \item $f$ lies in the kernel of $\psi_{p}$. \s
    \end{enumerate}
  \end{lemma}
\end{adjustwidth}

A polynomial $f \in \mathbb{Z}[x]$ is called \textbf{primative} if the GCD of its coefficents is $1$.

\begin{adjustwidth}{0.5cm}{}
  \begin{lemma}
    Let $f(x) = a_{n}x^{n} + \cdots + a_{0}$ have integer coefficents. Then the following are equivalent:
    \begin{enumerate}
      \item $f$ is primative.
      \item $f$ is not divisible by any prime $p$.
      \item $\psi_{p}(f) \ne 0$ for all primes $p$. \s
    \end{enumerate}
  \end{lemma}
\end{adjustwidth}

Observe that an integer $n \in \mathbb{Z}[x]$ is a prime element if and only if it is prime. Thus $fg \in (p)$ implies that $f \in (p)$ or $g \in (p)$: stated differently, $p \mid fg$ implies $p \mid f$ or $p \mid g$.

\begin{adjustwidth}{0.5cm}{}
  \begin{lemma}[Gauss' Lemma]
    The product of primative polynomials is primative.
  \end{lemma}
  \begin{proof}
    Suppose that $fg$ is not primative; then $p \mid fg$ for some prime integer $p$. Thus $p \mid f$ or $p \mid g$, so one of $f$ and $g$ must not be primative. Taking the contrapositive yields the desired result.
  \end{proof}
\end{adjustwidth}

That would be an insanely long number theory problem, in terms of a crazy sequence of equations --- and yet it falls so elegantly to the properties of prime ideals!

% --------------------------------------------- %

\section{The Gaussian Integers \texorpdfstring{$\mathbb{Z}[i]$}{Z[i]}}

Since $\mathbb{Z}[i]$ is isomorphic to $\mathbb{Z}[x] \, / \, (x^{2} + 1)$, we can use tools from polynomial rings to study Gaussian integers.

% --------------------------------------------- %

\subsection{A Euclidean Domain}

\begin{adjustwidth}{0.5cm}{}
  \begin{theorem}
    $\mathbb{Z}[i]$ is a Euclidean domain.
  \end{theorem}
  \begin{proof}
    Using the norm $\norm{a + bi} = a^{2} + b^{2}$, we will divide $a + bi$ by $c + di$. It is easy to deduce that there exist rationals $r, s$ such that
    \[
      \frac{a + bi}{c + di} \, = \, r + si.
    \]
    Approximate $r$ and $s$ by integers: namely define $n, m 
    \in \mathbb{Z}$ such that $\abs{r - n} \le \tfrac{1}{2}$ and $\abs{s - m} \le \tfrac{1}{2}$. Then we can express the above as
    \[
      r + si  = (n + mi ) \, + \, (r - n) + i(s - m).
    \]
    Expanding this out, we obtain a rather messy equation:
    \[
      a + bi \quad = \quad (n + ni )(c + di ) \quad + \quad \big( (r - n) + i(s - m)  \big)(c + di ).
    \]
    All that remains to be proven is that the right-most term has a norm less than $c + di$, which is equivalent to showing that $(r - n) + i (s - m) $ has a norm less than one:
    \[
      \norm{(r - n) + i(s - m) } = (r - n)^{2} + (s - m)^{2} \le \frac{1}{4} + \frac{1}{4} < 1.
    \]
    This completes the proof.
  \end{proof}
\end{adjustwidth}

\subsection{Gaussian Primes}

An irreducible element in $\mathbb{Z}[i]$ is called a \textbf{Gaussian prime}.

\begin{adjustwidth}{0.5cm}{}
  \begin{theorem}
    Let $\pi$ be a Gaussian prime. Then $\pi \cdot \conjugate{\pi}$ is either a prime integer or the square of a prime integer.
  \end{theorem}
  \begin{proof}
    For (1), let $\pi \cdot \conjugate{\pi}$ factor under $\mathbb{Z}$ as $p_{1}, \ldots, p_{n}$, and further factor each $p_{i}$ under the Gaussian integers. Since factorization in $\mathbb{Z}[i]$ is unique, this second factorization generates $n$ or more Gaussian primes. However, $\conjugate{\pi}$ is a Gaussian prime, since $z \mid \conjugate{\pi}$ implies $\conjugate{z} \mid \pi$. Hence $n$ is at most two.

    Consider when $n = 2$ --- that is, when $p_{1}p_{2}$ factors in $\mathbb{Z}[i]$ as $\pi \cdot \conjugate{\pi}$. Then $p_{1}$ is an associate of $\pi$ or $\conjugate{\pi}$, while $p_{2}$ is the negative of the above associate. Hence $\pi \cdot \conjugate{\pi} = p_{1}^{2}$.
  \end{proof}
\end{adjustwidth}

The following theorem characterizes the reverse direction:

\begin{adjustwidth}{0.5cm}{}
  \begin{theorem}
    Let $p$ be a prime integer. Then $p$ is either a Gaussian prime or factors as $\pi \cdot \conjugate{\pi}$ for some Gaussian prime $\pi$.
  \end{theorem}
  \begin{proof}
    Suppose that $p$ is not a Gaussian prime, and let $p = \pi z$ for some Gaussian prime $\pi$ and Gaussian integer $z$. It is clear that $z = n \conjugate{\pi}$ for some $n \in \mathbb{Z}$; since $n \mid z$ implies $n \mid p$, we must have $n = 1$.
  \end{proof}
\end{adjustwidth}

The following two theorems prepare for the debut of Fermat's Two-Square Theorem.

\begin{adjustwidth}{0.5cm}{}
  \begin{theorem}
    Let $p$ be a prime integer. Then the following are equivalent:
    \begin{enumerate}
      \item $p$ is a Gaussian prime.
      \item $\mathbb{Z}[i] \, / \, (p)$ is a field.
      \item $x^{2} + 1$ is irreducible in $\mathbb{Z}_{p}[x]$.
    \end{enumerate}
  \end{theorem}
  \begin{proof}
    From the properties of Euclidean domains, it is clear that
    \[
      p \text{ is a Gaussian prime } \iff (p) \text{ is maximal } \iff \mathbb{Z}[i] \, / \, (p) \text{ is a field}.
    \]
    Thus (1) and (2) are equivalent. For the equivalency of (2) and (3), we have
    \begin{align*}
      \mathbb{Z}[i] \, / \, (p) \text{ is a field } & \iff \big( \mathbb{Z}[x] \, / \, (x^{2} + 1) \big) \, / \, (p) \text{ is a field} \\
                                                    & \iff \mathbb{Z}_{p}[x] \, / \, (x^{2} + 1) \text{ is a field} \\
                                                    & \iff (x^{2} + 1) \text{ is maximal in } \mathbb{Z}_{p}[x] \\
                                                    & \iff x^{2} + 1 \text{ is irreducible in } \mathbb{Z}_{p}[x].
    \end{align*}
    The last equivalency follows from the fact that $\mathbb{Z}_{p}$ is a field, so $\mathbb{Z}_{p}[x]$ is a Euclidean Domain.
  \end{proof}
\end{adjustwidth}

The following proof uses Sylow's Theorem, found in AbstractAlgebra/artin7.tex:

\begin{adjustwidth}{0.5cm}{}
  \begin{theorem}
    Let $p$ be an odd prime. Then the following two facts hold:
    \begin{enumerate}
      \item $\mathbb{Z}_{p}^{\times}$ contains an element of order $4$ if and only if $p \, \equiv \, 1 \pmod{4}$.
      \item $x \in \mathbb{Z}_{p}$ has order $4$ if and only if $x^{2} \, \equiv \, -1 \pmod{p}$.
    \end{enumerate}
  \end{theorem}
  \begin{proof}
    We start with (1). Since $\mathbb{Z}_{p}$ is a finite field, $\mathbb{Z}_{p}^{\times} \, \cong \, C_{p - 1}$. Thus $\mathbb{Z}_{p}^{\times}$ has an element of order $4$ if and only if $4 \mid p - 1$, which entails $p \, \equiv \, 1 \pmod{4}$.

    For (2), suppose $x \in \mathbb{Z}_{p}$ has order $4$. Then
    \[
      (x^{2} + 1)(x^{2} - 1) \, = \, x^{4} - 1 \, = \, 0.
    \]
    Since $\mathbb{Z}_{p}[x]$ is a Euclidean domain, one of these polynomials must be $0$; since $x$ does not have order $2$, we deduce $x^{2} + 1 = 0$. The reverse direction is easy to prove.
  \end{proof}
\end{adjustwidth}

The following theorem is the culmination of this entire section:

\begin{adjustwidth}{0.5cm}{}
  \begin{theorem}[Fermat's Two-Square Theorem]
    Let $p$ be a prime integer. Then the following are equivalent:
    \begin{enumerate}
      \item $p$ is the product of complex conjugate Gaussian primes.
      \item $p = 2$ or $p$ is congruent to $1$ modulo $4$.
      \item $p$ is a sum of two integer squares.
      \item $-1$ is a quadratic residue modulo $p$.
    \end{enumerate}
  \end{theorem}
  \begin{proof}
    It is easy to see that (1) and (3) are equivalent. The equivalence of (2) and (4) is established by Theorem 12.

    Suppose (3), observe that the squares modulo $4$ are $0$ and $1$; therefore, $p = a^{2} + b^{2}$ must be $0$, $1$, or $2$ modulo $4$. Hence $p$ is either $2 = 1^{2} + 1^{2}$ or a prime congruent to $1 \pmod{4}$, which is (2).

    Suppose (4). Define $x$ such that $x^{2} \, \equiv \, -1 \pmod{p}$. Then $x^{4} \, \equiv \, 1 \pmod{p}$, so the polynomial $x^{4} + 1$ is reducible in the Euclidean domain $\mathbb{Z}_{p}[x]$. By the converse of Theorem 11, $p$ cannot be a Gauss prime --- hence by Theorem 10, it is the product of a Gauss prime and its conjugate. This entails (1).

    We conclude that (1), (2), (3), and (4) are equivalent conditions.
  \end{proof}
\end{adjustwidth}

This stunning and challenging theorem falls elegantly to the mechanics of Abstract Algebra. Isn't that fucking amazing?

% --------------------------------------------- %


\end{document}
