\documentclass[11pt]{article}
\usepackage[T1]{fontenc}
\usepackage{geometry, changepage, hyperref}
\usepackage{amsmath, amssymb, amsthm, bm}
\usepackage{physics, esint}

\hypersetup{colorlinks=true, linkcolor=blue, urlcolor=cyan}
\setlength{\parindent}{0pt}
\setlength{\parskip}{5pt}

\newtheorem{theorem}{Theorem}
\newtheorem{lemma}{Lemma}
\newtheorem{proposition}{Proposition}
\newtheorem{corollary}{Corollary}
\newtheorem{claim}{Claim}

\newcommand{\Hom}{\operatorname{Hom}}
\newcommand{\Ker}{\operatorname{Ker}}
\newcommand{\Coker}{\operatorname{Coker}}
\newcommand{\Ann}{\operatorname{Ann}}
\newcommand{\Spec}{\operatorname{Spec}}
\renewcommand{\longrightarrow}{\xrightarrow{\hspace*{0.7cm}}}

\title{Artin: Fields}
\author{James Pagan}
\date{March 2024}

% --------------------------------------------- %

\begin{document}

\maketitle
\tableofcontents
\newpage

% --------------------------------------------- %

\section{Fields}

A \textbf{field} is a commutative division ring. If $F \subseteq K$ are a pair of fields, we say $K$ is a \textbf{field extension} of $F$. This relation is denoted $K \, / \, F$; this is \textit{not} a quotient! Examples of fields are as follows:
\begin{enumerate}
  \item Subfields of $\mathbb{C}$ are called \textbf{number fields}. Any subfield of $\mathbb{C}$ contains the field $\mathbb{Q}$ of rational numbers. The most important number systems are \textbf{algebraic number fields}, whose elements are algebraic numbers.
  \item A \textbf{finite field} is a field that contains finitely many elements. Finite fields are gorgeous and colorful objects that obey beautiful, tight-knit properties.
  \item Extensions of the field $C(t)$ of rational functions are called \textbf{function fields}.
\end{enumerate}

% --------------------------------------------- %

\section{Algebraic and Transcendental Elements}

Let $K \, / \, F$ be a field extension and let $\alpha$ be an element of $K$. The element $\alpha$ is \textbf{algebraic over F} if is the root of a monic polynomial with coefficients in $F$ --- say, $f(\alpha) = 0$ for
\[
  f(x) \, = \, x^{n} + a_{n - 1}x^{n - 1} + \cdots + a_{0} \quad \text{for some} \quad a_{n - 1}, \ldots, a_{0} \in F,
\]
An element is \textbf{transcendental over F} if it is not algebraic. Both of these properties depend on the field $F$. Every element $\alpha \in F$ is algebraic over $F$ due to the monomial $x - \alpha$. We can elegantly describe this as a substitution homomorphism
\[
  \phi : F[x] \to X \quad \text{defined by} \quad x \leadsto \alpha.
\]
An element $\phi$ is transcendental if $\phi$ is injective and algebraic otherwise.

\begin{adjustwidth}{0.5cm}{}
  \begin{proposition}
    Let $\alpha \in K \, / \, F$ be an element of a field extension. The following conditions on a monic polynomial $f \in F[x]$ are equivalent:
    \begin{enumerate}
      \item $f$ is the unique monic polynomial of lowest degree in $F[x]$ with $\alpha$ as a root.
      \item $f$ is an irreducible element of $F[x]$ with $\alpha$ as a root.
      \item $f(\alpha) = 0$ and $(f)$ is a maximal ideal.
      \item If $g(\alpha) = 0$, then $f \mid g$.
    \end{enumerate}
  \end{proposition}
  \begin{proof}
    Since $F[x]$ is a Euclidean domain, the kernel of $\phi : F[x] \to K$ is a principal ideal generated by some polynomial $f$ of smallest degree. $f$ must be irreducible, or else a polynomial of smaller degree has a root at $\phi$; the other properties are easy to deduce.
  \end{proof}
\end{adjustwidth}
This polynomial is called the \textbf{minimal polynomial} of $\alpha$. Like before, the minimal polynomial depends on both $F$ and $\alpha$. The degree of the minimal polynomial of $\alpha$ is called the \textbf{degree} of $\alpha$. There are two distinct conversations at this point:
\begin{enumerate}
  \item The field $F(\alpha_{1}, \ldots, \alpha_{n})$ denotes the subfield of $K$ generated by $\alpha_{1}, \ldots, \alpha_{n}$.
  \[
    \text{$F(\alpha_{1}, \ldots, \alpha_{n})$ is the smallest subfield of $K$ that contains $F$ and $\alpha_{1}, \ldots, \alpha_{n}$}.
  \]
  \item The ring $F[\alpha_{1}, \ldots, \alpha_{n}]$ denotes the subring of $K$ generated by $\alpha_{1}, \ldots, \alpha_{n}$. The ring $F[\alpha]$ is isomorphic to the image of the substitution homomorphism $\phi : F[x] \to K$ as defined above.
\end{enumerate}
The field $F(\alpha)$ is isomorphic to the field of fractions of $F[\alpha]$. If $\alpha$ is transcendental, then $F[\alpha] \, \cong \, F[x]$ and $F(\alpha) \, \cong \, F(\alpha)$; otherwise,

\begin{adjustwidth}{0.5cm}{}
  \begin{proposition}
    Let $\alpha \in K \, / \, F$ be an element of a field extension which is algebraic over $F$. Let $f$ be the minimal polynomial of $\alpha$.
    \begin{enumerate}
      \item The canonical map $\phi : F[x] \, / \, (f) \to F[\alpha]$ an isomorphism.
      \item $F[\alpha]$ is a field, hence $F[\alpha] = F(\alpha)$.
      \item More generally, $F[\alpha_{1}, \ldots, \alpha_{n}] = F(\alpha_{1}, \ldots, \alpha_{n})$ if $\alpha_{1}, \ldots, \alpha_{n} \in K \, / \, F$ are algebraic.
    \end{enumerate}
  \end{proposition}
  \begin{proof}
    Let $\phi : F[x] \to K$ be the aforementioned substitution homomorphism. Then $F[x] \, / \, \Ker \phi \, \cong \, K$. By Proposition 1, the kernel of $\phi$ is a maximal ideal generated by the minimal polynomial $f$, which yields (1) and (2). As per (3), an induction argument proceeds something like
    \[
      F[\alpha_{1}, \ldots, \alpha_{n}] = F[\alpha_{1}, \ldots, \alpha_{n - 1}][\alpha_{n}] = F(\alpha_{1}, \ldots, \alpha_{n})[a_{k}] = F(\alpha_{1}, \ldots, \alpha_{n}).
    \]
    The omitted details are relatively easy to verify.
  \end{proof}
\end{adjustwidth}

The following proposition is a special case of one I omitted from Chapter 11.

\begin{adjustwidth}{0.5cm}{}
  \begin{proposition}
    Let $\alpha \in K \, / \, F$ be an algebraic element of a field extension. If $\deg \alpha = n$, then $\alpha_{1}, \ldots, \alpha_{n}$ is a basis for $F(\alpha)$ as a vector space over $F$.
  \end{proposition}
\end{adjustwidth}

A fundamental question is: given two elements $\alpha$ and $\beta$ --- or given their minimal polynomials --- when can one determine whether $\alpha$ and $\beta$ generate equal fields. Proposition three provides a necessary non-sufficient condition: that $\deg \alpha = \deg \beta$. The following proposition answers a special case.

% --------------------------------------------- %

\end{document}
