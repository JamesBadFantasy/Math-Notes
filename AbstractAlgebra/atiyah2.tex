\documentclass[11pt]{article}
\usepackage[T1]{fontenc}
\usepackage{geometry, changepage}
\usepackage{amsmath, amssymb, amsthm, bm}
\usepackage{physics}
\usepackage{hyperref}
\usepackage{tikz-cd}

\hypersetup{colorlinks=true, linkcolor=blue, urlcolor=cyan}
\setlength{\parindent}{0pt}
\setlength{\parskip}{5pt}

\newtheorem{theorem}{Theorem}
\newtheorem{lemma}{Lemma}
\newtheorem{claim}{Claim}
\newtheorem{corollary}{Corollary}
\newtheorem*{theorem*}{Theorem}
\newtheorem*{lemma*}{Lemma}
\newtheorem*{claim*}{Claim}

\newcommand{\Hom}{\operatorname{Hom}}
\newcommand{\Ker}{\operatorname{Ker}}
\newcommand{\Coker}{\operatorname{Coker}}
\newcommand{\Ann}{\operatorname{Ann}}
\renewcommand{\longrightarrow}{\xrightarrow{\hspace*{0.7cm}}}

\title{Atiyah-MacDonald: Modules}
\author{James Pagan}
\date{January 2024}

% --------------------------------------------- %

\begin{document}

\maketitle
\tableofcontents

\newpage

% --------------------------------------------- %

\section{Modules}

% --------------------------------------------- %

\subsection{Definition}

An \textbf{R-module} over a commutative ring $R$ is an abelian group $M$ (with operation written additively) endowed with a mapping $\mu : R \times M \to M$ (written multiplicatively) such that the following axioms are satisfied for all $x, y \in M$ and $a, b \in R$:
\begin{enumerate}
	\item $1x = x$;
	\item $(ab)x = a(bx)$;
	\item $a(x + y) = ax + ay$;
	\item $(a + b)x = ax + bx$.
\end{enumerate}

% --------------------------------------------- %

\subsection{Examples of Modules}

\begin{itemize}
	\item If $R$ is a ring, $R[x]$ is a module.
	\item All ideals $\mathfrak{a}$ of $R$ are $R$-modules using the same additive and multiplicative operations as $R$ --- in particular $R$ itself is an $R$-module.
	\item If $R$ is a field, $R$-modules are $R$-vector spaces. In fact, the axioms above are identical to the vector axioms, defined over commutative rings instead of fields.
	\item Abelian groups $G$ are precisely the modules over $\mathbb{Z}$.
\end{itemize}

% --------------------------------------------- %

\subsection{Homomorphisms of Modules}

A map $f: M \to N$ between two $R$-modules $M$ and $N$ is an \textbf{R-module homomorphism} (or is $R$-linear) if for all $a \in R$ and $x, y \in M$,
\begin{align*}
	f(x + y) & = f(x) + f(y) \\
	f(ax)    & = a f(x).
\end{align*}
Thus, an $R$-module homomorphism $f$ is a homomorphism of abelian groups that commutes with the action of each $a \in R$. If $R$ is a field, an $R$-module homomorphism is a linear transformation. A bijective $R$-homomorphism is called an $R$-isomorphism.

\newpage

The set $\Hom_{R}(M, N)$ denotes the set of all $R$-module homomorphisms from $M$ to $N$, and is a module if we define the following operations for $a \in R$ and $f, g \in \Hom_{R}(M, N)$:
\begin{align*}
	(f + g)(x) & = f(x) + g(x) \\
	(rf)(x)    & = r f(x).
\end{align*}
We denote $\Hom_{R}(M, N)$ by $\Hom(M, N)$ if there is no ambiguity about the commutative ring $R$.

\begin{adjustwidth}{1cm}{}
	\begin{theorem}
		$\Hom_{R}(R, M) \cong M$
	\end{theorem}
	\begin{proof}
		Consider the mapping $\phi : \Hom_{R}(R, M)$ defined by $\phi(f) = f(1)$ for $f \in \Hom_{R}(R, M)$. See that for all $f, g \in \Hom_{R}(R, M)$,
		\[
			\phi(f + g) = (f + g)(1) = f(1) + g(1) = \phi(f) + \phi(g)
		\]
		and for all $a \in R$,
		\[
			\phi(af) = (af)(1) = a f(1) = a \phi(f),
		\]
		so $\phi$ is an $R$-homomorphism. Now, suppose that $\phi(f) = \phi(g)$; then $f(1) = g(1)$. For all $a \in R$, we deduce that $f(a) = a f(1) = a g(1) = g(a)$, so $h = g$ and $\phi$ is injective.

		For all $x \in M$, consider the mapping $h : R \to M$ defined by $h(a) = ax$. As $h(ab) = abx = a h(b)$ for all $a, b \in R$, and
		\[
			h(a + b) = (a + b) h(1) = a h(1) + b h(1) = h(a) + h(b),
		\]
		we find that $h \in \Hom_{R}(R, M)$. Thus $\phi$ is surjective, as $\phi(h) = h(1) = m$, which establishes the desired isomrphism.
	\end{proof}
\end{adjustwidth}

Further observe that homomorphisms $u : M' \to M$ and $v : N \to N''$ induce mappings $\bar{u} : \Hom(M, N) \to \Hom(M', N)$ and $\bar{v} : \Hom(M, N) \to \Hom(M, N'')$ defined for $f \in \Hom(M, N)$ as follows
\[
	\bar{u}(f) = f \circ u \qquad \text{and} \qquad \bar{v}(f) = v \circ f.
\]
These identities translate between the landscapes of $R$-modules and homomorphisms between them. We will elaborate on this duality in Section 5.2.

\newpage

% --------------------------------------------- %

\section{Submodules and Quotient Modules}

% --------------------------------------------- %

\subsection{Definition}

A \textbf{submodule} $M'$ of $M$ is an abelian subgroup of $M$ closed under multiplication by elements of the commutative ring $R$. The following proof outlines a construction of \textbf{quotient modules}:

\begin{adjustwidth}{1cm}{}
	\begin{theorem}
		The abelian quotient group $M / M'$ is an $R$-module under the opreation $r(x + M') = rx + M'$.
	\end{theorem}
	\begin{proof}
		We must perform four rather routine calculations:
		\begin{enumerate}
			\item For all $x \in M$, we have that $1(x + M') = 1x + M' = x + M'$.
			\item For all $r, s \in R$ and $x \in M$, we have that $r(s(x + M')) = r(sx + M') = rsx + M' = (rs)(x + M')$.
			\item For all $r, s \in R$ and $x \in M$, we have that $(r + s)(x + M') = (r + s)x + M' = (rx + sx) + M' = (rx + M') + (sx + M') = r(x + M') + s(x + M')$.
			\item For all $r \in R$ and $x, y \in M$, we have that $r((x + M') + (y + M')) = r((x + y) + M') = r(x + y) + M' = (rx + M') + (ry + M') = r(x + M') + r(y + M)'$.
		\end{enumerate}
		Therefore, $M / M'$ is an $R$-module.
	\end{proof}
\end{adjustwidth}

% --------------------------------------------- %

\subsection{Assorted Submodules}

Let $f : M \to N$ be an $R$-module homomorphism. Then the \textbf{kernel} and \textbf{image} of $f$ are defined as follows:
\[
	\Ker f = \{ x \mid x \in M, f(x) = 0 \} \qquad \text{and} \qquad \Im f = f(M),
\]
and are submodules of $M$ and $N$ respectively. The \textbf{cokernel} of $f$ is defined as follows:
\[
	\Coker f = N \,/\, \Im f.
\]
If $M'$ is a submodule of $M$ such that $M' \subseteq \Ker f$, then $f$ induces a natural homomorphism $\bar{f} : M / M' \to N$ defined by $\bar{f}(x + M') = f(x)$. The kernel of $\bar{f}$ is $\Ker f \,/\, M'$ --- the distinct cosets of $M'$ in the kernel of $f$. Taking $M' = \Ker f$ yields the First Isomorphism Theorem for $R$-modules:
\[
	M \,/\, \Ker f \cong \Im f.
\]

% --------------------------------------------- %

\section{Operations on Submodules}

% --------------------------------------------- %

\subsection{Sums, Intersections, Products}

Let $M$ be an $R$-module with submodules $M_{1}, \ldots, M_{n}$. We can consider two crucial operations on these submodules:
\begin{enumerate}
	\item \textbf{Sum}: The sum $M_{1} + \cdots + M_{n}$ is the set of all sums $m_{1} + \cdots + m_{n}$, where $m_{i} \in M_{i}$ ($i \in \{ 1, \ldots, n \}$). It is the smallest submodule of $M$ that contains all $M_{1}, \ldots, M_{n}$.
	\item \textbf{Intersecrtion}: The intersection $M_{1} \cap \cdots \cap M_{n}$ is the largest submodule of $M$ that is contained inside each $M_{1}, \ldots, M_{n}$.
\end{enumerate}

The following result proves the Third Isomorphism Theorem for modules:

\begin{adjustwidth}{1cm}{}
	\begin{theorem}
		If $N \subseteq M \subseteq L$ are $R$-modules, then
		\[
			(L \,/\, N) \,/\, (M \,/\, N) \cong (L \,/\, M).
		\]
	\end{theorem}
	\begin{proof}
		Define the mapping $f : L \,/\, N \to L \,/\, M$ by $f(x + N) = x + M$. First, we prove $\phi$ is well-defined; if $x + N = y + N$ for $x, y \in L$, then $x - y \in N \subset M$, so $x + M = y + M$. Therefore,
		\[
			x + N = y + N \implies f(x + N) = f(y + N).
		\]
		It is trivial to verify that $f$ is a surjective $R$-homomorphism with kernel $M \,/\, N$ (the cosets of $N$ contained within $M$). We find by the First Isomorphism Theorem that $(L \,/\, N) \,/\, (M \,/\, N) \cong L \,/\, M$.
	\end{proof}
\end{adjustwidth}

For an ideal $\mathfrak{a}$ of $R$ and an $R$-module $M$, we define the \textbf{product} $\mathfrak{a} M$ as all finite sums $a_{1}x_{1} + \cdots + a_{n}x_{n}$ for $a_{i} \in \mathfrak{a}$ and $x_{i} \in M$  for each $i \in \{ 1, \ldots, n \}$. It is a submodule of $M$.

% --------------------------------------------- %

\subsection{The Annihilator}

If $N$ and $P$ are $R$-submodules of $M$, we define $(N : P)$ to be the set of all $r \in R$ such that $rP \subseteq N$.

\begin{adjustwidth}{1cm}{}
	\begin{theorem}
		$(N : P)$ is an ideal of $R$.
	\end{theorem}
	\begin{proof}
		If $r, s \in (N : P)$, then $rP, sP \subseteq N$; we must have that $rP + sP \subseteq N$. $(N : P)$ is nonempty, as $0P = (0) \subseteq N$; and clearly if $rP \in N$, then $-rP \in N$ as $N$ is an abelian group. Clearly $(N : P)$ satifies the multiplicative condition too.
	\end{proof}
\end{adjustwidth}

The \textbf{annihilator} of a module $M$ is $(0 : M)$, the ideal of all $r$ such that $rM = 0$, and is denoted $\Ann M$. If $I \subseteq \Ann M$, we may regard $M$ as an $R / I$-module. In particular, observe that if $\bar{r} \in R / I$, then $\bar{r}x$ is invariant by elements of $\bar{r}$; if $r_{1}$ and $r_{2}$ are any two elements of $\bar{r}$, then $r_{i} = r_{2} + i$ for some $i \in I \subseteq \Ann M$, so $r_{i}x = (r_{2} + i)x = r_{2}x + ix = r_{2}x$.

An $R$-module is \textbf{faithful} if $\Ann M = 0$. The annihalator of $R$ may change depending on the ring --- if $\Ann M = I$, then $M$ is faithful as an $R / I$ module.

\begin{adjustwidth}{1cm}{}
	\begin{theorem}
		If $M_{1}$ and $M_{2}$ are submodules of $M$, then $\Ann(M_{1} + M_{2}) = \Ann M_{1} \cap \Ann M_{2}$
	\end{theorem}
	\begin{proof}
		If $r \in \Ann M_{1} \cap \Ann M_{2} $, then for all $x_{1} + x_{2} \in M_{1} + M_{2}$, $r$ annihalates $x_{1}$ and $x_{2}$, so $r(x_{1} + x_{2}) = 0.$ Thus $r \in \Ann(M_{1} + M_{2})$.

		Now, suppose $r \notin \Ann M_{1} \cap \Ann M_{2} $. Then $r$ must fail to annnihilate an element in either $M_{1}$ or $M_{2}$ (or both) --- without loss of generality, let there exist $x_{1} \in M_{1}$ such that $rx_{1} \ne 0$.

		Then as $rx_{1} \in M_{1} + M_{2}$, we find that $r \notin \Ann(M_{1} + M_{2})$. By contraposition, we find that if $r \in \Ann(M_{1} + M_{2})$, then $r \in \Ann M_{1} \cap \Ann M_{2}$. This completes the proof.
	\end{proof}
\end{adjustwidth}

% --------------------------------------------- %

\subsection{Direct Sum and Product}

If $M$ and $N$ are $R$-modules, their \textbf{direct sum} $M \oplus N$ is the set of all pairs $(x, y)$ (with $x \in M$ and $y \in N$) endowed with the natural operations:
\begin{align*}
	(x_{1}, y_{1}) + (x_{2}, y_{2}) & = (x_{1} + x_{2}, y_{1} + y_{2}) \\
	r(x, y)                         & = (rx, ry)
\end{align*}
More generally, for a family of $R$-modules $(M_{j})_{j \in J}$ , we define their \textbf{direct sum} $\bigoplus_{j \in J} M_{j}$ as the families $(x_{j})_{j \in J}$ such that $x_{j} \in M_{j}$ for all $j \in J$, with the restriction that only finitely many $x_{j}$ are nonzero.

If we allow infinitely many $x_{j}$ to be nonzero, we attain the family's \textbf{direct product} $\prod_{j \in J} M_{j}$. Direct sums and direct products are equivalent if $J$ is finite, but not otherwise.

\newpage

\subsection{Direct Sums on Rings}

Suppose that a commutative ring $R$ is a direct product $R = R_{1} \times \cdots \times R_{n}$. Then $R$ has $n$ ideals of the form
\[
	\mathfrak{a}_{j} = (0, \ldots, 0, r_{j}, 0, \ldots, 0),
\]
where $a_{i} \in R_{i}$ for each $i \in \{ 1, \ldots, n \}$. Viewing this relation in terms of modules,
\[
	R \cong \mathfrak{a}_{1} \oplus \cdots \oplus \mathfrak{a}_{n}
\]
by the isomorphism $f(r_{1}, \ldots, r_{n}) = ((r_{1}, 0, \ldots), \ldots, (\ldots, 0, r_{n}))$.

Similarly, suppose $R$ is a commutative ring with ideals $\mathfrak{a}_{1}, \ldots, \mathfrak{a}_{n}$, and
\[
	R = \mathfrak{a}_{1} \oplus \cdots \oplus \mathfrak{a}_{n}.
\]
Then define $\mathfrak{b}_{i} = \bigoplus_{j \ne i} \mathfrak{a}_{j}$ for each $i \in \{ 1, \ldots, n \}$. We find that
\[
	R \cong (R \,/\, \mathfrak{b}_{1}) \times \cdots \times (R \,/\, \mathfrak{b}_{n})
\]
by the isomorphism $f(a_{1}, \ldots, a_{n}) = (a_{1} + \mathfrak{b}_{1}, \ldots, a_{n} + \mathfrak{b}_{n})$. We conclude that direct sums of ideals and direct products of subrings are dual notions. This is a critical strength of modules: it treats ideals and rings on equal footing, resulting in clarity and simplicity.

% --------------------------------------------- %

\section{Finitely Generated Modules}

% --------------------------------------------- %

\subsection{Definition}

An $R$-module $M$ is said to be \textbf{finitely generated} if there exist a set of \textbf{generators} $x_{1}, \ldots, x_{n}$ such that $M = Rx_{1} + \cdots + Rx_{n}$, where $Rx_{i}$ denotes the set $\{ rx_{i} \mid r \in R \}$ ($i \in \{ 1, \ldots, n \}$). In the following theorem, we denote $R \oplus \cdots \oplus R$ by $R^{n}$.

\begin{adjustwidth}{1cm}{}
	\begin{theorem}
		$M$ is a finitely-generated $R$-module if and only if $M$ is isomorphic to a quotient of $R^{n}$.
	\end{theorem}
	\begin{proof}
		Suppose $M$ is a finitely-generated $R$-module. Let $x_{1}, \ldots, x_{n} \in M$ generate $M$, and define $f : R^{n} \to M$ by $f(r_{1}, \ldots, r_{n}) = r_{1}x_{1} + \cdots + r_{n}x_{n}$. Then if $(r_{1}, \ldots, r_{n}), (s_{1}, \ldots, s_{n}) \in R^{n}$,
		\begin{align*}
			f(r_{1} + s_{1}, \ldots, r_{n} + s_{n}) &= (r_{1} + s_{1}) x_{1} + \cdots(r_{n} + s_{n}) \\
			&= (r_{1}x_{1} + \cdots + r_{n}x_{n}) + (s_{1}x_{1} + \cdots + s_{1}x_{1}) \\
			&= f(r_{1}, \ldots, r_{n}) + f(s_{1}, \ldots, s_{n}),
		\end{align*}
		and if $s \in R$,
		\[
			f(s r_{1}, \ldots, s r_{n}) = s r_{1} x_{1} + \cdots + s r_{n} x_{n} = s f(r_{1}, \ldots, r_{n}).
		\]
		Thus $f$ is a homomorphism; its surjectivity follows from the generation of $M$ by $x_{1}, \ldots, x_{n}$. Then if we set $\Ker f = \mathfrak{a}$, the First Isomorphism Theorem yields that
		\[
			R^{n} \,/\, \mathfrak{a} \, \cong \, M.
		\]
		Now, suppose that $R^{n} \,/\, \mathfrak{a} \cong M$ for some submodule $\mathfrak{r}$ of $R^{n}$ by the mapping $f$. The natural homomorphism $g : R^{n} \to R^{n} \,/\, \mathfrak{r}$ defined by $g(r_{1}, \ldots r_{n}) = (r_{1}, \ldots, r_{n}) + \mathfrak{r}$ is surjective, so $f \circ g : R^{n} \to M$ is a surjective $R$-module homomorhism.

		Denote $x_{i} = (f \circ g) (0, \ldots, 0, 1, 0, \ldots 0)$ for $i \in \{ 1, \ldots, n \}$. Then for all $x \in M$, there exist $r_{1}, \ldots, r_{n}$ such that
		\begin{align*}
			x &= (f \circ g)(r_{1}, \ldots, r_{n}) \\
			&= (f \circ g)(r_{1}, 0, \ldots, 0) + \cdots + (f \circ g)(0, \ldots, 0, r_{n}) \\
			&= r_{1} (f \circ g)(1, 0, \ldots, 0) + \cdots + r_{n} (f \circ g)(0, \ldots, 0, 1) \\
			&= r_{1} x_{1} + \cdots r_{n} x_{n}.
		\end{align*}
		We conclude that $x_{1}, \ldots, x_{n}$ generate $M$.
	\end{proof}
\end{adjustwidth}

The following proof is a transcription from Atiyah-MacDonald:

\begin{adjustwidth}{1cm}{}
	\begin{theorem}
		Let $M$ be a finitely-generated $R$-module, let $\mathfrak{a}$ be an ideal of $R$, and let $f : M \to M$ be an $R$-module endomorphism of $M$ such that $f(M) \subset \mathfrak{r}M$. Then $\phi$ satisfies an equation of the form
		\[
			f^{n} + r_{n - 1} f^{n - 1} + \cdots + r_{0} f^{0} = 0,
		\]
		where the $r_{i}$ are in $\mathfrak{a}$.
	\end{theorem}
	\begin{proof}
		Let $x_{1}, \ldots, x_{n}$ generate $M$. Then each $f(x_{i}) \in \mathfrak{a} M$, so we may define $r_{ij} \in \mathfrak{r}$ for $i, j \in \{ 1, \ldots, n \}$ by $f(x_{i}) = \sum_{j = 1}^{n} r_{ij}x_{j}$. This equation may be equivalently written for each $i \in \{ 1, \ldots, n \}$ as
		\[
			\sum\limits_{j = 1}^{n} (\delta_{ij} f - a_{ij} f^{0}) x_{j} = 0,
		\]
		where $\delta_{ij}$ is the Kronecker delta. By multiplying the left by the adjoint of the matrix $\delta_{ij} f - a_{ij} f^{0}$, it follows that $\det(\delta_{ij} f - a_{ij} f^{0})$ annihilates each $x_{ij}$ --- hence, it is the zero endomorphism of $M$. Expanding out the determinant yields an equation of the required form.
	\end{proof}
\end{adjustwidth}

\begin{adjustwidth}{1cm}{}
	\begin{corollary}
		Let $M$ be a finitely-generated $R$-module and let $\mathfrak{a}$ be an ideal of $R$ such that $\mathfrak{r} M = M$. Then there exists $r \in (1 + \mathfrak{r})$ such that $rM = 0$.
	\end{corollary}
	\begin{proof}
		Consider Theorem 7 under the identity transformation --- namely $f^{0}$ for some nonzero endomorphism $f$. Then there exist $r_{n - 1}, \ldots, r_{0} \in \mathfrak{a}$ such that
		\begin{align*}
			0 &= f^{n} + r_{n - 1} f^{n - 1} + \cdots + r_{0} f^{0} \\
			&= f^{0} + r_{n - 1} f^{0} + \cdots + r_{0} f^{0} \\
			&= (1 + r_{n - 1} + \cdots + r_{0}) f^{0}.
		\end{align*}
		Setting $r = 1 + r_{n - 1} + \cdots + r_{0}$ yields that $r f^{0}$ is the zero endomorphism, so $r f^{0}(x) = rx = 0$ for each $x \in M$. We conclude that $rM = 0$.
	\end{proof}
\end{adjustwidth}

A \textbf{free R-module} $M$ is a module such that $M \cong \bigoplus_{j \in J} M_{j}$, where $M_{j} \cong R$ for each $j \in J$. A finitely generated module $M$ is therefore free if $M$ is isomorphic to $R^{n}$ itself, in which case $M$ has a ``basis''.

\subsection{Relation to the Jacobson Radical}

% --------------------------------------------- %

The following lemma is called \textbf{Nakayama's Lemma} and has two proofs:

\begin{adjustwidth}{1cm}{}
	\begin{lemma}[Nakayama's Lemma]
		Let $M$ be a finitely generated $R$-module and $\mathfrak{a}$ an ideal of $R$ contained in the Jacobson radical $\mathfrak{J}$ of $R$. Then $\mathfrak{r}M = M$ implies $M = 0$.
	\end{lemma}
	\begin{proof}
		By Corollary 1, there exists $r \in 1 + j$ for $j \in \mathfrak{a}$ such that $rM = 0$. By the properties of the Jacobson radical, $1 + j$ (and thus $r$) is a unit. Hence,
		\[
			M = (r^{-1}r)M = r^{-1}(rM) = r^{-1}(0) = 0,
		\]
		as desired.
	\end{proof}
	\begin{proof}
		Suppose for contradiction that $\mathfrak{a}M = M$ and $M \ne 0$; let $x_{1}, \ldots, x_{n}$ be a set of generators of $M$ of shortest length. Then $x_{n} \in \mathfrak{r}M$, so $x_{n}$ satisfies an equation of the form
		\[
			 x_{n} = r_{1} x_{1} + \cdots + r_{n} x_{n}
		\]
		for $r_{1}, \ldots, r_{n} \in \mathfrak{a}$. None of these are $1$, since the Jacobson Radical is the intersection of maximal ideals, of which none contain $1$. We can therefore eliminate $x_{n}$;
		\[
			(1 - r_{n})x_{n} = r_{1}x_{1} + \cdots + r_{n - 1}x_{n - 1}.
		\]
		Since $r_{n}$ is in the Jacobson radical, $(1 - r_{n})$ is a unit. We may therefore multiply this equation by the inverse of $(1 - r_{n})$ to express $x_{n}$ as a linear combination of $x_{1}, \ldots, x_{n - 1}$. 

		Then $x_{1}, \ldots, x_{n - 1}$ generate $M$. This contradicts the minimality of the length of $x_{1}, \ldots, x_{n}$; we conclude that if $\mathfrak{a} M = M$, then $M = 0$.
	\end{proof}
\end{adjustwidth}

\begin{adjustwidth}{1cm}{}
	\begin{corollary}
		Let $M$ be a finitely-generated $R$-module, $N$ a submodule of $M$, and $\mathfrak{a}$ an ideal of $R$ contained in the Jacobson radical $\mathfrak{R}$ of $R$. Then $M = IM + N$ implies $M = N$.
	\end{corollary}
	\begin{proof}
		Suppose $M = \mathfrak{a} M + N$. Realize that for \textit{all} ideals $\mathfrak{r}$ of $R$,
		\[
			\mathfrak{a}(M \,/\, N) = (\mathfrak{r}M) \,/\, N = (\mathfrak{r}M + N) \,/\, N.
		\]
		Therefore, $\mathfrak{a}(M \,/\, N) = M \,/\, N$. By Nakayama's Lemma, $M \,/\, N = 0$, so $M = N$.
	\end{proof}
\end{adjustwidth}

% --------------------------------------------- %

\subsection{Relation to Local Rings}

Consider a finitely-generated module $M$ over a local ring $R$ with maximal ideal $\mathfrak{m}$. 

\begin{adjustwidth}{1cm}{}
	\begin{theorem}
		The elements of the $R$-module $M \,/\, \mathfrak{m} M$ and the $(R \,/\, \mathfrak{m})$-module $M$ are identical.
	\end{theorem}
	\begin{proof}
		It is relatively simple to verify that the function $f : M \,/\, \mathfrak{m} M \to M$ defined by
		\[
			f(r_{1} x_{1} + \cdots + r_{n} x_{n} + \mathfrak{m}M) = (r_{1} + \mathfrak{m})x_{1} + \cdots + (r_{n} + \mathfrak{m})x_{n}
		\]
		is bijective. It further satisfies $f(x + y) = f(x) + f(y)$ --- and $f(rx) = s f(x)$, where $s$ is the image of $r$ under the canonical surjection $\phi : R \to R \,/\, \mathfrak{m}$.
	\end{proof}
\end{adjustwidth}

This realization arises naturally, as $\mathfrak{m}$ annihalates the quotient module $M \,/\, \mathfrak{m} M$. Since $R \,/\, \mathfrak{m}$ is a field, $M \,/\, \mathfrak{m} M$ is actually a vector space.

\begin{adjustwidth}{1cm}{}
	\begin{theorem}
		Let $M$ be a module over a local ring. Then $x_{1}, \ldots, x_{n} \in M$ generate $M$ if and only if the images of $x_{1}, \ldots, x_{n}$ span the the vector space $M \,/\, \mathfrak{m} M$.
	\end{theorem}
	\begin{proof}
		Let $f : M \to M \,/\, \mathfrak{m} M$ be the natural homomorphism. Then
		\begin{align*}
			\text{$x_{1}, \ldots, x_{n}$ generate $M$} &\iff \text{$x_{1}, \ldots, x_{n}$ generate $M \,/\, \mathfrak{m} M$} \\
			&\iff \text{$x_{1}, \ldots, x_{n}$ are a basis of $M \,/\, \mathfrak{m} M$ over $R \,/\, \mathfrak{m}$}.
		\end{align*}
		The ommitted details are relatively simple to verify.
	\end{proof}
\end{adjustwidth}


\newpage

% --------------------------------------------- %

\section{Exact Sequences}

% --------------------------------------------- %

\subsection{Definition}

A sequence of $R$-modules and $R$-homomorphisms
\begin{equation}
	\cdots \longrightarrow M_{i - 1} \stackrel{f_{i}}{\longrightarrow} M_{i} \stackrel{f_{i + 1}}{\longrightarrow} M_{i + 1} \xrightarrow{\hspace*{1cm}} \cdots 
\end{equation}
is \textbf{exact} at $M_{i}$ if $\Im f_{i} = \Ker f_{i + 1}$. The sequence is \textbf{exact} if it is exact at each $M_{i}$. Such sequences induce a wealth of identities relating the modules and their images, kernels, and cokernels. Three examples of exact sequences are as follows:
\begin{align*}
	0 \to M' \stackrel{f}{\to} M \text{ is exact } &\iff \text{$f$ is injective, so $M' \cong \Im f$} \\
	M \stackrel{g}{\to} M'' \to 0 \text{ is exact } &\iff \text{$g$ is surjective, so $M \,/\, \Ker g \cong M'$.} \\
	0 \to M' \stackrel{f}{\to} M \stackrel{g}{\to} M'' \to 0 \text{ is exact} &\iff \text{$f$ is injective, $g$ is surjective, and} \\
	& \quad \qquad \text{$\Coker f = M \,/\, \Ker g \cong M'$}.
\end{align*}

An exact sequence of this third type is called a \textbf{short exact sequence}. Any long exact sequence, like that in equation (1), can be broken into numerous short exact sequences:
\[
	0 \longrightarrow \Coker f_{i - 1} \stackrel{f_{i}'}{\longrightarrow} M_{i} \stackrel{f_{i + 1}}{\longrightarrow} \Im f_{i + 1} \longrightarrow 0,
\]
where $f_{i}'$ is defined as $f_{i}'(x + \Ker f_{i}) = f_{i}(x)$. To save time, we will often write that $0 \to N_{i} \to M_{i} \to N_{i + 1} \to 0$ for some modules $N_{i}$ and $N_{i + 1}$ instead.

% --------------------------------------------- %

\subsection{Exact Sequences of Homomorphisms}

\begin{adjustwidth}{1cm}{}
	\begin{theorem}
		The sequence
		\begin{equation}
			M' \stackrel{u}{\longrightarrow} M \stackrel{v}{\longrightarrow} M'' \longrightarrow 0
		\end{equation}
		of $R$-modules and $R$-module homomorphisms is exact if and only if for all $R$-modules $N$, the sequence
		\begin{equation}
			0 \longrightarrow \Hom(M'', N) \stackrel{\bar{v}}{\longrightarrow} \Hom(M, N) \stackrel{\bar{u}}{\longrightarrow} \Hom(M', N)
		\end{equation}
		is exact, where $\bar{u}$ and $\bar{v}$ are as in Section 1.3.
	\end{theorem}
	\newpage
	\begin{proof}
		The exactness of (2) is equivalent to $\Im u = \Ker v$ and the surjectivity of $v$, while the exactness of (3) is equivalent to the injectivity of $\bar{v}$ and $\Im \bar{v} = \Ker \bar{u}$.
		\begin{adjustwidth}{1cm}{}
			\begin{claim}
				$v$ is surjective if and only if $\bar{v}$ is injective.
			\end{claim}
			\begin{proof}\renewcommand{\qedsymbol}{}
				Suppose that $v$ is surjective, and suppose $f \in \Ker \bar{v}$ for any $N$. Then $f$ is in $\Hom(M, M'')$ and
				\[
					\bar{v}(f) = f \circ v = 0.
				\]
				Since $v$ is surjective, $f$ must map the entirety of $M$ to zero; thus $f = 0$, and $\bar{v}$ is injective.

				Now, suppose that $v$ is not surjective. Then set $N = M'' \,/\, \Im v$ and let $f \in \Hom(M'', N)$ be the natural homomorphism; both are nonzero. Then
				\[
					\bar{v}(f)(M) = (f \circ v)(M) = f(\Im v) = 0.
				\]
				Then $f$ is a nonzero element of $\Ker \bar{v}$, so $\bar{v}$ is not surjective.
			\end{proof}
		\end{adjustwidth}
		The second claimed biconditional relation requires proof of the following claim.
		\begin{adjustwidth}{1cm}{}
			\begin{claim}
				$\Im u = \Ker v$ for surjective $v$ implies that $\Im \bar{v} = \Ker \bar{u}$.
			\end{claim}
			\begin{proof}\renewcommand{\qedsymbol}{}
				Suppose $\Im u = \Ker v$, so $v \circ u = 0$. Let $f \in \Im \bar{v}$; then there exists $g$ such that $\bar{v}(g) = g \circ v = f$. We conclude that
				\[
					\bar{u}(f) = f \circ u = g \circ v \circ u = g \circ 0 = 0,
				\]
				so $f \in \Ker \bar{u}$. Now, suppose $f \in \Ker \bar{u}$, so $\bar{u}(f) = f \circ u = 0$. Since $v$ is surjective, $m'' \in M''$ implies the existence of $m \in M$ such that $v(m) = m''$. Then define $g \in \Hom(M'', N)$ such that $g(m'') = f(m)$; it is relatively easy to demonstrate that $g$ is well-defined and a homomorphism. Thus, $f = g(u) = \bar{u}(g)$, so $f \in \Im \bar{v}$, so $\Im \bar{v} = \Ker \bar{u}$.
			\end{proof}
		\end{adjustwidth}
		If we suppose $\Im \bar{v} = \Ker \bar{u}$, then $\bar{u} \circ \bar{v} = 0$ --- that is, $v \circ u \circ f = 0$ for all $f \in \Hom(M'', N)$. This is equivalent to $\Im u = \Ker v$ in the start of Claim 2, with some added work to demonstrate that $\Im u \supseteq \Ker v$.
	\end{proof}
\end{adjustwidth}

Really, the natural language for these proofs is Homosexual Algebra and Abelian categories. Thus we will state the lemmas from this chapter without proof.

\newpage

The following theorem is asserted without proof, because I value my sanity.

\begin{adjustwidth}{1cm}{}
  \begin{theorem}
		The sequence
		\[
			0 \longrightarrow N' \stackrel{u}{\longrightarrow} N \stackrel{v}{\longrightarrow} N''
    \]
		of $R$-modules and $R$-module homomorphisms is exact if and only if for all $R$-modules $N$, the sequence
		\[
			0 \longrightarrow \Hom(M, N') \stackrel{\bar{u}}{\longrightarrow} \Hom(M, N) \stackrel{\bar{v}}{\longrightarrow} \Hom(M, N'')
    \]
		is exact, where $\bar{u}$ and $\bar{v}$ are as in Section 1.3.
  \end{theorem}
\end{adjustwidth}

The following theorem is called the Snake Lemma, a special case of exact homology sequences in Homosexual Algebra: 

\begin{adjustwidth}{1cm}{}
  \begin{theorem}[Snake Lemma]
    Suppose that
    \[ \begin{tikzcd}
      0 \arrow[r] & M' \arrow[r, "u"] \arrow[d, "f'"] & M \arrow[r, "v"] \arrow[d, "f"] & M'' \arrow[r] \arrow[d, "f''"] & 0 \\
      0 \arrow[r] & N' \arrow[r, "u'"]                & N \arrow[r, "v'"]               & N'' \arrow[r]                  & 0
    \end{tikzcd} \]
    is a commutative diagram of $R$-modules and homomorphisms, with the rows exact. Then there exists an exact sequence
    \[ \begin{tikzcd}
      0 \arrow[r] & \Ker(f') \arrow[r, "\bar{u}"] & \Ker(f) \arrow[r, "\bar{v}"]     & \Ker(f'') \arrow[ld, "d"']      &                       &   \\
                  &                               & \Coker(f') \arrow[r, "\bar{u}'"] & \Coker(f) \arrow[r, "\bar{v}'"] & \Coker(f'') \arrow[r] & 0
    \end{tikzcd} \]
    in which $\bar{u}$ and $\bar{v}$ are restrictions of $u$ and $v$, and in which $\bar{u}'$ and $\bar{v}'$ induced by $u'$ and $v'$. The \textbf{boundary homomorphism} $d$ is defined in Atiyah-MacDonald page 23.
  \end{theorem}
\end{adjustwidth}

We encourage the reader to adopt these results on faith. Homosexual Algebra is a complex subject of math that deserves its own set of notes.

% --------------------------------------------- %

\end{document}
