\documentclass[11pt]{article}
\usepackage[T1]{fontenc}
\usepackage{geometry, changepage}
\usepackage{amsmath, amssymb, amsthm, bm}
\usepackage{physics}
\usepackage{hyperref}
\usepackage{tikz-cd}

\hypersetup{colorlinks=true, linkcolor=blue, urlcolor=cyan}
\setlength{\parindent}{0pt}
\setlength{\parskip}{6pt}

\newtheorem{theorem}{Theorem}
\newtheorem{lemma}{Lemma}
\newtheorem{claim}{Claim}
\newtheorem{corollary}{Corollary}
\newtheorem*{theorem*}{Theorem}
\newtheorem*{lemma*}{Lemma}
\newtheorem*{claim*}{Claim}

\newcommand{\Hom}{\operatorname{Hom}}
\newcommand{\Ker}{\operatorname{Ker}}
\newcommand{\Coker}{\operatorname{Coker}}
\newcommand{\Ann}{\operatorname{Ann}}
\renewcommand{\longrightarrow}{\xrightarrow{\hspace*{0.7cm}}}

\newcommand{\nsg}{\mathrel{\lhd}}

\title{Atiyah-MacDonald: Modules}
\author{James Pagan}
\date{January 2024}

% --------------------------------------------- %

\begin{document}

\maketitle
\tableofcontents

\newpage

% --------------------------------------------- %

\section{Modules}

% --------------------------------------------- %

\subsection{Definition}

An \textbf{R-module} over a commutative ring $R$ is an abelian group $M$ (with operation written additively) endowed with a mapping $\mu : R \times M \to M$ (written multiplicatively) such that the following axioms are satisfied for all $x, y \in M$ and $a, b \in R$:
\begin{enumerate}
	\item $1x = x$;
	\item $(ab)x = a(bx)$;
	\item $a(x + y) = ax + ay$;
	\item $(a + b)x = ax + bx$.
\end{enumerate}

% --------------------------------------------- %

\subsection{Examples of Modules}

\begin{itemize}
	\item If $R$ is a ring, $R[x]$ is a module.
	\item All ideals $\mathfrak{a} \subseteq R$ are $R$-modules using the same additive and multiplicative operations as $R$ --- in particular $R$ itself is an $R$-module.
	\item If $R$ is a field, $R$-modules are $R$-vector spaces. In fact, the axioms above are identical to the vector axioms, defined over commutative rings instead of fields.
	\item Abelian groups $G$ are precisely the modules over $\mathbb{Z}$.
\end{itemize}

% --------------------------------------------- %

\subsection{R-Module Homomorphisms}

A map $f: M \to N$ between two $R$-modules $M$ and $N$ is an \textbf{R-module homomorphism} (or is \textbf{R-linear}) if for all $a \in R$ and $x, y \in M$,
\begin{align*}
	f(x + y) & = f(x) + f(y) \\
	f(ax)    & = a f(x).
\end{align*}
Thus, an $R$-module homomorphism $f$ is a homomorphism of abelian groups that commutes with the action of each $a \in R$. If $R$ is a field, an $R$-module homomorphism is a linear map. A bijective $R$-homomorphism is called an $R$-isomorphism.

The set $\Hom_{R}(M, N)$ denotes the set of all $R$-module homomorphisms from $M$ to $N$, and is a module if we define the following operations for $a \in R$ and $f, g \in \Hom_{R}(M, N)$:
  \begin{align*}
	(f + g)(x) & = f(x) + g(x) \\
	(af)(x)    & = a f(x).
\end{align*}
We denote $\Hom_{R}(M, N)$ by $\Hom(M, N)$ if the ring $R$ is unambiguous.

\begin{adjustwidth}{0.5cm}{}
	\begin{theorem}
		$\Hom_{R}(R, M) \cong M$
	\end{theorem}
	\begin{proof}
		The mapping $\phi : \Hom_{R}(R, M) \to M$ defined by $\phi(f) = f(1)$ is a homomorphism, as verified by a routine computation: for all $f, g \in \Hom_{R}(M, N)$ and $a \in R$,
		\begin{align*}
      \phi(f + g) = (f + g)(1) &= f(1) + g(1) = \phi(f) + \phi(g) \\
            \phi(af) = (af)(1) &= a f(1) = a \phi(f),
		\end{align*}
		so $\phi$ is an $R$-homomorphism. This mapping is injective, since each $f$ is uniquely determined by $f(1)$. It is also surjective; for each $m \in M$, set define a homomorphism by $h(1) = m$. Thus $\phi$ is the desired isomorphism.
	\end{proof}
\end{adjustwidth}

Homomorphisms $u : M' \to M$ and $v : N \to N''$ induce mappings $\bar{u} : \Hom(M, N) \to \Hom(M', N)$ and $\bar{v} : \Hom(M, N) \to \Hom(M, N'')$ defined for $f \in \Hom(M, N)$ as follows
\[
	\bar{u}(f) = f \circ u \qquad \text{and} \qquad \bar{v}(f) = v \circ f.
\]
I do not know why such a manipulation is noteworthy. The formulas above are quite easy to memorize if the time ever comes to invoke them.

% --------------------------------------------- %

\subsection{Submodules}

A \textbf{submodule} $M'$ of $M$ is an abelian subgroup of $M$ closed under multiplication by elements of the commutative ring $R$. 

\begin{adjustwidth}{0.5cm}{}
  \begin{lemma}
    $\mathfrak{a}$ is an ideal of $R$ if and only if it is an $R$-submodule of $R$.
  \end{lemma}
  \begin{proof}
    The proof evolves from a fundamental observation:
    \[
      R \mathfrak{a} = \mathfrak{a} \, \iff \, \text{scalar multiplication in the $R$-module $\mathfrak{a}$ is closed}.
    \]
    The rest of the multiplicative module conditions follow from the ring axioms.
  \end{proof}
\end{adjustwidth}

The following proof outlines the construction of \textbf{quotient modules}:

\begin{adjustwidth}{0.5cm}{}
	\begin{theorem}
		The abelian quotient group $M \, / \, M'$ is an $R$-module under the opreation $a(x + M') = ax + M'$.
	\end{theorem}
	\begin{proof}
		We must perform four rather routine calculations: for all $x, y \in M$ and $a, b \in R$,
		\begin{enumerate}
      \item \textbf{Identity}: $1(x + M') = 1x + M' = x + M'$.
      \item \textbf{Compatability}: $a(b(x + M')) = a(bx + M') = abx + M' = (ab)(x + M')$.
			\item \textbf{Left Distributivity}: $(a + b)(x + M') = (a + b)x + M' = (ax + bx) + M' = (ax + M') + (bx + M') = a(x + M') + b(x + M')$.
			\item \textbf{Right Distriutivity}: $a((x + M') + (y + M')) = a((x + y) + M') = a(x + y) + M' = (ax + M') + (ay + M') = a(x + M') + a(y + M)'$.
		\end{enumerate}
		Therefore, $M / M'$ is an $R$-module. Also, this operation is naturally well-defined.
	\end{proof}
\end{adjustwidth}

$R$-module homomorphisms $f : M \to N$ induce three notable submodules: 
\begin{enumerate}
  \item \textbf{Kernel}: $\Ker f \, = \, \{ x \in M \, \mid \, f(x) = 0 \}$, a submodule of $M$.
  \item \textbf{Image}: $\Im f \, = \, \{ f(x) \, \mid \, x \in M \}$, a submodule of $N$.
  \item \textbf{Cokernel}: $\Coker f \, = \, N \, / \, \Im f$, a quotient of $N$.
\end{enumerate}
The cokernel is perhaps an unfamiliar face. Such a quotient is not possible for rings or groups; images of homomorphisms need not be ideals of $R$ nor normal subgroups of $G$. 

\begin{adjustwidth}{0.5cm}{}
  \begin{theorem}[First Isomorphism Theorem]
     $N \, / \, \Ker f \, \cong \, \Im f$.
  \end{theorem}
  \begin{proof}
    Let $K = \Ker f$, and define a mapping $g : M \, / \, N \to \Im f$ by $g(x + K) = f(x)$. We have for arbitrary $x, y \in N$ and $a \in R$ that
    \begin{align*}
      g(x + y + K) \, = \, f(x + y) \, &= \, f(x) + f(y) \, = \, g(x + K) + g(y + K). \\
            g(ax + K) \, = \, f(ax) \, &= \, a f(x) \, = \, a g(x + K).
    \end{align*}
    Hence $g$ is a homomorphism. For injectivity, suppose that $g(x + K) = g(y + K)$ --- that is, $f(x) = f(y)$. Then
    \[
      f(y - x) \, = \, f(y) - f(x) \, = \, 1,
    \]
    so $y - x \in K$. Thus $x + K = y + K$. Surjectivity is quite clear. We conclude that $g$ is the desired isomorhism.
  \end{proof}
\end{adjustwidth}

Let $f : M \to N$ be an $R$-module homomorphism. Here are two special cases of the prior theorem:
\begin{enumerate}
  \item If $f$ is a monomorphism, them $M \, \cong \, \Im f$.
  \item If $f$ is an epimorphism, then $M \, / \, \Ker f \, \cong \, N$.
\end{enumerate}

For a submodule $N' \subseteq \Im f$, I call $M' = \{ x \in M \, \mid \, f(a) \in N' \}$ the \textbf{contraction module}.

\begin{adjustwidth}{0.5cm}{}
  \begin{theorem}[Correspondence Theorem]
    Submodules of $G$ which contain $\Ker f$ correspond one-to-one with submodules of $\Im f$.
  \end{theorem}
  \begin{proof}
    For each submodule $N' \subseteq \Im f$ consider the contraction module $M' \, = \, \{ x \, \mid \, f(x) \in N' \}$. Since this is an Abelian subgroup, we need only check for multiplicative closure: for all $x \in M'$ and $a \in R$, we have
    \[
      f(ax) = a f(x) \in N' \implies ax \in N'.
    \]
    Hence $M'$ is a submodule. It is clear that $\Ker f \subseteq M'$, so the First Isomorphism Theorem yields that
    \[
      N' \, / \, \Ker f \, \cong \, M'.
    \]
    Thus this construction is injective. It is surjective, since for each $\Ker \subseteq N' \subseteq N$, the subgroup $N'$ is contracted by $f(N')$. The correspondence is now established.
  \end{proof}
\end{adjustwidth}

The Second Isomorphism Theorem utilizes the definitioins of Section 2.1:

\begin{adjustwidth}{0.5cm}{}
  \begin{theorem}[Second Isomorphism Theorem]
    If $M_{1}, M_{2} \subseteq M$ are submodules, then $(M_{1} + M_{2}) \, / \, M_{1} \, \cong \, M_{2} \, / \, (M_{1} \cap M_{2})$.
  \end{theorem}
  \begin{proof}
    Define a mapping $f : M_{2} \to (M_{1} + M_{2}) \, / \, M_{1}$ by $\phi(m_{2}) = m_{2} + M_{1}$. Clearly $\phi$ is well-defined; it is an $R$-module homomorphism, since $x_{1}, x_{2} \in M_{2}$ and $a \in R$ implies
    \begin{align*}
      f(x_{1} + x_{2}) \, = \, x_{1} + x_{2} + x_{1} \, &= \, (x_{1}N)(x_{2}N) \, = \, f(x_{1}) f(x_{2}) \\
                    f(ax_{1}) \, = \, ax_{1} + x_{1} \, &= \, a(x_{1} + x_{1}) \, = \, a f(x_{1}).
    \end{align*}
    $f$ is surjective, since for all $x_{1} + M_{1} \in (M_{1} + M_{2}) \, / \, M_{2}$, we have $f(x_{1}) = x_{1} + X_{1}$. The kernel of $f$ is all $x \in M_{1}$ --- namely, $M_{1} \cap M_{2}$. We conclude by the First Isomorphism Theorem that
    \[
      (M_{1} + M_{2}) \, / \, M_{1} \quad = \quad M_{2} \, / \, (M_{1} \cap M_{2}),
    \]
    which completes the proof.
  \end{proof}
\end{adjustwidth}

\begin{adjustwidth}{0.5cm}{}
  \begin{theorem}[Third Isomorphism Theorem]
    If $L \nsg M$ and $L \subseteq N \nsg M$, then $M \, / \, N \, = \, (M \, / \, L) \, / \, (N \, / \, L)$.
  \end{theorem}
  \begin{proof}
    Let $\phi : M \to M \, / \, L$ be the canonical epimorphism. Define $\psi : M \to \phi(M) \, / \, \phi(N)$ by the rule $\psi(a) = \phi(a)\phi(M)$. It is clear that $\psi$ is well-defined and surjective; it is an $R$-module homomorphism since
    \[
      \psi(ab) \, = \, \phi(ab) \phi(N) \, = \, \big(\phi(a) \phi(M)\big) \, \big(\phi(b) \phi(M) \big) \, = \, \psi(a) \psi(b).
    \]
    The kernel of $\phi$ is all $a \in N$. The First Isomorphism Theorem yields that
    \[
      M \, / \, N \quad \cong \quad \phi(M) \, / \, \phi(M).
    \]
    Since the kernel of $\phi$ is $L$, we have that $\phi(M) \, \cong \, M \, / \, L$ and $\phi(N) \, \cong \, N \, / \, L$; substituting yields the desired $M \, / \, N \, = \, (M \, / \, L) \, / \, (N \, / \, L)$.
  \end{proof}
\end{adjustwidth}

% --------------------------------------------- %

\section{Operations on Submodules}

% --------------------------------------------- %

\subsection{Sums, Intersections, Products}

Let $M$ be an $R$-module with submodules $M_{1}, \ldots, M_{n}$. We can consider two crucial operations on these submodules:
\begin{enumerate}
	\item \textbf{Sum}: The sum $M_{1} + \cdots + M_{n}$ is the set of all sums $m_{1} + \cdots + m_{n}$, where $m_{i} \in M_{i}$ ($i \in \{ 1, \ldots, n \}$). It is the smallest submodule of $M$ that contains all $M_{1}, \ldots, M_{n}$.
	\item \textbf{Intersecrtion}: The intersection $M_{1} \cap \cdots \cap M_{n}$ is the largest submodule of $M$ that is contained inside each $M_{1}, \ldots, M_{n}$.
\end{enumerate}

For an ideal $\mathfrak{a}$ of $R$ and an $R$-module $M$, we define the \textbf{product} $\mathfrak{a} M$ as all finite sums $a_{1}x_{1} + \cdots + a_{n}x_{n}$ for $a_{i} \in \mathfrak{a}$ and $x_{i} \in M$  for each $i \in \{ 1, \ldots, n \}$. It is a submodule of $M$.

% --------------------------------------------- %

\subsection{The Annihilator}

If $N$ and $P$ are $R$-submodules of $M$, we define $(N : P)$ to be the set of all $a \in R$ such that $aP \subseteq N$.

\newpage

\begin{adjustwidth}{0.5cm}{}
	\begin{theorem}
		$(N : P)$ is an ideal of $R$.
	\end{theorem}
	\begin{proof}
		If $a, b \in (N : P)$, then $aP, bP \subseteq N$; we must have that $aP + bP \subseteq N$. Observe that $(N : P)$ is nonempty, as $0P = (0) \subseteq N$; and clearly if $aP \in N$, then $-aP \in N$ as $N$ is an abelian group. $(N : P)$ satifies the multiplicative condition too.
	\end{proof}
\end{adjustwidth}

The \textbf{annihilator} of a module $M$ is $(0 : M)$, the ideal of all $a$ such that $aM = 0$, and is denoted $\Ann M$. If $\mathfrak{a} \subseteq \Ann M$, we may regard $M$ as an $R \, / \, \mathfrak{a}$-module. In particular, observe that if $\bar{a} \in R \, / \, \mathfrak{a}$, then $a_{1}, a_{2} \in \bar{a}$ implies $a_{1}x = a_{2}x$ --- so $\bar{a}$ is well-defined.

An $R$-module is \textbf{faithful} if $\Ann M = 0$. The annihalator of $R$ may change depending on the ring --- if $\Ann M = \mathfrak{a}$, then $M$ is faithful as an $R \, / \, \mathfrak{a}$ module.

\begin{adjustwidth}{0.5cm}{}
	\begin{theorem}
		If $M_{1}, M_{2} \subseteq M$ are submodules, then $\Ann(M_{1} + M_{2}) = \Ann M_{1} \cap \Ann M_{2}$.
	\end{theorem}
	\begin{proof}
		Suppose $a \in \Ann M_{1} \cap \Ann M_{2}$. Then $a$ annihalates all $x_{1} + x_{2} \in M_{1} + M_{2}$, so $a \in \Ann(M_{1} + M_{2})$.

		Suppose $a \notin \Ann M_{1} \cap \Ann M_{2} $. Then $a$ must fail to annnihilate an element in either $M_{1}$ or $M_{2}$ (or both) --- without loss of generality, let there exist $x_{1} \in M_{1}$ such that $ax_{1} \ne 0$. Then as $ax_{1} \in M_{1} + M_{2}$, we find that $a \notin \Ann(M_{1} + M_{2})$. This completes the proof.
	\end{proof}
\end{adjustwidth}

% --------------------------------------------- %

\subsection{Direct Sum and Product}

If $M$ and $N$ are $R$-modules, their \textbf{direct sum} $M \oplus N$ is the set of all pairs $(x, y)$ (with $x \in M$ and $y \in N$) endowed with the natural operations:
\begin{align*}
	(x_{1}, y_{1}) + (x_{2}, y_{2}) & = (x_{1} + x_{2}, y_{1} + y_{2}) \\
	                        a(x, y) & = (ax, ay)
\end{align*}
More generally, for a family of $R$-modules $(M_{\alpha})$, we define their \textbf{direct sum} $\bigoplus_{\alpha} M_{\alpha}$ as the families $(x_{\alpha})$ such that $x_{\alpha} \in M_{\alpha}$ for each $\alpha$ with the restriction that only finitely many $x_{\alpha}$ are nonzero.

If we allow infinitely many $x_{\alpha}$ to be nonzero, we attain the family's \textbf{direct product} $\prod_{\alpha} M_{\alpha}$. Direct sums and direct products are equivalent if there are finitely many $\alpha$, but not otherwise.

\newpage

\subsection{Direct Sums on Rings}

Direct products of rings and direct sums of ideals are equivalent notions:
\begin{enumerate}
  \item Suppose that $R \, = \, R_{1} \times \cdots \times R_{n}$. Then $R$ has $n$ ideals of the form $(\delta_{n, m})$ for each $m \in \{ 1, \ldots, n \}$. Viewing this relation in terms of modules,
  \[
  	R \cong \mathfrak{a}_{1} \oplus \cdots \oplus \mathfrak{a}_{n}
  \]
  by the isomorphism $f(r_{1}, \ldots, r_{n}) = ((r_{1}, 0, \ldots), \ldots, (\ldots, 0, r_{n}))$.
  \item Suppose that $R = \mathfrak{a}_{1} \oplus \cdots \oplus \mathfrak{a}_{n}$, where $\mathfrak{a}_{i} \subseteq R$ are ideals. Defining $\mathfrak{b}_{i} = \bigoplus_{j \ne i} \mathfrak{a}_{j}$ for each $i$, we find that
  \[
	  R \cong (R \,/\, \mathfrak{b}_{1}) \times \cdots \times (R \,/\, \mathfrak{b}_{n})
  \]
  by $f(a_{1}, \ldots, a_{n}) = (a_{1} + \mathfrak{b}_{1}, \ldots, a_{n} + \mathfrak{b}_{n})$. Each $\mathfrak{a}_{i}$ is a ring isomorphic to $R \, / \, \mathfrak{b}_{i}$.
\end{enumerate}

This illustrates how one may define a direct sum of ideals.

% --------------------------------------------- %

\section{Finitely Generated Modules}

% --------------------------------------------- %

\subsection{Definition}

An $R$-module $M$ is said to be \textbf{finitely generated} if there exist a set of \textbf{generators} $x_{1}, \ldots, x_{n}$ such that $M = Rx_{1} + \cdots + Rx_{n}$. The $R$-module $R \oplus \cdots \oplus R$ is denoted by $R^{n}$.

\begin{adjustwidth}{0.5cm}{}
	\begin{theorem}
		$M$ is finitely-generated by $n$ elements if and only if $M$ is isomorphic to a quotient of $R^{n}$.
	\end{theorem}
	\begin{proof}
    Suppose $M$ is finitely-generated by $x_{1}, \ldots, x_{n}$. Define a mapping $f : R^{n} \to M$ by $f(a_{1}, \ldots, a_{n}) = a_{1}x_{1} + \cdots + a_{n}x_{n}$. Then for all $(a_{1}, \ldots, a_{n}), (b_{1}, \ldots, b_{n}) \in R^{n}$, we have
		\begin{align*}
			f(a_{1} + b_{1}, \ldots, a_{n} + b_{n}) \, &= \, (a_{1} + b_{1}) x_{1} \, + \, \cdots \, + \, (a_{n} + b_{n})x_{n} \\
			&= \, (a_{1}x_{1} + \cdots + a_{n}x_{n}) \, + \, (b_{1}x_{1} + \cdots + b_{n}x_{n}) \\
			&= \, f(a_{1}, \ldots, a_{n}) \, + \, f(s_{1}, \ldots, s_{n}),
		\end{align*}
    and for all $(a_{1}, \ldots, a_{n}) \in R^{n}$ and $b \in R$, we have
		\[
			f(b a_{1}, \ldots, b a_{n}) \, = \, b a_{1} x_{1} + \cdots + b a_{n} x_{n} \, = \, b \, f(a_{1}, \ldots, a_{n}).
		\]
    Thus $f$ is an $R$-module homomorphism. Its surjectivity follows from the generation of $M$ by $x_{1}, \ldots, x_{n}$. Thus if we set $\Ker f = \mathfrak{a}$, the First Isomorphism Theorem yields that
		\[
			R^{n} \,/\, \mathfrak{a} \, \cong \, M.
		\]
		Suppose that $R^{n} \,/\, \mathfrak{a} \, \cong \, M$ by an isomorphism $f$. Two definitions are in order:
    \begin{enumerate}
      \item Let $g : R^{n} \to R^{n} \, / \, \mathfrak{a}$ be the canonical epimorphism. Then the composition $f \circ g$ is a surjective $R$-module homomorphism.
      \item For each $(\delta_{mi}) \in R^{n}$, Let $x_{i} = (f \circ g)(\delta_{mi})$ across each $i \in \{ 1, \ldots, n \}$. We claim that the $x_{i}$ generate $M$.
    \end{enumerate}
    From the surjectivity of $f \circ g$, all $x \in M$, there exists $r = (r_{1}, \ldots, r_{n}) \in R^{n}$ such that $(f \circ g)(r) = x$. Therefore 
		\begin{align*}
			x &= (f \circ g)(r_{1}, \ldots, r_{n}) \\
        &= (f \circ g)(r_{1} \delta_{m1}) + \cdots + (f \circ g)(r_{n} \phi_{mn}) \\
			  &= r_{1} (f \circ g)(\delta_{m1}) + \cdots + r_{n} (f \circ g)(\delta_{mn}) \\
			  &= r_{1} x_{1} + \cdots r_{n} x_{n}.
		\end{align*}
		We conclude that $x_{1}, \ldots, x_{n}$ generate $M$.
	\end{proof}
\end{adjustwidth}

The following proof is a transcription from Atiyah-MacDonald:

\begin{adjustwidth}{0.5cm}{}
	\begin{theorem}
		Let $M$ be a finitely-generated $R$-module, let $\mathfrak{a}$ be an ideal of $R$, and let $f : M \to M$ be an $R$-module endomorphism of $M$ such that $f(M) \subset \mathfrak{a}M$. Then $f$ satisfies an equation of the form
		\[
			f^{n} \, + \, a_{n - 1} f^{n - 1} \, + \, \cdots \, + \, a_{0} f^{0} \, = \, 0,
		\]
		where each $a_{i}$ lies in $\mathfrak{a}$.
	\end{theorem}
	\begin{proof}
		Let $x_{1}, \ldots, x_{n}$ generate $M$. Then each $f(x_{i}) \in \mathfrak{a} M$, so we may define $a_{ij} \in \mathfrak{a}$ for $i, j \in \{ 1, \ldots, n \}$ by $f(x_{i}) = \sum_{j = 1}^{n} a_{ij}x_{j}$. This equation may be equivalently written for each $i \in \{ 1, \ldots, n \}$ as
		\[
			\sum\limits_{j = 1}^{n} (\delta_{ij} f - a_{ij} f^{0}) x_{j} = 0,
		\]
    By multiplying the left by the adjoint of the matrix $\delta_{ij} f - a_{ij} f^{0}$, it follows that the determinant of $\delta_{ij} f - a_{ij} f^{0}$ annihilates each $x_{ij}$ --- hence it is the zero endomorphism. Expanding out the determinant yields an equation of the required form.
	\end{proof}
\end{adjustwidth}

\begin{adjustwidth}{0.5cm}{}
	\begin{corollary}
		Let $M$ be a finitely-generated $R$-module and let $\mathfrak{a}$ be an ideal of $R$ such that $\mathfrak{a} M = M$. Then there exists $a \in (1 + \mathfrak{a})$ such that $a \in \Ann M$.
	\end{corollary}
	\begin{proof}
		Consider Theorem 7 under the identity transformation $f$. Then there exist $a_{n - 1}, \ldots, a_{0} \in \mathfrak{a}$ such that
		\begin{align*}
			0 &= f^{n} + a_{n - 1} f^{n - 1} + \cdots + a_{0} f^{0} \\
			&= f + a_{n - 1} f + \cdots + a_{0} f \\
			&= (1 + a_{n - 1} + \cdots + a_{0}) f.
		\end{align*}
		Setting $a \, = \, 1 + a_{n - 1} + \cdots + a_{0}$ yields that $a f$ is the zero endomorphism, so $a f(x) = ax = 0$ for each $x \in M$. We conclude that $a \in \Ann M$.
	\end{proof}
\end{adjustwidth}

A \textbf{free R-module} $M$ is a module such that $M \cong \bigoplus_{j \in J} M_{j}$, where $M_{j} \cong R$ for each $j \in J$. A finitely generated module $M$ is therefore free if $M$ is isomorphic to $R^{n}$ itself.

\subsection{Relation to the Jacobson Radical}

% --------------------------------------------- %

The following lemma is called \textbf{Nakayama's Lemma} and has two proofs:

\begin{adjustwidth}{0.5cm}{}
	\begin{lemma}[Nakayama's Lemma]
		Let $M$ be a finitely generated $R$-module and $\mathfrak{a}$ an ideal of $R$ contained in the Jacobson radical $\mathfrak{J}$ of $R$. Then $\mathfrak{a}M = M$ implies $M = 0$.
	\end{lemma}
	\begin{proof}
		By Corollary 1, there exists $a \in (1 + \mathfrak{J})$ such that $a \in \Ann M$. By the properties of the Jacobson radical, $a$ is a unit. Hence,
		\[
			M = (a^{-1}a)M = a^{-1}(aM) = a^{-1}(0) = 0,
		\]
    which completes the proof.
	\end{proof}
	\begin{proof}
		Suppose for contradiction that $\mathfrak{a}M = M$ and $M \ne 0$; let $x_{1}, \ldots, x_{n}$ be a set of generators of $M$ of shortest length. Then $x_{n} \in \mathfrak{a}M$, so $x_{n}$ satisfies an equation of the form
		\[
			 x_{n} = a_{1} x_{1} + \cdots + a_{n} x_{n}
		\]
		for $a_{1}, \ldots, a_{n} \in \mathfrak{a}$. Since none of these scalars are $1$, we have
		\[
			(1 - a_{n})x_{n} = a_{1}x_{1} + \cdots + a_{n - 1}x_{n - 1}.
		\]
    Since $a_{n} \in \mathfrak{J}$, the element $(1 - a_{n})$ is a unit. Thus $x_{1}, \ldots, x_{n - 1}$ generate $M$, which yields the desired contradiction.
	\end{proof}
\end{adjustwidth}

\begin{adjustwidth}{0.5cm}{}
	\begin{corollary}
		Let $M$ be a finitely-generated $R$-module, $N \subseteq M$ be a submodule, and $\mathfrak{a} \subseteq \mathfrak{J}$ be an ideal of $R$ in the Jacobson radical. Then $M = \mathfrak{a}M + N$ implies $M = N$.
	\end{corollary}
	\begin{proof}
    Suppose that $M = \mathfrak{a}M + N$. Then
    \[
      \mathfrak{a}(M \, / \, N) \, = \, \mathfrak{a}M \, / \, N \, = \, (\mathfrak{a}M + N) \, / \, N \, = \, M \, / \, N.
    \]
    Thus applying Nakayama's Lemma to $M \, / \, N$, we have $M \, / \, N = 0$; hence $M = N$.
	\end{proof}
\end{adjustwidth}

% --------------------------------------------- %

\subsection{Relation to Local Rings}

Consider a finitely-generated module $M$ over a local ring $R$ with maximal ideal $\mathfrak{m}$. 

\begin{adjustwidth}{0.5cm}{}
	\begin{theorem}
		The elements of the $R$-module $M \,/\, \mathfrak{m} M$ and the $(R \,/\, \mathfrak{m})$-module $M$ are identical.
	\end{theorem}
	\begin{proof}
		It is relatively simple to verify that the function $f : M \,/\, \mathfrak{m} M \to M$ defined by
		\[
			f(a_{1} x_{1} + \cdots + a_{n} x_{n} + \mathfrak{m}M) = (a_{1} + \mathfrak{m})x_{1} + \cdots + (a_{n} + \mathfrak{m})x_{n}
		\]
		is bijective. It further satisfies $f(x + y) = f(x) + f(y)$ and $f(ax) = b f(x)$, where $b$ is the image of $a$ under the canonical epimorphism $\phi : R \to R \,/\, \mathfrak{m}$.
	\end{proof}
\end{adjustwidth}

This realization arises naturally, as $\mathfrak{m}$ annihalates the quotient module $M \,/\, \mathfrak{m} M$. Since $R \,/\, \mathfrak{m}$ is a field, $M \,/\, \mathfrak{m} M$ is actually a vector space.

\begin{adjustwidth}{0.5cm}{}
	\begin{theorem}
		Let $M$ be a module over a local ring. Then $x_{1}, \ldots, x_{n} \in M$ generate $M$ if and only if the images of $x_{1}, \ldots, x_{n}$ span the the vector space $M \,/\, \mathfrak{m} M$.
	\end{theorem}
	\begin{proof}
		Let $f : M \to M \,/\, \mathfrak{m} M$ be the canonical epimorphism. Then
		\begin{align*}
			\text{$x_{1}, \ldots, x_{n}$ generate $M$} &\iff \text{$x_{1}, \ldots, x_{n}$ generate $M \,/\, \mathfrak{m} M$} \\
			&\iff \text{$x_{1}, \ldots, x_{n}$ are a basis of $M \,/\, \mathfrak{m} M$ over $R \,/\, \mathfrak{m}$}.
		\end{align*}
		The ommitted details are relatively simple to verify.
	\end{proof}
\end{adjustwidth}


\newpage

% --------------------------------------------- %

\section{Exact Sequences}

% --------------------------------------------- %

\subsection{Definition}

A sequence of $R$-modules and $R$-homomorphisms
\begin{equation}
	\cdots \longrightarrow M_{i - 1} \stackrel{f_{i}}{\longrightarrow} M_{i} \stackrel{f_{i + 1}}{\longrightarrow} M_{i + 1} \xrightarrow{\hspace*{0.5cm}} \cdots 
\end{equation}
is \textbf{exact} at $M_{i}$ if $\Im f_{i} = \Ker f_{i + 1}$. The sequence is \textbf{exact} if it is exact at each $M_{i}$. Such sequences induce a wealth of identities relating the modules and their images, kernels, and cokernels. Three examples of exact sequences are as follows:
\begin{align*}
	0 \to M' \stackrel{f}{\to} M \text{ is exact } &\iff \text{$f$ is injective, so $M' \cong \Im f$} \\
	M \stackrel{g}{\to} M'' \to 0 \text{ is exact } &\iff \text{$g$ is surjective, so $M \,/\, \Ker g \cong M'$.} \\
	0 \to M' \stackrel{f}{\to} M \stackrel{g}{\to} M'' \to 0 \text{ is exact} &\iff \text{$f$ is injective, $g$ is surjective, and} \\
	& \quad \qquad \text{$\Coker f = M \,/\, \Ker g \cong M'$}.
\end{align*}

An exact sequence of this third type is called a \textbf{short exact sequence}. Any long exact sequence, like that in equation (1), can be broken into numerous short exact sequences:
\[
	0 \longrightarrow \Coker f_{i - 1} \stackrel{f_{i}'}{\longrightarrow} M_{i} \stackrel{f_{i + 1}}{\longrightarrow} \Im f_{i + 1} \longrightarrow 0,
\]
where $f_{i}'$ is defined as $f_{i}'(x + \Ker f_{i}) = f_{i}(x)$. To save time, we will often write that $0 \to N_{i} \to M_{i} \to N_{i + 1} \to 0$ for some modules $N_{i}$ and $N_{i + 1}$ instead.

% --------------------------------------------- %

\subsection{Exact Sequences of Homomorphisms}

\begin{adjustwidth}{0.5cm}{}
	\begin{theorem}
		The sequence
		\begin{equation}
			M' \stackrel{u}{\longrightarrow} M \stackrel{v}{\longrightarrow} M'' \longrightarrow 0
		\end{equation}
		of $R$-modules and $R$-module homomorphisms is exact if and only if for all $R$-modules $N$, the sequence
		\begin{equation}
			0 \longrightarrow \Hom(M'', N) \stackrel{\bar{v}}{\longrightarrow} \Hom(M, N) \stackrel{\bar{u}}{\longrightarrow} \Hom(M', N)
		\end{equation}
		is exact, where $\bar{u}$ and $\bar{v}$ are as in Section 1.3.
	\end{theorem}
	\newpage
	\begin{proof}
		The exactness of (2) is equivalent to $\Im u = \Ker v$ and the surjectivity of $v$, while the exactness of (3) is equivalent to the injectivity of $\bar{v}$ and $\Im \bar{v} = \Ker \bar{u}$.
		\begin{adjustwidth}{0.5cm}{}
			\begin{claim}
				$v$ is surjective if and only if $\bar{v}$ is injective.
			\end{claim}
			\begin{proof}\renewcommand{\qedsymbol}{}
				Suppose that $v$ is surjective, and suppose $f \in \Ker \bar{v}$ for any $N$. Then $f$ is in $\Hom(M, M'')$ and
				\[
					\bar{v}(f) = f \circ v = 0.
				\]
				Since $v$ is surjective, $f$ must map the entirety of $M$ to zero; thus $f = 0$, and $\bar{v}$ is injective.

				Now, suppose that $v$ is not surjective. Then set $N = M'' \,/\, \Im v$ and let $f \in \Hom(M'', N)$ be the canonical epimorphism; both are nonzero. Then
				\[
					\bar{v}(f)(M) = (f \circ v)(M) = f(\Im v) = 0.
				\]
				Then $f$ is a nonzero element of $\Ker \bar{v}$, so $\bar{v}$ is not surjective.
			\end{proof}
		\end{adjustwidth}
		The second claimed biconditional relation requires proof of the following claim.
		\begin{adjustwidth}{0.5cm}{}
			\begin{claim}
				$\Im u = \Ker v$ for surjective $v$ implies that $\Im \bar{v} = \Ker \bar{u}$.
			\end{claim}
			\begin{proof}\renewcommand{\qedsymbol}{}
				Suppose $\Im u = \Ker v$, so $v \circ u = 0$. Let $f \in \Im \bar{v}$; then there exists $g$ such that $\bar{v}(g) = g \circ v = f$. We conclude that
				\[
					\bar{u}(f) = f \circ u = g \circ v \circ u = g \circ 0 = 0,
				\]
				so $f \in \Ker \bar{u}$. Now, suppose $f \in \Ker \bar{u}$, so $\bar{u}(f) = f \circ u = 0$. Since $v$ is surjective, $m'' \in M''$ implies the existence of $m \in M$ such that $v(m) = m''$. Then define $g \in \Hom(M'', N)$ such that $g(m'') = f(m)$; it is relatively easy to demonstrate that $g$ is well-defined and a homomorphism. Thus, $f = g(u) = \bar{u}(g)$, so $f \in \Im \bar{v}$, so $\Im \bar{v} = \Ker \bar{u}$.
			\end{proof}
		\end{adjustwidth}
		If we suppose $\Im \bar{v} = \Ker \bar{u}$, then $\bar{u} \circ \bar{v} = 0$ --- that is, $v \circ u \circ f = 0$ for all $f \in \Hom(M'', N)$. This is equivalent to $\Im u = \Ker v$ in the start of Claim 2, with some added work to demonstrate that $\Im u \supseteq \Ker v$.
	\end{proof}
\end{adjustwidth}

Really, the natural language for these proofs is Homosexual Algebra and Abelian categories. Thus we will state the lemmas from this chapter without proof.

\newpage

The following theorem is asserted without proof, because I value my sanity.

\begin{adjustwidth}{0.5cm}{}
  \begin{theorem}
		The sequence
		\[
			0 \longrightarrow N' \stackrel{u}{\longrightarrow} N \stackrel{v}{\longrightarrow} N''
    \]
		of $R$-modules and $R$-module homomorphisms is exact if and only if for all $R$-modules $N$, the sequence
		\[
			0 \longrightarrow \Hom(M, N') \stackrel{\bar{u}}{\longrightarrow} \Hom(M, N) \stackrel{\bar{v}}{\longrightarrow} \Hom(M, N'')
    \]
		is exact, where $\bar{u}$ and $\bar{v}$ are as in Section 1.3.
  \end{theorem}
\end{adjustwidth}

The following theorem is called the Snake Lemma, a special case of exact homology sequences in Homosexual Algebra: 

\begin{adjustwidth}{0.5cm}{}
  \begin{theorem}[Snake Lemma]
    Suppose that
    \[ \begin{tikzcd}
      0 \arrow[r] & M' \arrow[r, "u"] \arrow[d, "f'"] & M \arrow[r, "v"] \arrow[d, "f"] & M'' \arrow[r] \arrow[d, "f''"] & 0 \\
      0 \arrow[r] & N' \arrow[r, "u'"]                & N \arrow[r, "v'"]               & N'' \arrow[r]                  & 0
    \end{tikzcd} \]
    is a commutative diagram of $R$-modules and homomorphisms, with the rows exact. Then there exists an exact sequence
    \[ \begin{tikzcd}
      0 \arrow[r] & \Ker(f') \arrow[r, "\bar{u}"] & \Ker(f) \arrow[r, "\bar{v}"]     & \Ker(f'') \arrow[ld, "d"']      &                       &   \\
                  &                               & \Coker(f') \arrow[r, "\bar{u}'"] & \Coker(f) \arrow[r, "\bar{v}'"] & \Coker(f'') \arrow[r] & 0
    \end{tikzcd} \]
    in which $\bar{u}$ and $\bar{v}$ are restrictions of $u$ and $v$, and in which $\bar{u}'$ and $\bar{v}'$ induced by $u'$ and $v'$. The \textbf{boundary homomorphism} $d$ is defined in Atiyah-MacDonald page 23.
  \end{theorem}
\end{adjustwidth}

We encourage the reader to adopt these results on faith. Homosexual Algebra is a complex subject of math that deserves its own set of notes.

% --------------------------------------------- %

\end{document}
