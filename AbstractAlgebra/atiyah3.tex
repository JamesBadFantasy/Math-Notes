\documentclass[11pt]{article}
\usepackage[T1]{fontenc}
\usepackage{geometry, changepage, hyperref}
\usepackage{amsmath, amssymb, amsthm, bm}
\usepackage{physics, esint}

\hypersetup{colorlinks=true, linkcolor=blue, urlcolor=cyan}
\setlength{\parindent}{0pt}
\setlength{\parskip}{5pt}

\newtheorem{theorem}{Theorem}
\newtheorem{lemma}{Lemma}
\newtheorem{proposition}{Proposition}
\newtheorem{corollary}{Corollary}
\newtheorem{claim}{Claim}

\title{Atiyah-MacDonald: Rings and Modules of Fractions}
\author{James Pagan}
\date{February 2024}

% --------------------------------------------- %

\begin{document}

\maketitle
\tableofcontents
\newpage

% --------------------------------------------- %

\section{Construction of Rings of Fractions}

% --------------------------------------------- %

\subsection{Equality of Fractions}

Let $R$ be a ring. A \textbf{multiplicatively closed subset} $S \subseteq R$ is a subset that contains $1$ and is closed under multiplication --- that is, if $(S, \times)$ is a submonoid of $(R, \times)$. 

\begin{adjustwidth}{1cm}{}
  \begin{lemma}
    Define a relation $\equiv$ on $R \times S$ as follows:
    \[
      (a, s) \equiv (b, t) \iff (at - bs)x = 0 \text{ for some } x \in S.
    \]
    Then $\equiv$ is an equivalence relation.
  \end{lemma}
  \begin{proof}
    Let $(a, s)$, $(b, t)$, and $(c, u)$ be any elements of $R \times S$. We must verify three properties:
    \begin{enumerate}
      \item \textbf{Reflexivity}: It is clear that $(a, b) 
        \equiv (a, b)$, since $(ab - ab)1 = 0$.
      \item \textbf{Symmetry}: The proof is as simple as multiplying by $-1$:
      \begin{align*}
        (a, s) \equiv (b, t) &\iff (at - bs)x = 0 \text{ for some } x \in S \\
                             &\iff (bs - at)x = 0 \text{ for some } x \in S \\
                             &\iff (b, t) \equiv (a, s).
      \end{align*}
      \item \textbf{Transitivity}: Suppose $(a, s) \equiv (b, t)$ and $(b, t) \equiv (c, u)$. Then there exists $x, y \in S$ such that
      \[
        (at - bs)x = 0 \qquad \text{and} \qquad (bu - ct)y = 0.
      \]
      Muliplying these equations by $uy$ and $sx$ respectively and add them: we obtain that $(aut - cts)xy = 0$. Hence $(au - cs)xyt = 0$; since $xyt \in S$ by closure, we find $(a, s) \equiv (c, u)$. 
    \end{enumerate}
    Therefore, $\equiv$ is an equivalence relation.
  \end{proof}
\end{adjustwidth}

The equivalence class of $(a, s)$ is denoted $\tfrac{a}{s}$ and called a \textbf{fraction}. The set $S^{-1}R$ denotes all the equivalence classes on $S \times R$. Observe that we do not impose $0 \notin S$; if $S$ contains zero, then all fractions are equivalent and $S^{-1}R = 0$.

\subsection{Operations on Fractions}

We endow $S^{-1}R$ with ring structure by defining addition and multiplication of fractions:
\begin{align*}
  \frac{a}{s} + \frac{b}{t} &= \frac{at + bs}{st} \\
  \frac{a}{s} \times \frac{b}{t} &= \frac{ab}{st}.
\end{align*}
\begin{adjustwidth}{1cm}{}
  \begin{lemma}
    Addition and multiplication of fractions is well-defined.
  \end{lemma}
  \begin{proof}
    Suppose $\tfrac{a}{s} = \tfrac{c}{u}$ and $\tfrac{b}{t} = \tfrac{d}{v}$. First, we demonstrate that $\tfrac{a}{s} + \tfrac{b}{t} = \tfrac{c}{u} + \tfrac{d}{v}$; there exist $x, y \in S$ such that
    \begin{equation}
      (au - cs)x = 0 \qquad \text{and} \qquad (bv - dt)y = 0.
    \end{equation}
    Multiply these equations by $tvy$ and $sux$ respectively and add them: we obtain that $(atuv + bsuv - cvst - dust)xy = 0$. Thus
    \[
      \frac{a}{s} + \frac{b}{s} \, = \, \frac{at + bs}{st} \, = \, \frac{cv + du}{uv} \, = \, \frac{c}{u} + \frac{d}{v}.
    \]
    Demonstrating that $\tfrac{a}{s} \times \tfrac{b}{t} = \tfrac{c}{u} \times \tfrac{d}{v}$ is a simpler story. Multiplying our equations in (1) by $bvy$ and $csx$ respectively and adding yields $(abuv - cdst)xy = 0$, so
    \[
      \frac{a}{s} \times \frac{b}{s} \, = \, \frac{ab}{st} \, = \, \frac{cd}{uv} \, = \, \frac{c}{u} \times \frac{d}{v};
    \]
    hence addition and multiplication of fractions is well-defined. 
  \end{proof}
\end{adjustwidth}

Naturally, $S^{-1}R$ is a commutative ring. The proof of this assertion is identical to the verification that $\mathbb{Q}$ is a field --- with one exception. It may not be true that each fraction has a multiplicative inverse.

\begin{adjustwidth}{1cm}{}
  \begin{corollary}
    If $R$ is an integral domain and $S = R \setminus \{ 0 \}$, then $S^{-1}R$ is the field of fractions of $R$.
  \end{corollary}
\end{adjustwidth}

% --------------------------------------------- %



% --------------------------------------------- %

\end{document}
