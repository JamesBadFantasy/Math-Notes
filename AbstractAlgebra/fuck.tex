\documentclass[11pt]{article}
\usepackage[T1]{fontenc}
\usepackage{geometry, changepage, hyperref}
\usepackage{amsmath, amssymb, amsthm, bm}
\usepackage{physics, esint}

\hypersetup{colorlinks=true, linkcolor=blue, urlcolor=cyan}
\setlength{\parindent}{0pt}
\setlength{\parskip}{5pt}

\newtheorem{theorem}{Theorem}
\newtheorem{lemma}{Lemma}
\newtheorem{proposition}{Proposition}
\newtheorem{corollary}{Corollary}
\newtheorem{claim}{Claim}

\title{Fuck}
\author{Fuck}
\date{Fuck}

% --------------------------------------------- %

\begin{document}

\maketitle
\tableofcontents
\newpage

% --------------------------------------------- %

\section{Day 1}

An operation $\ast$ on a set $S$ is something that takes in two values of $S$ and spits out one value of $S$. For instance,
\begin{enumerate}
  \item Addition, subtraction, multiplication, and division are operations on rational numbers $\mathbb{Q}$.
  \item Exponentiation is an operation on positive integers: let $a \ast b \, = \, a^{b}$.
  \item Logic gates are an operation: they take in two numbers (each either $0$ or $1$) and spit out $0$ or $1$.
\end{enumerate}
An operation $\ast$ on $S$ is called \textbf{closed} if for all $a, b \in S$, the element $a \ast b \in S$ too. In other words, $\ast$ never "leaves" $S$. The operation $\ast$ is associative if for all $a, b, c \in S$, we have $a \ast (b \ast c) = (a \ast b) \ast c$. Furthermore, $\ast$ is \textbf{commutative} if $a \ast b = b \ast a$ for all $a, b \in S$. It's possible to have one of these without the other! 

% THE PROBLEMS: No formal proofs required.
% 1. Is division of complex numbers $\mathbb{C}$ closed? Is it associative or commutative?
% 2. Explain why the operation $\ast$ on reals $\mathbb{R}$ defined by $x \ast y = 2^{x + y}$ is commutative, but not associative.
% 3. Let $M$ be the set of all $2$-by-$2$ matrices. Do you see why multiplication of matrices is associative, but not commutative?

% --------------------------------------------- %

% \section{Day 2}
%
% We often denote $S$ with the operation $&$ by the pair $(S, &)$. There are two other properties of operations we care about:
% \begin{enumerate}
%   \item $(S, &)$ has an \textbf{identity element} if there exists $e$ in $S$ such that the operation ``does nothing'' with $e$ --- for all $a$ in $S$, we have $a & e \, = \, e & a \, = \, a$.
%   \item $(S, &)$ has \textbf{inverses} if for \textit{all} $a$ in $S$, there exists some $b$ in $S$ that ``cancels'' $a$ --- namely $a & b \, = \, b & a \, = \, e$, where $e$ is the identity. 
% \end{enumerate}
% We often write the inverse of $a$ as $a^{-1}$, where this is \textit{not necessarily $1$ divided by $a$}. It's like how the inverse of a matrix $\mat{T}$ is written $\mat{T}^{-1}$. Notice that to have inverses, there must be an identity.
%
% Drill these five properties --- closure, associativity, identity, inverses, commutativity --- into your head! They will appear in basically every problem from now on
%
% 1. Find the identity of complex numbers $\mathbb{C}$ with respect to addition. Find the identity of \textit{nonzero} complex numbers with respect to multiplication.
% 2. 
%
% 3. $\mathbb{Q}$ has an additive identity. Does each rational number $\mathbb{Q}$ under $+$ have an inverse?
%
% % --------------------------------------------- %
%
% \section{Day 3}
%
% Listing out the properties of $(S, &)$ can get tiring --- so mathematicians gave bundles of these properties special names. $(S, &)$ has \textbf{group structure} if $&$ satisfies four properties: closure, associativity, identity, and inverses. If $(S, &)$ is also commutative (thus has all five), it has \textbf{Abelian group structure}.
%
% The set of polynomials with rational coefficients is denoted $\mathbb{Q}[x]$. Similarly, polynomials with integer, real, or complex coefficients are denoted $\mathbb{Z}[x]$, $\mathbb{R}[x]$, and $\mathbb{C}[x]$ respectively.
% \begin{enumerate}
%   \item The set of polynomials with rational coefficients is denoted $\mathbb{Q}[x]$. Does $\mathbb{Q}[x]$ with $+$ have group structure? Is it Abelian?
%   \item Does $\mathbb{Q}[x]$ under $\times$ have group structure? Is it Abelian?
% \end{enumerate}
%
% % --------------------------------------------- %
%
% \section{Day 4}
%
% Instead of saying $(S, +)$ ``has group structure'', we often simply call $S$ a \textbf{group} under $+$. In this sense, a group isn't really a ``thing'' --- it's just a list of properties something can have. We similarly refer to $(S, +)$ with Abelian group structure as an \textbf{Abelian group}. Over the following days, we will build examples of groups from diverse fields. 
%
% # Geometry and Symmetry
%
% SYMMETRIC GROUP PDF
%
% 1. Is $S_{n}$ a group? If so, is it an Abelian group?
% 2. How many elements are in $S_{n}$ for each positive integer $n$?
%
% % --------------------------------------------- %
%
% \section{Day 5}
%
% A group $(S, &)$ is called \textbf{finite} if $S$ is a finite set, and \textbf{infinite} otherwise. For instance, $(\mathbb{Z}, +)$ is an infinite group, while the set of permutations $S_{n}$ is a finite group. It's called the \textbf{symmetric group}, and is possibly the most important (and complicated) finite group.
%
% % --------------------------------------------- %
%
% \section{Day 6}
%
% The \textbf{dihedral group}
%
% % --------------------------------------------- %
%
% \section{Day 7}
%
% We now take a minute to examine groups from number theory: in particular, the integers $0, 1, \ldots, n - 1 \pmod{n}$.
%
% % --------------------------------------------- %
%
% \section{Day 8}
%
% The \textbf{cyclic group of order n} is the set $\{ e, a, \ldots, a^{n - 1} \}$ with multiplication, where $e$ is the identity and $a^{n} = e$. Here's an example of arithmetic in the cyclic group of order $4$:
% \[
%   a^{9} = (a^{4})(a^{4})(a) = (e)(e)(a) = a.
% \]
% The cyclic group is among the simplest examples of a finite group. It should be easy to see that cyclic groups are Abelian, since $(a^{n})(a^{m}) = (a^{m})(a^{n})$.
% 1. Explain why $\{ 1, i, -1, -i \}$ with multiplication is a cyclic group of order $4$.
% 2. Explain why $0, 1, \ldots, 5 \pmod{6}$ with addition is a cyclic group of order $6$.
%
% % --------------------------------------------- %
%
% \section{Day 9}
%
% There is one last example of a group I'd like to foster: matrix multiplication.
% \begin{enumerate}
%   \item Let $B$ be the set of $2$-by-$2$ matrices (with rational entries) under multiplication. Explain why $B$ is not a group.
%   \item The set of all $n$-by-$n$ \textit{invertible} matrices under multiplication is a group --- it's called the \textbf{general linear group}, and denoted $GL_{n}(\mathbb{R})$. Is the general linear group Abelian?
% \end{enumerate}
%
% % --------------------------------------------- %
%
% % SEND IN THE DISCORD:
% I am considering bringing back daily problems.
% - For several months, they'll discuss one single topic: group theory. Each problem would be bite-sized and digestible.
% - They will introduce new topics I didn't cover in lectures. I would attach a PDF to these questions that explains these concepts.
%
% % --------------------------------------------- %
%
% I want to first get a wide repository of groups under their belts.
%
% FIRST: Examples of groups.
% - Number systems. % Arithmetic operations on number systems. The 4th roots of unity.
% - Symmetry. % Symmetric group, dihedral groups.
% - Number Theory. % Modular arithmetic under addition or multiplication. Introduce the cyclic group.
% - Matrices. % The group of invertible matrices.
%
% NEXT: General properties. 
% % These problems will now take a radical turn. Instead of analyzing examples, we will prove properties about all groups.
%
% % Over the last week, we developed examples of groups from diverse mathematical fields: from arithmetic, symmetry and geometry, number theory, and linear algebra. Group Theory itself doesn't analyze these examples individually --- it examines all of them at once. It asks: what properties are derived from the group axioms alone? These properties automatically apply to ALL groups.
% - 1. \textbf{Uniqueness}: Prove that in any group, the identity is unique. Prove that the inverse is unique.
% - 2. \textbf{Properties about Inverses}: Remember that the inverse is unique. Prove that $(a^{-1})^{-1} = 1$. Prove that in any group, $()b)^{-1} b b^{-a}a^{-1}$.
% - 3. \textbf{Sanity Check}: We proved the four results above for any group. This means they apply to all our previous examples. State the four above results for invertible matrices in $GL_{n}(\mathbb{R})$ and for $\mathbb{Z}_{p}^{\times }$.
%
% NEXT: Homomorphisms and Isomorphisms
% - 4. \textbf{Homomorphisms}: Define a function from $\mathbb{Z}_{4}^{\times}$ to $\{ 1, i, -1, -i \}$ that is an isomorphism.
% - 6. \textbf{Isomorphisms}: 
%
% % --------------------------------------------- %

\end{document}
