\documentclass[11pt]{article}
\usepackage[T1]{fontenc}
\usepackage{geometry, changepage, hyperref}
\usepackage{amsmath, amssymb, amsthm, bm}
\usepackage{physics, esint}

\hypersetup{colorlinks=true, linkcolor=blue, urlcolor=cyan}
\setlength{\parindent}{0pt}
\setlength{\parskip}{5pt}

\newtheorem{theorem}{Theorem}
\newtheorem{lemma}{Lemma}
\newtheorem{proposition}{Proposition}
\newtheorem{corollary}{Corollary}
\newtheorem{claim}{Claim}
\newtheorem*{ex1}{Example: Symmetries of a Cube}

\renewcommand{\vec}[1]{\mathbf{#1}}
\newcommand{\uvec}[1]{\mathop{} \!\hat{\textbf{#1}}}
\newcommand{\mat}[1]{\mathbf{#1}}
\newcommand{\tensor}[1]{\mathsf{#1}}
\newcommand{\nll}{\operatorname{null}}
\newcommand{\range}{\operatorname{range}}
\newcommand{\cof}{\operatorname{cof}}

\newcommand{\s}{$\text{ } \\ \text{ }$}

\title{Fuck}
\author{Fuck}
\date{Fuck}

% --------------------------------------------- %

\begin{document}

\maketitle
\tableofcontents
\newpage

% --------------------------------------------- %

\section{Day 1}

An operation $\ast$ on a set $S$ is something that takes in two values of $S$ and spits out one value of $S$. For instance,
\begin{enumerate}
  \item Addition, subtraction, multiplication, and division are operations on rational numbers $\mathbb{Q}$.
  \item Exponentiation is an operation on positive integers: let $a \ast b \, = \, a^{b}$.
  \item Logic gates are an operation: they take in two numbers (each either $0$ or $1$) and spit out $0$ or $1$.
\end{enumerate}
An operation $\ast$ on $S$ is called \textbf{closed} if for all $a, b \in S$, the element $a \ast b \in S$ too. In other words, $\ast$ never "leaves" $S$. The operation $\ast$ is associative if for all $a, b, c \in S$, we have $a \ast (b \ast c) = (a \ast b) \ast c$. Furthermore, $\ast$ is \textbf{commutative} if $a \ast b = b \ast a$ for all $a, b \in S$. It's possible to have one of these without the other! 

% THE PROBLEMS: No formal proofs required.
% 1. Is division of complex numbers $\mathbb{C}$ closed? Is it associative or commutative?
% 2. Explain why the operation $\ast$ on reals $\mathbb{R}$ defined by $x \ast y = 2^{x + y}$ is commutative, but not associative.
% 3. Let $M$ be the set of all $2$-by-$2$ matrices. Do you see why multiplication of matrices is associative, but not commutative?

% --------------------------------------------- %

\section{Day 2}

We often denote $S$ with the operation $\ast$ by the pair $(S, \ast)$. There are two other properties of operations we care about:
\begin{enumerate}
  \item $(S, \ast)$ has an \textbf{identity element} if there exists $e$ in $S$ such that the operation ``does nothing'' with $e$ --- for all $a$ in $S$, we have $a \ast e \, = \, e \ast a \, = \, a$.
  \item $(S, \ast)$ has \textbf{inverses} if for \textit{all} $a$ in $S$, there exists some $b$ in $S$ that ``cancels'' $a$ --- namely $a \ast b \, = \, b \ast a \, = \, e$, where $e$ is the identity. 
\end{enumerate}
We often write the inverse of $a$ as $a^{-1}$, where this is \textit{not necessarily $1$ divided by $a$}. It's like how the inverse of a matrix $\mat{T}$ is written $\mat{T}^{-1}$. Notice that to have inverses, there must be an identity. Drill these five properties --- closure, associativity, identity, inverses, commutativity --- into your head! 

\newpage

% --------------------------------------------- %

\section{Day 3}

Listing out the properties of $(S, \ast)$ can get tiring --- so mathematicians gave bundles of these properties special names. $(S, \ast)$ has \textbf{group structure} if $\ast$ satisfies the following properties: for all $a, b, c \in G$,
\begin{enumerate}
  \item \textbf{Closure}: $a \ast b \in G$.
  \item \textbf{Associativity}: $a \ast (b \ast c) \, = \, (a \ast b) \ast c$.
  \item \textbf{Identity}: There exists $e \in G$ such that $a \ast e \, = \, e \ast a \, = \, a$.
  \item \textbf{Inverses}: There exists $a^{-1} \in G$ such that $a \ast a^{-1} \, = \, a^{-1} \ast a \, = \, e$.
\end{enumerate}
The operation $\ast$ need not be commutative --- that is, a set where $a \ast b$ isn't always $b \ast a$ can still be a group. If it \textit{is} commutative, we say $(S, \ast)$ has \textbf{Abelian group structure} (pronounced ah-BELL-ee-an).

% 1. Does $(Q[x], +)$ have group structure? If so, is it Abelian?
% 2. Does $(C[x], *)$ have group structure? If so, is it Abelian?

% Z[x] denotes the set of polynomials with integer coefficients. Similarly, Q[x], R[x], and C[x] denote the set of polynomials with rational, real, and complex coefficients. *You will see this notation again*.

% % --------------------------------------------- %

\section{Day 4}

% Instead of saying $(S, \ast)$ ``has group structure'', we often simply call $(S, \ast)$ a \textbf{group}. In this sense, a group isn't really a ``thing'' --- it's just a list of properties something can have. Similarly, $(S, +)$ is an \textbf{Abelian group} if it has Abelian group structure.

These examples of groups have been pretty shit, in my opinion. We're gonna examine much more interesting ones from geometry, number theory, and linear algebra. \textbf{This begins with geometry --- specifically symmetry}.

\begin{adjustwidth}{0.5cm}{}
  $\\$
  \textbf{Example: The Cube}. Paint each 8 faces of a cube a different color, and imagine the set of distinct rotations of the cube. We can compose two rotations by performing one after the other. Let's denote the set of rotations by $G$ and the operation by $\ast$.
  
  $(G, \ast)$ is closed and associate, because duh. There's an identity rotation --- namely doing nothing --- and each rotation can be ``performed in reverse'' to yield inverses. So, the symmetries $(G, \ast)$ of a cube constitute a group. 
  $\text{ } \\ \text{ }$
\end{adjustwidth}

In fact, \textbf{the symmetries of \textit{every} shape form a group}: octahedrons (this is an AMC problem!), spheres, you name it. You can always compose rotations of the shape, performing one after the other, and expect the four group properties to hold.

% https://www.google.com/search?q=rotations%20of%20a%20cube&tbm=isch&tbs=itp:animated&client=firefox-b-1-d&hl=en&sa=X&ved=0CAQQpwVqFwoTCMj8te-J64QDFQAAAAAdAAAAABAD&biw=1324&bih=725#imgrc=6WDXWosfGQOtJM

% 1. Calculate the number of elements of this group --- namely, the distinct rotations of a cube that leave it looking identical.
% 2. Show that (G, *) is not Abelian by a counterexample.

% In fact, the symmetries of subatomic particles and molecules form a group too. This is called \textit{gauge symmetry}.

% --------------------------------------------- %

\newpage

\section{Day 5}

Groups were discovered in the 1820s by the prodigy Évariste Galois at 18-years-old. Unfortunately, he died only one year later. Now it is your turn to study group theory. For fifty years after its inception, group theory studied one group: the \textbf{symmetric group}.

% 1. Let n be a positive integer. How many elements are in the group S_n?
% 2. Is S_n ever Abelian?
%
% --------------------------------------------- %
%
% \section{Day 6}
%
% The \textbf{dihedral group}
%
% % --------------------------------------------- %
%
% \section{Day 7}
%
% We now take a minute to examine groups from number theory: in particular, the integers $0, 1, \ldots, n - 1 \pmod{n}$.
%
% % --------------------------------------------- %
%
% \section{Day 8}
%
% The \textbf{cyclic group of order n} is the set $\{ e, a, \ldots, a^{n - 1} \}$ with multiplication, where $e$ is the identity and $a^{n} = e$. Here's an example of arithmetic in the cyclic group of order $4$:
% \[
%   a^{9} = (a^{4})(a^{4})(a) = (e)(e)(a) = a.
% \]
% The cyclic group is among the simplest examples of a finite group. It should be easy to see that cyclic groups are Abelian, since $(a^{n})(a^{m}) = (a^{m})(a^{n})$.
% 1. Explain why $\{ 1, i, -1, -i \}$ with multiplication is a cyclic group of order $4$.
% 2. Explain why $0, 1, \ldots, 5 \pmod{6}$ with addition is a cyclic group of order $6$.
%
% % --------------------------------------------- %
%
% \section{Day 9}
%
% There is one last example of a group I'd like to foster: matrix multiplication.
% \begin{enumerate}
%   \item Let $B$ be the set of $2$-by-$2$ matrices (with rational entries) under multiplication. Explain why $B$ is not a group.
%   \item The set of all $n$-by-$n$ \textit{invertible} matrices under multiplication is a group --- it's called the \textbf{general linear group}, and denoted $GL_{n}(\mathbb{R})$. Is the general linear group Abelian?
% \end{enumerate}
%
% % --------------------------------------------- %
%
% % SEND IN THE DISCORD:
% I am considering bringing back daily problems.
% - For several months, they'll discuss one single topic: group theory. Each problem would be bite-sized and digestible.
% - They will introduce new topics I didn't cover in lectures. I would attach a PDF to these questions that explains these concepts.
%
% % --------------------------------------------- %
%
% I want to first get a wide repository of groups under their belts.
%
% FIRST: Examples of groups.
% - Number systems. % Arithmetic operations on number systems. The 4th roots of unity.
% - Symmetry. % Symmetric group, dihedral groups.
% - Number Theory. % Modular arithmetic under addition or multiplication. Introduce the cyclic group.
% - Matrices. % The group of invertible matrices.
%
% NEXT: General properties. 
% % These problems will now take a radical turn. Instead of analyzing examples, we will prove properties about all groups.
%
% % Over the last week, we developed examples of groups from diverse mathematical fields: from arithmetic, symmetry and geometry, number theory, and linear algebra. Group Theory itself doesn't analyze these examples individually --- it examines all of them at once. It asks: what properties are derived from the group axioms alone? These properties automatically apply to ALL groups.
% - 1. \textbf{Uniqueness}: Prove that in any group, the identity is unique. Prove that the inverse is unique.
% - 2. \textbf{Properties about Inverses}: Remember that the inverse is unique. Prove that $(a^{-1})^{-1} = 1$. Prove that in any group, $()b)^{-1} b b^{-a}a^{-1}$.
% - 3. \textbf{Sanity Check}: We proved the four results above for any group. This means they apply to all our previous examples. State the four above results for invertible matrices in $GL_{n}(\mathbb{R})$ and for $\mathbb{Z}_{p}^{\times }$.
%
% NEXT: Homomorphisms and Isomorphisms
% - 4. \textbf{Homomorphisms}: Define a function from $\mathbb{Z}_{4}^{\times}$ to $\{ 1, i, -1, -i \}$ that is an isomorphism.
% - 6. \textbf{Isomorphisms}: 
%
% % --------------------------------------------- %

\end{document}
