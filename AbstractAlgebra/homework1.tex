\documentclass[11pt]{article}
\usepackage[T1]{fontenc}
\usepackage{geometry, changepage, hyperref}
\usepackage{amsmath, amssymb, amsthm, bm}
\usepackage{physics, esint}
\usepackage{enumitem}

\hypersetup{colorlinks=true, linkcolor=blue, urlcolor=cyan}
\setlength{\parindent}{0pt}
\setlength{\parskip}{5pt}

\newtheorem{theorem}{Theorem}
\newtheorem{lemma}{Lemma}
\newtheorem{corollary}{Corollary}
\newtheorem{claim}{Claim}

\newcommand{\Hom}{\operatorname{Hom}}
\newcommand{\Ker}{\operatorname{Ker}}
\newcommand{\Coker}{\operatorname{Coker}}
\newcommand{\Ann}{\operatorname{Ann}}

\title{MATH-UA 349: Homework 1}
\author{James Pagan}
\date{Professor Kleiner}

% --------------------------------------------- %

\begin{document}

\maketitle
\tableofcontents
\newpage

% --------------------------------------------- %

\section{Problem 1}

\begin{proof}
  The number $7 + \sqrt[3]{2}$ is a root of the polynomial $x^{3} - 21x^{2} + 147x - 345 = (x - 7)^{3} - 2$ in $\mathbb{Z}[x]$, as verified by the following computation:
  \begin{align*}
	  \big( (7 + \sqrt[3]{2}) - 7 \big)^{3} - 2 &= (\sqrt[3]{2})^{3} - 2 \\
  	&= 2 - 2 \\
  	&= 0.
  \end{align*}
  We conclude that $7 + \sqrt[3]{2}$ is algebraic.
\end{proof} 

% --------------------------------------------- %

\section{Problem 2}

\begin{proof} 
  We begin by characterzing the subring $R$ generated by $\alpha = \tfrac{i}{2}$. For all $n \in \mathbb{Z}_{\ge 0}$:
    \begin{enumerate}
   \item If $n \equiv 0 \pmod{4}$, then $\alpha^{n} = \frac{1}{2^{n}}$.
    \item If $n \equiv 1 \pmod{4}$, then $\alpha^{n} = \frac{i}{2^{n}}$.
   \item If $n \equiv 2 \pmod{4}$, then $\alpha^{n} = - \frac{1}{2^{n}}$.
   \item If $n \equiv 3 \pmod{4}$, then $\alpha^{n} = - \frac{i}{2^{n}}$.
  \end{enumerate}
  
  \begin{claim}
    If $n$ is an odd integer and $b + ci$ is a Gaussian integer, then $\tfrac{a + bi}{2^{n}} \in R$
  \end{claim}
  \begin{adjustwidth}{1cm}{}
    \begin{proof}\renewcommand{\qedsymbol}{}
      Suppose that $n \equiv 1 \pmod{4}$. Then 
      \[
        \frac{b + ci}{2^{n}} = \frac{b}{2^{n}} + \frac{ci}{2^{n}} = - 2b \left( - \frac{1}{2^{n + 1}} \right) + c \left( \frac{i}{2^{n}} \right) = -2b \alpha^{n + 1} + b \alpha^{n},
      \]
      which lies in $R$ by closure. Similarly, $n \equiv 3 \pmod{4}$ yields that
     \[
        \frac{b + ci}{2^{n}} = \frac{b}{2^{n}} + \frac{ci}{2^{n}} = 2b \left( \frac{1}{2^{n + 1}} \right) - c \left( -\frac{i}{2^{n}} \right) = 2b \alpha^{n + 1} - b \alpha^{n},
     \]
     which also lies in $R$.
   \end{proof}
  \end{adjustwidth}
  
We now examine closure. Let $z$ be a complex number, and define $S_{n} = \left\{ \tfrac{a + bi}{2^{n}} \mid a, b \in \mathbb{Z} \right\}$ for odd integers $n$; observe that $S_{n} \subset R$. It is a trivial exercise in geometry that the farthest $z$ lie away from an element of $S$ is ``half of the main diagonal'' of the smallest square --- more formally,
  \[
    \min \{ \abs{z - s} \, \mid \, s \in S_{n} \} \le \frac{1}{2^{n}} \left( \frac{\sqrt{2}}{2} \right) = \frac{\sqrt{2}}{2^{n + 1}}.
  \]
  For all $\epsilon > 0$, the Archmedian Property ensures the existence of an integer $N$ such that $\tfrac{\sqrt{2}}{2^{N + 1}} < \epsilon$. Then $N < n$ implies
  \[
    \min \{ \abs{z - s} \, \mid \, s_{n} \in S_{n} \} = \frac{\sqrt{2}}{2^{n + 1}} < \frac{\sqrt{2}}{2^{N + 1}} < \epsilon,
  \]
  so there exists $s \in S_{n} \subset R$ such that $\abs{z - s} < \epsilon$ for all $\epsilon$. We conclude that the subring generated by $\alpha$ is dense in $R$.
\end{proof}

% --------------------------------------------- %

\section{Problem 3}

\begin{proof}
  We tackle Part (c) first:
  \begin{adjustwidth}{1cm}{}
  	\begin{claim}
  		A number $m \in \mathbb{Z}_{n}$ is a unit if and only if $\gcd(n, m) = 1$.
  	\end{claim}
  	\begin{proof}\renewcommand{\qedsymbol}{}
  		Realize the following:
  		\begin{align*}
  			m \text{ is a unit in } \mathbb{Z}_{n} &\iff ma \equiv 1 \pmod{n} \text{ for some } a \in \mathbb{Z} \\
  			&\iff n \mid (ma - 1) \text{ for some } a \in \mathbb{Z} \\
  			&\iff nb = ma - 1 \text{ for some } a, b \in \mathbb{Z} \\
  			&\iff nb - ma = 1 \text{ for some } a, b \in \mathbb{Z} \\
  			&\iff \gcd(n, m) = 1.
  		\end{align*}
  		The last step is a direct application of Bézout's Identity.
  	\end{proof}
  \end{adjustwidth}

  We deduce the following answers for each part:
  \begin{enumerate}[label=(\alph*)]
  	\item The units are $1$, $5$, $7$, and $11$.
  	\item The units are $1$, $3$, $5$, and $7$.
  	\item The units are all $m \in \mathbb{Z}_{n}$ such that $\gcd(n, m) = 1$,
  \end{enumerate}
  as desired.
\end{proof}

% --------------------------------------------- %

\section{Problem 4}

\begin{proof}
  Performing polynomial divison yields that
  \[
    x^{4} + 3x^{3} + x^{2} + 7x + 5 = (x^{2} + 2x - 2)(x^{2} + x +1) + (7x + 7).
  \]
  If $x^{4} + 3x^{3} + x^{2} + 7x + 5$ divides $x^{2} + x + 1$ in $\mathbb{Z}_{n}$, then $7x + 7$ is must be the zero polynomial --- which occurs if and only if $7 \equiv 0 \pmod{n}$. The answer is $\boxed{\text{all positive $n$ such that $7 \mid n$}}$, \textbf{with a \textit{potential} inclusion of $n = 1$ if deemed a valid modulus}.
\end{proof}

% --------------------------------------------- %

\section{Problem 5}

% --------------------------------------------- %

\subsection{Part (a)}

\begin{proof}
  Rather routine calculations verify that $F[[x]]$ is a ring. We must first prove that $(F[[x]], +)$ is an Abelian group:
  \begin{enumerate}
    \item \textbf{Closure}: It is clear that if $f, g \in F[[x]]$, then $f + g \in F[[x]]$.
    \item \textbf{Associativity}: Since $F$ is a field, $f, g, h \in F[x]$ implies
    \[
      \big( (f + g) + h \big)(k) = \big( f(k) + g(k) \big) + h(k) = f(k) + \big( g(k) + h(k) \big) = \big( f + (g + h) \big)(k)
    \]
    for all $k \in \mathbb{Z}_{\ge 0}$; thus $(f + g) + h = f + (g + h)$
    \item \textbf{Identity}: It is easy to verify that $f(k) = 0$ is an additive identity of $F[[x]]$.
    \item \textbf{Invertability}: For $f \in F[[x]]$, define $-f$ by $(-f)(k) = -f(k)$. Then
    \[
      (-f)(k) + f(k) = -f(k) + f(k) = 0 = f(k) - f(k) = f(k) + (-f)(k)
    \]
    for all $k \in \mathbb{Z}_{\ge 0}$; thus $-f + f = 0$.
    \item \textbf{Commutativity}: See that $(f + g)(k) = f(k) + g(k) = g(k) + f(k) = (g + f)(k)$ for all $k \in \mathbb{Z}_{\ge 0}$; hence $f = g$.
  \end{enumerate}
  The multiplicative axioms are as follows:
  \begin{enumerate}\addtocounter{enumi}{5}
    \item \textbf{Closure}: It is clear that if $f, g \in F[[x]]$, then $fg \in F[[x]]$.
    \item \textbf{Associativity}: Observe that for all $k \in F$,
      \begin{align*}
        \big( (fg)h \big)(k) &= \sum\limits_{i + j = k} (fg)(i) h(j) = \sum\limits_{i + j = k} \left( \sum\limits_{a + b = i} f(a)g(b) \right) g(j) \\
        &= \sum\limits_{a + b + c = k} f(a)g(b)h(c) \\
        &= \sum\limits_{i + j = k} f(i) \left( \sum\limits_{a + b = j} g(a)h(b)  \right) = \sum\limits_{i + j = k} f(i)(gh)(j)  \\
        &= \big( f(gh) \big)(k).
      \end{align*}
      Therefore, $f(gh) = fg(h)$.
      \item \textbf{Identity}: Let $g(k) = 0$ if $k \ne 0$ and $g(0) = 1$. Then for all $f \in F[[x]]$, and $k \in F$,
      \[
        (fg)(k) = \sum\limits_{i + j = k} f(i)g(j) = f(k) = \sum\limits_{i + k = k} g(i)f(j) = (gf)(k).
      \]
      We conclude that $fg = gf = f$ for all $f \in F[[x]]$, so $g$ is a multiplicative identity.
  \end{enumerate}
  The two distributive axioms are as follows:
  \begin{enumerate}\addtocounter{enumi}{8}
    \item \textbf{Left Distributivity}: For all $f, g, h \in F[[x]]$ and $k \in F$, we have
    \begin{align*}
      \big( f(g + h) \big)(k) &= \sum\limits_{i + j = k} f(i)(g + h)(j) \\
                              &= \sum\limits_{i + j = k} f(i)g(j) + \sum\limits_{i + k = k} f(i)h(j) \\
                              &= (fg)(k) + (fh)(k).
    \end{align*}
    Thus $f(g + h) = fg + fh$.
    \item \textbf{Right Distributivity}: For all $f, g, h \in F[[x]]$ and $k \in F$, we have
    \begin{align*}
      \big( (f + g)h \big)(k) &= \sum\limits_{i + j = k} (f + g)(i)h(j) \\
                              &= \sum\limits_{i + j = k} f(i)h(j) + \sum\limits_{i + k = k} g(i)h(j) \\
                              &= (fh)(k) + (gh)(k).
    \end{align*}
    Thus $(f + g)h = fg + gh$.
  \end{enumerate}
  Therefore, $F[[x]]$ is a ring.
\end{proof}

\newpage

% --------------------------------------------- %

\subsection{Part (b)}

\begin{proof}
  Recall that the identity function of $F[[x]]$ is $1$ when $k = 0$ and $0$ otherwise. If $f \in F[[x]]$ has a multiplicative inverse $g$, then expanding these equations across all $k \ge 0$ yields
  \begin{align*}
   1 &= f(0)g(0)\\
   0 &= f(1)g(0) + f(0)g(1)  \\
   0 &= f(2)g(0) + f(1)g(1) + f(0)g(1) \\
     & \,\,\, \vdots \\
   0 &= f(k)g(0) + \cdots + f(0)g(k) \\
     & \, \, \, \vdots
  \end{align*}
  Solving for $g$ along each equation, we obtain a recursive formula:
  \begin{align*}
   g(0) &= \frac{1}{f(0)} \\
   g(1) &= - \frac{f(1)g(0)}{f(0)} \\
   g(2) &= - \frac{f(2)g(0) + f(1)g(1)}{f(0)} \\
        & \, \, \, \vdots \\
   g(k) &= \frac{f(k)g(0) + \cdots + f(1)g(k - 1)}{f(0)} \\
        & \, \, \, \vdots
  \end{align*}
  A straightforward induction verifies that this formula produces a multiplicative inverse. Naturally, this recursion can occur if and only if $\boxed{f(0) \ne 0}$.
\end{proof}

% --------------------------------------------- %

\section{Problem 6}

\begin{proof} 
  Suppose $\mathfrak{a}$ is a nonzero ideal of the Gaussian integers, and let $a + bi \in \mathfrak{a}$ for $a, b \in \mathbb{Z}$, not both equal to zero. Then
  \[
  	(a + bi)(a - bi) = a^{2} + b^{2} \in \mathfrak{a};
  \]
  noting that $a^{2} + b^{2} \in \mathbb{Z}+{> 0}$ completes the proof.
\end{proof}

% --------------------------------------------- %

\section{Problem 7}

% --------------------------------------------- %

\begin{proof}
  Since the operations upon $F$ are pointwise, verifying that $\Phi$ is a homomorphism is easy: for all $f, g \in R$ and $a \in F$, we have
  \begin{align*}
    \Phi(f + g)(a) = (f + g)(a) &= f(a) + g(a) = \Phi(f)(a) + \Phi(g)(a) \\
   \Phi(fg)(a) = (fg)(a) &= f(a)g(a) = \Phi(f)(a) \Phi(g)(a).
  \end{align*}
  As for the injectivity of $\Phi$, suppose that $\Phi(f)(a) = f(a) = 0$ for all $a \in F$. Consider $f$ in the algebraically closed extension of $F$: it has more roots than its degree, since the former is finite while the latter is infinite.
  
  We conclude that $f$ it must be the zero polynomial in this algebraically closed extension. Thus $f = 0$ in $F[x]$ as well, so $\Phi$ is injective.
\end{proof}

% --------------------------------------------- %

\section{Problem 8}

\begin{proof}
  Suppose $\phi : \mathbb{Z}[x] \to \mathbb{Z}[x]$ is an automorphism, and let $p = a_{n}x^{n} + \cdots + a_{1}x + a_{0}$ be any polymonial of $\mathbb{Z}[x]$, where $a_{n} \ne 0$. Then 
  \[
    \phi(p) = \phi(a_{n}x^{n} + \cdots + a_{1}x + a_{0}) = a_{n} \phi(x)^{n} + \cdots + a_{1} \phi(x) + a_{0} \phi(1).
  \]
  Hence, $\phi$ is uniquely determined by $\phi(x)$ and $\phi(1)$. \textbf{We claim the answer is as follows}: $\phi$ is an automorphism if and only if $\phi$ is an endomorphism and $\phi(x)$ is an affine function with leading coefficent $1$ or $-1$.
  
  Let $\phi$ be an automorphism. We wish to demonstrate that $\phi(x)$ is an affine function with leading coefficent $1$ or $-1$.
  
  \begin{adjustwidth}{1cm}{}
   \begin{claim}
     $\phi(x)$ is an affine function with leading coefficent $1$ or $-1$.
   \end{claim}
   \begin{proof}\renewcommand{\qedsymbol}{}
     Suppose for contradiction that $\deg \phi(x) > 1$; then the degree of all nonconstant polynomials in $\phi(\mathbb{Z}[x])$ is an integer multiple of $\phi(x)$, violating the injectivity of $\phi$. Hence $\phi(x)$ is an affine function of the form $bx + c$ for some $b, c \in \mathbb{Z}$.
  
     As noted in Claim 1, $b \ne 0$. Realize that the leading term of $\phi(p)$ is the leading term of $a_{n} \phi(x)^{n}$, which is
     \[
       a_{n} b^{n} x^{n}.
     \]
     We must have that $b^{n} = \pm 1$ in order for $\phi$ to be injective; thus $b = \pm 1$.
   \end{proof}
  \end{adjustwidth}

  Now, let $\phi$ be an endomorphism of $\mathbb{Z}[x]$ such that $\phi(x) = bx + c$, where $b \in \{ -1, 1 \}$ and $c \in \mathbb{Z}$. We wish to demonstrate that $\phi$ is an automorphism.

  \begin{adjustwidth}{1cm}{}
    \begin{claim}
      $\phi$ is injective.
    \end{claim}
    \begin{proof}\renewcommand{\qedsymbol}{}
      Let $p = a_{n}x^{n} + \cdots + a_{1}x + a_{0}$ be a polynomial in $\mathbb{Z}[x]$ of degree $n$ such that $\phi(p) = 0$; suppose for contradiction that $n \ge 1$. Then the leading coefficent $\pm a_{n}x^{n}$ of $\phi(p)$ of must be zero; hence $a_{n} = 0$, which violates the degree of $n$.
  
      Thus $p$ must be constant. Since $\phi(p) = p$ for all constant polynomials (a consequence of the fact $\phi(1) = 1$), we must have $p = 0$; thus $\Ker \phi = 0$, so $\phi$ is injective.
    \end{proof}
  \end{adjustwidth}

  \begin{adjustwidth}{1cm}{}
    \begin{claim}
      $\phi$ is surjective.
    \end{claim}
    \begin{proof}\renewcommand{\qedsymbol}{}
      We prove that for all $p \in \mathbb{Z}[x]$, there exists $s \in \mathbb{Z}[x]$ such that $\phi(s) = p$ by induction on the degree of $p$. For the base clase: clearly if $p$ is constant, then $\phi(p) = p$.
  
      For the inductive step: suppose that all polyomials of degree $n - 1$ or smaller lie within $\phi(\mathbb{Z}[x])$, and let $p = a_{n}x^{n} + \cdots + a_{n}x^{n} + a_{0}$, where $a_{n} \ne 0$. Then
      \[
        p \pm a_{n} (bx + c)^{n}
      \]
      is of degree $n - 1$ or smaller, where $\pm$ cancels out the leading coefficent of $p$, sign beng dependent on $b$ and the parity of $n$. Our inductive hypothesis guarantees the existence of a polynomial $s \in \mathbb{Z}[x]$ such that
      \[
        \phi(s) = p \pm a_{n}(bx + c)^{n}.
      \]
      Hence, we deduce that 
      \[
        \phi(s \mp a_{n}x^{n}) = \phi(s) \mp a_{n} \phi(x)^{n} = p \pm a_{n}(bx + c)^{n} \mp a_{n}(bx + c)^{n} = p.
      \]
      Thus all polynomials of degree $n$ lie within $\phi(\mathbb{Z}[x])$. This completes the induction.
    \end{proof}
  \end{adjustwidth}

  We conclude that $\phi$ is an automorphism, which implies the required result.
\end{proof}

\newpage

% --------------------------------------------- %

\section{Problem 9}

% --------------------------------------------- %

\subsection{Part (a)}
\begin{proof}
We claim that $\boxed{\mathbb{Z}[i] \,/\, (2 + i) \, \cong \, \mathbb{Z}_{5}}$, by the isomorphism: $\phi(a + bi) = a + 3b \pmod{5}$. Verifying that $\phi$ is a homomorphism is straightforward: if $a + bi$ and $c + di$ are Gaussian integers,
\begin{align*}
  \phi(a + bi) + \phi(c + di) &= a + 3b + c + 3d \pmod{5} \\
  &\equiv (a + c) + 3(b + d) \pmod{5} \\
  &= \phi \big( a + c + i(b + d) \big) \\
  &= \phi \big( (a + bi) + (c + di) \big).
\end{align*}
As per the multiplicative condition,
\begin{align*}
  \phi(a + bi) \phi(c + di) &= (a + 3b)(c + 3d) \pmod{5} \\
                            &\equiv ac + 3(ad + bc) + 9bd \pmod{5} \\
                            &\equiv (ac - bd) + 3(ad + bc) \pmod{5} \\
                            &= \phi \big( (ac - bd) + i(ad + bc) \big) \\
                            &= \phi \big( (a + bi)(c + di) \big).
\end{align*}
It is clear that $\phi(1) = 1$, so $\phi$ is a homomorphism; it is surjective, as $\phi(n) = n$ for $n \in \{ 0, \ldots, 4 \}$. We need only demonstrate that $\phi$ is injective. Suppose that $\phi(a + bi) = 0$, which implies $a + 3b \equiv 0 \pmod{5}$; we wish to demonstrate that $a + bi = (2 - i)z$ for a Gaussian integer $z$. See that
\[
  2a + b \equiv 2(a + 3b) \cong 0 \pmod{5} \qquad \text{and} \qquad -a + 2b = -1(a + 3b) \cong 0 \pmod{5}.
\]
Then let us divide $a + bi$ by $2 - i$: define
\[
  z = \frac{a + bi}{2 - i} = \frac{(a + bi)(2 - i)}{2^{2} - i^{2}} = \frac{(2a + b) + (-a + 2b)i}{5}.
\]
Since both the real and imaginary components of this fraction are divisible by $5$, we deduce that $z$ is a Gaussian integer. Then $a + bi$ is $0$ modulo $(2 + i)$, so $\Ker \phi = 0$. We conclude that $\phi$ is injective, which implies the desired isomorphism.
\end{proof}
% --------------------------------------------- %

\subsection{Part (b)}

\begin{proof}
We claim that $\boxed{\mathbb{Z}[x] \,/\, (x^{2} + 3, 5) \cong \mathbb{F}_{25}}$, the field with $25$ elements; since all finite fields of the same order are isomorphic, we need only demonstrate that $\mathbb{Z}[x] \,/\, (x^{2} + 3, 5)$ is a field with $25$ elements. 

Our proof will utilize the equivalent notation $\mathbb{Z}[x] \,/\, (x^{2} + 3, 5) = \mathbb{Z}_{5}[x] \,/\, (x^{2} + 3)$.

\newpage

Naturally, the elements of $\mathbb{Z}_{5}[x] \,/\, (x^{2} + 3)$ are the $25$ polynomials of the form $ax + b$, for $a,b \in \{ 0, \ldots, 4 \}$; this is because if $\deg p \ge 2$, there exist polynomials $q$ and $r$ with integer coefficents (ensured since $x^{2} + 3$ is monic) such that
\[
  p = (x^{2} + 3)q + r
\]
where $r$ is zero or $\deg r < 2$. Hence $p \equiv r \pmod{x^{2} + 3}$, and $r$ is of the aforementioned form $ax + b$. Hence the commutative ring $\mathbb{Z}_{5}[x] \,/\, (x^{2} + 5)$ has order $25$.

\begin{adjustwidth}{1cm}{}
  \begin{claim}
    Every nonzero polynomial in $\mathbb{Z}_{5}[x] \,/\, (x^{2} + 3)$ is a multiplicative unit.
  \end{claim}
  \begin{proof}\renewcommand{\qedsymbol}{}
    The nonzero constant polynomials $1, 2, 3, 4$ are units by Problem 3. Now, consider $ax$ for $a \ne 0$; since $x^{2} \equiv 2 \pmod{x^{2} + 3}$, we have that
    \[
      (ax)(3a^{-1}x) \equiv (a a^{-1})3x^{2} \equiv 3x^{2} \equiv 6 \equiv 1 \pmod{x^{2} + 3}.
    \]
    where $a^{-1}$ denotes the modular inverse of $a$; thus $ax$ is a unit. Now, consider $ax + b$ for $a, b \ne 0$; define $n$ as the multiplicative inverse of $2 - (a^{-1}b)^{2}$ (since squares modulo $5$ are congruent to $0$, $1$, or $4$, this quantity is never zero and is thus a unit). Then
    \begin{align*}
      (ax + b) \Big( n(a^{-1}x - a^{-2}b) \Big) &\equiv n(ax + b)(a^{-1}x - a^{-2}b) \pmod{x^{2} + 3} \\
      &\equiv n(x^{2} - a^{-2}b^{2}) \pmod{x^{2} + 3} \\
      &\equiv n(2 - (a^{-1}b)^{2}) \pmod{x^{2} + 3} \\
      &\equiv 1 \pmod{x^{2} + 3}.
    \end{align*}
    Thus every nonzero polynomial in $\mathbb{Z}_{5}[x] \,/\, (x^{2} + 3)$ is a unit.
  \end{proof}
\end{adjustwidth}

We conclude that $\mathbb{Z}_{5}[x] \,/\, (x^{2} + 3)$ is a field with 25 elements, so it is isomorphic to $\mathbb{F}_{25}$.
\end{proof}

% --------------------------------------------- %

\section{Problem 10}

\begin{proof}
Suppose for contradiction that $\phi : \mathbb{Z}[x] \,/\, (2x^{2} + 7) \to \mathbb{Z}[x] \,/\, (x^{2} + 7)$ is an isomorphism. Observe that $(2x^{2} + 7)$ and $(x^{2} + 7)$ are prime ideals of $\mathbb{Z}[x]$, so both quotient rings are integral domains. Since $\phi(n) = n$ for all constant polynomials $n$, we have
\begin{align*}
  \phi(0) = 0 &\implies \phi(x^{2} + 7) = 2x^{2} + 7 \\
  &\implies \phi(x^{2}) + \phi(7) = 2x^{2} + 7 \\
  &\implies \phi(x)^{2} + 7 = 2x^{2} + 7.
\end{align*}
We deduce that $\phi(x)^{2} = 2x^{2}$. Let $\phi(x) = ax + b$ for integers $a, b$; then
\[
  2x^{2} = (ax + b)^{2} = a^{2}x^{2} + 2abx + b^{2}.
\]
We must have that $2abx = 0$; thus $a = 0$ or $b = 0$. If $a = 0$, then $\phi(x)$ is a constant; the image of $\phi$ consists of constant polynomials, violating the injectivity of $\phi$. Thus $b = 0$ and $\phi(x) = ax$. This leaves us with the equation
\[
  2x^{2} = a^{2}x^{2} \implies (2 - a^{2})x^{2} = 0.
\]
Then $2 - a^{2} = 0$; however, no integer $a$ satisfies this equation. Any possibility of the value $\phi(x)$ leads to a contradiction, so $\boxed{\text{the rings are not isomorphic}}$.
\end{proof}

% --------------------------------------------- %


\end{document}
