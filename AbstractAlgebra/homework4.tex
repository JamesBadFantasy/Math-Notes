\documentclass[11pt]{article}
\usepackage[T1]{fontenc}
\usepackage{geometry, changepage, hyperref}
\usepackage{amsmath, amssymb, amsthm, bm}
\usepackage{physics, esint}

\hypersetup{colorlinks=true, linkcolor=blue, urlcolor=cyan}
\setlength{\parindent}{0pt}
\setlength{\parskip}{5pt}

\newtheorem{theorem}{Theorem}
\newtheorem{lemma}{Lemma}
\newtheorem{proposition}{Proposition}
\newtheorem{corollary}{Corollary}
\newtheorem{claim}{Claim}

\newcommand{\Hom}{\operatorname{Hom}}
\newcommand{\Ker}{\operatorname{Ker}}
\newcommand{\Coker}{\operatorname{Coker}}
\newcommand{\Ann}{\operatorname{Ann}}
\newcommand{\Spec}{\operatorname{Spec}}
\renewcommand{\longrightarrow}{\xrightarrow{\hspace*{0.7cm}}}

\newcommand{\conjugate}[1]{\overline{#1}}

\title{MATH-UA 349: Homework 4}
\author{James Pagan, February 2024}
\date{Professor Kleiner}

% --------------------------------------------- %

\begin{document}

\maketitle
\tableofcontents
\newpage

% --------------------------------------------- %

\section{Problem 1}

\begin{proof}
  For the first polynomial, we have that
  \begin{align*}
    x^{9} - x \, &= \, x(x^{8} + 1) \\
                 &= \, x(x^{8} + 8x^{7} + 28x^{6} + 56x^{5} + 70x^{4} + 56x^{3} + 28x^{2} + 8x + 1) \\
                 &= \, \boxed{x(x + 1)^{8}}.
  \end{align*}
  The second polynomial is a similar story:
  \begin{align*}
    x^{9} - 1 \, &= \, (x^{3} - 1)(x^{6} + x^{3} + 1) \\
                 &= \, \boxed{(x + 1)(x^{2} + x + 1)(x^{6} + x^{3} + 1)}.
  \end{align*}
  One can verify that $x^{6} + x^{3} + 1$ is irreducible through the Sieve of Eratosthenes.
\end{proof}

% --------------------------------------------- %

\section{Problem 2}

\begin{proof}
  Let $p(x) = x^{4} + 6x^{3} + 9x + 3$. We claim that $\boxed{\text{$p$ generates a maximal ideal}}$ in $\mathbb{Q}[x]$.
  
  \begin{adjustwidth}{0.5cm}{}
    \begin{lemma}
      $p$ is irreducible in $\mathbb{Q}[x]$.
    \end{lemma}
    \begin{proof}\renewcommand{\qedsymbol}{}
      By the Rational Root Theorem, the only possible rational roots of $p(x)$ are $-3$, $-1$, $1$, and $3$. A quick check verifies that none of these are roots of $p$ --- hence it has no rational roots.
  
      By the Factor Theorem, this ensures that no polynomial of the form $(x - q)$ for $q \in \mathbb{Q}$ divides $p$. Since these are the prime elements of the Euclidean domain $\mathbb{Q}[x]$, we conclude that $p$ is irreducible.
    \end{proof}
  \end{adjustwidth}
  
  Hence $p$ is a prime element of $\mathbb{Q}[x]$. Since $\mathbb{Q}[x]$ is a principal ideal domain, the ideal $(p)$ must be maximal.
\end{proof}

% --------------------------------------------- %

\section{Problem 3}

\begin{proof}
  We claim that the argument of $\pi$ in polar coordinates is a muliple of $\tfrac{\pi}{4}$.

  Suppose that $\pi = a + bi = r e^{i \theta}$ is a Gauss prime such that $\conjugate{\pi}$ and $\pi$ are associates --- that is, $\conjugate{\pi} = u \pi$ for some $u \in \{ 1, i, -1, -i \}$. This yields the equation
  \[
    r e^{-i \theta} = r e^{i \left( \theta + \tfrac{n \pi}{2} \right) }
  \]
  for some $n \in \mathbb{Z}$; equivalently, we find $-\theta = \theta + \tfrac{n \pi}{2}$; the solutions to this equation are of the form $\tfrac{4 k \pi}{8}$ for $k \in \{ 0, \ldots, 7 \}$. Hence in rectangular form, the Gauss prime $\pi$ has three forms:
  \begin{enumerate}
    \item $\pi$ is purely real.
    \item $\pi$ is purely imaginary.
    \item $\pi$ is on a diagonal of the complex plane --- that is, $\pi$ is of the form $a + ai$, $a - ai$, $-a - ai$, or $-a + ai$ for some $a \in \mathbb{Z}$.
  \end{enumerate}
  
  If $\pi$ is purely real, it must be a prime congruent to $3 \pmod{4}$. If $\pi$ is purely imaginary, then its associate is of the previous form. If $\pi$ lies on a diagonal, it is quite clear that $\pi$ is $1 + i$ or one of its associates; hence $\pi \cdot \conjugate{\pi} = 2$.
\end{proof}

% --------------------------------------------- %

\section{Problem 4}

\textbf{Case 1}: If $p \equiv 3 \pmod{4}$, then $p$ is a Gaussian prime; hence $\mathbb{Z}[i] \, / \, (p)$ is a field with $p^{2}$ elements: they have the form $a + bi$ for $a, b \in \{ 0, \ldots, p - 1 \}$. We conclude that $\boxed{\mathbb{Z}[i] \, / \, (p) \, \cong \, \mathbb{F}_{p^{2}}}$.

\textbf{Case 2}: If $p \equiv 1 \pmod{4}$, then $x^{2} + 1$ is not irreducible in $\mathbb{Z}_{p}$; hence there exist $a, b \in \mathbb{Z}_{p}$ such that $(x^{2} + 1) = (x + a)(x + b)$. As elaborated in Case 3, we have $a \ne b$. Hence we claim $\boxed{\mathbb{Z}[i] \, / \, (p) \, \cong \, \mathbb{Z}_{p}[x] \, / \, (x + a) \times \mathbb{Z}_{p}[x] \, / \, (x + b)}$. It is easy to see that
\begin{align*}
  \mathbb{Z}[i] \, / \, (p)  \, & \cong \,  \big( \mathbb{Z}[x] \, / \, (x^{2} + 1) \big) \, / \, (p) \\
                                 & = \, \big( \mathbb{Z}[x] \, / \, (p) \big) \, / \, (x^{2} + 1) \\
                                 & \cong \, \mathbb{Z}_{p}[x] \, / \, (x^{2} + 1) \\
                                 & = \, \mathbb{Z}_{p}[x] \, / \, (x + a)(x + b).
\end{align*}
Since $\mathbb{Z}_{p}[x]$ is a Euclidean domain, the ideals $(x + a)$ and $(x + b)$ are maximal. Hence $(x + a) + (x + b) = \mathbb{Z}_{p}[x]$; we conclude by the Chinese Remainder Theorem the desired
\[
  \mathbb{Z}[i] \, / \, (p) \, \cong \, \mathbb{Z}_{p}[x] \, / \, (x + a)(x + b) \, \cong \, \mathbb{Z}_{p}[x] \, / \, (x + a) \times \mathbb{Z}_{p}[x] \, / \, (x + b).
\]

\textbf{Case 3}: We claim $p = 2$ if and only if $a = b$. This is because if $(x + a)^{2} = (x + 1)$, then $a^{2} = 1$ and $a + a = 0$; thus
\[
  0 = 0(a) = (a + a)a = a^{2} + a^{2} = 2.
\]
The other direction is trivial since $(x + 1)^{2} = x^{2} + 1$ in $\mathbb{Z}_{2}$. Similar logic to the above yields that $\boxed{\mathbb{Z}[i] \, / \, (2) \, \cong \, \mathbb{Z}_{2}[x] \, / \, (x + 1)^{2}}$.

% --------------------------------------------- %

\section{Problem 5}
\begin{proof}
  Let $n = p_{1}^{e_{1}} \cdots p_{n}^{e_{n}} q_{1}^{f_{1}} \cdots q_{m}^{f_{m}}$, where the $p_{i}$ are prime integers congruent to $1 \pmod{4}$ and the $q_{i}$ are prime integers congruent to $3 \pmod{4}$. We claim that
  \[
    \boxed{\text{$n$ is a sum of squares if and only if $f_{1}, \ldots, f_{n}$ are even}}.
  \]
  First, we demonstrate that if $f_{1}, \ldots, f_{n}$ are even, then $n$ is a sum of two squares.
  
  \begin{adjustwidth}{0.5cm}{}
    \begin{lemma}[Brahmagupta-Diophantus Identity]
      Let $j, k \in \mathbb{Z}$ be sums of two squares. Then $jk$ is a sum of two squares.
    \end{lemma}
    \begin{proof}
      This follows from the identity
      \[
        (a^{2} + b^{2})(c^{2} + d^{2}) = (ac + bd)^{2} + (ad - bc)^{2}
      \]
      for integers $a$, $b$, $c$, and $d$.
    \end{proof}
  \end{adjustwidth}
  Since $f_{i}$ is even for all $i$, each $q_{i}^{f_{i}}$ is a square; thus the product $q_{1}^{f_{1}} \cdots q_{m}^{f_{m}}$ is a sum of squares by Lemma 2. By Fermat's Two-Square Theorem, all the primes $p_{i}$ are a sum of squares. Thus their product $p_{1}^{e_{1}} \cdots p_{n}^{e_{n}}$ is a sum of squares. Multiplying the $p_{i}$ and $q_{i}$ together, we obtain that $n$ is a sum of two squares.
  
  Next, we demonstrate that if $n$ is a sum of two squares, then $f_{i}$ are even. Suppose for contradiction that $f_{k_{1}}, \cdots f_{k_{j}}$ are odd; then
  \[
    q_{k_{1}} \cdots q_{k_{j}} \mid n.
  \]
  Expressing $n$ as a product of Gaussian primes, it is easy to attain that $n$ cannot be a square since $q_{k_{1}}, \ldots, q_{k_{j}}$ are Gaussian primes, using Euclid's Lemma.
\end{proof}

% --------------------------------------------- %

\section{Problem 6}

% --------------------------------------------- %

\subsection{Part (a)}

\begin{proof}
  Let $M$ be a simple $R$-module. Since $(M, +)$ is a simple Abelian group, it must be isomorphic to a finite cyclic group of prime order --- say $C_{p}$. Hence $M$ is generated by one element $x$.
  
  \begin{adjustwidth}{0.5cm}{}
    \begin{lemma}
      $M$ is isomorphic to a quotient of $R$.
    \end{lemma}
    \begin{proof}\renewcommand{\qedsymbol}{}
      Define a mapping $f : R \to M$ by the rule $f(a) = ax$. This is an $R$-module homomorphism, since $a, b \in R$ implies
      \begin{align*}
        f(a + b) \, = \, (a + b)x \, &= \, ax + bx \, = \, f(a) + f(b) \\
        f(ab) \, = \, abx \, &= \, a(bx) \, = \, a f(b).
      \end{align*}
      $\phi$ is surjective, since $f(1 + \cdots + 1) = x + \cdots + x$, which generates the entirety of $M$. Thus if we let $\mathfrak{m} = \Ker f$, the First Isomorphism Theorem yields the desired $R \, / \, \mathfrak{m} \, \cong \, M$.
    \end{proof}
  \end{adjustwidth}
  
  Because the ring $R \, / \, \mathfrak{m}$ has prime order, it contains no proper nonzero ideals. Thus the quotient is a field, so $\mathfrak{m}$ is maximal. This completes the proof.
\end{proof}

% --------------------------------------------- %

\subsection{Part (b)}

\begin{proof}
  Suppose $\phi : V \to V'$ is a homomorphism of simple $R$-modules. Then $V$ and $V'$ must be finite, and the submodules $\Ker \phi \subseteq V$ and $\Im \phi \subseteq V'$ must be either $0$ or the module itself.
  \begin{enumerate}
    \item If $\Ker \phi = V$: then $\phi$ is the zero homomorphism.
    \item If $\Ker \phi = 0$: then $V \, \cong \, \Im \phi$ by the First Isomorphism Theorem. Thus $\Im \phi$ is a nonzero submodule of $V'$, so $\Im \phi = V'$. We conclude that $V \, \cong \, V'$.
  \end{enumerate}
  This yields the desired result.
\end{proof}

% --------------------------------------------- %

\section{Problem 7}

% --------------------------------------------- %

\subsection{Part (a)}

For convenience, we denote by $x_{i}$ for each $i \in \{ 1, \ldots, n \}$ as the canonical basis $(\delta_{ik})_{k = 1}^{n}$ for each $i \in \{ 1, \ldots, n \}$, where $\delta$ is the Kronecker delta. Then the list
\[
  \phi(x_{1}), \ldots, \phi(x_{n}) \quad \in \quad R^{n}
\]
is linearly independent and has length $n$, so it must constitute a basis of $R^{n}$. Thus for $r_{1}, \ldots, r_{n} \in R$,
\[
  r_{1} \phi(x_{1}) + \cdots + r_{n} \phi(x_{n}) = 0 \quad \implies \quad r_{1} = \cdots = r_{n} = 0.
\]
Since the right-hand side equals $\phi(r_{1} x_{1} + \cdots + r_{n} x_{n})$, we conclude that $\phi$ has kernel $0$ --- thus $\phi$ is an isomorphism.

% --------------------------------------------- %

\subsection{Part (b)}

$\boxed{\text{No}}$. Consider the ring $\mathbb{Z}[x]$ and the prime ideals $(2) \subset (2, x)$. These ideals are free $\mathbb{Z}[x]$-modules; hence let $\phi : (2, x) \to (2, x)$ be the $\mathbb{Z}[x]$-module homommorphism defined by $\phi(y) = 2y$. Two observations:
\begin{enumerate}
  \item \textbf{Injectivity}: Holds. Clearly $\Ker \phi = 0$, since $\mathbb{Z}[x]$ is an integral domain.
  \item \textbf{Surjectivity}: Fails. $\phi$ maps the entirety of $(2, x)$ to $(2)$.
\end{enumerate}
Since $\phi$ is not an automorphism, it constitutes a counterexample to the stated claim.

% --------------------------------------------- %

\end{document}
