\documentclass[11pt]{article}
\usepackage[T1]{fontenc}
\usepackage{geometry, changepage, hyperref}
\usepackage{amsmath, amssymb, amsthm, bm}
\usepackage{physics, esint}

\hypersetup{colorlinks=true, linkcolor=blue, urlcolor=cyan}
\setlength{\parindent}{0pt}
\setlength{\parskip}{5pt}

\newtheorem{theorem}{Theorem}
\newtheorem{lemma}{Lemma}
\newtheorem{proposition}{Proposition}
\newtheorem{corollary}{Corollary}
\newtheorem{claim}{Claim}

\newcommand{\Hom}{\operatorname{Hom}}
\newcommand{\Ker}{\operatorname{Ker}}
\newcommand{\Coker}{\operatorname{Coker}}
\newcommand{\Ann}{\operatorname{Ann}}
\newcommand{\Spec}{\operatorname{Spec}}
\renewcommand{\longrightarrow}{\xrightarrow{\hspace*{0.7cm}}}

\title{MATH-UA 349: Homework 4}
\author{James Pagan, February 2024}
\date{Professor Kleiner}

% --------------------------------------------- %

\begin{document}

\maketitle
\tableofcontents
\newpage

% --------------------------------------------- %

\section{Problem 1}

% --------------------------------------------- %

\section{Problem 2}

% --------------------------------------------- %

\section{Problem 3}

% --------------------------------------------- %

\section{Problem 4}

% --------------------------------------------- %

\section{Problem 5}

% --------------------------------------------- %

\section{Problem 6}

% --------------------------------------------- %

\subsection{Part (a)}

Let $M$ be a simple $R$-module. Since $(M, +)$ is a simple Abelian group, it must be isomorphic to a finite cyclic group of prime order --- say $C_{p}$. Hence $M$ is generated by one element $x$.

\begin{adjustwidth}{0.5cm}{}
  \begin{lemma}
    $M$ is isomorphic to a quotient of $R$.
  \end{lemma}
  \begin{proof}\renewcommand{\qedsymbol}{}
    Define a mapping $f : R \to M$ by the rule $f(a) = ax$. This is an $R$-module homomorphism, since $a, b \in R$ implies
    \begin{align*}
      f(a + b) \, = \, (a + b)x \, &= \, ax + bx \, = \, f(a) + f(b) \\
      f(ab) \, = \, abx \, &= \, a(bx) \, = \, a f(b).
    \end{align*}
    $\phi$ is surjective, since $f(1 + \cdots + 1) = x + \cdots + x$, which generates the entirety of $M$. Thus if we let $\mathfrak{m} = \Ker f$, the First Isomorphism Theorem yields the desired $R \, / \, \mathfrak{m} \, \cong \, M$.
  \end{proof}
\end{adjustwidth}

Because the ring $R \, / \, \mathfrak{m}$ has prime order, it contains no proper nonzero ideals. Thus the quotient is a field, so $\mathfrak{m}$ is maximal. This completes the proof.

% --------------------------------------------- %

\subsection{Part (b)}

% --------------------------------------------- %

\section{Problem 7}

% --------------------------------------------- %

\subsection{Part (a)}

% --------------------------------------------- %

\subsection{Part (b)}

% --------------------------------------------- %

\end{document}
