\documentclass[11pt]{article}
\usepackage[T1]{fontenc}
\usepackage{geometry, changepage, hyperref}
\usepackage{amsmath, amssymb, amsthm, bm}
\usepackage{physics, esint}

\hypersetup{colorlinks=true, linkcolor=blue, urlcolor=cyan}
\setlength{\parindent}{0pt}
\setlength{\parskip}{5pt}

\renewcommand{\vec}[1]{\mathbf{#1}}
\newcommand{\uvec}[1]{\mathop{} \!\hat{\textbf{#1}}}
\newcommand{\mat}[1]{\mathbf{#1}}
\newcommand{\tensor}[1]{\mathsf{#1}}
\newcommand{\nll}{\operatorname{null}}
\newcommand{\range}{\operatorname{range}}

\newtheorem{theorem}{Theorem}
\newtheorem{lemma}{Lemma}
\newtheorem{proposition}{Proposition}
\newtheorem{corollary}{Corollary}
\newtheorem{claim}{Claim}

\title{MATH-UA 349: Honors Algebra II}
\author{James Pagán, March 2024}
\date{Professor Kleiner}

% --------------------------------------------- %

\begin{document}

\maketitle
\tableofcontents
\newpage

% --------------------------------------------- %

\section{Problem 1}

% --------------------------------------------- %

\subsection{Part (a)}

Performing reduction, it is easy to verify that
\[
  \begin{bmatrix} 1 & 0 & 0 \\ 0 & 2 & 0 \end{bmatrix} \quad = \quad \begin{bmatrix} 1 & -2 \\ 0 & 1 \end{bmatrix} \begin{bmatrix} 4 & 7 & 2 \\ 2 & 4 & 6 \end{bmatrix} \begin{bmatrix} 2 & 1 & 17 \\ -1 & 0 & -10 \\ 0 & 0 & 1
\end{bmatrix}
\]
Observe that $(x, y, z) \in \mathbb{Z}^{3}$ is mapped to zero by the left-hand side if and only if $x = y = 0$. Therefore, the solutions are 
\[
  \begin{bmatrix} 2 & 1 & 17 \\ -1 & 0 & -10 \\ 0 & 0 & 1 \end{bmatrix} \begin{bmatrix} 0 \\ 0 \\ z \end{bmatrix} \quad = \quad \boxed{\begin{bmatrix} 17z \\ -10z \\ z \end{bmatrix}}
\]
across all $z \in \mathbb{Z}$.

% --------------------------------------------- %

\end{document}
