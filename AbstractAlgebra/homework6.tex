\documentclass[11pt]{article}
\usepackage[T1]{fontenc}
\usepackage{geometry, changepage, hyperref}
\usepackage{amsmath, amssymb, amsthm, bm}
\usepackage{physics, esint}

\hypersetup{colorlinks=true, linkcolor=blue, urlcolor=cyan}
\setlength{\parindent}{0pt}
\setlength{\parskip}{5pt}

\newtheorem{theorem}{Theorem}
\newtheorem{lemma}{Lemma}
\newtheorem{proposition}{Proposition}
\newtheorem{corollary}{Corollary}
\newtheorem{claim}{Claim}

\title{MATH-UA 349: Homework 6}
\author{James Pagan, March 2024}
\date{Professor Kleiner}

% --------------------------------------------- %

\begin{document}

\maketitle
\tableofcontents
\newpage

% --------------------------------------------- %

\section{Problem 1}

Select nonzero $r \in R$ arbitrarily. Since $R$ is a vector space over $f$, we have $RF \subseteq F$; hence selecting nonzero $f \in F$, we have
\[
  rf \in F.
\]
Since $rf$ is nonzero, it has a multiplicative inverse $f^{\ast}$. We conclude that $r(f f^{\ast}) = (f f^{\ast})r = 0$, so $r$ is a unit. Hence $R$ is a field.

% --------------------------------------------- %

\end{document}
