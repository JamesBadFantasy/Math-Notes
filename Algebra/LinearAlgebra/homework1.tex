\documentclass[11pt]{article}
\usepackage[T1]{fontenc}
\usepackage{geometry, changepage}
\usepackage{amsmath, amssymb, amsthm, bm}
\usepackage{physics}
\usepackage{hyperref}

\hypersetup{colorlinks=true, linkcolor=blue, urlcolor=cyan}
\setlength{\parindent}{0pt}
\setlength{\parskip}{5pt}

\newtheorem{theorem}{Theorem}
\newtheorem{lemma}{Lemma}
\newtheorem{claim}{Claim}
\newtheorem*{theorem*}{Theorem}
\newtheorem*{lemma*}{Lemma}
\newtheorem*{claim*}{Claim}

\renewcommand{\vec}[1]{\mathbf{#1}}
\newcommand{\uvec}[1]{\mathop{} \!\hat{\mathbf{#1}}}
\newcommand{\mat}[1]{\mathbf{#1}}
\newcommand{\tensor}[1]{\mathsf{#1}}

\renewcommand{\div}{\nabla \cdot}
\renewcommand{\curl}{\nabla \cross}
\renewcommand{\grad}{\nabla}
\renewcommand{\laplacian}{\nabla^{2}}

\title{MATH-UA 148: Homework 1}
\author{James Pagan, September 2023}
\date{Professor Weare}

% --------------------------------------------- %

\begin{document}

\maketitle
\tableofcontents

\newpage

% --------------------------------------------- %

\section{1B Problems}

\subsection{Problem 1B.2}

If $a \vec{v} = 0$ and $a \ne 0$, then $\tfrac{1}{a}$ exists; we deduce that
\[
	\vec{v} = 1 \vec{v} = \left( \tfrac{1}{a} \cdot a \right) \vec{v} = \tfrac{1}{a} (a \vec{v}) = \tfrac{1}{a} (\vec{0}) = \vec{0}, 
\]
as desired.

\subsection{Problem 1B.5}

Denote a \textit{quasi-space} as a set $V$ over a field $F$ that satisfies all the vector axioms --- except \textit{potentially} the existence of additive inverses --- and where $0 \vec{v} = \vec{0}$ for all $\vec{v} \in V$. If $V$ is a quasi-space, $-1 \in \mathbb{F}$, so $(-1)\vec{v} \in V$; we have that
\[
	(-1)\vec{v} +\vec{v} = (-1) \vec{v} + 1 \vec{v} = (-1 + 1) \vec{v} = 0 \vec{v} = \vec{0}.
\]
All elements of $V$ thus have an additive inverse, which implies that $V$ is a vector space. Conversely, let $V$ be a vector space. For all $\vec{v} \in V$, we have that $0 \vec{v} = (0 + 0) \vec{v} = 0 \vec{v} + 0 \vec{v}$. Adding $-0 \vec{v}$ to both sides yields $\vec{0} = 0 \vec{v}$, so $V$ is a quasi-space.

We conclude that quasi-spaces and vector spaces are equivalent structures.

\subsection{Problem 1B.6}

The set $R \cup \infty \cup - \infty$ is not a vector space, as addition of infinities is not associative:
\begin{align*}
	\infty + (- \infty + (- \infty)) &= \infty + (- \infty) = 0 \\ (\infty + (- \infty)) + (- \infty) &= 0 + (- \infty) = - \infty.
\end{align*}

% --------------------------------------------- %

\section{1C Problems}

\subsection{Problem 1C.4}

Let the set of all continuous real-valued functions with integral $b$ from $0$ to $1$ be $X_{b}$.

For $X_{b}$ to be a vector space, we must have that $f(x) + g(x) \in X_{b}$ --- or equivalently, that the integral of $f(x) + g(x)$ is $b$ --- for all $f(x), g(x) \in X_{b}$. However, the integral of $f(x) + g(x)$ from $0$ to $1$ is $2b$ --- so $2b$ must equal $b$, and $b = 0$. We deduce that only $X_{0}$ can be a vector space.

\newpage

We now prove that $X_{0}$ is a subspace of $\mathbb{R}^{[0, 1]}$: 
\begin{itemize}
	\item \textbf{Additive Closure}: If $f, g \in X_{0}$, then $f + g$ is continuous, real-valued, and the integral of $f + g$ is $0 + 0 = 0$, so $f + g \in X_{0}$.
	\item \textbf{Multiplicative Closure}: If $a \in \mathbb{R}$ and $f \in X_{0}$, then $af$ is continuous, real-valued, and its integral from $0$ to $1$ is $a0 = 0$, so $af \in X_{0}$.
	\item \textbf{Additive Identity}: The function $h = 0$ is trivially in $X_{0}$, and for all $f \in X_{0}$, we have that $f + 0 = 0 + f = f$.
\end{itemize}
Therefore, $X_{0}$ is a subspace of $\mathbb{R}^{[0, 1]}$. We deduce that the set of continuous real-valued functions $f$ on the interval $[0, 1]$ such that $\int_{0}^{1} f = b$ is a subspace of $\mathbb{R}^{[0, 1]}$ if and only if $b = 0$.

\subsection{Problem 1C.6}

(a) \textbf{Yes}: Let $S = \{ (a, b, c) \in \mathbb{R}^{3} \mid a^{3} = b^{3} \}$. If $a^{3} = b^{3}$ for $a, b \in \mathbb{R}$, we have that $a = b$. Thus, $S = \{ (a, a, b) \in \mathbb{R}^{3} \}$. Now, we verify the conditions that $S$ is a subspace of $\mathbb{R}^{3}$:
\begin{itemize}
	\item \textbf{Additive Closure}: If $(a, a, b)$ and $(c, c, d) \in S$, then their sum is $(a + c, a + c, b + d)$; as the first and second coordinates of this vector are the same, it belongs to $S$. 
	\item \textbf{Multiplicative Closure}: If $(a, a, b) \in S$ and $x \in \mathbb{R},$ then their product is $(xa, xa, xb)$; as the first and second coordinates of this vector are the same, it belongs to $S$.
	\item \textbf{Additive Identity}: Clearly $(0, 0, 0) \in S$ such that $\vec{v} + (0, 0, 0) = (0, 0, 0) + \vec{v} = \vec{v}$ for all $\vec{v} \in S$.
\end{itemize}
Hence, $S$ is a subspace of $\mathbb{R}^{3}$.

(b) \textbf{No}: Let $S = \{ (a, b, c) \in \mathbb{C}^{3} \mid a^{3} = b^{3} \}$ and let $\omega$ be a nontrivial cubic root of unity as follows: 
\[
	\omega = -\frac{1}{2} + \frac{\sqrt{3}}{2} i \qquad \text{and} \qquad \omega^{2} = -\frac{1}{2} -\frac{\sqrt{3}}{2} i.
\]

Clearly $\omega^{3} = 1$ and $\omega - \omega^{2} = \sqrt{3} i$. See that $(-1, -1, 0)$ and $(\omega^{2}, \omega, 0)$ are elements of $S$. However, their sum $(\omega^{2} - 1, \omega - 1, 0)$ fails to meet the conditions to belong in $S$, as the cubes of its first and second coordinates are not equal:
\begin{align*}
	(\omega - 1)^{3} &= \omega^{3} - 3 \omega^{2} + 3 \omega - 1 = 3 \omega - 3 \omega^{2} = 3 \sqrt{3} i \\ 
	(\omega^{2} - 1)^{3} &= \omega^{6} - 3 \omega^{4} + 3 \omega^{2} - 1 = 3 \omega^{2} - 3 \omega = -3 \sqrt{3} i.
\end{align*}

As $S$ is not closed under addition, it cannot be a subspace of $\mathbb{C}^{3}$.

\subsection{Problem 1C.12}

Let $U$ and $W$ be subspaces of $V$. We procceed by the contrapositive --- namely, we prove $U \cup W$ is \textit{not} a subspace if and only if one of $U$ and $W$ does \textit{not} contain the other.

Suppose $U \cup W$ is not a subspace. Consider which conditions to be a subspace have not been met: clearly $\vec{0} \in U \cup W$, and if $a \in \mathbb{F}$, then $\vec{v} \in U \cup W$ implies that $\vec{v}$ lies in at least one of the two subspaces. Thus $a \vec{v}$ lies in the same subspace, and $a \vec{v} \in U \cup W$.

We conclude that the violated condition is additive closure --- namely, there exists $\vec{u}, \vec{v} \in U \cup W$ such that $\vec{u} + \vec{w} \notin U \cup W$. Note that $\vec{u}$ and $\vec{w}$ cannot both lie in $U$, as their sum would be in $U$; similarly, $\vec{u}$ and $\vec{w}$ cannot lie both in $W$. Then one of the two vectors lies \textit{only} in $U$ and the other lies \textit{only} in $W$ --- which implies that one of $U$ and $W$ cannot contain the other.

Conversely, suppose one of $U$ and $W$ does not contain the other. Then there exists $\vec{u} \in U$ and $\vec{w} \in W$ such that $\vec{u} \notin W$, and $\vec{w} \notin U$. Thus the sum $\vec{u} + \vec{w}$ cannot lie in either $U$ or $W$, so $\vec{u} + \vec{w} \notin U \cup W$. We deduce that $U \cup W$ is not closed under vector addition, and cannot be a subspace.

Taking the contrapositive yields the desired result.

\subsection{Problem 1C.20}

Define the subspace $W \subseteq \mathbb{F}^{4}$ as follows:
\[
	W = \{ (0, x, 0, y) \in \mathbb{F}^{4} \mid x, y \in \mathbb{F} \}.
\]
Observe that $(0, 0, 0, 0) \in W$, and $W$ is closed by addition and scalar multiplication, so it is a subspace of $\mathbb{F}^{4}$. Clearly, $U$ is a subspace of $\mathbb{F}^{4}$ as well.

\begin{adjustwidth}{1cm}{}
	\begin{claim*}
		$\mathbb{F}^{4} = U + W$
	\end{claim*}
    \begin{proof}\renewcommand{\qedsymbol}{}
		Let $(a, b, c, d)$ be any element of $\mathbb{F}^{4}$. Then $(a, a, c, c) \in U$ and $(0, b - a, 0, d - c) \in W$ such that 
		\[
			(a, b, c, d) = (a, a, c, c) + (0, b - a, 0, d - c) \in U + W.
		\]
		Therefore, $\mathbb{F}^{4} = U + W$ 
	\end{proof}
\end{adjustwidth}

Trivially, $U \cap W = (0, 0, 0, 0) = \{ \vec{0} \}$. Therefore, $U + W$ is a direct sum and $U \oplus W = \mathbb{F}^{4}$.

\subsection{Problem 1C.23}

The given result is \textbf{not true}. Consider the vector space $\mathbb{R}^{2}$, and define the following subspaces: 
\begin{align*}
	W &= \{ (a, 0) \in \mathbb{R}^{2} \mid a \in \mathbb{R} \}, \\
	U_{1} &= \{ (0, b) \in \mathbb{R}^{2} \mid b \in \mathbb{R} \}, \\
	U_{2} &= \{ (b, b) \in \mathbb{R}^{2} \mid b \in \mathbb{R} \}.
\end{align*}
It is trivial to verify that all three sets are subspaces of $\mathbb{R}^{2}$ and that $U_{1} \ne U_{2}$. We will now prove that $\mathbb{R}^{2} = W \oplus U_{1} = W \oplus U_{2}$.

Let $(x, y)$ be any vector in $\mathbb{R}^{2}$. We have that $(x, 0) \in W$ and $(0, y) \in U_{1}$ such that
\[
	(x, y) = (x, 0) + (0, y) \in W + U.
\] 
Thus, $W + U_{1} = \mathbb{R}^{2}$. Comparatively, note that $(x - y, 0) \in W$ and $(y, y) \in U_{2}$ such that
\[
	(x , y) = (x - y, 0) + (y, y) \in W + U_{2}.
\]
Thus, $W + U_{2} = \mathbb{R}^{2}$. Clearly $W \cap U_{1} = W \cap U_{2} = \{ (0, 0) \} = \{ \vec{0} \}$, as if $(a, 0) = (0, b)$, then $a = 0$ and $b = 0$, and if $(a, 0) = (b, b)$, then $a = b = 0$. Therefore,
\[
	\mathbb{R}^{2} = W \oplus U_{1} = W \oplus U_{2} \qquad \text{and} \qquad  U_{1} \ne U_{2}.
\]
We conclude that the given result is false.

% --------------------------------------------- %

\end{document}
