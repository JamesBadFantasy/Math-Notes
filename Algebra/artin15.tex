\documentclass[11pt]{article}
\usepackage[T1]{fontenc}
\usepackage{geometry, changepage, hyperref}
\usepackage{amsmath, amssymb, amsthm, bm}
\usepackage{physics, esint}
\usepackage{tgpagella, eulervm} 

\hypersetup{colorlinks=true, linkcolor=blue, urlcolor=cyan}
\setlength{\parindent}{0pt}
\setlength{\parskip}{5pt}

\newtheorem{theorem}{Theorem}
\newtheorem{lemma}{Lemma}
\newtheorem{proposition}{Proposition}
\newtheorem{corollary}{Corollary}
\newtheorem{claim}{Claim}

\renewcommand{\vec}[1]{\mathbf{#1}}
\newcommand{\uvec}[1]{\mathop{} \!\hat{\textbf{#1}}}
\newcommand{\mat}[1]{\mathbf{#1}}
\newcommand{\tensor}[1]{\mathsf{#1}}
\newcommand{\nll}{\operatorname{null}}
\newcommand{\range}{\operatorname{range}}
\newcommand{\cof}{\operatorname{cof}}

\newcommand{\Hom}{\operatorname{Hom}}
\newcommand{\Ker}{\operatorname{Ker}}
\newcommand{\Coker}{\operatorname{Coker}}
\newcommand{\Ann}{\operatorname{Ann}}
\newcommand{\Spec}{\operatorname{Spec}}
\renewcommand{\longrightarrow}{\xrightarrow{\hspace*{0.7cm}}}

\newcommand{\s}{$\text{ } \\ \text{ }$}

\title{Artin: Fields}
\author{James Pagan}
\date{March 2024}

% --------------------------------------------- %

\begin{document}

\maketitle
\tableofcontents
\newpage

% --------------------------------------------- %

\section{Fields}

A \textbf{field} is a commutative division ring. If $F \subseteq K$ is a pair of fields, we say $K$ is a \textbf{field extension} of $F$. This relation is denoted $K \, / \, F$; this is \textit{not} a quotient! Examples of fields are as follows:
\begin{enumerate}
  \item The motivation for examining field extensions originates from \textbf{number fields}: subfields of $\mathbb{C}$. All number fields are extensions of $\mathbb{Q}$. The classical questions regarding number fields concerned \textbf{algebraic number fields}, whose elements are algebraic.
  \item A \textbf{finite field} is a field that contains finitely many elements. Finite fields are gorgeous and colorful objects that obey beautiful, tight-knit properties.
  \item Extensions of the field $F(t)$ of rational functions are called \textbf{function fields}.
\end{enumerate}

% --------------------------------------------- %

\section{Algebraic and Transcendental Elements}

Let $K \, / \, F$ be a field extension and let $\alpha$ be an element of $K$. The element $\alpha$ is \textbf{algebraic over F} if is the root of a monic polynomial with coefficients in $F$ --- say, $f(\alpha) = 0$ for
\[
  f(x) \, = \, x^{n} + a_{n - 1}x^{n - 1} + \cdots + a_{0}, \quad \text{where} \quad a_{n - 1}, \ldots, a_{0} \in F,
\]
An element is \textbf{transcendental over F} if it is not algebraic. Both of these properties depend on the field $F$. Every element $\alpha \in F$ is algebraic over $F$ due to the monomial $x - \alpha$. We can elegantly describe this as a substitution homomorphism
\[
  \phi : F[x] \to X \quad \text{defined by} \quad x \leadsto \alpha.
\]
An element $\phi$ is transcendental if $\phi$ is injective and algebraic otherwise.

\begin{adjustwidth}{0.5cm}{}
  \begin{proposition}
    Let $\alpha \in K \, / \, F$ be an element of a field extension. The following conditions on a monic polynomial $f \in F[x]$ are equivalent:
    \begin{enumerate}
      \item $f$ is the unique monic polynomial of lowest degree in $F[x]$ with $\alpha$ as a root.
      \item $f$ is an irreducible element of $F[x]$ with $\alpha$ as a root.
      \item $f(\alpha) = 0$ and $(f)$ is a maximal ideal.
      \item If $g(\alpha) = 0$, then $f \mid g$.
    \end{enumerate}
  \end{proposition}
  \begin{proof}
    Since $F[x]$ is a Euclidean domain, the kernel of $\phi : F[x] \to K$ is a principal ideal generated by some polynomial $f$ of smallest degree. $f$ must be irreducible, or else a polynomial of smaller degree has a root at $\phi$; the other properties are easy to deduce.
  \end{proof}
\end{adjustwidth}
This polynomial is called the \textbf{minimal polynomial} of $\alpha$. The degree of the minimal polynomial of $\alpha$ is called the \textbf{degree} of $\alpha$. Whatever the case, the goal of this section is to examine the fields and rings generated by algebraic elements:

\begin{enumerate}
  \item The field $F(\alpha_{1}, \ldots, \alpha_{n})$ denotes the subfield of $K$ generated by $\alpha_{1}, \ldots, \alpha_{n}$.
  \[
    \text{$F(\alpha_{1}, \ldots, \alpha_{n})$ is the smallest subfield of $K$ that contains $F$ and $\alpha_{1}, \ldots, \alpha_{n}$}.
  \]
  \item The ring $F[\alpha_{1}, \ldots, \alpha_{n}]$ denotes the subring of $K$ generated by $\alpha_{1}, \ldots, \alpha_{n}$. The ring $F[\alpha]$ is isomorphic to the image of the substitution homomorphism $\phi : F[x] \to K$ as defined above.
\end{enumerate}

The field $F(\alpha)$ is isomorphic to the field of fractions of $F[\alpha]$. If $\alpha$ is transcendental, then $F[\alpha] \, \cong \, F[x]$ and $F(\alpha) \, \cong \, F(x)$; nowhere does a polynomial or rational function in $\alpha$ reduce. If $\alpha$ is algebraic,

\begin{adjustwidth}{0.5cm}{}
  \begin{proposition}
    Let $\alpha \in K \, / \, F$ be an algebraic element of a field extension. Let $f$ be the minimal polynomial of $\alpha$. Then the following holds:
    \begin{enumerate}
      \item The canonical map $\phi : F[x] \, / \, (f) \to F[\alpha]$ an isomorphism.
      \item $F[\alpha]$ is a field, hence $F[\alpha] = F(\alpha)$.
      \item More generally, $F[\alpha_{1}, \ldots, \alpha_{n}] = F(\alpha_{1}, \ldots, \alpha_{n})$ if $\alpha_{1}, \ldots, \alpha_{n} \in K \, / \, F$ are algebraic.
    \end{enumerate}
  \end{proposition}
  \begin{proof}
    Let $\phi : F[x] \to K$ be the aforementioned substitution homomorphism. Then $F[x] \, / \, \Ker \phi \, \cong \, K$. By Proposition 1, the kernel of $\phi$ is a maximal ideal generated by the minimal polynomial $f$, which yields (1) and (2). As per (3), an induction argument proceeds along these lines:
    \[
      F[\alpha_{1}, \ldots, \alpha_{n}] = F[\alpha_{1}, \ldots, \alpha_{n - 1}][\alpha_{n}] = F(\alpha_{1}, \ldots, \alpha_{n})[a_{k}] = F(\alpha_{1}, \ldots, \alpha_{n}).
    \]
    The omitted details are relatively easy to verify.
  \end{proof}
\end{adjustwidth}

The following proposition is a special case of one I omitted from Chapter 11.

\begin{adjustwidth}{0.5cm}{}
  \begin{proposition}
    Let $\alpha \in K \, / \, F$ be an algebraic element of a field extension. If $\deg \alpha = n$, then $1, \alpha \ldots, \alpha^{n - 1}$ is a basis for $F(\alpha)$ as a vector space over $F$. \s
  \end{proposition}
\end{adjustwidth}

Given two algebraic elements $\alpha \in K \, / \, F$ and $\beta \in L \, / \, F$ --- or given their minimal polynomials --- how can one determine whether $\alpha$ and $\beta$ generate the same field? We answer this question in Proposition 5.

\begin{adjustwidth}{0.5cm}{}
  \begin{proposition}
    Let $\alpha \in K \, / \, F$ and $\beta \in L \, / \, F$ be algebraic elements of field extensions. Then $\alpha$ and $\beta$ have the same minimal polynomial if and only if $F(\alpha) \, \cong \, F(\beta)$ --- in which case, the isomorphism is the identity on $F$ and maps $\alpha \leadsto \beta$
  \end{proposition}
  \begin{proof}
    Suppose that $\alpha$ and $\beta$ share the same minimal polynomial $f \in F[x]$. By Proposition 2, $F(\beta) \, \cong \, F[x] \, / \, (f) \, \cong \, F(\alpha)$; the additional conditions imposed upon the isomorphism are easy to verify.

    For the other direction,  suppose $F(\alpha) \, \cong \, F(\beta)$ by the described isomorphism. Let the minimal polynomial of $f$ be $\alpha$; by Proposition 5, $f(\alpha) = 0$ implies $f(\beta) = 0$ too --- hence the minimal polynomial of $\alpha$ divides the minimal polynomial of $\beta$. Observing that they're monic and share the same degree implies they are equal.
  \end{proof}
\end{adjustwidth}

Let $K \, / \, F$ and $K' \, / \, F$ be field extensions. An \textbf{F-isomorphism} is an isomorphism $\phi : K \to K'$ that restricts $F$ to the identity; the fields $K$ and $K'$ are \textbf{isomorphic field extensions}.

\begin{adjustwidth}{0.5cm}{}
  \begin{proposition}
    Let $\phi : K \to K'$ be an isomorphism of field extensions, and suppose $f \in F[x]$. Then $f(\alpha) = 0$ if and only if $f(\alpha') = 0$.
  \end{proposition}
  \begin{proof}
    It suffices to prove the theorem for the minimal polynomial of $\alpha$ --- thus redefine $f$ as such. The canonical epimorphism $K' \to K' \, / \, (f)$ may be decomposed as
    \[
      K' \longrightarrow K \longrightarrow K \, / \, (f) \longrightarrow K' \, / \, (f),
    \]
    of which $\phi(\alpha)$ vanishes; thus $f(\phi(\alpha)) = 0$. Alternatively, we could let $f(x) = a_{n}x^{n} + \cdots + a_{0}$, and observe that
    \[
      a_{n} \phi(\alpha)^{n} + \cdots + a_{0} \, = \, \phi(a_{n} \alpha^{n} + \cdots + a_{0}) \, = \, \phi(0) = 0.
    \]
    The symmetry of isomorphisms entails the desired bicondition.
  \end{proof}
\end{adjustwidth}

My intuition is that Proposition 5 should constrain the structure of field extensions --- but hell, what do I know. The following lemma regards the \textbf{characteristic} of a field.

\begin{adjustwidth}{0.5cm}{}
  \begin{lemma}
    The characteristic of a field is either $0$ or prime.
  \end{lemma}
  \begin{proof}
    If $F$ has characteristic $n = ab$ for $a, b \in \{ 2, \ldots, n - 1 \}$, we attain the following equation:
    \[
      \left( \sum\limits_{i = 1}^{a} 1 \right) \left( \sum\limits_{i = 1}^{b} 1 \right) \, = \, \sum\limits_{i = 1}^{n} 1 \, = \, 0
    \]
    Since $F$ is an integral domain, one of these is zero --- violating the minimality of $n$.
  \end{proof}
  \newpage
\end{adjustwidth}

% --------------------------------------------- %

\section{The Degree of a Field Extension}

Any field extension $K \, / \, F$ may be regarded as an $F$-vector space $K$. The \textbf{degree} $[K : F]$ of this field extension is the dimension of this vector space. 

\begin{adjustwidth}{0.5cm}{}
  \begin{theorem}[Multiplicative Property of the Degree]
    Let $L \, / \, K \, / \, F$ be field extensions. Then $[L : F] \, = \, [L : K] \, [K : F]$; hence each of $[L : K]$ and $[K : F]$ divides $[L : F]$.
  \end{theorem}
  \begin{proof}
    Let $\ell_{1}, \ldots, \ell_{n}$ be a basis of $L$ over $K$; let $k_{1}, \ldots, k_{n}$ be  basis of $K$ over $F$. We claim the products $\ell_{i}k_{j}$ constitute a basis of $L$ over $F$ --- which starts with demonstrating that they span $L$. For all $\ell \in L$, there exist $j_{1}, \ldots, j_{n}$ such that
    \[
      \ell \, = \, j_{1} \ell_{1} + \cdots + j_{n} \ell_{n}.
    \]
    Similarly, each $j_{i}$ factors in $K$ for $f_{i1}, \ldots, f_{im}$ as 
    \[
      j_{i} \, = \, f_{i1} k_{1} + \cdots + f_{im} k_{m}.
    \]
    Substituting this equation into the prior one yields a linear combination of $\ell$ into the terms $\ell_{i}k_{j}$. What remains to be demonstrated is their independence; suppose that
    \[
      0 \, = \, \sum\limits_{i = 1}^{n} \sum\limits_{j = 1}^{m} f_{ij} \ell_{i} k_{j} \, = \, \ell_{1} \left( \sum\limits_{j = 1}^{m} f_{1j} k_{j} \right) \, + \, \cdots \, + \, \ell_{n} \left( \sum\limits_{j = 1}^{n} f_{mj}k_{j} \right).
    \]
    Since $\ell_{1}, \ldots, \ell_{n}$ are a basis, each of these sums must be zero; since $k_{1}, \ldots, k_{m}$ are a basis, each $f_{ij}$ must be zero. The lengths of these bases imply the desired result.
  \end{proof}
\end{adjustwidth}

A field extension $K \, / \, F$ is a \textbf{finite extension} if $K$ its degree is finite. As we will see, finite extensions are an equivalent way to characterize extensions generated by algebraic elements:

\begin{adjustwidth}{0.5cm}{}
  \begin{lemma}
    Let $\alpha \in K \, / \, F$ be an element of a field extension. Then the following holds:
    \begin{enumerate}
      \item If $\alpha$ is algebraic, then $[F(\alpha) : F] \, = \, \deg \alpha$.
      \item $\alpha$ is algebraic if and only if $[F(\alpha) : F]$ is finite.
    \end{enumerate}
  \end{lemma}
  \begin{proof}
    Since $\alpha$ is algebraic, no linear combinations of $1, \alpha, \ldots, \alpha^{n - 1}$ yield zero; by the division algorithm, they span $F(\alpha)$. This yields (1). As per (2), $\alpha$ being algebraic implies $1, \alpha, \alpha^{2} \ldots, \alpha^{n - 1}$ spans $F[\alpha]$; otherwise, $1, \alpha, \alpha^{2}, \ldots$ is an infinite basis of $F[\alpha]$.
  \end{proof}
\end{adjustwidth}

Unfortunately, the reverse direction is more complex: a finite extension is generated by finitely many algebraic elements, but these may be distinct.

\begin{adjustwidth}{0.5cm}{}
  \begin{lemma}
    Suppose that $K \, / \, F$ is a field extension. Then the following holds:
    \begin{enumerate}
      \item $K$ is finite if and only if it is generated by finitely many algebraic elements.
      \item If $K$ is finite, then $\alpha$ is algebraic and $\deg \alpha$ divides $[K : F]$.
    \end{enumerate}
  \end{lemma}
  \begin{proof}
    If $K$ is finite, then let $\deg K = n$. There exists $\alpha_{1}, \ldots \alpha_{n}$ that constitute a basis of the $F$-vector space $K$ --- hence $K = F(\alpha_{1}, \ldots, \alpha_{n})$. On the contrary: if $K$ is generated by finitely many algebraic elements $\alpha_{1}, \ldots, \alpha_{n}$, then
    \begin{align*}
      [F(\alpha_{1}, \ldots, \alpha_{n}) : F] \, &= \, [F(\alpha_{1}, \ldots, \alpha_{n}) : F(\alpha_{1}, \ldots, \alpha_{n - 1})] \cdots [F(\alpha_{1}) : F(\alpha)] \\
                                                 &\le \, \deg \alpha_{n} \times \cdots \times \deg \alpha_{0} \\
                                                 &< \, \infty,
    \end{align*}
    so $K \, / \, F$ is a finite extension. For (2), the fact that $\alpha$ is algebraic follows from the fact that $F(\alpha)$ is an $F$-subspace of the finite-dimensional $F$-vector space $K$. As per the degree: if $\deg \alpha = n$:
    \[
      [K : F] \, = \, [K : F(\alpha)] \, [F(\alpha) : F] \, = \, [K : F(\alpha)] \deg \alpha.
    \]
    Hence $\deg \alpha$ divides $[K : F]$. This completes the proof.
  \end{proof}
\end{adjustwidth}

We now examine the relationship between the degrees of iterated field extensions.

\begin{adjustwidth}{0.5cm}{}
  \begin{corollary}
    Let $L \, / \, K \, / \, F$ be field extensions. If $\alpha \in L$ is $F$-algebraic, then $\alpha$ is $K$-algebraic and $\deg_{K} \alpha \, \le \, \deg_{F} \alpha$.
  \end{corollary}
  \begin{proof}
    If $\alpha \in L$ is algebraic over $F$, then there exist $f_{1}, \ldots, f_{n} \in L$ such that
    \[
      \alpha^{n} + f_{n - 1} \alpha^{n - 1} + \cdots + f_{0} \, = \, 0.
    \]
    Since $L \subseteq K$, this means $\alpha$ is a root of a polynomial in $K[x]$ --- hence $\alpha$ is algebraic over $K$. The degree is smaller than $n$ if the above polynomial reduces in $K$, and equal otherwise.
  \end{proof}
\end{adjustwidth}

Unfortunately, it is not true that $K$-algebraic elements are $F$-algebraic: consider $\mathbb{C} \, / \, \mathbb{R} \, / \, \mathbb{Q}$ with the $\mathbb{R}$-algebraic element $\pi$.

\begin{adjustwidth}{0.5cm}{}
  \begin{corollary}
    Let $F \, \subseteq \, K, K' \, \subseteq \, \mathcal{K}$ be field extensions, and let $F'$ be the field generated by $K$ and $K'$. Then
    \[
      [K : F] \, [K' : F] \, \ge \, [F' : F],
    \]
    yet both $[K : F]$ and $[K' : F]$ divide $[F' : F]$.
  \end{corollary}
  \begin{proof}
    Since $F \, \subseteq \, K, K' \, \subseteq \, F'$ are field extensions, the multiplicative property of the degree yields that $[K : F]$ and $[K' : F]$ divide $[F' : F]$. Now, let $\alpha_{1}, \ldots, \alpha_{n}$ and $\beta_{1}, \ldots, \beta_{n}$ be $F$-bases of $K$ and $K'$. Then the products $\alpha_{i} \beta_{j}$ \textit{span} $F'$, yielding the desired inequality.
  \end{proof}
\end{adjustwidth}
Two corollaries of the above result are as follows:
\begin{enumerate}
  \item $\operatorname{lcm}\big( [K : F], \, [K' : F] \big) \, \le \, [F' : F]$.
  \item If $[K : F]$ and $[K' : F]$ are relatively prime integers, then $[K : F] \, [K' : F] = [F' : F]$.
\end{enumerate}
This latter component is quite important.

% --------------------------------------------- %

\subsection{Low-Degree Field Extensions}

If $[K : F] = 2$, then $K \, / \, F$ is a \textbf{quadratic extension}. Similarly $[K : F] = 3$ entails that $K \, / \, F$ is a \textbf{cubic extension}. Quadratic and quintic extensions are defined similarly.

\begin{adjustwidth}{0.5cm}{}
  \begin{lemma}
    Let $\alpha \in K \, / \, F$ is an element of a field extension. Then the following holds:
    \begin{enumerate}
      \item $[K : F] = 1$ if and only if $K = F$.
      \item $\deg \alpha = 1$ if and only if $\alpha \in F$.
    \end{enumerate}
  \end{lemma}
  \begin{proof}
    If there was some element $\alpha \in K \setminus F$, then $1, \alpha$ would be independent in $K$ --- hence $[K : F] \ge 2$. The contrapositive yields (1). For (2), we have
    \[
      \deg \alpha = 1 \, \iff \, x - \alpha \text{ is the minimal polynomial of } \alpha \, \iff \, \alpha \in F.
    \]
    This concludes the proof.
  \end{proof}
\end{adjustwidth}

This classifies extensions with degree $1$. Extensions of degree $2$ have a simple story as well:

\begin{adjustwidth}{0.5cm}{}
  \begin{proposition}
    Suppose that the characteristic of $F$ is not $2$. Then an extension $K \, / \, F$ is quadratic if and only if adjoining $\delta^{2} = a \in F$ not in $F$ obtains $K$.
  \end{proposition}
  \begin{proof}
    Suppose that $K \, / \, F$ is quadratic. Then there exists $\alpha \in K \setminus F$, in which case $(1, \alpha)$ is a basis of $K$. Thus there exist $b, c \in F$ such that $\alpha^{2} = b \alpha + c$. Deriving the quadratic formula by completing the square, we find
    \[
      \alpha = \frac{-b \pm \sqrt{b^{2} - 4c}}{2}.
    \]
    Because $\alpha \notin F$, the element $b^{2} - 4c$ must not be a square in $F$. If $\delta$ is one of these square roots, it is clear that $(1, \delta)$ spans $K$ --- hence $F(\delta) = K$. The contrary is trivial.
  \end{proof}
\end{adjustwidth}
We give the reader three sets of exercises:
\begin{enumerate}
  \item Let the two complex roots of $x^{3} - 2 = 0$ be $\alpha$ and $\alpha^{2}$. Calculate $[\mathbb{Q}(\alpha, \alpha^{2}) : \mathbb{Q}]$.
  \item Let $\beta$ be a root of the irreducible polynomial $x^{4} + x + 1$ over $\mathbb{Q}$. Prove that $\sqrt[3]{2} \notin \mathbb{Q}(\beta)$.
  \item Calculate (with proof) the degree of $i$ over $\mathbb{Q}(\sqrt{2})$.
\end{enumerate}
These are easy corollaries from the above theorems.

% --------------------------------------------- %

\end{document}
