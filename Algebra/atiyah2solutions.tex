\documentclass[11pt]{article}
\usepackage[T1]{fontenc}
\usepackage{geometry, changepage, hyperref}
\usepackage{amsmath, amssymb, amsthm, bm}
\usepackage{physics, esint}
\usepackage{tgpagella, eulervm}

\hypersetup{colorlinks=true, linkcolor=blue, urlcolor=cyan}
\setlength{\parindent}{0pt}
\setlength{\parskip}{5pt}

\newtheorem{theorem}{Theorem}
\newtheorem{lemma}{Lemma}
\newtheorem{corollary}{Corollary}
\newtheorem{proposition}{Proposition}
\newtheorem{claim}{Claim}

\newcommand{\Hom}{\operatorname{Hom}}
\newcommand{\Ker}{\operatorname{Ker}}
\newcommand{\Coker}{\operatorname{Coker}}
\newcommand{\Ann}{\operatorname{Ann}}
\newcommand{\Spec}{\operatorname{Spec}}
\renewcommand{\longrightarrow}{\xrightarrow{\hspace*{0.7cm}}}

\newcommand{\s}{$\text{ } \\ \text{ }$}

\title{Atiyah-MacDonald: Modules Exercises}
\author{James Pagan}
\date{March 2024}

% --------------------------------------------- %

\begin{document}

\maketitle
\tableofcontents
\newpage

% --------------------------------------------- %

\section{Problem 1}

\begin{proof}
  Let the modular inverse of $m \pmod{n}$ be $m^{-1}$. Then for all $a, b \in \mathbb{Z}_{n} \otimes \mathbb{Z}_{m}$,
  \[
    a \otimes b \, = \, mm^{-1}a \otimes b \, = \, m_{-1}a \otimes mb \, = \, m^{-1}a \otimes 0 \, = \, 0.
  \]
  We conclude that $\mathbb{Z}_{n} \otimes \mathbb{Z}_{m} \, = \, 0$.
\end{proof}


% --------------------------------------------- %

\section{Problem 2}

\begin{proof}
  Consider the exact sequence
  \[
    \mathfrak{a} \stackrel{i}{\longrightarrow} A \stackrel{\pi}{\longrightarrow} A \, / \, \mathfrak{a} \longrightarrow 0,
  \]
  where $i$ is the inclusion map and $\pi$ is the canonical epimorphism. Tensoring with $M$, we find that
  \[
    \mathfrak{a} \otimes_{A} M \stackrel{i \otimes_{A} 1}{\longrightarrow} A \otimes_{A} M \stackrel{\pi \otimes_{A} 1}{\longrightarrow} (A \, / \, \mathfrak{a}) \otimes_{A} M \longrightarrow 0
  \]
  is an exact sequence. Observe that $A \otimes_{A} M \, \cong \, M$ by the mapping $f(a, x) = ax$; hence there exists a surjective mapping
  \[
    M \stackrel{(\pi \otimes_{A} 1) \circ g}{\longrightarrow} (A \, / \, \mathfrak{a}) \otimes_{A} M,
  \]
  where $g(x) = 1 \otimes_{A} x$. It is easy to verify that the kernel of this homomorphism is all elements of the form $a \otimes_{A} x$ for all elements $a \in \mathfrak{a}$ --- in other words, $\mathfrak{a} M$. Hence the First Isomorphism Theorem yields
  \[
    M \, / \, \mathfrak{a} M \, \cong \, (A \, / \, \mathfrak{a}) \otimes_{A} M.
  \]
  This completes the proof.
\end{proof}

% --------------------------------------------- %

\section{Problem 3}

\begin{proof}
  Let $\mathfrak{m}$ be the sole maximal ideal of $A$. Realize that $M \otimes_{A} N \, = \, 0$ implies that 
  \[
    (A \, / \, \mathfrak{m}) \otimes_{A} (M \otimes_{A} N) \otimes_{A} (A \, / \, \mathfrak{m}) \, = \, 0 \, \implies \, M_{(A / \mathfrak{m})} \otimes_{A} N_{(A / \mathfrak{m})} \, = \, 0.
  \]
  However, $M_{(A / \mathfrak{m})}$ are vector spaces over the field $A \, / \, \mathfrak{m}$. Thus we have (probably)
  \[
    0 \, = \, \dim \big( M_{(A / \mathfrak{m})} \otimes_{A} N_{(A / \mathfrak{m})} \big) \, = \, \dim M_{(A / \mathfrak{m})} \times \dim N_{(A / \mathfrak{m})}.
  \]
  Thus one of $M_{(A / \mathfrak{m})}$ or $N_{A / \mathfrak{m}}$ must be zero. Without loss of generality, let $M_{(A / \mathfrak{m})}$ be zero; thus by exercise 2,
  \[
    M_{(A / \mathfrak{m})} \, = \, 0 \, \implies \, (A \, / \, \mathfrak{m}) \otimes_{A} M \, = \, 0 \, \implies \, M \, / \, \mathfrak{m} M \, = \, 0.
  \]
  Thus since $M$ is finitely-generated, $M = \mathfrak{m} M$. By Nakayama's Lemma, we conclude $M = 0$. This completes the proof.
\end{proof}

% --------------------------------------------- %

\section{Problem 4}

\begin{proof}
  We utilize the following lemma. The proof is straightforward, omitted for brevity:
  \begin{adjustwidth}{0.5cm}{}
    \begin{lemma}
      Let $P_{i} \stackrel{f_{i}}{\longrightarrow}  Q_{i}$ be homomorphisms of $A$-modules. Then
      \[
        \bigoplus\limits_{i} P_{i} \stackrel{\bigoplus\limits_{i} f_{i}}{\longrightarrow} \bigoplus\limits_{i} Q_{i}
      \]
      is injective if and only if each $f_{i}$ is injective.
    \end{lemma}
  \end{adjustwidth}
  We are ready to tackle the problem at hand. Let $N_{1} \stackrel{f}{\longrightarrow} N_{2}$ be any monomorphism of $A$-modules. Then
  \begin{align*}
    \text{$\bigoplus\limits_{i} M_{i}$ is flat} \, &\iff \, N_{1} \otimes \bigoplus\limits_{i} M_{i} \stackrel{f \otimes \sum\limits_{i} 1_{i}}{\longrightarrow} N_{2} \otimes \bigoplus\limits_{i} N_{i} \text{ is injective} \\
                                                   &\iff \, \bigoplus\limits_{i} (N_{1} \otimes M_{i}) \stackrel{\bigoplus\limits_{i} (f \otimes 1_{i})}{\longrightarrow} \bigoplus\limits_{i} (N_{2} \otimes M_{i}) \text{ is injective} \\
                                                   &\iff \, N_{1} \otimes M_{i} \stackrel{f \otimes 1_{i}}{\longrightarrow} N_{2} \otimes M_{i} \text{ is injective for each } i \\ \text{ } \\
                                                   &\iff \, M_{i} \text{ is flat for each } i.
  \end{align*}
  This completes the proof. Hence, all free modules are flat.
\end{proof}

% --------------------------------------------- %

\end{document}
