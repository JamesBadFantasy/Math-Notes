\documentclass[11pt]{article}
\usepackage[T1]{fontenc}
\usepackage{geometry, changepage, hyperref}
\usepackage{amsmath, amssymb, amsthm, bm}
\usepackage{physics, esint}
\usepackage{tgpagella, eulervm}

\hypersetup{colorlinks=true, linkcolor=blue, urlcolor=cyan}
\setlength{\parindent}{0pt}
\setlength{\parskip}{5pt}

\newtheorem{theorem}{Theorem}
\newtheorem{lemma}{Lemma}
\newtheorem{proposition}{Proposition}
\newtheorem{corollary}{Corollary}
\newtheorem{claim}{Claim}

\renewcommand{\vec}[1]{\mathbf{#1}}
\newcommand{\uvec}[1]{\mathop{} \!\hat{\textbf{#1}}}
\newcommand{\mat}[1]{\mathbf{#1}}
\newcommand{\tensor}[1]{\mathsf{#1}}
\newcommand{\nll}{\operatorname{null}}
\newcommand{\range}{\operatorname{range}}

\renewcommand{\grad}{\nabla}
\renewcommand{\div}{\nabla \cdot}
\renewcommand{\curl}{\nabla \cross}

\title{Axler: Vector Spaces}
\author{James Pagan}
\date{March 2024}
% I learned these much, much earlier than March 2024...

% --------------------------------------------- %

\begin{document}

\maketitle
\tableofcontents
\newpage

% --------------------------------------------- %

\section{Vector Spaces}

% --------------------------------------------- %

An \textbf{vector space} over a field $F$ is an Abelian group $V$ (with operation written additively) endowed with a mapping $\mu : F \times V \to V$ (written multiplicatively) such that the following axioms are satisfied for all $\vec{v}, \vec{w} \in V$ and $a, b \in R$:
\begin{enumerate}
	\item $1 \vec{v} = \vec{v}$;
	\item $(ab)\vec{v} = a(b \vec{v})$;
	\item $a(\vec{v} + \vec{w}) = a \vec{v} + a \vec{w}$;
	\item $(a + b) \vec{v} = a \vec{v} + b \vec{v}$.
\end{enumerate}

Elements of $V$ are called \textbf{vectors}. Since $(V, +)$ is an Abelian group, it has a unique additive identity, unique inverses, and satisfies $-(- \vec{v}) = \vec{v}$ and $- (\vec{v} + \vec{w}) = - \vec{v} - \vec{w}$ for all $\vec{v}, \vec{w} \in V$. The additive identity of $V$ is denoted $\vec{0}$ and the additive inverse of $\vec{v}$ is denoted $-\vec{v}$.

\begin{adjustwidth}{0.5cm}{}
  \begin{theorem}
    Let $V$ be a $F$-vector space. Then the following holds for all $\vec{v}, \vec{w} \in V$ and $a \in F$:
    \begin{enumerate}
      \item $0 \vec{v} = \vec{0}$.
      \item $a \vec{0} = \vec{0}$.
      \item $(-1) \vec{v} = - \vec{v}$.
    \end{enumerate}
  \end{theorem}
  \begin{proof}
    All three properties follow from the distributive laws. For (1), we have that
    \[
      0 \vec{v} + 0 \vec{v} \, = \, (0 + 0) \vec{v} = 0 \vec{v}.
    \]
    Subtracting both sides by $0 \vec{v}$ yields that $0 \vec{v} = \vec{0}$. For (2), a similar proof holds:
    \[
      a \vec{0} + a \vec{0} = a(\vec{0} + \vec{0}) = a \vec{0},
    \]
    hence $a \vec{0} = \vec{0}$. The third property is quite easy as well: we have that
    \[
      -1 \vec{v} + \vec{v} \, = \, (-1 + 1) \vec{v} \, = \, 0 \vec{v} \, = \, \vec{0} \, = \, 0 \vec{v} \, = \, (1 - 1) \vec{v} \, = \, \vec{v} + (-1 \vec{v})
    \]
    Hence $-1 \vec{v}$ is the unique inverse of $\vec{v}$, that being $-\vec{v}$.
  \end{proof}
\end{adjustwidth}

A vector space over $\mathbb{R}$ is a \textbf{real vector space}, while vector spaces over $\mathbb{C}$ are \textbf{complex vector spaces}.

% --------------------------------------------- %

\section{Subspaces}

A subset $U \subseteq V$ is a \textbf{subspace} if it is a vector space under the field and operations of $V$.

\begin{adjustwidth}{0.5cm}{}
  \begin{theorem}
    A subset $U \subseteq V$ is a subspace if and only if $\vec{0} \in U$ and $U$ is closed under addition and scalar multiplication.
  \end{theorem}
  \begin{proof}
    Suppose $U$ satisfies the three desired properties. Then $(U, +) \subseteq (V, +)$ is an Abelian subgroup; once multiplicative closure is ensured, the four other properties are inherited from $V$.
  \end{proof}
\end{adjustwidth}

Let $V$ be an $F$-vector space with subspaces $V_{1}, \ldots, V_{m}$. We consider two crucial operations on these subspaces: 

\begin{enumerate}
	\item \textbf{Sum}: The sum $V_{1} + \cdots + V_{n}$ is the set of all sums $m_{1} + \cdots + m_{n}$, where $m_{i} \in V_{i}$ ($i \in \{ 1, \ldots, n \}$). It is the smallest subspace of $V$ that contains all $V_{1}, \ldots, V_{n}$.
	\item \textbf{Intersection}: The intersection $V_{1} \cap \cdots \cap V_{n}$ is the largest subspace of $V$ that is contained inside each $V_{1}, \ldots, V_{n}$.
\end{enumerate}

Let $V_{1}, \ldots, V_{n}$ be $F$-vector spaces. The \textbf{direct sum} $V_{1} \oplus \cdots \oplus V_{n}$ is the set of all formal pairs $(\vec{v}_{1}, \ldots, \vec{v}_{n})$, with addition and scalar multiplication defined componentwise.

\begin{adjustwidth}{0.5cm}{}
  \begin{theorem}
    Let $V_{1}, \ldots, V_{n} \subseteq V$ be $F$-subspaces. Then the following holds:
    \begin{enumerate}
      \item $V_{1} + \cdots + V_{n} \, \cong \, V_{1} \oplus \cdots \oplus V_{n}$ if and only if $\vec{v}_{1} + \cdots + \vec{v}_{n} = \vec{0}$ for $\vec{v}_{i} \in V_{i}$ implies that $\vec{v}_{1} = \cdots = \vec{v}_{n} = \vec{0}$.
      \item $V_{1} + \cdots + V_{n} \, \cong \, V_{1} \oplus \cdots \oplus V_{n}$ if and only if each vector in the former decomposes as a sum of $\vec{v}_{i} \in V_{i}$ uniquely.
    \end{enumerate} 
  \end{theorem}
  \begin{proof}
    Suppose that $\vec{v}_{1} + \cdots + \vec{v}_{n} = \vec{0}$ for $\vec{v}_{i} \in V_{i}$ implies that $\vec{v}_{1} = \cdots = \vec{v}_{n} = \vec{0}$. Then define a linear map
    \[
      V_{1} \oplus \cdots \oplus V_{n} \, \mapsto \, V_{1} + \cdots + V_{n} \quad \text{by} \quad \vec{v}_{1}, \ldots, \vec{v}_{n} \, \leadsto \, \vec{v}_{1} + \cdots + \vec{v}_{n}.
    \]
    By definition, this map is surjective; it is injective by our hypothesis. Hence the two vector spaces are isomorphic. The converse is easy to deduce --- while result (2) is a mere corollary of the first.
  \end{proof}
\end{adjustwidth}


% --------------------------------------------- %

\end{document}
