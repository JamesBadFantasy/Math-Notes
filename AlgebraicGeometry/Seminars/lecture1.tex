\documentclass[11pt]{article}
\usepackage[T1]{fontenc}
\usepackage{geometry, changepage, hyperref}
\usepackage{amsmath, amssymb, amsthm, bm}
\usepackage{physics, esint}

\hypersetup{colorlinks=true, linkcolor=blue, urlcolor=cyan}
\setlength{\parindent}{0pt}
\setlength{\parskip}{5pt}

\newtheorem{theorem}{Theorem}
\newtheorem{lemma}{Lemma}
\newtheorem{corollary}{Corollary}
\newtheorem{claim}{Claim}

\newcommand{\Spec}{\operatorname{Spec}}

\title{Algebraic Geometry: Seminar 1}
\author{James Pagan, Ferbuary 2024}

% --------------------------------------------- %

\begin{document}

\maketitle
\tableofcontents
\newpage

% --------------------------------------------- %

\section{Introduction}

The undergraduate Algebraic Geometry seminar will follow two texts:
\begin{enumerate}
  \item Hartshorne: Algebraic Geometry, and 
  \item THE RISING SEA: Foundations of Algebraic Geometry.
\end{enumerate}
Both texts can be found on Google. The prior seminar examined the Theory of Varieties --- this knowledge is critical to the Algebraic Geometry as a whole, but will not be assumed.


% --------------------------------------------- %

\section{Motivation for Schemes}

% --------------------------------------------- %

\subsection{Spectrum of a Commutative Ring}

Inutuitively, there are ``three components'' to a scheme.
\begin{enumerate}
  \item Points. It is a function, after al of a Commutative Ringl.
  \item A topology, specifically the Zariski topology.
  \item The \textbf{structure sheaf} of the scheme.
\end{enumerate}
The \textbf{spectrum} of a commutative ring denotes the set of all prime ideals of the ring:
\[
  \Spec R = \{ \mathfrak{p} \subseteq R \mid \text{$\mathfrak{p}$ is a prime ideal of $R$} \}.
\]
The prime ideals constitute the ``points'' of a scheme. Then what are elements $r \in R$? They are \textbf{functions}, that map points $\mathfrak{p}$ to $r$ modulo $\mathfrak{p}$ (notably $\mathfrak{r} \in R \,/\, \mathfrak{p}$). We say that the function $r$ \textbf{vanishes} at $\mathfrak{p}$ if $r \in \mathfrak{p}$.

% --------------------------------------------- %

\subsection{Examples}

Consider $\Spec \mathbb{Z}$. The prime ideals are $(p)$ for prime $p$. Under the function $8$, the prime ideals become $8 \pmod {p}$.

Consider $\Spec \big( k[\epsilon] \,/\, \epsilon^{2} \big)$, where $k$ is a field. Consider a point $\mathfrak{p}$ in the spectrum; since $0 \in \mathfrak{p}$, we have $\epsilon^{2} \in \mathfrak{p}$. Since $\mathfrak{p}$ is prime, we have $\epsilon \in \mathfrak{p}$ for all points $\mathfrak{p}$ (duh, because $\epsilon$ is nilpotent). $\epsilon$ vanishes everywhere on $\Spec$, yet it clear doesn't for the ring itself.

\begin{theorem}
  Let $R$ be a ring. Then $\Spec R \,/\, \mathfrak{N} \, \cong \, \Spec R$.
\end{theorem}
\begin{adjustwidth}{1cm}{}
  \begin{proof}
    prove it yourself dumbass, i think this is in atiyah-macdonald
  \end{proof}
\end{adjustwidth}

Let $k$ be an algebraically closed field, and consider $\Spec k[x_{1}, \ldots, x_{n}]$. It is a principal ideal domain; the prime/maximal ideals of this ring are exactly the principal ideals generated by linear terms.

If $\mathfrak{a}$ is an ideal of the ring $R$, then $\Spec (R \,/\, \mathfrak{a}) \subseteq \Spec R$. Any localization $\Spec (S^{-1}R)$ is also an order-preserving subset of $\Spec R$.

Question: Why not maximal ideals? Well, this falls apart when we consider topology: the intersection of maximal ideals is not a maximal ideal.

% --------------------------------------------- %

\end{document}
