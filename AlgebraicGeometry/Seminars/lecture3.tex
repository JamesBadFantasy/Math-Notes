\documentclass[11pt]{article}
\usepackage[T1]{fontenc}
\usepackage{geometry, changepage, hyperref}
\usepackage{amsmath, amssymb, amsthm, bm}
\usepackage{physics, esint}

\hypersetup{colorlinks=true, linkcolor=blue, urlcolor=cyan}
\setlength{\parindent}{0pt}
\setlength{\parskip}{5pt}

\newtheorem{theorem}{Theorem}
\newtheorem{lemma}{Lemma}
\newtheorem{proposition}{Proposition}
\newtheorem{corollary}{Corollary}
\newtheorem{claim}{Claim}

\newcommand{\Hom}{\operatorname{Hom}}
\newcommand{\Ker}{\operatorname{Ker}}
\newcommand{\Coker}{\operatorname{Coker}}
\newcommand{\Ann}{\operatorname{Ann}}
\newcommand{\Spec}{\operatorname{Spec}}

\title{Algebraic Geometry: Lecture 3}
\author{James Pagan}

% --------------------------------------------- %

\begin{document}

\title
\newpage

% --------------------------------------------- %

\section{Introduction}

Let $X$ be the spectrum of some ring. For all $f \in R$, this function can be regarded as a map
\[
  f : \Spec R \to \bigcup\limits_{p \in \Spec R} R \, / \, \mathfrak{p},
\]
where that is hte disjoint union. Unfortunately I will not be able to understand this lecture... the Zariski Topology has already been defined. Though this isn't to hard to gather if you just use Atiyah-MacDonald's definition.

% --------------------------------------------- %



% --------------------------------------------- %

\end{document}
