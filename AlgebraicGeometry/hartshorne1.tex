\documentclass[11pt]{article}
\usepackage[T1]{fontenc}
\usepackage{geometry, changepage, hyperref}
\usepackage{amsmath, amssymb, amsthm, bm}
\usepackage{physics, esint}

\hypersetup{colorlinks=true, linkcolor=blue, urlcolor=cyan}
\setlength{\parindent}{0pt}
\setlength{\parskip}{6pt}

\newtheorem{theorem}{Theorem}
\newtheorem{lemma}{Lemma}
\newtheorem{proposition}{Proposition}
\newtheorem{corollary}{Corollary}
\newtheorem{claim}{Claim}

\title{Hartshorne: Varieties}
\author{February 2024}

% --------------------------------------------- %

\begin{document}

\maketitle
\tableofcontents
\newpage

% --------------------------------------------- %

\section{Affine Varieties}

% --------------------------------------------- %

\subsection{Familiar Definitions}

Let $k$ be an algebraically closed field. We define \textbf{affine space} over $k$, denoted $\mathbb{A}_{k}^{n}$ or $\mathbb{A}^{n}$, as the set of all $n$-tuples with components in $k$. The elements $P \in \mathbb{A}^{n}$ are called \textbf{points} --- and if $P = (a_{1}, \ldots, a_{n})$, the elements $a_{1}, \ldots, a_{n}$ are called \textbf{components}.

The set $R = k[x_{1}, \ldots, x_{n}]$ denotes the commutative ring of polynomials with variables $x_{1}, \ldots, x_{n}$ with coefficents in $k$. We may interpret each $f \in R$ as a function from $\mathbb{A}^{n}$ to $k$, defined by $f(P) = f(a_{1}, \ldots, a_{n})$. We may thus define the \textbf{zeroes} of $f$, given by the set $Z(f) = \{ P \in \mathbb{A}_{k}^{n} \, \mid \, f(P) = 0 \}$. More generally, for any subset $T$ of polynomials $R$, its \textbf{zeroes} are given by
\[
  Z(T) = \{ P \in \mathbb{A}^{n} \, \mid \, f(P) = 0 \text{ for each } f \in T \}.
\]
Ideals in $R$ are quite elegant: since $R$ is Noetherian, each ideal $\mathfrak{a}$ has a finite set of generators $f_{1}, \ldots, f_{n}$. Thus $\mathfrak{a}$ may be expressed as the common zeroes of $f_{1}, \ldots, f_{n}$. It is easy to verify that if $\mathfrak{b}$ is the ideal generated by $T$, then $Z(T) = Z(\mathfrak{b})$.

% --------------------------------------------- %

\subsection{The Zariski Topology}

A subset $Y \subseteq \mathbb{A}^{n}$ is an \textbf{algebraic set} if there exists a subset $T \subseteq Y$ such that $Y = Z(T)$.

\begin{adjustwidth}{1cm}{}
  \begin{theorem}
    The following two results hold:
    \begin{enumerate}
      \item The union of two algebraic sets $X, Y \subseteq \mathbb{A}^{n}$ is algebraic.
      \item The intersection of any family of algebraic sets $Y_{\alpha}$ is an algebraic set.
    \end{enumerate}
  \end{theorem}
  \begin{proof}
    For (1), let $X = Z(T)$ and $Y = Z(S)$. We claim that $Z(T \cup S) = Z(TS)$, where $TS$ is the set of all $ts$ with $t \in T$ and $s \in S$.
    \begin{enumerate}
      \item Suppose $P \in Z(TS)$ --- that is, $ts(P) = 0$ for all $ts \in TS$. Since $R$ is an integral domain, we have $t(P)$ or $s(P)$ = 0, so $P \in Z(T)$ or $P \in Z(S)$. Hence $P \in Z(T) \cup Z(S)$.
      \item Suppose $P \in Z(T) \cup Z(S)$. Then $P \in Z(T)$ or $P \in S(T)$ --- in which case, $t(P) = 0$ for all $t$ or $s(P) = 0$ for all $s$. In either case, $ts(P) = 0$ for all $ts \in TS$, so $P \in Z(TS)$.
    \end{enumerate}
    Thus $X \cup Y = Z(T) \cup Z(S) = Z(TS)$. We deduce that $X \cup Y$ is an algebraic set. For (2), let $Y_{\alpha} = T_{a}$. It is easy to verify that $\bigcap_{\alpha \in A} Z(T_{\alpha}) = Z \big( \bigcup_{\alpha \in A} T_{\alpha} \big)$; hence $\bigcup_{\alpha \in A} Y_{\alpha}$ is an algebraic set.
  \end{proof}
\end{adjustwidth}

Noting that $\varnothing$ and $\mathbb{A}^{n}$ are algebraic sets (since $\varnothing = Z(1)$ and $\mathbb{A}^{n} = Z(0)$), we deduce that algebraic sets in $R$ satisfy the axioms for closed sets in a topological space. The ensuing topology is called the \textbf{Zariski topology}.

% --------------------------------------------- %

\end{document}
