\documentclass[11pt]{article}
\usepackage[T1]{fontenc}
\usepackage{geometry, changepage}
\usepackage{amsmath, amssymb, amsthm, bm}
\usepackage{physics}
\usepackage{hyperref, bookmark}

\hypersetup{colorlinks=true, linkcolor=blue, urlcolor=cyan}
\setlength{\parindent}{0pt}
\setlength{\parskip}{5pt}

\newtheorem{theorem}{Theorem}
\newtheorem{lemma}{Lemma}
\newtheorem{claim}{Claim}
\newtheorem*{theorem*}{Theorem}
\newtheorem*{lemma*}{Lemma}
\newtheorem*{claim*}{Claim}

\renewcommand{\vec}[1]{\mathbf{#1}}
\newcommand{\uvec}[1]{\mathop{} \!\hat{\mathbf{#1}}}
\newcommand{\mat}[1]{\mathbf{#1}}
\newcommand{\tensor}[1]{\mathsf{#1}}

\renewcommand{\div}{\nabla \cdot}
\renewcommand{\curl}{\nabla \cross}
\renewcommand{\grad}{\nabla}
\renewcommand{\laplacian}{\nabla^{2}}

\title{Rudin: Integration of Differential Forms}
\author{James Pagan}
\date{November 2023}

% --------------------------------------------- %

\begin{document}

\maketitle
\tableofcontents

% --------------------------------------------- %

\section{Integration}

\subsection{Definition}

Suppose $I_{n}$ is a $\mathbf{n}$\textbf{-cell} in $\mathbb{R}^{n}$ --- a multi-dimension analogue of a box, namely
\[
	I_{n} = \{ (x_{1}, \ldots, x_{n}) \mid a_{i} \le x_{i} \le b_{i} \text{ for } i \in \{ 1, \ldots, n \}\}.
\]
We define $I^{j}$ as the $j$-cell in $\mathbb{R}^{j}$ defined by the first $j$ inequalities of $I^{n}$. For a $C^{1}$ function $f$, put $f = f_{n}$, and define $f_{n - 1}$ as
\[
	f_{n - 1}(x_{1}, \ldots, x_{n - 1}) = \int_{a_{n}}^{b_{n}} f_{n}(x_{1}, \ldots, x_{n - 1}, x_{n}) \dd{x_{n}}.
\]
We repeat this process and optain functions $f_{j}$, continuous on $f_{j - 1}$, until we arrive at a \textit{number} $f_{0}$, which is called the \textbf{integral} of $f$ over $I^{n}$.

% --------------------------------------------- %

\section{Differential Forms}

% --------------------------------------------- %

\subsection{Prerequisites}

In preparation for the construction of differential forms, we develop the notion of a compact set. If $f : D \subset \mathbb{R}^{k} \to \mathbb{R}^{n}$ is $C^{1}$, then $D$ is \textbf{compact set} if there exists an open set $W$ containing $D$ and a $C^{1}$ mapping $g : W \to \mathbb{R}^{n}$ such that $f(\vec{x}) = g(\vec{x})$ for all $x \in D$.

A \textbf{k-surface} in an open set $U \subset \mathbb{R}^{n}$ is a mapping from a compact set $D \subset \mathbb{R}^{k}$ into $U$.

\subsection{Definition}

Let $U \subset \mathbb{R}^{n}$ be an open set, let $D \subset \mathbb{R}^{k}$ be a compact set, and let $\Phi : D \to U$ be a surface. A \textbf{differential form of order k}, briefly called a \textbf{k-form}, is a mapping from surfaces $\Phi$ to real numbers, notated as
\[
	\omega = \sum f_{i_{1}, \ldots, i_{k}}(\vec{x}) \dd{x_{i_{1}}} \wedge \cdots \wedge \dd{x_{i_{n}}}
\]
that assignes to each $\Phi$ a number $\omega(\Phi) = \int_{\Phi} \omega$ according to the rule
\[
	\int_{\Phi} \omega = \int_{D} \sum f_{i_{1}, \ldots, i_{k}}(\vec{x}) \pdv{(x_{i_{1}}, \ldots, x_{i_{k}})}{(u_{1}, \ldots, u_{k})} \dd{\vec{u}},
\]
where if the components of $\Phi$ are $\Phi_{1}, \ldots, \Phi_{n}$, the Jacobian is determined by the mapping $(u_{1}, \ldots, u_{k}) \to (\Phi_{i_{1}} (\vec{u}), \ldots, \Phi_{i_{n}} (\vec{u}))$ and the functions $f_{i_{1}, \ldots, i_{n}}$ are real and continuous.

A $k$-form $\omega$ is $C^{m}$ for some $m \in \mathbb{Z}_{> 0} $ if every function $f_{i_{1}, \ldots, i_{n}}$ is $C^{m}$. A $0$-form in $U$ is defined to be a continuous funciton in $U$.

% Remark: Differential forms are functions from surfaces to numbers, defined *by* integration and notated *through* integration. The act of mapping a surface to a number by $\omega$ --- denoted by $\omega(\Phi)$ --- is notated by integration, namely $\int_{\Phi} \omega$. 

% --------------------------------------------- %

\subsection{Basic Properties}

Let $\omega$, $\omega_{1}$, and $\omega_{2}$ be $k$-forms. We write $\omega_{1} = \omega_{2}$ if and only if $\omega_{1}(\Phi) = \omega_{2}(\Phi)$ for all $k$-surfaces $\Phi$ in $U$. We define $c \omega$ for $c \in \mathbb{R}^{n}$ by
\[
	\int_{\Phi} c \omega = c \int_{\Phi} \omega,
\]
and we write $\omega = \omega_{1} + \omega_{2}$ if and only if
\[
	\int_{\Phi} \omega = \int_{\Phi} \omega_{1} + \int_{\Phi} \omega_{2}.
\]
Now, consider the $k$-form $\omega = f(\vec{x}) \dd{x_{i_{1}}} \wedge \cdots \wedge \dd{x_{i_{n}}}$. If we define $\overline{\omega}$ as the $k$-form obtained by swapping some pair of subscripts of $\omega$, we swap the sign of the Jacobian --- thus finding that $\overline{\omega} = -\omega$. A specal case of this is the \textit{anticommutative relation}
\[
	\dd{x_{i}} \wedge \dd{x_{j}} = - \dd{x_{j}} \wedge \dd{x_{i}}.
\]
With this in mind, we define $\dd{x_{i}} \wedge \dd{x_{i}} = 0$. More generally, if $i_{r} = i_{k}$ for some indicies over the $k$-form $\omega = f(\vec{x}) \dd{x_{i_{1}}} \wedge \cdots \wedge \dd{x_{i_{n}}}$, we swap these indicies and obtain that $\omega = \overline{\omega} = - \omega$, so $\omega = 0$.

% Remark: This is why I do not specify whether the $i_{1}, \ldots, i_{n}$ are ordered or unordered --- if ordered, we can swap the indicies and simplify to yield an equivalent form with unordered indicies.

Consequentially, the only $k$-form in the open set $U \subset \mathbb{R}^{n}$ if $k > n$ is $0$.

% --------------------------------------------- %

\subsection{Basic k-forms}

If $I = (i_{1}, \ldots, i_{k})$ is an increasing sequence of $k$ integers from $\{ 1, \ldots, n \}$, we call $I$ an \textbf{increasing k-index} and use the notation
\[
	\dd{x_{I}} = \dd{x_{i_{1}}} \wedge \cdots \wedge \dd{x_{i_{k}}}.
\]
The form $\dd{x_{I}}$ is called a \textbf{basic k-form}.

There are clearly $\binom{n}{k}$ basic $k$-forms. Clearly we can "swap the indicies" of a form to express it as a sum of basic $k$-forms: namely, for all distinct $j_{1}, \ldots, j_{k} \in \{ 1, \ldots, n \}$, we may call its increasing permutation $J$ and observe that
\[
	\dd{x_{j_{1}}} \wedge \cdots \wedge \dd{x_{j_{k}}} = \epsilon(j_{1, \ldots, j_{k}}) \dd{x_{J}},
\]
where $\epsilon(j_{1}, \ldots, j_{k})$ is $1$ or $-1$. Thus, from this point on, we will use the \textbf{standard presentation} of $\omega$ by writing
\[
	\omega = \sum_{I} f_{I}(\vec{x}) \dd{x_{I}}.
\]
This is a \textit{unique} way to represent a $k$-form, as we will prove in the following result:

\begin{theorem}
	Suppose a differential $k$-form $\omega$ in an open set $U \subset \mathbb{R}^{n}$ has the standard presentation
	\[
		\omega = \sum_{I} f_{i}(\vec{x}) \dd{x_{I}}.
	\]	
	If $\omega = 0$ in $U$, then $f_{I}(\vec{x}) = 0$ for every increasing $k$-index $I$ and $\vec{x} \in U$.
\end{theorem}
\begin{adjustwidth}{1cm}{}
    \begin{proof}\renewcommand{\qedsymbol}{}
		
	\end{proof}
\end{adjustwidth}
% --------------------------------------------- %

\end{document}
