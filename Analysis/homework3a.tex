\documentclass[11pt]{article}
\usepackage[T1]{fontenc}
\usepackage{geometry, changepage, hyperref}
\usepackage{amsmath, amssymb, amsthm, bm}
\usepackage{physics, esint}

\hypersetup{colorlinks=true, linkcolor=blue, urlcolor=cyan}
\setlength{\parindent}{0pt}
\setlength{\parskip}{5pt}

\newtheorem{theorem}{Theorem}
\newtheorem{lemma}{Lemma}
\newtheorem{proposition}{Proposition}
\newtheorem{corollary}{Corollary}
\newtheorem{claim}{Claim}
\newtheorem{definition}{Definition}

\renewcommand{\vec}[1]{\mathbf{#1}}
\newcommand{\uvec}[1]{\mathop{} \!\hat{\textbf{#1}}}
\newcommand{\mat}[1]{\mathbf{#1}}
\newcommand{\tensor}[1]{\mathsf{#1}}
\newcommand{\nll}{\operatorname{null}}
\newcommand{\range}{\operatorname{range}}

\renewcommand{\grad}{\nabla}
\renewcommand{\div}{\nabla \cdot}
\renewcommand{\curl}{\nabla \cross}

\title{MATH-UA 329: Homework 3A}
\author{James Pagan, March 2024}
\date{Professor Güntürk}

% --------------------------------------------- %

\begin{document}

\maketitle
\tableofcontents
\newpage

% --------------------------------------------- %

\section{Problem 1}
 
Let $\vec{x}$ be the vector in $X$ such that $\norm{\vec{x}}_{X} = 1$ and $\norm{\mat{ST} \vec{x}}_{Z} = \norm{\mat{ST}}_{X \to Z}$. The existence of this vector is ensured by Extreme Value Theorem, since $\norm{\mat{ST}}_{X \to Z}$ is a supremum of the image of a compact set. Thus we have
\begin{align*}
  \norm{\mat{ST}}_{X \to Z} \, & = \, \norm{\mat{ST} \vec{x}}_{Z} \\
                               & \le \, \norm{\mat{S}}_{Y \to Z} \norm{\mat{Tx}}_{Y} \\
                               & \le \, \norm{\mat{S}}_{Y \to Z} \norm{\mat{T}}_{X \to Y} \norm{\vec{x}}_{X} \\
                               & = \, \norm{\mat{S}}_{Y \to Z} \norm{\mat{T}}_{X \to Y}.
\end{align*}
This completes the proof.

% --------------------------------------------- %

\section{Problem 2}

% --------------------------------------------- %

\subsection{Part (a)}

Let the matrix $\mat{M}$ have the form
\[
  \mat{M} \quad \stackrel{\text{def}}{=} \quad \begin{bmatrix} M_{11} & \cdots & M_{1n} \\ \vdots & \ddots & \vdots \\ M_{m1} & \cdots & M_{mn} \end{bmatrix}
\]
For each $i \in \{ 1, \ldots, m \}$, define the constants $s_{i1}, \ldots, s_{in} \in \{ 1, -1 \}$ such that $s_{ij} M_{ij} = \abs{M_{ij}}$. Let 
\begin{equation}
  L \, \stackrel{\text{def}}{=} \, \max\limits_{1 \le i \le m} \sum\limits_{j = 1}^{n} \abs{M_{ij}} \, = \, \max\limits_{1 \le i \le m} \sum\limits_{j = 1}^{n} M_{ij}s_{ij}.
\end{equation}
We will show that for all $\vec{x} \in \mathbb{R}^{n}$ such that $\norm{\vec{x}}_{\infty} = 1$, we have $\norm{\mat{M}\vec{x}} \le L$ and equality is attained. So, suppose that $\vec{x} = (x_{1}, \ldots, x_{n}) \in \mathbb{R}^{n}$ has supremum norm $1$. Hence $\abs{x_{i}} \le 1$ for all $i$; this yields
\[
  \mat{M} \vec{x} \, = \, \begin{bmatrix} M_{11} & \cdots & M_{1n} \\ \vdots & \ddots & \vdots \\ M_{m1} & \cdots & M_{mn} \end{bmatrix} \begin{bmatrix} x_{1} \\ \vdots \\ x_{n} \end{bmatrix} = \begin{bmatrix} \sum\limits_{j = 1}^{n} M_{1j}x_{j} \\ \vdots \\ \sum\limits_{j = 1}^{n} M_{mj}x_{j} \end{bmatrix}.
\]
We deduce that
\begin{align*}
  \norm{\mat{M}\vec{x}}_{\infty} \, = \, \max\limits_{1 \le i \le m} \sum\limits_{j = 1}^{n}  M_{ij} x_{j} \, \le \, \max\limits_{1 \le i \le m} \sum\limits_{j = 1}^{n} \abs{M_{ij}} \abs{x_{j}} \, \le \, \max\limits_{1 \le i \le m} \sum\limits_{j = 1}^{n} \abs{M_{ij}} \, = \, L.
\end{align*}
We conclude that $L$ is an upper bound of $\norm{\mat{M}\vec{x}}$ across all $\vec{x} \in \mathbb{R}^{n}$ with supremum norm $1$. To demonstrate that $L$ is achieved, let $k$ be an integer in $\{ 1, \ldots, m \}$ such that the maximum in Equation (1) is achieved. Consider the vector $\vec{s} = (s_{k1}, \ldots, s_{kn})$:
\[
  \mat{M} \vec{s} \, = \, \begin{bmatrix} M_{11} & \cdots & M_{1n} \\ \vdots & \ddots & \vdots \\ M_{m1} & \cdots & M_{mn} \end{bmatrix} \begin{bmatrix} s_{k1} \\ \vdots \\ s_{kn} \end{bmatrix} = \begin{bmatrix} \sum\limits_{j = 1}^{n} M_{1j}s_{kj} \\ \vdots \\ \sum\limits_{j = 1}^{n} M_{mj}s_{kj} \end{bmatrix}.
\]
Observe that one of the entries of $\mat{M} \vec{s}$ is
\[
  \sum\limits_{j = 1}^{n} M_{kj}s_{kj} \, = \, \sum\limits_{j = 1}^{n} \abs{M_{kj}} \, = \, L.
\]
Thus the supremum norm of $\mat{M} \vec{s}$ is $\abs{L} = L$ or greater; since $\vec{s}$ clearly has supremum norm $1$, we proved it must be precisely $L$. We conclude that
\[
  \max\limits_{1 \le i \le m} \sum\limits_{j = 1}^{n} \abs{M_{ij}} \, = \, L \, = \, \sup\limits_{\norm{\vec{x}}_{\infty} = 1} \norm{\mat{M} \vec{x}}_{\infty} \, = \, \norm{\mat{M}}_{\infty, \infty}.
\]
We now describe all $\vec{x} \in \mathbb{R}^{n}$ such that $\norm{\vec{x}}_{\infty} = 1$ and $\norm{\mat{M}\vec{x}}_{\infty} = \norm{\mat{M}}_{\infty, \infty}$. It is quite simple: they are all vectors with components satisfying the following properties. For each $j \in \{ 1, \ldots, n \}$, select a $k$ such that the maximum of Equation 2 is satisfied.
\begin{enumerate} 
  \item If $M_{kj} = 0$, then the $j$-th coordinate of $\vec{x}$ can be any number from $-1$ to $1$. We impose these bounds on the coordinate so that $\vec{x}$ has $\infty$-norm $1$.
  \item If $M_{kj} \ne 0$, then the $j$-th coordinate of $\vec{x}$ must be $s_{kj}$ --- namely, the element of $\{ 1, -1 \}$ such that $M_{kj}s_{kj} = \abs{M_{kj}}$.
\end{enumerate}
This completes the proof.

% --------------------------------------------- %

\subsection{Part (b)}

Let $\mat{M}$ have the same form as above. Again, define the constant
\begin{equation}
  L \quad \stackrel{\text{def}}{=} \quad \max\limits_{1 \le j \le n} \sum\limits_{i = 1}^{m} \abs{M_{ij}}.
\end{equation}
Define $\vec{x} = (x_{1}, \ldots, x_{n}) \in \mathbb{R}^{n}$ such that $1 = \norm{\vec{x}}_{1} = \abs{x_{1}} + \cdots + \abs{x_{n}}$. This yields
\[
  \mat{M} \vec{x} \, = \, \begin{bmatrix} M_{11} & \cdots & M_{1n} \\ \vdots & \ddots & \vdots \\ M_{m1} & \cdots & M_{mn} \end{bmatrix} \begin{bmatrix} x_{1} \\ \vdots \\ x_{n} \end{bmatrix} = \begin{bmatrix} \sum\limits_{j = 1}^{n} M_{1j}x_{j} \\ \vdots \\ \sum\limits_{j = 1}^{n} M_{mj}x_{j} \end{bmatrix}.
\]
We deduce through computation that
\begin{align*}
  \norm{\mat{M}\vec{x}}_{1} \, &= \, \sum\limits_{i = 1}^{m} \sum\limits_{j = 1}^{n} \abs{M_{ij} x_{j}} \\ 
                               &= \, \sum\limits_{j = 1}^{n} \left( \abs{x_{j}} \sum\limits_{i = 1}^{m} \abs{M_{ij}} \right) \\ 
                               &\le \, \sum\limits_{j = 1}^{n} \left( \abs{x_{j}} \max\limits_{1 \le j \le n} \sum\limits_{i = 1}^{m} \abs{M_{ij}} \right) \\
                               &\le \, \left( \max\limits_{1 \le j \le n} \sum\limits_{i = 1}^{m} \abs{M_{ij}} \right) \left( \sum\limits_{j = 1}^{n} \abs{x_{j}} \right) \\
                               &= \, \left( \max\limits_{1 \le j \le n} \sum\limits_{i = 1}^{m} \abs{M_{ij}} \right) \\
                               &= \, L.
\end{align*}
Showing that $L$ is achieved is again quite easy. If $k$ is an integer in the set $\{ 1, \ldots, n \}$ such that the maximum in Equation (2) is achieved, the vector $\vec{e}_{k}$ does the job: since $\mat{M} \vec{e_{k}}$ is the vector $(M_{1k}, \ldots, M_{mk})$, we have
\[
  \norm{\mat{M}\vec{e}_{k}}_{1} \, = \, \sum\limits_{i = 1}^{m} \abs{M_{mk}} \, = \, \max\limits_{1 \le j \le n} \sum\limits_{i = 1}^{m} \abs{M_{ij}} \, = \, L.
\]
We conclude our identification of the supremum:
\[
  \max\limits_{1 \le j \le n} \sum\limits_{i = 1}^{m} \abs{M_{ij}} \, = \, L \, = \,  \sup\limits_{\norm{\vec{x}}_{1} = 1} \norm{\mat{M} \vec{x}}_{1} \, = \, \norm{\mat{M}}_{1, 1}.
\]
As per the vectors $\vec{x} \in \mathbb{R}^{n}$ such that $\norm{\vec{x}}_{1} = 1$ and $\norm{\mat{M} \vec{x}}_{1} = \norm{\mat{M}}_{1, 1}$: if there are distinct values $k_{1}, \ldots, k_{\ell}$ such that the maximum in Equation (2) is satisfied, we can construct all such vectors $\vec{x}$ as follows: The entries in $k_{1}, \ldots, k_{\ell}$ may be any positive real numbers that sum to $1$. All other entries must be zero.

The verification of this fact is relatively easy to deduce from the above equations.

% --------------------------------------------- %

\subsection{Part (c)}

The key is to examine the sums of the absolute values of all entries in $\mat{M}$:
\begin{align*}
  m \norm{\mat{M}}_{\infty, \infty} \, = \, m \left( \max\limits_{1 \le i \le m} \sum\limits_{j = 1}^{n} \abs{M_{ij}} \right) \, \ge \, \sum\limits_{i = 1}^{m} \sum\limits_{j = 1}^{n} \abs{M_{ij}} \, \ge \, \max\limits_{1 \le j \le n} \sum\limits_{i = 1}^{m} \abs{M_{ij}} \, = \, \norm{\mat{M}}_{1, 1}.
\end{align*}
Equality is achieved here if $\mat{M}$ contains one nonzero column in which all entries are equal. Similarly,
\begin{align*}
  n \norm{\mat{M}}_{1, 1} \, = \, n \left( \max\limits_{1 \le j \le n} \sum\limits_{i = 1}^{m} \abs{M_{ij}} \right) \, \ge \, \sum\limits_{i = 1}^{m} \sum\limits_{j = 1}^{n} \abs{M_{ij}} \, \ge \, \max\limits_{1 \le i \le m} \sum\limits_{j = 1}^{n} \abs{M_{ij}} \, = \, \norm{\mat{M}}_{\infty, \infty}.
\end{align*}
Equality here is similarly achieved if $\mat{M}$ contains one nonzero row in which all entries are equal. We conclude the existence of the desired constants:
\[
  \boxed{\tfrac{1}{m} \norm{\mat{M}}_{1, 1} \, \le \, \norm{\mat{M}}_{\infty, \infty} \, \le \, n\norm{\mat{M}}_{1, 1}}.
\]

% --------------------------------------------- %

\end{document}
