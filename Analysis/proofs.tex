\documentclass[11pt]{article}
\usepackage[T1]{fontenc}
\usepackage{geometry, changepage}
\usepackage{amsmath, amssymb, amsthm, bm}
\usepackage{physics}
\usepackage{hyperref}

\hypersetup{colorlinks=true, linkcolor=blue, urlcolor=cyan}
\setlength{\parindent}{0pt}
\setlength{\parskip}{5pt}

\newtheorem{theorem}{Theorem}
\newtheorem{lemma}{Lemma}
\newtheorem{claim}{Claim}
\newtheorem*{theorem*}{Theorem}
\newtheorem*{lemma*}{Lemma}
\newtheorem*{claim*}{Claim}

\title{Real Analysis: Self-Discovered Proofs}
\author{James Pagan}
\date{}

% --------------------------------------------- %

\begin{document}

\maketitle
\tableofcontents

% --------------------------------------------- %

\section{Bolzano-Weierstrauss Theorem}

\begin{theorem*}
	All bounded sequences $a_n$ in $\mathbb{R}$ contain a convergent subsequence.
\end{theorem*}

\begin{proof}

Let the first term of $a_n$ be $a_{0}$, the lower bound of $a_{n}$ be $m$, and the upper bound of $a_{n}$ be $M$ such that $m \le a_n \le M$ for all $n \in \mathbb{Z}_{\ge 0}$. If $m = M$, then $a_n$ trivially converges.

Otherwise, $m \ne M$; let the set $S_n$ for $n \in \mathbb{Z}_{\ge 0}$ be the $2^{n}$ closed intervals of size $\tfrac{M-m}{2^{n}}$ in the interval $[m, M]$, and $I_n$ for $n \in \mathbb{Z}_{\ge 0}$ be the unique interval --- of those in $S_n$ that contain infinitely many values of $a_n$ --- that has the greatest upper endpoint. 

\begin{adjustwidth}{1cm}{}
	\begin{claim*}
		$I_N$ exists for all $\mathbb{Z}_{\ge 0}$
	\end{claim*}
	\begin{proof}\renewcommand{\qedsymbol}{}
		Suppose for contradiction that all the intervals in $S_{n}$ for some $n \in \mathbb{Z}_{\ge 0}$ contain finitely many values of $a_n$. Then the union of all these intervals --- $[m, M]$ itself --- must contain finitely many values of $a_n$, which contradicts the definition of $a_n$. Hence, at least one interval in $S_{n}$ must contain infinitely many values of $a_n$ for all $n \in \mathbb{Z}_{\ge 0}$.

		As all the intervals in $S_{n}$ for each $n$ have distinct upper endpoints, there exists a unique interval among those in $S_{n}$ that contains infinitely many values of $a_n$ that has the greatest upper endpoint. $I_n$ thus exists for all $n \in \mathbb{Z}_{\ge 0}$.
	\end{proof}
\end{adjustwidth}

\newpage

We will now establish that $I_n$ meets the two conditions of the Nested Intervals Theorem. 

\begin{adjustwidth}{1cm}{}
	\begin{claim}
		$I_{n+1} \subseteq I_n$ for all $n \in \mathbb{Z}_{\ge 0}$
	\end{claim}
	\begin{proof}\renewcommand{\qedsymbol}{}
		Of all the intervals in $S_{n+1}$, it is trivial that no interval with upper endpoint greater than that of $I_n$ contains infinitely many values of $a_n$. Now, at least one of the two intervals of size $\tfrac{M-m}{2^{n+1}}$ inside $I_n$ must contain infinitely many values of $a_n$; therefore, one of these is $I_{n+1}$, and so $I_{n+1} \subseteq I_{n}$ for all $n \in \mathbb{Z}_{\ge 0}$ 
	\end{proof}
	\begin{claim}
		The limit of the width of $I_n$ as $n$ approaches infinity is $0$.
	\end{claim}
	\begin{proof}\renewcommand{\qedsymbol}{}
		$I_n$ has width $\tfrac{M-m}{2^{n}}$, which trivially approaches $0$ as $n$ approaches infinity.
	\end{proof}
\end{adjustwidth}

Then by the Nested intervals Theorem, there exists a unique real number $x$ such that $x \in I_{n}$ for all $n \in \mathbb{Z}_{\ge 0}$. 

We now construct an explicit convergent subsequence of $a_n$. Consider $a_{b_{n}}$, where $b_{n}$ is a sequence of integers defined for $n \in \mathbb{Z}_{\ge 0}$ recursively: $b_0 = 0$ and $b_n = \min \{ k \mid k \in \mathbb{Z}_{\ge 0}, a_k \in I_n, k > b_{n-1} \}$ for $n \in \mathbb{Z}_{>0}$. It is trivial to prove that $b_{n}$ always exists for $n \in \mathbb{Z}_{\ge 0}$, and that as $b_n$ is strictly increasing, $a_{b_{n}}$ is an infinite subsequence. By definition, $a_{b_{n}} \in I_n$ for all $n \in \mathbb{Z}_{\ge 0}$.

\begin{adjustwidth}{1cm}{}
	\begin{claim*}
		$\displaystyle{\lim_{n \to \infty}} a_{b_{n}} = x$.
	\end{claim*}
	\begin{proof}\renewcommand{\qedsymbol}{}
		For all $n \in \mathbb{Z}_{\ge 0}$, observe that $a_{b_{n}}$ and $x$ both lie inside $I_n$ --- so $\abs{a_{b_{n}} - x} < \tfrac{M-m}{2^{n}}$. Then for all $\epsilon > 0$, we have that $\log_2 (\tfrac{M-m}{\epsilon}) < n$ implies
		
		\[ \abs{a_{b_{n}} - x} < \frac{M-m}{2^n} < \frac{M-m}{2^{\log_2 \left(\tfrac{M-m}{\epsilon}\right)}} = \frac{M-m}{\tfrac{M-m}{\epsilon}} = \epsilon \].
		
		Hence, $\displaystyle{\lim_{n \to \infty}} a_{b_{n}} = x$. 
	\end{proof}
\end{adjustwidth}

Therefore, if $a_n$ is a bounded sequence, it is always possible to construct $a_{b_{n}}$ --- a convergent subsequence of $a_n$.

\end{proof}

% --------------------------------------------- %

\section{Boundedness Theorem}

\begin{theorem*}
	If $f \colon \mathbb{R} \to \mathbb{R}$ is continuous on $[a, b]$, then $f(x)$ is bounded on $[a, b]$.
\end{theorem*}

\begin{proof}

If $a = b$, then $f(x)$ is bounded above by $f(a) + 1$ and below by $f(a) - 1$. If $a \ne b$, suppose for contradiction that $f(x)$ is not bounded on $[a, b]$ --- then for all $M \in \mathbb{Z}_{\ge 0}$, there exists a $y \in [a, b]$ such that $f(y) > M$. We will prove that this assumption implies the existence of a subinterval of $[a, b]$ that is both bounded and unbounded.

Let $S_n$ for $n \in \mathbb{Z}_{\ge 0}$ be the set of the $2^{n}$ closed intervals of size $\tfrac{b-a}{2^{n}}$ between $a$ and $b$. Define $I_n$ as the unique closed interval that --- among those in $S_n$ such that $f(x)$ is not bounded on the interval --- has the largest possible upper endpoint.

\begin{adjustwidth}{1cm}{}
    \begin{claim*}
    	$I_n$ exists for all $n \in \mathbb{Z}_{\ge 0}$.
    \end{claim*}
    \begin{proof}\renewcommand{\qedsymbol}{}	
		Suppose for contradiction that $f(x)$ is bounded over all the closed intervals of $S_{n}$. Then $f(x)$ is bounded over the union of all these intervals --- $[a, b]$ itself --- which is a contradiction. Hence, $f(x)$ must be unbounded over least one closed interval in $S_{n}$.

		As all the intervals in $S_n$ have distinct upper endpoints, there exists a unique interval that --- among those in $S_{n}$ such that $f(x)$ is not bounded --- has the largest possible upper endpoint. $I_n$ thus exists for all $n \in \mathbb{Z}_{\ge 0}$.
    \end{proof}
\end{adjustwidth}

We will now establish that $I_n$ meets the two conditions of the Nested Intervals Theorem. 

\begin{adjustwidth}{1cm}{}
	\begin{claim*}
		$I_{n+1} \subseteq I_n$ for all $n \in \mathbb{Z}_{\ge 0}$
	\end{claim*}
	\begin{proof}\renewcommand{\qedsymbol}{}
		Of all the intervals in $S_{n+1}$, it is trivial that $f(x)$ is bounded over all interval with greater upper endpoint than $I_n$. Now, $f(x)$ must not be bounded over at least one of the two intervals of size $\tfrac{b-a}{2^{n+1}}$ inside $I_n$; therefore, one of these is $I_{n+1}$, and so $I_{n+1} \subseteq I_{n}$ for all $n \in \mathbb{Z}_{\ge 0}$ 
	\end{proof}
	\begin{claim*}
		The limit of the width of $I_n$ as $n$ approaches infinity is $0$.
	\end{claim*}
	\begin{proof}\renewcommand{\qedsymbol}{}
		$I_n$ has width $\tfrac{1}{2^{n}}$, which trivially approaches $0$ as $n$ approaches infinity.
	\end{proof}
\end{adjustwidth}

Therefore, the Nested Intervals Theorem guarantees the existence of a unique real number $r$ such that $r \in I_{n}$ for all $n \in \mathbb{Z}_{\ge 0}$. 

Because $f(x)$ is continuous at $r$, we have that there exists $\delta$ such that 
\[  
0 < |x - r | < \delta \implies \abs{f(x) - f(r)} < 1
\]
Therefore, the interval $(r - \delta, r + \delta)$ is bounded by $f(r) + 1$.

\begin{adjustwidth}{1cm}{}
    \begin{claim*}
    	There exists $I_m$ for some $n \in \mathbb{Z}_{\ge 0}$ such that $I_m \subset (r - \delta, r+ \delta)$.
    \end{claim*}
    \begin{proof}\renewcommand{\qedsymbol}{} 
		Consider the closed interval $I_m$ for $m = \Bigl\lceil \log_2 \left( \tfrac{b-a}{\delta} \right) \Bigr\rceil + 1$. Note that the width of $I_m$ satisfies 
		\[
			\frac{b-a}{2^{\Bigl\lceil \log_2 \left( \tfrac{b-a}{\delta} \right) \Bigr\rceil + 1}} < \frac{b-a}{2^{\log_2 \left( \tfrac{b-a}{\delta} \right) }} = \frac{b-a}{\tfrac{b-a}{\delta}} = \delta.
		\]
		Suppose for contradiction that $I_m \not\subset (r - \delta, r + \delta)$. Then there is a real number $x$  such that $x \in I_m$ and $x \not\in (r - \delta, r + \delta)$.

		As $r \in I_M$, the width of $I_m$ is greater than or equal to $\abs{x - r}$ --- however, the width of $I_M$ is less than $\delta$, so $\abs{x - r} < \delta$. Then $x$ is inside $(r - \delta, r + \delta)$, which contradicts the definition of $x$.
		
		Therefore, no such $x$ exists; $I_m \in (r - \delta, r + \delta)$.
	\end{proof}
\end{adjustwidth}

Hence, $I_m$ is bounded by $f(r) + 1$ as well. This contradicts the definition if $I_m$ --- namely, that $f(x)$ is unbounded on $I_m$.

We deduce that $f(x)$ has an upper bound over $[a, b]$. As $-f(x)$ also has an upper bound over $[a, b]$, $f(x)$ also has a lower bound over $[a, b]$. Therefore, $f(x)$ is bounded over $[a, b]$.

\end{proof}

% --------------------------------------------- %

\end{document}
