\documentclass[11pt]{article}
\usepackage[T1]{fontenc}
\usepackage{geometry, changepage}
\usepackage{amsmath, amssymb, amsthm, bm}
\usepackage{physics}
\usepackage{hyperref}

\hypersetup{colorlinks=true, linkcolor=blue, urlcolor=cyan}
\setlength{\parindent}{0pt}
\setlength{\parskip}{5pt}

\newtheorem{theorem}{Theorem}
\newtheorem{lemma}{Lemma}
\newtheorem{claim}{Claim}
\newtheorem*{theorem*}{Theorem}
\newtheorem*{lemma*}{Lemma}
\newtheorem*{claim*}{Claim}

\renewcommand{\vec}[1]{\mathbf{#1}}
\newcommand{\uvec}[1]{\mathop{} \!\hat{\textbf{#1}}}
\newcommand{\mat}[1]{\mathbf{#1}}
\newcommand{\tensor}[1]{\mathsf{#1}}

\renewcommand{\div}{\nabla \cdot}
\renewcommand{\curl}{\nabla \cross}
\renewcommand{\grad}{\nabla}
\renewcommand{\laplacian}{\nabla^{2}}

\title{MATH-UA 129: Homework 10}
\author{James Pagan, November 2023}
\date{Professor Serfaty}

% --------------------------------------------- %

\begin{document}

\maketitle
\tableofcontents

\newpage

% --------------------------------------------- %

\section{Section 7.3}

% --------------------------------------------- %

\subsection*{Problem 7}

\textbf{Part (a)}: As a cross section along the $xy$-plane should yield circles, as the function is unbounded along the $z$-axis, and as the cone is used in Part (d), the answer must be $\boxed{\textbf{(iii)}}$.

\textbf{Part (b)}: The similarity of this function to a unit ball in spherical coordinates --- with stretches of the $x$ and $y$ coordinates --- indicates that the answer is an ellipsoid. The correct graph is thus $\boxed{\textbf{(i)}}$

\textbf{Part (c)}: As a cross section along the $xz$-plane should yield a parabola, the answer is $\boxed{\textbf{(ii)}}$

\textbf{Part (d)}: This is a common parametrization: that of a cone. The answer is $\boxed{\textbf{(iv)}}$.

% --------------------------------------------- %

\subsection*{Problem 8}

\textbf{Part (a)}: As this is the only problem with a constricted domain --- and as we should expect the answer to resemble a circle --- the answer is $\boxed{\textbf{(i)}}$.

\textbf{Part (b)}: As the $z$-coordinate is bounded above by $4$, the answer should be $\boxed{\textbf{(ii)}}$.

\textbf{Part (c)}: As all components of the output vector are linear, we should expect the result to be a plane --- so the answer is $\boxed{\textbf{(ii)}}$.

\textbf{Part (d)}: By process of elimination, the answer should be $\boxed{\textbf{(iv)}}$.

% --------------------------------------------- %

\subsection*{Problem 9}

The surface is the $\boxed{\text{unit ball}}$ in spherical coordinates. Therefore, a unit normal to the ball is $\boxed{(\cos(v)\sin(u), \sin(v)\sin(u), \cos(u))}$ itself.

% --------------------------------------------- %

\subsection*{Problem 15}

As we seek to parametrize a function, the answer is clearly $\Phi(u, v) = (u, v, 3u^{2} + 8uv)$. An easy calculation verifies that $(1, 0)$ maps to $(1, 0, 3)$. Now, as
\[
	\mathbf{T}_{u} = (1, 0, 6u + 8v) \qquad \text{and} \qquad \mathbf{T}_{v} = (0, 1, 8u),
\]
the tangent plane should be given by
\begin{align*}
	\vec{v} &= (1, 0, 3) + t(1, 0, 6) + s(0, 1, 8) \\
			&= (1 + t, s, 3 + 6t + 8s),
\end{align*}
which is equivalent to the plane $\boxed{6x + 8y - z = 3}$. 

% --------------------------------------------- %

\subsection*{Problem 22}

\textbf{Part (a)}: Let a point on the image of $\Phi$ be $(a \sin(u) \cos(v), b \sin(u) \sin(v), c \cos(u))$ for $0 \le u \le \pi$ and $0 \le v \le 2\pi$, where $b < a$. Then
\begin{align*}
	\frac{x^{2}}{a^{2}} + \frac{y^{2}}{b^{2}} + \frac{z^{2}}{c^{2}} &= \frac{(a \sin(u) \cos(v))^{2}}{a^{2}} + \frac{(b \sin(u) \sin(v))^{2}}{b^{2}} + \frac{(c \cos(u))^{2}}{c^{2}} \\
	&= \sin^{2}(u) \cos^{2}(v) + \sin^{2}(u) \sin^{2}(v) + \cos^{2}(u) \\
	&= \sin^{2}(u) (\cos^{2} (v) + \sin^{2}(v)) + \cos^{2}(u) \\
	&= \sin^{2}(u) + \cos^{2} (u) \\
	&= 1,
\end{align*}
as desired.

\textbf{Part (b)}: We have that
\begin{align*}
	\mathbf{T}_{u} &= (a \cos(u) \cos(v), b \cos(u) \sin(v), -c \sin(u)) \\
	\mathbf{T}_{v} &= (- a \sin(u) \sin(v), b \sin(u) \cos(v), 0).
\end{align*}
Thus, 
\begin{align*}
	\mathbf{T}_{u} \times \mathbf{T}_{v} =& \begin{vmatrix}  \uvec{\i} & a \cos(u) \cos(v) & - a \sin(u) \sin(v) \\ \uvec{\j} & b \cos(u) \sin(v) & b \sin(u) \cos(v) \\ \uvec{k} & -c \sin(u) & 0 \end{vmatrix} \\
	=& (bc \sin^{2}(u) \cos(v)) \uvec{\i} + (ac \sin^{2}(u)\sin(v)) \uvec{\j} \\ 
	 &+ (ab \sin(u) \cos(u) \cos^{2}(v) + ab \sin(u) \cos(u) \sin^{2}(v)) \uvec{k} \\
	=& (bc \sin^{2}(u) \cos(v)) \uvec{\i} + (ac \sin^{2}(u)\sin(v)) \uvec{\j} + (ab \sin(u) \cos(u)) \uvec{k}.
\end{align*}
As this vector is never zero within the given region, the surface is regular at all points.

% --------------------------------------------- %

\section{Section 7.4}

% --------------------------------------------- %

\subsection*{Problem 10}

We compute the area \textit{for a sphere of radius $1$} by an integral --- one must break this integral into two parts: a sector of a sphere and a cone. We have that
\begin{align*}
	\int_{0}^{\tfrac{\sqrt{2}}{2}} \pi z^{2} \dd{z} + \int_{\tfrac{\sqrt{2}}{2}}^{1} \pi (\sqrt{1 - z^{2}})^{2} \dd{z} &= \pi \Big[ \frac{z^{3}}{3} \Big]_{0}^{\tfrac{\sqrt{2}}{2}} + \pi \Big[ z - \frac{z^{3}}{3} \Big]_{\tfrac{\sqrt{2}}{2}}^{1}\\
	&= \pi \left( \frac{\sqrt{2}}{12} \right) + \pi \left( \frac{2}{3} - \frac{\sqrt{2}}{2} + \frac{\sqrt{2}}{12} \right) \\
	&= \pi \left( \frac{2 -  \sqrt{2}}{3} \right).
\end{align*}

To compute the area for a sphere of radius $k$, we simply multiply this by $R^{3}$ to get the answer:
\[
	\boxed{\pi R^{3} \left( \frac{2 - \sqrt{2}}{3} \right)}.
\]

% --------------------------------------------- %

\subsection*{Problem 13}

Inspired by spherical coordinates, one such parametrization is
\[
	\boxed{\Phi(\theta, \phi) = (a \cos(\theta)\sin(\phi), b \sin(\theta)\sin(\phi), c \cos(\phi))}
\]
for $\theta \in [0, 2\pi)$ and $\phi \in [0, \pi)$. Now, observe that 
\begin{align*}
	\mathbf{T}_{\theta} &= (-a \sin(\theta)\sin(\phi), b \cos(\theta) \sin(\phi), 0) \\
	\mathbf{T}_{\phi} &= (a \cos(\theta)\cos(\phi), b \sin(\theta)\cos(\phi), -c \sin(\phi)),
\end{align*}
so
\begin{align*}
	\mathbf{T}_{\theta} \times \mathbf{T}_{\phi} &= \begin{vmatrix} \uvec{\i} & -a \sin(\theta)\sin(\phi) & a \cos(\theta) \cos(\phi) \\ \uvec{\j} & b \cos(\theta) \sin(\phi) & b \sin(\theta) \cos(\phi) \\ \uvec{k} & 0 & -c \sin(\phi) \end{vmatrix} \\
	&= (-bc \cos(\theta) \sin^{2}(\phi)) \uvec{\i} + (-ca \sin(\theta)\sin^{2}(\phi)) \uvec{\j} \\
	& \quad + (-ab \sin^{2}(\theta)\sin(\phi)\cos(\phi) - ab \cos^{2}(\theta) \sin(\phi) \cos(\phi)) \uvec{k} \\
	&= (-bc \cos(\theta) \sin^{2}(\phi)) \uvec{\i} + (-ca \sin(\theta)\sin^{2}(\phi)) \uvec{\j} + (-ab \sin(\phi) \cos(\phi)) \uvec{k}.
\end{align*}
Thus, we have that the area of the surface integral is the integral over this curl --- namely,
\[
	\boxed{\int_{0}^{\pi} \int_{0}^{2\pi} \sqrt{b^{2}c^{2}\cos^{2}(\theta)\sin^{4}(\phi) + c^{2}a^{2}\sin^{2}(\phi)\sin^{4}(\phi) + a^{2}b^{2}\sin^{2}(\phi)\cos^{2}(\phi)} \dd{\theta} \dd{\phi}}.
\]

% --------------------------------------------- %

\subsection*{Problem 25}

We can define a parametrized surface $\Phi$ for $f$ as $\Phi(x, y) = \left(x, y, \tfrac{2}{3} (x^{3/2} + y^{3/2})\right)$. Then $\mathbf{T}_{x} = (1, 0, x^{1/2})$ and $\mathbf{T}_{y} = (0, 1, y^{1/2})$, so $\mathbf{T}_{x} \times \mathbf{T}_{y} = \left( -x^{1/2}, -y^{1/2}, 1 \right)$ and $\norm{\mathbf{T}_{x} \times  \mathbf{T}_{y}} = \sqrt{x + y + 1}$ The area of this surface is thus
\begin{align*}
	\int_{0}^{1} \int_{0}^{1} \norm{\mathbf{T}_{x} \times \mathbf{T}_{y}} \dd{x} \dd{y} &= \int_{0}^{1} \int_{0}^{1} \sqrt{x + y + 1} \dd{x} \dd{y} \\
	&= \int_{0}^{1} \Big[ \tfrac{2}{3} \left( x + y + 1 \right)^{3/2} \Big]_{0}^{1} \dd{y} \\
	&= \frac{2}{3} \int_{0}^{1} (y + 2)^{3/2} - (y + 1)^{3/2} \dd{y} \\
	&= \frac{2}{3} \Big[ \tfrac{2}{5} (y + 2)^{5/2} - \tfrac{2}{5} (y + 1)^{5/2} \Big]_{0}^{1} \\
	&= \frac{4}{15} \left( \sqrt{3^{5}} - \sqrt{2^{5}} - \sqrt{2^{5}} + 1  \right) \\
	&= \boxed{\frac{36 \sqrt{3} - 32 \sqrt{2} + 4}{15}}.
\end{align*}

% --------------------------------------------- %

\section{Section 7.5}

% --------------------------------------------- %

\subsection*{Problem 6}

The surface can be parametrized by the mapping $\Phi(r, \theta) = (r \cos(\theta), r \sin(\theta), 4 + r \cos(\theta) + r \sin(\theta))$ under the domain $\{ (r, \theta) \mid r \in [0, 2], \theta \in [0, 2\pi) \}$. Now, see that
\begin{align*}
	\mathbf{T}_{r} &= (\cos(\theta), \sin(\theta), \cos(\theta) + \sin(\theta)) \\
	\mathbf{T}_{\theta} &= (-r \sin(\theta), r \cos(\theta), -r \sin(\theta) + r \cos(\theta)).
\end{align*}
Therefore,
\begin{align*}
	\mathbf{T}_{r} \times \mathbf{T}_{\theta} &= \begin{bmatrix} \sin(\theta) (-r \sin(\theta) + r \cos(\theta)) - r \cos(\theta) (\cos(\theta) - \sin(\theta)) \\ (\cos(\theta) + \sin(\theta))(-r \sin(\theta)) - \cos(\theta)(-r \sin(\theta) + r \cos(\theta)) \\ \cos(\theta)(r \cos(\theta)) - \sin(\theta) (- r \sin(\theta) )\end{bmatrix} \\
	&= \begin{bmatrix} -r \\ - r \\ r \end{bmatrix},
\end{align*}
so $\norm{\mathbf{T}_{r} \times \mathbf{T}_{\theta}} = r \sqrt{3}$. We are now ready to compute the surface area of the function:
\begin{align*}
	\iint_{S} x^{2}z + y^{2}z &= \int_{0}^{2\pi} \int_{0}^{2} \left( r^{2} \cos^{2}(\theta) + r^{2} \sin(\theta) \right) \left( 4 + r \cos(\theta) + r \sin(\theta) \right)  \left( \sqrt{3} r \right) \dd{r} \dd{\theta} \\
	&= \sqrt{3} \int_{0}^{2\pi} \int_{0}^{2} 4r^{3} + r^{4}(\cos(\theta) + \sin(\theta)) \dd{r} \dd{\theta} \\
	&= \sqrt{3} \int_{0}^{2\pi} \left[ r^{4} + \frac{r^{5}}{5}(\cos(\theta) + \sin(\theta)) \right]_{0}^{2} \dd{\theta} \\
	&= \sqrt{3} \int_{0}^{2\pi} 16 + \frac{32}{5} \left( \sin(\theta) + \cos(\theta) \right) \dd{\theta} \\
	&= \sqrt{3} \left[ 16\theta + \frac{32}{5} (\sin(\theta) - \cos(\theta)) \right]_{0}^{2\pi} \\
	&= \boxed{32 \sqrt{3}\pi}.
\end{align*}

% --------------------------------------------- %

\subsection*{Problem 10}

Let $X$ be the portion of $B$ below the $xy$-plane and let $Y$ be the portion of $B$ above teh $xy$-plane. Then
\begin{align*}
	\iint_{S}(x + y + z) \dd{S} &= \iint_{X} (x + y + z) \dd{S} + \iint_{Y} (x + y + z) \dd{S} \\
	&= \iint_{X} (x + y + z) \dd{S} - \iint_{X} (x + y + z) \dd{S} \\
	&= 0.
\end{align*}

% --------------------------------------------- %

\subsection*{Problem 16}

A parametrization of the sphere is $\Phi(\theta, \phi) = (R\cos(\theta)\sin(\phi), R \sin(\theta)\sin(\phi), R \cos(\phi))$. Using our work in Problem 22 Part (b), we find that
\begin{align*}
	\norm{\mathbf{T}_{\phi} \times \mathbf{T}_{\theta}} &= \norm{(R^{2} \sin^{2}(\phi) \cos(\theta)) \uvec{\i} + (R^{2} \sin^{2}(\phi)\sin(\theta)) \uvec{\j} + (R^{2} \sin(\phi) \cos(\phi)) \uvec{k}} \\
	&= R^{2} \sqrt{\sin^{4}(\phi)\cos^{2}(\theta) + \sin^{4}(\phi)\sin^{2}(\theta) + \sin^{2}(\phi) \cos^{2}(\phi)} \\
	&= R^{2} \sin(\phi),
\end{align*}
where $\abs{\sin(\phi)} = \sin(\phi)$ as $\phi \in [0, \pi/2]$. Then as $x^{2} + y^{2} = R^{2}\sin^{2}(\phi)$, the mass density is given by the integral
\begin{align*}
	\int_{0}^{\pi/2} \int_{0}^{2\pi} (R^{2}\sin^{2}(\phi))(R^{2} \sin(\phi)) &= 2\pi R^{4} \int_{0}^{\pi/2} \sin^{3}(\phi) \dd{\phi} \\
	&= 2\pi R^{4} \left( \frac{2}{3} \right) \\
	&= \boxed{\frac{4\pi R^{3}}{3}}/
\end{align*}

% --------------------------------------------- %

\subsection*{Problem 17}

\textbf{Part (a)}: Realize that if we merely apply a rotation to the sphere, the Change of Variables formula may yield that
\[
	\iint_{S} x^{2} \dd{S} = \iint_{S} y^{2} \dd{S}.
\]
Similarly, another roation yields that
\[
	\iint_{S} y^{2} \dd{S}  = \iint_{S} z^{2} \dd{S}.
\]

\textbf{Part (b)}: For all points $(x, y, z)$ on the unit sphere, $x^{2} + y^{2} + z^{2} = R^{2}$. Thus,
\begin{align*}
	\iint_{S} x^{2} \dd{S} &= \frac{1}{3} \iint_{S} x^{2} \dd{S} + \frac{1}{3} \iint_{S} x^{2} \dd{S} + \frac{1}{3} \iint_{S} x^{2} \dd{S} \\
	&= \frac{1}{3} \iint_{S} x^{2} \dd{S} + \frac{1}{3} \iint_{S} y^{2} \dd{S} + \frac{1}{3} \iint_{S} z^{2} \dd{S} \\
	&= \frac{1}{3} \iint_{S} x^{2} + y^{2} + z^{2} \dd{S} \\
	&= \frac{1}{3} \iint_{S} R^{2} \dd{S} \\
	&= \frac{R^{2}}{3} \iint_{S} \dd{S} \\
	&= \boxed{\frac{4\pi R^{4}}{3}}.
\end{align*}

\textbf{Part (c)}: Yes, it does. Realize that if the sphere is $S$ and the semisphere is $X$,
\begin{align*}
	\iint_{X} x^{2} + y^{2} \dd{S} &= \frac{1}{2} \iint_{S} x^{2} + y^{2} \dd{S} \\
								   &= \iint_{S} x^{2} \dd{S} \\
								   &= \boxed{\frac{4\pi R^{4}}{3}},
\end{align*}
which matches our answer in Problem 16.

% --------------------------------------------- %

\subsection*{Problem 26}

Without loss of generality, we can assume the point lies on the $z$-axis; this is because the desired integral only computes the distance from $\vec{p}$ to the unit sphere, which is symmetric across rotaions. We may also declare $S$ to be situated at the origin.

$S$ is parametrized by the function $\Phi(\theta, \phi) = (r \cos(\theta) \sin(\phi), r \sin(\theta) \sin(\phi), r \cos(\phi))$ for $\theta \in [0, 2\pi)$ and $\phi \in [0, \pi]$ Then if we represent $\vec{x}$ in spherical coordinates as above, we have that
\begin{align*}
	\norm{\vec{x} - \vec{p}} &= \sqrt{r^{2} \cos^{2}(\theta) \sin^{2}(\phi) + r^{2} \sin^{2}(\theta)\sin^{2}(\phi) + (r \cos(\phi) - d)^{2}} \\
	&= \sqrt{r^{2}\sin^{2}(\phi) + r^{2}\cos^{2}(\phi) - 2rd \cos(\phi) + d^{2}} \\
	&= \sqrt{r^{2} - 2 rd \cos(\phi) + d^{2}}.
\end{align*}
Therefore, the Change of Variables Theorem yields that (since $p \ne 0$)
\begin{align*}
	\iint_{S} \frac{1}{\norm{\vec{x} - \vec{p}}} \dd{S} &= \int_{0}^{2\pi} \int_{0}^{\pi} \frac{r^{2} \sin(\phi)}{\sqrt{r^{2} - 2rd \cos(\phi) + d^{2}}} \dd{\phi} \dd{\theta} \\
	&= \frac{r}{2d}\int_{0}^{2\pi} \int_{0}^{\pi} \frac{2rd \sin(\phi)}{\sqrt{r^{2} - 2rd \cos(\phi) + d^{2}}} \dd{\phi} \dd{\theta} \\
	&= \frac{r\pi}{d} \left[ 2 \sqrt{r^{2} - 2rd \cos(\phi) + d^{2}} \right]_{0}^{\pi} \\
	&= \frac{2r\pi}{d} \left( \sqrt{r^{2} + 2rd - d^{2}} + \sqrt{r^{2} - 2rd + d^{2}} \right) \\
	&= \frac{2r \pi}{d} \big( \abs{r + d} - \abs{r - d} \big).
\end{align*}
If $r > d$, then $\abs{r - d} = r - d$, so
\[
	\frac{2r \pi}{d} \big( \abs{r + d} - \abs{r - d} \big) = \frac{2r \pi}{d} \big( r + d - (r - d) \big) = \frac{2r \pi}{d} (2d) = \boxed{4\pi r}.
\]
If $r < d$, then $\abs{r - d} = d - r$, so
\[
	\frac{2r \pi}{d} \big( \abs{r + d} - \abs{r - d} \big) = \frac{2r \pi}{d} \big( r + d - (d - r) \big) = \frac{2r \pi}{d} (2r) = \boxed{\frac{4\pi r^{2}}{d}}.
\]
% --------------------------------------------- %

\section{Section 7.6}

% --------------------------------------------- %

\subsection*{Problem 4}

The cylinder may be parametrized by the mapping $\Phi(\theta, z) = (2 \cos(\theta), 2 \sin(\theta), z)$ for $\theta \in [0, 2\pi)$ and $z \in [0, 1]$. Then 
	\begin{align*}
	\mathbf{T}_{\theta} &= (-2 \sin(\theta), 2 \cos(\theta), 0) \\
	\mathbf{T}_{z} &= (0, 0, 1).
\end{align*}
Thus, $\mathbf{T}_{\theta} \times \mathbf{T}_{z} = (2 \cos(x), 2 \sin(x), 0)$. By substitution, $\mathbf{F}(x, y, z) = (4\cos(\theta), -4\sin(\theta), z^{2})$ We are now ready to compute the required integral:
\begin{align*}
	\iint_{S} \mathbf{F} \cdot \dd{\mathbf{S}} &= \int_{0}^{1} \int_{0}^{2\pi} (4\cos(\theta), -4\sin(\theta), z^{2}) \cdot (2 \cos(x), 2 \sin(x), 0) \dd{\theta} \dd{z} \\
	&= \int_{0}^{2\pi} 8 \cos^{2}(\theta) - 8 \sin^{2}(\theta) \dd{\theta} \\
	&= 4 \int_{0}^{2\pi} 2 \cos(2\theta) \dd{\theta} \\
	&= 4 \left[ \sin(2\theta) \right]_{0}^{2\pi} \\
	&= \boxed{0}.
\end{align*}

% --------------------------------------------- %

\subsection*{Problem 7}

Divide the surface $S$ into two parts:
\begin{itemize}
	\item $S_{1}$: The upper hemisphere. $\Phi_{1}(\phi, \theta) = (\cos(\theta)\sin(\phi), \sin(\theta)\sin(\phi), \cos(\phi))$ for $\theta \in [0, 2\pi)$ and $\phi \in [0, \pi/2)$ is one parametrization. \textbf{The variables are flipped!}
	\item $S_{2}$: The unit disc. $\Phi_{2}(r, \theta) = (r \sin(\theta), r\cos(\theta), 0)$ for $r \in [0, 1]$ and $\theta \in [0, 2\pi)$ is one parametrization.
\end{itemize}
It is easy to verify that these possess the same orientation. We thus have that
\begin{equation}
	\iint_{S} \mathbf{E} \cdot \dd{\vec{S}} = \iint_{S_{1}} \mathbf{E} \cdot \dd{\vec{S}} + \iint_{S_{2}} \mathbf{E} \cdot \dd{\vec{S}}.
\end{equation}
First, we tackle $S_{1}$. Realize that for $\Phi_{1}$,
\begin{align*}
	\mathbf{T}_{\phi} &= (\cos(\theta)\cos(\phi), \sin(\theta)\cos(\phi), -\sin(\phi)) \\
	\mathbf{T}_{\theta} &= (-\sin(\theta)\sin(\phi), \cos(\theta)\sin(\phi), 0)
\end{align*}
Therefore,
\[
	\mathbf{T}_{\phi} \times \mathbf{T}_{\theta} = (\cos(\theta)\sin^{2}(\phi), \sin(\theta)\sin^{2}(\phi), \sin(\phi)\cos(\phi)).
\]
We are almost ready to integrate the flux of $S_{1}$:
\begin{align*}
	\mathbf{E}(\Phi_{1}) \cdot (\mathbf{T}_{\phi} \times \mathbf{T}_{\theta}) &= \begin{bmatrix} 2\cos(\theta)\sin(\phi) \\ 2\sin(\theta)\sin(\phi) \\ 2\cos(\phi) \end{bmatrix} \cdot \begin{bmatrix} \cos(\theta)\sin^{2}(\phi) \\ \sin(\theta)\sin^{2}(\phi) \\ \sin(\phi)\cos(\phi) \end{bmatrix} \\
	&= 2\cos^{2}(\theta)\sin^{3}(\phi) + 2\sin^{2}(\theta)\sin^{3}(\phi) + 2\sin(\phi)\cos^{2}(\phi) \\
	&= 2 \sin^{3}(\phi)(\sin^{2}(\theta) + \cos^{2}(\theta)) + 2 \sin(\phi)\cos^{2}(\phi) \\
	&= 2\sin^{3}(\phi) + 2\sin(\phi)\cos^{2}(\phi) \\
	&= 2\sin(\phi)(\sin^{2}(\phi) + \cos^{2}(\phi)) \\
	&= 2\sin(\phi).
\end{align*}
Therefore,
\begin{align*}
	\iint_{S_{1}} \mathbf{E} \cdot \dd{\vec{S}} &= \iint_{S_{1}} \mathbf{E}(\Phi_{1}) \cdot (\mathbf{T}_{\phi} \times \mathbf{T}_{\theta}) \dd{\theta} \dd{\phi} \\
	&= \int_{0}^{2\pi} \int_{0}^{\pi/2} 2\sin(\phi) \dd{\theta} \dd{\phi} \\
	&= 2\pi \Big[ -2\cos(\phi) \Big]_{0}^{\pi/2} \\
	&= 4\pi.
\end{align*}
Now, we calculate the flux over $S_{2}$. Realize that for $\Phi_{2}$,
\begin{align*}
	\mathbf{T}_{r} &= (\sin(\theta), \cos(\theta), 0) \\
	\mathbf{T}_{\theta} &= (r\cos(\phi), -r\sin(\phi), 0).
\end{align*}
Therefore,
\[
	\mathbf{T}_{r} \times \mathbf{T}_{\theta} = (0, 0, -r),
\]
so
\[
	\mathbf{E}(\Phi_{2}) \cdot (\mathbf{T}_{r} \times \mathbf{T}_{\theta}) = (2r\cos(\theta), 2r\sin(\theta), 0) \times (0, 0, -r) = 0.
\]
Hence,
\begin{align*}
	\iint_{S_{2}} \mathbf{E} \cdot \dd{\vec{S}}  &= \iint_{S_{2}} \mathbf{E}(\Phi_{2}) \cdot (\mathbf{T}_{r} \times \mathbf{T}_{\theta}) \dd{\theta} \dd{r} \\
	&= \iint_{S_{2}} 0 \dd{\theta} \dd{r} \\
	&= 0.
\end{align*}
Finally, we combine this with Equation (1) to yield that
\[
	\iint_{S} \mathbf{E} \cdot \dd{\vec{S}} = 4\pi + 0 = \boxed{4\pi}.
\]
% --------------------------------------------- %

\subsection*{Problem 9}

The surface may be parametrized by $\Phi(\theta, \phi) = \left( \cos(\theta)\sin(\phi), \sin(\theta)\sin(\phi), \tfrac{\sqrt{3}}{3} \cos(\phi) \right)$. for $\theta \in [0, 2\pi)$ and $\phi \in [0, \pi/2]$. Thus, 
\begin{align*}
	\mathbf{T}_{\theta} &= (-\sin(\theta)\sin(\phi), \cos(\theta)\sin(\phi), 0) \\
	\mathbf{T}_{\phi} &= \left( \cos(\theta)\cos(\phi), \sin(\theta)\cos(\phi), -\tfrac{\sqrt{3}}{3} \sin(\phi) \right).
\end{align*}
Therefore, we have that
\[
	\mathbf{T}_{\theta} \times \mathbf{T}_{\phi} = \left( -\tfrac{\sqrt{3}}{3} \cos(\theta)\sin^{2}(\phi), -\tfrac{\sqrt{3}}{3} \sin(\theta) \sin^{2}(\phi), -\sin(\phi)\cos(\phi) \right),
\]
We also have that $\curl \mathbf{F} = (2x^{3}yz, -3x^{2}y^{2}z, -2)$, so
\[
	\curl \mathbf{F} (\Phi) = \begin{bmatrix} -\tfrac{2\sqrt{3}}{3}  \sin(\theta) \cos^{3}(\theta)\sin^{8}(\phi) \\ \sqrt{3} \sin^{2}(\theta)\cos^{2}(\theta)\sin^{9}(\phi)\cos(\phi) \\ -2 \end{bmatrix}.
\]
Thus, 
\begin{align*}
	(\curl \mathbf{F}(\Phi)) \cdot (\mathbf{T}_{\theta} \times \mathbf{T}_{\phi}) &= \left( \tfrac{2}{3} \sin(\theta) \cos^{4}(\theta) \sin^{10}(\phi) \right) + \left( -\sin^{3}(\theta) \cos^{2}(\theta) \sin^{11}(\phi)\cos(\phi) \right) \\
	& \quad - (2 \sin(\phi)\cos(\phi)).
\end{align*}
When we integrate this quantity, all terms with a $\theta$ will cancel, as this quantity is being integrated from $0$ to $2\pi$. Then only the last term will remain, and 
\begin{align*}
	\iint_{S} (\curl \mathbf{F}) \cdot \dd{\vec{S}} &= \iint_{S} (\curl \mathbf{F}(\Phi)) \cdot (\mathbf{T}_{\theta} \times \mathbf{T}_{\phi}) \dd{\phi} \dd{\theta} \\
	&= \int_{0}^{2\pi} \int_{0}^{\pi/2} -2\sin(\phi)\cos(\phi) \dd{\phi} \dd{\theta} \\
	&= -2\pi \int_{0}^{\pi/2} \sin(2\phi) \dd{\phi} \\
	&= -2 \pi \Big[ -\frac{\cos(\phi)}{2} \Big]_{0}^{\pi/2} \\
	&= \boxed{-2\pi}.
\end{align*}

\newpage

% --------------------------------------------- %

\subsection*{Problem 11}

The following setup is identical to Problem 7. Divide the surface $S$ into two parts:
\begin{itemize}
	\item $S_{1}$: The upper hemisphere. $\Phi_{1}(\phi, \theta) = (\cos(\theta)\sin(\phi), \sin(\theta)\sin(\phi), \cos(\phi))$ for $\theta \in [0, 2\pi)$ and $\phi \in [0, \pi/2)$ is one parametrization. \textbf{The variables are flipped!}
	\item $S_{2}$: The unit disc. $\Phi_{2}(r, \theta) = (r \sin(\theta), r\cos(\theta), 0)$ for $r \in [0, 1]$ and $\theta \in [0, 2\pi)$ is one parametrization.
\end{itemize}
It is easy to verify that these possess the same orientation. Thus,
\begin{equation}
	\iint_{S} \mathbf{E} \cdot \dd{\vec{S}} = \iint_{S_{1}} \mathbf{E} \cdot \dd{\vec{S}} + \iint_{S_{2}} \mathbf{E} \cdot \dd{\vec{S}}.
\end{equation}
Using our work from Problem 7,
\[
	\mathbf{T}_{\phi} \times \mathbf{T}_{\theta} = (\cos(\theta)\sin^{2}(\phi), \sin(\theta)\sin^{2}(\phi), \sin(\phi)\cos(\phi)).
\]
Furthermore, realize that
\[
	\mathbf{F}(\Phi_{1}) = \begin{bmatrix} \cos(\theta)\sin(\phi) + 3\sin^{5}(\theta)\sin^{5}(\phi) \\ \sin(\theta)\sin(\phi) + 10\cos(\theta)\sin(\phi)\cos(\phi) \\ \cos(\phi) - \sin(\theta)\cos(\theta)\sin^{2}(\phi) \end{bmatrix},
\] 
so we have that
\begin{align*}
	\mathbf{F}(\Phi_{1}) \cdot (\mathbf{T}_{\phi} \times \mathbf{T}(\theta)) &= \left( \cos^{2}(\theta)\sin^{3}(\phi) + 3 \sin^{5}(\theta)\cos(\theta)\sin^{7}(\phi)) \right) \\
	& \quad + \left( \sin^{2}(\theta)\sin^{3}(\phi) + 10 \sin(\theta)\cos(\theta)\sin^{3}(\phi)\cos(\phi) \right) \\
	& \quad + \left( 3\sin(\phi)\cos^{2}(\phi) - \sin(\theta)\cos(\theta)\sin^{3}(\phi)\cos(\phi) \right).
\end{align*}
When we integrate this quantity, all terms with a $\theta$ will cancel, as this quantity is being integrated from $0$ to $2\pi$. Then only the term $\sin(\phi)\cos^{2}(\phi)$ will remain, and 
\begin{align*}
	\iint_{S_{1}} (\curl \mathbf{F}) \cdot \dd{\vec{S}} &= \iint_{S_{1}} (\curl \mathbf{F}(\Phi_{1})) \cdot (\mathbf{T}_{\phi} \times \mathbf{T}_{\theta}) \dd{\phi} \dd{\theta} \\
	&= \int_{0}^{2\pi} \int_{0}^{\pi/2} 3\sin(\phi)\cos^{2}(\phi) \dd{\phi} \dd{\theta} \\
	&= 2\pi \left[ -\cos^{3}(\phi) \right]_{0}^{\pi/2} \\
	&= 2\pi.
\end{align*}
Now, we tackle $S_{2}$. For $\Phi_{2}$, our work in Problem 7 yields that
\[
	\mathbf{T}_{r} \times \mathbf{T}_{\theta} = (0, 0, -r),
\]
Furthermore, realize that
\[
	\mathbf{F}(\Phi_{2}) = \begin{bmatrix} r \cos(\theta) + 3 r^{5} \sin^{5}(\theta) \\ r \cos(\theta) \\ -r^{2} \sin(\theta)\cos(\theta) \end{bmatrix}.
\]
Thus,
\[
	\mathbf{F}(\Phi^{2}) \cdot (\mathbf{T}_{\phi} \times \mathbf{T}_{\theta}) = r^{3}\sin(\theta)\cos(\theta).
\]
We are now ready to compute the integral for $S_{2}$.
\begin{align*}
	\iint_{S_{2}} \mathbf{F} \cdot \dd{\vec{S}}  &= \iint_{S_{2}} \mathbf{E}(\Phi_{2}) \cdot (\mathbf{T}_{r} \times \mathbf{T}_{\theta}) \dd{\theta} \dd{r} \\
	&= \int_{0}^{1} \int_{0}^{2\pi} r^{3} \sin(\theta)\cos(\theta)  \dd{\theta} \dd{r} \\
	&= 0.
\end{align*}
Finally, we combine this with Equation (1) to yield that
\[
	\iint_{S} \mathbf{F} \cdot \dd{\vec{S}} = 2\pi + 0 = \boxed{2\pi}.
\]
% --------------------------------------------- %

\end{document}
