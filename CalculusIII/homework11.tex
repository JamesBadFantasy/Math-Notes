\documentclass[11pt]{article}
\usepackage[T1]{fontenc}
\usepackage{geometry, changepage}
\usepackage{amsmath, amssymb, amsthm, bm}
\usepackage{physics}
\usepackage{hyperref}

\hypersetup{colorlinks=true, linkcolor=blue, urlcolor=cyan}
\setlength{\parindent}{0pt}
\setlength{\parskip}{5pt}

\newtheorem{theorem}{Theorem}
\newtheorem{lemma}{Lemma}
\newtheorem{claim}{Claim}
\newtheorem*{theorem*}{Theorem}
\newtheorem*{lemma*}{Lemma}
\newtheorem*{claim*}{Claim}

\renewcommand{\vec}[1]{\mathbf{#1}}
\newcommand{\uvec}[1]{\mathop{} \!\hat{\mathbf{#1}}}
\newcommand{\mat}[1]{\mathbf{#1}}
\newcommand{\tensor}[1]{\mathsf{#1}}

\renewcommand{\div}{\nabla \cdot}
\renewcommand{\curl}{\nabla \cross}
\renewcommand{\grad}{\nabla}
\renewcommand{\laplacian}{\nabla^{2}}

\title{MATH-UA 129: Homework 11}
\author{James Pagan, December 2023}
\date{Professor Serfaty}

% --------------------------------------------- %

\begin{document}

\maketitle
\tableofcontents
\newpage

% --------------------------------------------- %

\section{Section 8.1}

% --------------------------------------------- %

\subsection*{Problem 10}

Let $C$ be the closed simple curve that bounds the disc with radius $R$ --- namely, $C(\theta) = (R\cos(\theta), R\sin(\theta))$. Then by Green's Theorem, the area of this region is
\[
	\frac{1}{2} \int_{0}^{2\pi} (R \cos(\theta))(R \cos(\theta)) - (R \sin(\theta))(-R \sin(\theta)) \dd{\theta} = \frac{2\pi R^{2}}{2} = \boxed{\pi R^{2}}.
\]

% --------------------------------------------- %

\subsection*{Problem 12}

By the Divergence Theorem,
\begin{align*}
	\int_{\partial D} \mathbf{F} \cdot \vec{n} \dd{\vec{s}} &= \iint_{D} (\div \mathbf{F}) \dd{A} \\
	&= \iint_{D} \pdv{x} y - \pdv{y} (-x) \dd{y} \dd{x} \\
	&= \iint_{D} 0 - 0 \dd{y} \dd{x} \\
	&= 0.
\end{align*}

% --------------------------------------------- %

\subsection*{Problem 13}

Using Green's Theorem, the area of the region bounded by the curve is 
\begin{align*}
	\frac{1}{2} \int_{0}^{2\pi} x \dd{y} - y \dd{x} &= \frac{1}{2} \int_{0}^{2\pi} a(\theta - \sin(\theta))(a \sin(\theta)) - a(1 - \cos(\theta))(a - a \cos(\theta)) \dd{\theta} \\
	&= \frac{a^{2}}{2} \int_{0}^{2\pi} \theta \sin(\theta) - \sin^{2}(\theta) - (1 - 2 \cos(\theta) + \cos^{2}(\theta)) \dd{\theta} \\
	&= \frac{a^{2}}{2} \int_{0}^{2\pi} \theta \sin(\theta) + 2\cos(\theta) - 2 \dd{\theta} \\
	&= \frac{a^{2}}{2} \big[ 3\sin(\theta) - \theta \cos(\theta) - 2 \theta \big]_{0}^{2\pi} \\
	&= \frac{a^{2}}{2} \big[ -2\pi - 4\pi \big] \\
	&= -3 \pi a^{2}.
\end{align*}
The answer is the absolute value of this ---namely $\boxed{3\pi a^{2}}$.

% --------------------------------------------- %

\subsection*{Problem 15}

If $D$ is the unit disc, then Green's Theorem yields that
\begin{align*}
	\int_{C} (2x^{3} - y^{3}) \dd{x} + (x^{3} + y^{3}) \dd{y} &= \iint_{D} \left( \pdv{x} x^{3} + y^{3} \right) - \left( \pdv{y} (2x^{3} - y^{3}) \right) \dd{y} \dd{x} \\
	&= \iint_{D} 3x^{2} + 3y^{2} \dd{x} \dd{y} \\
	&= \int_{0}^{1} \int_{0}^{2\pi} 3r^{2}(r) \dd{\theta} \dd{r} \\
	&= 6\pi \left[ \frac{r^{4}}{4} \right]_{0}^{1} \\
	&= \boxed{\frac{3\pi}{2}}.
\end{align*}

% --------------------------------------------- %

\section{Section 8.2}

% --------------------------------------------- %

\subsection*{Problem 11}

\textbf{Verification by flux}: Realize that
\[
	\curl \mathbf{F} = \begin{bmatrix} \pdv{y} (z) - \pdv{z} (y) \\ \pdv{z} (x) - \pdv{x} (z) \\ \pdv{x} (y) - \pdv{y} (x) \end{bmatrix} = \vec{0}.
\]
Then if we let the upper hemisphere be $\Sigma$,
\[
	\iint_{\Sigma} \curl \mathbf{f} \cdot \dd{\vec{S}} = \iint_{\Sigma} \vec{0} \cdot \dd{\vec{S}} = \boxed{0}.
\]
\textbf{Verification by circulation}: The oriented boundary of the upper hemisphere is given by $\vec{c}(\theta) = (\cos(\theta), \sin(\theta), 0)$ for $\theta \in [0, 2\pi)$. Then if the upper hemisphere is $\Sigma$, 
\begin{align*}
	\int_{\partial \Sigma} \mathbf{F} \cdot \dd{\vec{s}} &= \int_{0}^{2\pi} \mathbf{F}(\vec{c}(\theta)) \cdot \vec{c}'(\theta) \dd{\theta} \\
	&= \int_{0}^{2\theta} (\cos(\theta), \sin(\theta), 0) \cdot (-\sin(\theta), \cos(\theta), 0) \dd{\theta} \\
	&= \int_{0}^{2\pi} -\sin(\theta)\cos(\theta) + \sin(\theta)\cos(\theta) \dd{\theta}  \\
	&= \int_{0}^{2\pi} 0 \dd{\theta} \\
	&= \boxed{0}.
\end{align*}

These two integrals match, as stated by Stokes' Theorem.

% --------------------------------------------- %

\subsection*{Problem 24}

Realize that if $\vec{v} = (v_{1}, v_{2}, v_{3})$, then
\begin{align*}
	\vec{v} \times \vec{r} &= \begin{bmatrix} v_{1} \\ v_{2} \\ v_{3} \end{bmatrix} \times \begin{bmatrix} x \\ y \\ z \end{bmatrix} \\
						   &= \begin{bmatrix} v_{2} z - v_{3} y \\ v_{3} x - v_{1} z \\ v_{1} y - v_{2} x \end{bmatrix}.
\end{align*}
To apply Stokes' Theorem, we must calculate the curl of this vector:
\begin{align*}
	\curl (\vec{v} \times \vec{r}) &= \begin{bmatrix} \pdv{y} (v_{1}y - v_{2}z) - \pdv{z} (v_{3}x - v_{1}z) \\ \pdv{z} (v_{2}z - v_{3}y) - \pdv{x} (v_{1}y - v_{2}x) \\ \pdv{x} (v_{3}x - v_{1}z) - \pdv{z} (v_{2}z - v_{3}y) \end{bmatrix} \\
	&= \begin{bmatrix} v_{1} + v_{1} \\ v_{2} + v_{2} \\ v_{3} + v_{3} \end{bmatrix} \\
	&= 2 \begin{bmatrix} v_{1} \\ v_{2} \\ v_{3} \end{bmatrix} \\
	&= 2 \vec{v}.
\end{align*}
Then by Stokes' Theorem,
\[
	\int_{\partial S} (\vec{v} \times \vec{r}) \cdot \dd{\vec{s}} = \iint_{S} \curl (\vec{v} \times \vec{r}) = \iint 2 \vec{v} \cdot \dd{\vec{S}} = 2 \iint_{S} \vec{v} \cdot \vec{n} \dd{S}.
\]

% --------------------------------------------- %

\subsection*{Problem 25}

Consider removing a ``small hole of circumference $\epsilon$'' to yield the surface $\Sigma$. As $\epsilon$ goes to zero, we expect the circulation
\[
	\int_{\partial \Sigma} \mathbf{F} \cdot \dd{\vec{s}}
\]
to go to zero. By Stokes' Theorem, we expect the equivalent quantity
\[
	\int_{\Sigma} (\curl \mathbf{F}) \cdot \dd{\vec{S}} 
\]
to approach $0$ as well.

% --------------------------------------------- %

\section{Section 8.3}

% --------------------------------------------- %

\subsection*{Problem 4}

\textbf{Part (a)}: For $f(x) = e^{x} \cos(y) + z \pi$, observe that $\mathbf{F} = \grad f$. Hence, $\boxed{\text{$f$ is a gradient}}$ and is thus $\boxed{\text{not a curl}}$.

\textbf{Part (b)}: For $f(x) = \tfrac{xy}{z^{2} + 4}$, observe that $\mathbf{F} = \grad f$. Hence, $\boxed{\text{$f$ is a gradient}}$ and is thus $\boxed{\text{not a curl}}$.

\textbf{Part (c)}: Observe that
\[
	\curl \mathbf{F} = \left( x \cos(z), 2x^{2}y^{2}z - y \cos(z), ye^{x} - 2x^{2}yz^{2} \right)
\]
is nonzero, and 
\[
	\div \mathbf{F} = xy^{2}z^{2} + e^{x} - xy \sin(z)j
\]
is nonzero, so $\boxed{\text{$f$ is neither a gradient nor a curl}}$.

\textbf{Part (d)}: See that $\div F = \pdv{x} (6z^{5}y^{5}) + \pdv{y} (9x^{8}z^{2}) + \pdv{z} (4x^{3}y^{3}) = 0 + 0 + 0 = 0$, so $\boxed{\text{$f$ is a curl}}$ and is thus $\boxed{\text{not a gradient}}$.

% --------------------------------------------- %

\subsection*{Problem 5}

Suppose that $f$ and $g$ are two potential fields of $\mathbf{F}$ --- namely, that $\grad f = \grad g = \mathbf{F}$. Then
\[
	\grad (f - g) = \grad f - \grad g = \mathbf{F} - \mathbf{F} = \vec{0},
\]
so $f - g$ is constant (this is a well-known result we discussed in class), which completes the proof.

% --------------------------------------------- %

\subsection*{Problem 18}

\textbf{Part (a)}: Realize that as
\[
	\pdv{y} (2x + y^{2} - y \sin(x)) = 2y - \sin(x) \ne 2yz - \sin(x) = \pdv{x} (2xyz + \cos(x)),
\]
the vector field $\boxed{\text{$\mathbf{F}$ is not a gradient}}$.

\textbf{Part (b)}: Realize that as
\[
	\pdv{z} 6x^{2}z^{2} = 12x^{2}z \ne 0 = \pdv{x} 4y^{2}z^{2},
\]
the vector field $\boxed{\text{$\mathbf{F}$ is not a gradient}}$.

\textbf{Part (c)}: Realize that if $f(x, y) = xy^{3} + x + y$, that $\mathbf{F} = \grad f$. Thus, $\boxed{\grad (xy^{3} + x + y) = \mathbf{F}}$ --- and of course, we can add a constant to $f$.

\textbf{Part (d)}: Realize that as
\[
	\pdv{y} (x e^{x^{2} + y^{2}} + 2xy) = 2xy e^{x^{2} + y^{2}} + 2x \ne 2xy e^{x^{2} + y^{2}} = \pdv{x} (y e^{x^{2} + y^{2}} + 4y^{3}z),
\]
the vector field $\boxed{\text{$\mathbf{F}$ is not a gradient}}$.

% --------------------------------------------- %

\subsection*{Problem 22}

We have that
\[
	\div \mathbf{F} = \pdv{x} (xz) + \pdv{y} (-yz) + \pdv{z} (y) = z - z + 0 = 0.
\]
We conclude that $\mathbf{F}$ is the curl of some vector field --- an example of such a field is $\boxed{\mathbf{G}(x, y, z) = (0, xy, xyz)}$, as verified by a trivial computation.

% --------------------------------------------- %

\subsection*{Problem 29}

We have that if $\vec{r} = (x, y, z)$,
\[
	\mathbf{F} = -\frac{GmM \vec{r}}{r^{3}} = -\frac{GmM}{\sqrt{(x^{2}+y^{2}+z^{2})^{3}}} (x, y, z).
\]
Thus,
\begin{align*}
	\div \mathbf{F} &= -\frac{GmM(-2x^{2} + y^{2} + z^{2})}{\sqrt{(x^{2} + y^{2} + z^{2})^{5}}}  -\frac{GmM(x^{2} - 2y^{2} + z^{2})}{\sqrt{(x^{2} + y^{2} + z^{2})^{5}}} -\frac{GmM(x^{2} + y^{2} - 2z^{2})}{\sqrt{(x^{2} + y^{2} + z^{2})^{5}}} \\
					&= - \frac{GmM(0)}{\sqrt{(x^{2} + y^{2} + z^{2})^{5}}} \\
					&= 0.
\end{align*}

% --------------------------------------------- %

\section{Section 8.4}

% --------------------------------------------- %

\subsection*{Problem 5}

Let the unit sphere be $W$. Then by the Divergence Theorem, we have that the flux on the surface is
\begin{align*}
	\iint_{\partial W} \mathbf{F} \cdot \dd{\vec{S}} &= \iiint_{W} (\div \mathbf{F}) \dd{V} \\
	&= \iiint_{W} \left( \pdv{x} (x - y) + \pdv{y} (y - z) + \pdv{z} (z - x) \right) \\
	&= \iiint_{@} 3 \dd{V} \\
	&= 3 \left( \frac{4\pi}{3} \right) \\
	&= \boxed{4\pi}.
\end{align*}

% --------------------------------------------- %

\subsection*{Problem 11}

Let the box be $W$. By Gauss' Theorem, we have that
\begin{align*}
	\iint_{\partial W} \mathbf{F} \cdot \dd{\vec{S}} &= \iiint_{W} (\div \mathbf{F}) \dd{V} \\
	&= \iiint_{W} \left( \pdv{x} (x - y^{2}) + \pdv{y} (y) + \pdv{z} (x^{3}) \right) \dd{V} \\
	&= \int_{0}^{1} \int_{1}^{2} \int_{1}^{4} 2 \dd{z} \dd{y} \dd{x} \\
	&= \boxed{6}.
\end{align*}

% --------------------------------------------- %

\subsection*{Problem 12}

Let the unit sphere be $W$. Then by Gauss' Theorem, we have that
\begin{align*}
	\iint_{S} \mathbf{F} \cdot \dd{\vec{S}} &= \iiint_{W} (\div \mathbf{F}) \dd{V} \\
	&= \iiint_{W} \left( \pdv{x} (3xy^{2}) + \pdv{y} (3x^{2}y) + \pdv{z} (3z^{2}) \right) \dd{V} \\
	&= \iiint_{W} 3x^{2} + 3y^{2} + 3z^{2} \dd{V} \\
	&= \int_{0}^{1} \int_{0}^{2\pi} \int_{0}^{\pi} 3\rho^{2} (\rho^{2} \sin(\phi)) \dd{\phi} \dd{\theta} \dd{\rho} \\
	&= 3 \Big[ \frac{\rho^{5}}{5} \Big]_{0}^{1} (2\pi) \Big[ -\cos(\phi) \Big]_{0}^{\pi} \\
	&= \boxed{\frac{12\pi}{5}}
\end{align*}



% --------------------------------------------- %

\subsection*{Problem 16}

By Gauss' Theorem, we have that
\begin{align*}
	\iint_{\partial W} \mathbf{F} \cdot \vec{n}\dd{A} &= \iiint_{W} (\div \mathbf{F}) \dd{V} \\
	&= \iiint_{W} \left( \pdv{x} (1) + \pdv{y} (1) + \pdv{z} (z(x^{2} + y^{2})^{2}) \right) \\
	&= \iiint_{W} (x^{2} + y^{2})^{2} \dd{V} \\
	&= \int_{0}^{1} \int_{0}^{1} \int_{0}^{2\pi} (r^{2})^{2}(r) \dd{\theta} \dd{\rho} \dd{z} \\
	&= 2\pi \Big[ \frac{r^{6}}{6} \Big]_{0}^{1} \\
	&= \boxed{\frac{\pi}{3}}.
\end{align*}

% --------------------------------------------- %

\subsection*{Problem 17}

By the properties of divergence (and by Gauss' Theorem), we have that
\begin{align*}
	\iiint_{W} (\grad f) \cdot \mathbf{F} \dd{x} \dd{y} \dd{z} &= \iiint_{W} (\grad f) \cdot \mathbf{F} \dd{x} \dd{y} \dd{z} \\
	& \quad + \iiint_{W} f (\div \mathbf{F}) \dd{x} \dd{y} \dd{z} - \iiint_{W} f (\div \mathbf{F}) \dd{x} \dd{y} \dd{z} \\
	&= \iiint_{W} \grad f \cdot \mathbf{F} + f (\div \mathbf{F}) \dd{x} \dd{y} \dd{z} - \iiint_{W} f (\div \mathbf{F}) \dd{x} \dd{y} \dd{z} \\
	&= \iiint_{W} \div (f \mathbf{F}) \dd{x} \dd{y} \dd{z} - \iiint_{W} f (\div \mathbf{F}) \dd{x} \dd{y} \dd{z} \\
	&= \iint_{\partial W} f \mathbf{F} \cdot \dd{S} - \iiint_{W} f(\div \mathbf{F}) \dd{x} \dd{y} \dd{z} \\
	&= \iint_{\partial W} f \mathbf{F} \cdot \vec{n} \dd{S} - \iiint_{W} f(\div \mathbf{F}) \dd{x} \dd{y} \dd{z} \\
\end{align*}
as desired.

% --------------------------------------------- %

\subsection*{Problem 21}

For the first formula: we have that $\div (f \grad g) = f(\div \grad g) + \grad g \cdot \grad f = f \laplacian g + \grad f \cdot \grad g$. Then by Gauss' Theorem,
\[
	\iint_{\partial W} f \grad g \cdot \vec{n} \dd{S} = \iiint_{W} \div (f \grad g) \dd{V} = \iiint_{W} (f \laplacian g + \grad f \cdot \grad g) \dd{V}.
\]
Now to prove the second formula: realize that by substituting $f$ for $g$ and vice versa, 
\[
	\iint_{\partial W} g \grad f \cdot \vec{n} \dd{S} = \iiint_{W} (g \laplacian f + \grad g \cdot \grad f).
\]
We thus deduce by substition and Gauss' Theorem that
\begin{align*}
	\iint_{\partial W} (f \grad g - g \grad f) \cdot \vec{n} \dd{S} &= \iint_{\partial W} f \grad g \cdot \vec{n} \dd{S} - \iint_{\partial W} g \grad f \cdot \vec{n} \dd{S} \\
	&= \iiint_{W} (f \laplacian g + \grad f \cdot \grad g) \dd{V} - \iiint_{W} (g \laplacian f + \grad f \cdot \grad g) \dd{V} \\
	&= \iiint_{W} (f \laplacian g - g \laplacian f + \grad f \cdot \grad g - \grad g \cdot \grad f) \dd{V} \\
	&= \iiint_{W} (f \laplacian g - g \laplacian f) \dd{V} 
\end{align*}

% --------------------------------------------- %

\subsection*{Problem 28}

Let the region enclosed by $S$ be $W$. Then by Gauss' Theorem (and because the curl is divergence-free),
\begin{align*}
	\iint_{S} (\curl \mathbf{F}) \cdot \dd{\vec{S}} &= \iiint_{W} \div (\curl \mathbf{F}) \dd{V} \\
	&= \iiint_{W} 0 \dd{V} \\
	&= 0.
\end{align*}


% --------------------------------------------- %

\end{document}
