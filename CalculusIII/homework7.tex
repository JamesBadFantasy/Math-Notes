\documentclass[11pt]{article}
\usepackage[T1]{fontenc}
\usepackage{geometry, changepage}
\usepackage{amsmath, amssymb, amsthm, bm}
\usepackage{physics}
\usepackage{hyperref}

\hypersetup{colorlinks=true, linkcolor=blue, urlcolor=cyan}
\setlength{\parindent}{0pt}
\setlength{\parskip}{5pt}

\newtheorem{theorem}{Theorem}
\newtheorem{lemma}{Lemma}
\newtheorem{claim}{Claim}
\newtheorem*{theorem*}{Theorem}
\newtheorem*{lemma*}{Lemma}
\newtheorem*{claim*}{Claim}

\renewcommand{\vec}[1]{\mathbf{#1}}
\newcommand{\uvec}[1]{\mathop{} \!\hat{\mathbf{#1}}}
\newcommand{\mat}[1]{\mathbf{#1}}
\newcommand{\tensor}[1]{\mathsf{#1}}

\renewcommand{\div}{\nabla \cdot}
\renewcommand{\curl}{\nabla \cross}
\renewcommand{\grad}{\nabla}
\renewcommand{\laplacian}{\nabla^{2}}

\title{MATH-UA 129: Homework 7}
\author{James Pagan, November 2023}
\date{Professor Serfaty}

% --------------------------------------------- %

\begin{document}

\maketitle
\tableofcontents

% --------------------------------------------- %

\section{Secction 4.2}

% --------------------------------------------- %

\subsection*{Problem 4}

If we define $\vec{c}(t) = (2 \cos(t), 2 \sin(t), t)$, then we seek 
\[
	\int_{0}^{2\pi} \norm{\vec{c}'(t)} \dd{t} = \int_{0}^{2\pi} \sqrt{(-2 \sin(t))^{2} + (2 \cos(t))^{2} + 1^{2}} \dd{t} = \int_{0}^{2\pi} \sqrt{5} \dd{t} = t \sqrt{5} \Big|_{0}^{2\pi} = \boxed{2\pi\sqrt{5}}.
\]

% --------------------------------------------- %

\subsection*{Problem 6}

If we define $\vec{c}(t) = (t, t \sin(t), t \cos(t))$, then we seek
\[
	\int_{0}^{2\pi} \norm{\vec{c}'(t)} \dd{t} = \int_{0}^{2\pi} \sqrt{t^{2} + (t \sin(t))^{2} + (t \cos(t))^{2}} \dd{t} = \int_{0}^{2\pi} \sqrt{2t^{2}} \dd{t} = \int_{0}^{2\pi} 2 \abs{t} \dd{t} = \boxed{4\pi}.
\]

% --------------------------------------------- %

\subsection*{Problem 14}

We seek to compute two integrals:
\[
	\int_{0}^{t} \norm{\vec{\alpha}'(\tau)} \dd{\tau} \qquad \text{and} \qquad \int_{0}^{t} \norm{\vec{\beta}'(t)} \dd{\tau} 
\]
For the first integral, observe that the arc length is
\[
	\int_{0}^{t} \norm{\vec{\alpha}'(t)} \dd{\tau} = \int_{0}^{t} \sqrt{\sinh^{2}(\tau) + \cosh^{2}(\tau) + \tau^{2}} \dd{\tau} = \boxed{ \int_{0}^{t} \sqrt{2 \cosh^{2}(\tau) - 1 + \tau^{2}} \dd{\tau}}.
\]
For the second integral,
\[
	\int_{0}^{t} \norm{\vec{\beta}'(t)} \dd{\tau} = \int_{0}^{t} \sqrt{\cos^{2}(\tau) + \sin^{2}(\tau) + \tau^{2}} \dd{\tau} = \boxed{\int_{0}^{t} \sqrt{1 + \tau^{2}} \dd{\tau}}.
\]

% --------------------------------------------- %

\subsection*{Problem 15}

% --------------------------------------------- %

\section{Section 4.3}

% --------------------------------------------- %

\subsection*{Problem 9}

\textbf{Part (a)}: Clearly, $\vec{V}(x, y) = x \vec{i} + y \vec{j}$ is represented by $\boxed{\text{Graph (ii)}}$.

\textbf{Part (b)}: Clearly, $\vec{V}(x, y) = y \vec{i} - x \vec{x}$ is respresented by $\boxed{\text{Graph (i)}}$.

% --------------------------------------------- %

\subsection*{Problem 10}

\textbf{Part (a)}: Clearly, $\vec{V}(x, y) = \tfrac{y}{\sqrt{x^{2} + y^{2}}} \vec{i} - \tfrac{x}{\sqrt{x^{2} + y^{2}}} \vec{j}$ is represented by $\boxed{\text{Graph (i)}}$.

\textbf{Part (a)}: Clearly, $\vec{V}(x, y) = \tfrac{x}{\sqrt{x^{2} + y^{2}}} \vec{i} + \tfrac{y}{\sqrt{x^{2} + y^{2}}} \vec{j}$ is represented by $\boxed{\text{Graph (ii)}}$.

\textbf{Part (c)}: These two fields are the $\boxed{\text{normalized vector fields of Problem $9$}}$. They are thus not defined at $\boxed{(x, y) = (0, 0)}$.

% --------------------------------------------- %

\subsection*{Problem 15} 

We have that
\[
	\vec{c}'(t) = \left( \dv{t} e^{2t}, \dv{t} \ln \abs{t}, \dv{t} \tfrac{1}{t} \right) = \left( 2 e^{2t}, \tfrac{1}{t}, -\tfrac{1}{t^{2}} \right) = \mathbf{F}(e^{2t}, \ln \abs{t}, \tfrac{1}{t}) = \mathbf{F}(\vec{c}(t)).
\]
Thus, $\vec{c}(t)$ is a flow line of the given velocity vector field $\mathbf{F}(x, y, z)$.

% --------------------------------------------- %

\subsection*{Problem 20}

We have that
\begin{align*}
	\vec{c}'(t) &= \left( \dv{t} a \cos(t) - b \sin(t), \dv{t} a \sin(t) - b \cos(t) \right) \\
	&= \left( -a \sin(t) - b \cos(t), a \cos(t) - b \sin(t) \right) \\
	&= \mathbf{F}(a \cos(t) - b \sin(t), a \sin(t) + b \cos(t)) \\
	&= \mathbf{F}(\vec{c}(t)).
\end{align*}
Thus, $\vec{c}(t)$ is a flow line of the given velocity vector field $\mathbf{F}(x, y, z)$.

% --------------------------------------------- %

\section{Section 4.4}

% --------------------------------------------- %

\subsection*{Problem 2}

If we let $\vec{V}(x, y, z) = u(x, y, z) \vec{i} + v(x, y, z) \vec{j} + w(x, y, z) \vec{k}$, then 
\begin{align*}
	\grad \cdot \vec{V}(x, y, z) &= \pdv{u}{x} + \pdv{v}{y} + \pdv{w}{z} \\
	&= \pdv{x} yz + \pdv{y} zx + \pdv{z} xy \\
	&= 1 + 1 + 1 \\
	&= \boxed{3}.
\end{align*}

% --------------------------------------------- %

\subsection*{Problem 4}

If we let $\vec{V}(x, y, z) = u(x, y, z) \vec{i} + v(x, y, z) \vec{j} + w(x, y, z) \vec{k}$, then 
\begin{align*}
	\grad \cdot \vec{V}(x, y, z) &= \pdv{u}{x} + \pdv{v}{y} + \pdv{w}{z} \\
	&= \pdv{x} x^{2} + \pdv{y} (x + y)^{2} + \pdv{z} (x + y + z)^{2} \\
	&= 2x + 2(x + y) + 2(x + y + z) \\
	&= \boxed{6x + 4y + 2z}.
\end{align*}

% --------------------------------------------- %

\subsection*{Problem 17}

If we let $\vec{F}(x, y) = f(x, y) \vec{i} + g(x, y) \vec{j}$, then the scalar curl is as follows:
\[
	\grad \times \vec{F} = \left( \pdv{g}{x} - \pdv{f}{y} \right) \vec{k} = \left( \pdv{x} \cos(x) - \pdv{y} \sin(x) \right) \vec{k} = \left( -\sin(x) - 0 \right) \vec{k} = \boxed{- \sin(x) \vec{k}}.
\]

% --------------------------------------------- %

\subsection*{Problem 18}

If we let $\vec{F}(x, y) = f(x, y) \vec{i} + g(x, y) \vec{j}$, then the scalar curl is as follows:
\[
	\grad \times \vec{F} = \left( \pdv{g}{x} - \pdv{f}{y} \right) \vec{k} = \left( \pdv{x} (-x) - \pdv{y} y \right) \vec{k} = \left( -1 - 1 \right) \vec{k} = \boxed{-2 \vec{k}}.
\]

% --------------------------------------------- %

\subsection*{Problem 21}

\textbf{Part (a)}: If we let $F(x, y, z) = f(x, y, z) \vec{i} + g(x, y, z) \vec{j} + h(x, y, z) \vec{k}$, then
\[
	\grad \times f(x, y, z) = \begin{bmatrix} \pdv{h}{y} - \pdv{g}{z} \\ \pdv{f}{z} - \pdv{h}{x} \\ \pdv{g}{x} - \pdv{f}{y} \end{bmatrix} = \begin{bmatrix} \pdv{y} (z + zx) - \pdv{z} x^{2} y \\ \pdv{z} x^{2} - \pdv{x} (z + zx) \\ \pdv{x} x^{2}y - \pdv{y} x^{2} \end{bmatrix} = \begin{bmatrix} 0 \\ -z \\ 2xy \end{bmatrix}.
\]
Therefore,
\[
	\grad \cdot (\grad \times f) = \grad \cdot \begin{bmatrix} 0 \\ -z \\ 2xy \end{bmatrix} = \pdv{x} 0 - \pdv{y} z + \pdv{z} 2xy = 0.
\]
\textbf{Part (b)}: Suppose for contradiction that there exists a function $f : \mathbb{R}^{3} \to \mathbb{R}$ such that $\vec{F} = \grad f$. Then 
\begin{align*}
	\pdv{f}{x} &= x^{2} \\
	\pdv{f}{y} &= x^{2} y \\
	\pdv{f}{z} &= z + zx
\end{align*}
We may integrate each of these equations with respect to $x$, $y$, and $z$ respectively to yield that
\begin{align*}
	f &= \frac{x^{3}}{3} + P(y, z) \\
	f &= \frac{x^{2}y^{2}}{2} + Q(z, x) \\
	f &= \frac{z^{2}}{2} + \frac{z^{2}x}{2} + S(x, y)
\end{align*}
for some functions $P(y, z)$, $Q(z, x)$, and $S(x, y)$. Observe the term $\tfrac{z^{2}x}{2}$ in the third equation; it cannot be cancelled by $S(x, y)$, as $S$ does not contain terms with $z$. Therefore, $\tfrac{z^{2}x}{2}$ is a term of $f$.

However, the first equation reveals that $f$ cannot contain any terms with both $z$ and $x$ --- the only term with an $x$ is $\tfrac{x^{3}}{3}$, and all other terms exclusively contain $y$ and $z$. This implies that $\tfrac{z^{2}x}{2}$ cannot be a term of $f$, which yields the desired contradiction.

We conclude that $\boxed{\text{$f$ does not exist}}$.

% --------------------------------------------- %

\subsection*{Problem 24}

Observe that $\grad f$ maps $\mathbb{R}^{3}$ to $\mathbb{R}^{3}$, $\grad \cdot f$ maps $\mathbb{R}^{3}$ to $\mathbb{R}$, and $\grad \times f$ is not defined.

\textbf{Part (a)}: The expression $\grad \times (\grad f)$ is $\boxed{\text{a meaningful vector-valued function}}$.

\textbf{Part (b)}: The expression $\grad (\grad \times f)$ is $\boxed{\text{not meaningful}}$, as $\grad \times f$ is only defined if $f$ maps to $\mathbb{R}^{3}$.

\textbf{Part (c)}: The expression $\grad \cdot (\grad f)$ is $\boxed{\text{a meaningful scalar-valued function}}$.

\textbf{Part (d)}: The expression $\grad (\grad \cdot f)$ is $\boxed{\text{a meaningful vector-valued function}}$. 

\textbf{Part (e)}: The expression $\grad \times (\grad \cdot f)$ is $\boxed{\text{not meaningul}}$, as the curl accepts 3-D vectors and $\grad \cdot f$ maps to $\mathbb{R}$.

\textbf{Part (f)}: The expression $\grad \cdot (\grad \times f)$ is $\boxed{\text{not meaningful}}$, as $\grad \times f$ is only defined if $f$ maps to $\mathbb{R}^{3}$.

% --------------------------------------------- %

\subsection*{Problem 25}

Observe that $\grad f$ does not exist, $\grad \cdot f$ maps $\mathbb{R}^{3}$ to $\mathbb{R}$, and $\grad \times f$ maps $\mathbb{R}$ to $\mathbb{R}^{3}$.

\textbf{Part (a)}: The expression $\grad \times (\grad f)$ is $\boxed{\text{not meaningful}}$, as the gradient only accepts scalar-valued functions.

\textbf{Part (b)}: The expression $\grad (\grad \times f)$ is $\boxed{\text{not meaningful}}$.

\textbf{Part (c)}: The expression $\grad \cdot (\grad f)$ is $\boxed{\text{not meaningful}}$, as the gradient only accepts scalar-valued functions.

\textbf{Part (d)}: The expression $\grad (\grad \cdot f)$ is $\boxed{\text{a meaningful vector-valued function}}$.

\textbf{Part (e)}: The expression $\grad \times (\grad \cdot f)$ is $\boxed{\text{not meaningful}}$, as the curl accepts 3-D vectors and $\grad \cdot f$ maps to $\mathbb{R}$.

\textbf{Part (f)}: The expression $\grad \cdot (\grad \times f)$ is $\boxed{\text{a meaningful scalar-valued function}}$.

% --------------------------------------------- %

\subsection*{Problem 31}

We have that
\[
	\grad f = \begin{bmatrix} \pdv{x} \tfrac{1}{x^{2} + y^{2} + z^{2}} \\ \pdv{y} \tfrac{1}{x^{2} + y^{2} + z^{2}} \\ \pdv{z} \tfrac{1}{x^{2} + y^{2} + z^{2}} \end{bmatrix} = \begin{bmatrix} - \tfrac{2x}{(x^{2} + y^{2} + z^{2})^{2}} \\ -\tfrac{2y}{(x^{2} + y^{2} + z^{2})^{2}} \\ -\tfrac{2z}{(x^{2} + y^{2} + z^{2})} \end{bmatrix} = -\frac{2}{(x^{2} + y^{2} + z^{2})^{2}} \begin{bmatrix} x \\ y \\ z \end{bmatrix}.
\]
Therefore,
\begin{align*}
	\grad \times (\grad f) &= \grad \times \left( -\frac{2}{(x^{2} + y^{2} + z^{2})^{2}} \begin{bmatrix} x \\ y \\ z \end{bmatrix} \right) \\
	&= -\frac{2}{(x^{2} + y^{2} + z^{2})^{2}} \left( \grad \times \begin{bmatrix} x \\ y \\ z \end{bmatrix} \right) - \left( \grad \frac{2}{x^{2} + y^{2} + z^{2}} \right) \times \begin{bmatrix} x \\ y \\ z \end{bmatrix} \\
	&= -\frac{2}{(x^{2} + y^{2} + z^{2})^{2}} \left( \begin{bmatrix} 0 \\ 0 \\ 0 \end{bmatrix} \right) + \frac{4}{(x^{2} + y^{2} + z^{2})^{2}} \left( \begin{bmatrix} x \\ y \\ z \end{bmatrix} \times \begin{bmatrix} x \\ y \\ z \end{bmatrix} \right). \\
	&= \begin{bmatrix} 0 \\ 0 \\ 0 \end{bmatrix} + \frac{4}{(x^{2} + y^{2} + z^{2})^{2}} \left( \begin{bmatrix} 0 \\ 0 \\ 0 \end{bmatrix} \right) \\
	&= \begin{bmatrix} 0 \\ 0 \\ 0 \end{bmatrix}.
\end{align*}

% --------------------------------------------- %

\subsection*{Problem 34}

Suppose for contradiction that there exists $f : \mathbb{R}^{3} \to \mathbb{R}$ such that $\mathbf{F} = \grad f$. Then
\begin{align*}
	\pdv{f}{x} = x^{2} + y^{2} \\
	\pdv{f}{y} = -2xy.
\end{align*}
However,
\[
	\pdv[2]{f}{x}{y} = \pdv{x} -2xy = -2y \ne 2y = \pdv{y} (x^{2} + y^{2}) = \pdv[2]{f}{y}{x}.
\]
This violates the equality of mixed partial derivatives, implying the given result.

% --------------------------------------------- %

\subsection*{Problem 36}

\textbf{Part (a)}: We have that
\[
	\grad (\mathbf{F} + \mathbf{G}) = \grad \mathbf{F} + \grad \mathbf{G} = 0,
\]
so the divergence of $\mathbf{F} + \mathbf{G}$ $\boxed{\text{necessarily zero}}$.

\textbf{Part (b)}: The divergence of $\mathbf{F} \times \mathbf{G}$ is $\boxed{\text{not necessarily zero}}$.

% --------------------------------------------- %

\subsection*{Problem 37}

\textbf{Part (a)}: We have that
\[
	\grad f = \begin{bmatrix} 2xy \\ x^{2} \end{bmatrix}
\]
\textbf{Part (d)}: We have that
\[
	\mathbf{F} \cdot (\grad f) = F \cdot \begin{bmatrix} 2xy  \\ x^{2} \end{bmatrix} = \pdv{x} 2xy + \pdv{y} x^{2} = 2y.
\]

% --------------------------------------------- %

\subsection*{Problem 39}

$\boxed{\text{No}}$, it is not true that the curl of a vector field is perpendicular to the vector field. This is because the curl describes the activity of the vectors \textit{around} a vector --- this can be rigorously demonstrated using a counterexample.

% --------------------------------------------- %

\subsection*{Problem 40}

\textbf{Part (a)}: We have that
\[
	\grad \times f = \begin{bmatrix} \pdv{y} 0 - \pdv{z} (x^{3} + y^{3}) \\ \pdv{z} (3x^{2}y) - \pdv{x} 0 \\ \pdv{x} (x^{3} + y^{3}) - \pdv{y} (3x^{2}y) \end{bmatrix} = \begin{bmatrix} 0 \\ 0 \\ 0 \end{bmatrix}.
\]
\textbf{Part (b)}: The function $f(x, y) = x^{3}y + \tfrac{1}{4}y^{4}$ is one such function, as
\[
	\pdv{f}{x} = 3x^{2}y
\]
and
\[
	\pdv{f}{y} = x^{3} + y^{3}.
\]

% --------------------------------------------- %

\end{document}
