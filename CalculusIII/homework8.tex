\documentclass[11pt]{article}
\usepackage[T1]{fontenc}
\usepackage{geometry, changepage}
\usepackage{amsmath, amssymb, amsthm, bm}
\usepackage{physics}
\usepackage{hyperref}

\hypersetup{colorlinks=true, linkcolor=blue, urlcolor=cyan}
\setlength{\parindent}{0pt}
\setlength{\parskip}{5pt}

\newtheorem{theorem}{Theorem}
\newtheorem{lemma}{Lemma}
\newtheorem{claim}{Claim}
\newtheorem*{theorem*}{Theorem}
\newtheorem*{lemma*}{Lemma}
\newtheorem*{claim*}{Claim}

\renewcommand{\vec}[1]{\mathbf{#1}}
\newcommand{\uvec}[1]{\mathop{} \!\hat{\mathbf{#1}}}
\newcommand{\mat}[1]{\mathbf{#1}}
\newcommand{\tensor}[1]{\mathsf{#1}}

\renewcommand{\div}{\nabla \cdot}
\renewcommand{\curl}{\nabla \cross}
\renewcommand{\grad}{\nabla}
\renewcommand{\laplacian}{\nabla^{2}}

\title{MATH-UA 129: Homework 8}
\author{James Pagan, November 2023}
\date{Professor Serfaty}

% --------------------------------------------- %

\begin{document}

\maketitle
\tableofcontents

% --------------------------------------------- %

\section{Section 5.2}
% --------------------------------------------- %

\subsection*{Problem 8}

The volume is equivalent to the volume of $f(x, y) = x^{2} + y^{4}$ over the unit square, which is
\[
	\int_{0}^{1} \int_{0}^{1} (x^{2} + y^{4}) \dd{x} \dd{y} = \int_{0}^{1} \int_{0}^{1} x^{2} \dd{x} \dd{y} + \int_{0}^{1} \int_{0}^{1} y^{4} \dd{x} \dd{y} = \frac{1}{3} + \frac{1}{5} = \boxed{\frac{8}{15}}.
\]

% --------------------------------------------- %

\subsection*{Problem 14}

We seek to evaluate the integral defined as $f$:
\begin{align*}
	f(m, n) &= \int_{-\pi}^{\pi} \int_{-\pi}^{\pi} \cos(nx) \sin(ny) \dd{x} \dd{y} \\
	&= \left(\int_{-\pi}^{\pi} \sin(my) \dd{y} \right) \left( \int_{-\pi}^{\pi} \cos(nx) \dd{x} \right) \\
	&= \left( \frac{-\cos(m \pi) + \cos(-m \pi)}{m} \right) \left( \frac{\sin(n \pi) - \sin(-n \pi)}{n} \right) \\
	&= (0) \left( \frac{2 \sin(n \pi)}{n} \right) \\
	&= 0.
\end{align*}
Therefore, $\lim\limits_{m, n \to \infty} f(m, n) = 0$.

% --------------------------------------------- %

\subsection*{Problem 18}

Suppose for contradiction that there exists $\vec{a} \in R$ such that $f(\vec{a}) > 0$. As $f$ is continuous and $R$ is an open set, there exists $\delta$ such that
\[
	\norm{\vec{x} - \vec{a}} < \delta \implies \vec{x} \in R \quad \text{and} \quad \abs{f(\vec{x}) - f(\vec{a})} < \frac{f(\vec{a})}{2}
\]
Denote the open ball defined by $\norm{\vec{x} - \vec{a}} < \delta$ as $B$. We then have that for all $\vec{x} \in B$, 
\[
	-\frac{f(\vec{a})}{2} < f(\vec{x}) - f(\vec{a}) < \frac{f(\vec{a})}{2} \implies \frac{f(\vec{a})}{2} < f(\vec{x}) < \frac{3f(\vec{a})}{2},
\]
so $f(\vec{x}) > 0$ on $B$. We defined that $B \subset R$, so $B \cup (R \setminus B) = R$; hence,
\[
	0 < \iint_{B} f \dd{A} \le \iint_{B} f \dd{A} + \iint_{B \setminus R} f \dd{A} = \iint_{R} f \dd{A},
\]
which yields the desired contradiction. We conclude that $f = 0$ on $R$.

% --------------------------------------------- %

\section{Section 5.3}

% --------------------------------------------- %

\subsection*{Problem 6}

Observe that the ellipse is given by $\tfrac{x^{2}}{a^{2}} + \tfrac{y^{2}}{b^{2}} = 1$. Therefore, the area is given by evaluating the following integral:
\begin{align*}
	\int_{-a}^{a} \int_{-b \sqrt{1 - \tfrac{x^{2}}{a^{2}}}}^{b \sqrt{1 - \tfrac{x^{2}}{a^{2}}}} 1 \dd{y} \dd{x} &=\int_{-a}^{a} 2b \sqrt{1 - \frac{x^{2}}{a^{2}}} \dd{x} \\
	&= \Big[ ab \arcsin \left( \tfrac{x}{a} \right) + bx \sqrt{1 - \tfrac{x^{2}}{a^{2}}} \Big]_{-a}^{a} \\
	&= ab \arcsin(1) + 0 - ab \arcsin(-1) - 0 \\
	&= \boxed{ab \pi}.
\end{align*}


% --------------------------------------------- %

\subsection*{Problem 9}

The area is given by evaluating the following integral:
\begin{align*}
	\iint_{D} x^{3}y \dd{x} \dd{y} &= \int_{-\frac{\sqrt{3}}{2}}^{\frac{\sqrt{3}}{3}} \int_{0}^{-4y^{2} + 3} x^{3}y \dd{x} \dd{y} \\
	&= \int_{-\frac{\sqrt{3}}{2}}^{\frac{\sqrt{3}}{2}} \Big[ \frac{x^{4}y}{4} \Big]_{0}^{-4y^{2} + 3} \dd{y} \\
	&= \int_{-\frac{\sqrt{3}}{2}}^{\frac{\sqrt{3}}{2}} \frac{y(3-4y^{2})^{4}}{4} \dd{y} \\
	&= -\frac{1}{32} \int_{-\frac{\sqrt{3}}{2}}^{\frac{\sqrt{3}}{2}} (-8y)(3 - 4y^{2})^{4} \dd{y} \\
	&= -\frac{1}{32} \Big[ \frac{(3 - 4y^{2})^{5}}{5} \Big]_{-\frac{\sqrt{3}}{2}}^{\frac{\sqrt{3}}{2}} \\
	&= \boxed{0}.
\end{align*}

% --------------------------------------------- %

\subsection*{Problem 15}

Note that for a given $z$, the 2D region bounded by $x^{2} + y^{2} = z$ is a circle with radius $\sqrt{z}$, so it has area $\pi z$. The volume is thus
\[
	\int_{0}^{10} \pi z \dd{z} = \Big[ \frac{\pi z^{2}}{2} \Big]_{0}^{10} = \boxed{50\pi}.
\]

% --------------------------------------------- %

\subsection*{Problem 16}

This is equivalent to the solid bounded by $z = 0$, $z = 10$, and $x^{2} + y^{2} = (10 - z)^{2}$. The integral for this equation is given by
\[
	\boxed{\int_{0}^{10} \pi (10 - z)^{2} \dd{z}} \qquad \text{or} \qquad \boxed{\int_{0}^{10} \int_{z - 10}^{10 - z} 2\sqrt{(10 - z)^{2} - x^{2}} \dd{x} \dd{z}}
\]

% --------------------------------------------- %

\section{Section 5.4}

% --------------------------------------------- %

\subsection*{Problem 4}

\textbf{Part (a)}: We have that
\begin{align*}
	\int_{-1}^{1} \int_{\abs{y}}^{1} (x + y)^{2} \dd{x} \dd{y} &= \int_{0}^{1} \int_{-x}^{x} (x + y)^{2} \dd{y} \dd{x} \\
	&= \int_{0}^{1} \Big[ \frac{(x + y)^{3}}{3} \Big]_{-x}^{x} \dd{x} \\
	&= \int_{0}^{1} \frac{8x^{3}}{3} \dd{x} \\
	&= \Big[ \frac{2x^{4}}{3} \Big]_{0}^{1} \\
	&= \boxed{\frac{2}{3}}.
\end{align*}

\textbf{Part (b)}: We have that
\begin{align*}
	\int_{-3}^{1} \int_{-\sqrt{9-y^{2}}}^{\sqrt{9-y^{2}}} x^{2} \dd{x} \dd{y} &= \int_{-3}^{1} \Big[ \frac{x^{3}}{3} \Big]_{-\sqrt{9-y^{2}}}^{\sqrt{9-y^{2}}} \dd{y} \\
	&= \int_{-3}^{1} \frac{2\sqrt{(9 - y^{2})^{3}}}{3} \dd{y} \\
\end{align*}
which can be simlpified to quite a complex answer involving inverse trigonometric functions. 

\textbf{Part (c)}: We have that
\begin{align*}
	\int_{0}^{4} \int_{\frac{y}{2}}^{2} e^{x^{2}} \dd{x} \dd{y} &= \int_{0}^{2} \int_{0}^{2x} e^{x^{2}} \dd{y} \dd{x} \\
	&= \int_{0}^{2} \Big[ y e^{x^{2}} \Big]_{0}^{2x} \dd{x} \\
	&= \int_{0}^{2} 2x e^{x^{2}} \dd{x} 
	&= \Big[ e^{x^{2}} \Big]_{0}^{2}
	&= \boxed{e^{4} - 1}.
\end{align*}

\textbf{Part (d)}: We have that
\begin{align*}
	\int_{0}^{1} \int_{\arctan(y)}^{\pi/4} \sec^{5}(x) \dd{x} \dd{y} &= \int_{0}^{\pi/4} \int_{0}^{\tan(x)} \sec^{5}(x) \dd{y} \dd{x} \\
	&= \int_{0}^{\pi/4} \sec^{5}(x) \tan(x) \dd{x} \\
	&= \Big[ \frac{\sec^{5}(x)}{5} \Big]_{0}^{\pi/4} \\
	&= \boxed{\frac{\sqrt{2}}{40}}.
\end{align*}

% --------------------------------------------- %

\subsection*{Problem 5}
We have that
\begin{align*}
	\int_{0}^{1} \int_{\sqrt{y}}^{1} e^{x^{3}} \dd{x} \dd{y} &= \int_{0}^{1} \int_{0}^{x^{2}} e^{x^{3}} \dd{y} \dd{x} \\
	&= \int_{0}^{1} x^{2} e^{x^{3}} \dd{x} \\
	&= \Big[ \frac{e^{x^{3}}}{3} \Big]_{0}^{1} \\
	&= \boxed{\frac{e - 1}{3}}
\end{align*}

% --------------------------------------------- %

\subsection*{Problem 9}

Observe that the minimmum and maximum values of $\tfrac{1}{x^{2} + y^{2} + 1}$ on $D$ are $\tfrac{1}{6}$ and $1$. Thus
\[
	1 = \iint_{D} \frac{\dd{x} \dd{y}}{6} \le \iint_{D} \frac{\dd{x} \dd{y}}{x^{2} + y^{2} + 1} \le \iint_{D} \dd{x} \dd{y} = 6.
\]

% --------------------------------------------- %

\subsection*{Problem 10}

Observe that the minimum and maximum values of $\tfrac{1}{y - x + 3}$ on $D$ are $\tfrac{1}{3}$ and $\frac{1}{2}$ respectively. Thus,
\[
	\frac{1}{6} = \iint_{D} \frac{\dd{A}}{3} \le \iint_{D} \frac{\dd{A} }{y - x + 3} \le \iiint_{D} \frac{\dd{A} }{2} \le \frac{1}{4}.
\]

% --------------------------------------------- %

\subsection*{Problem 11}

Observe that an ellipsoid with axes $a$, $b$, and $c$ is a unit sphere under the linear transformation
\[
	\begin{bmatrix} a & 0 & 0 \\ 0 & b & 0 \\ 0 & 0 & c \end{bmatrix}.
\]
The volume of any figure under a linear transformation (or more generally, the Lebesgue measure of a measurable subset of $\mathbb{R}^{n}$) is scaled precisely by the absolute value of determininat of the transformation: the volume we seek is thus
\[
	\frac{4\pi}{3} \begin{vmatrix} a & 0 & 0 \\ 0 & b & 0 \\ 0 & 0 & c \end{vmatrix} = \boxed{\frac{4\pi abc}{3}}
\]
(I am indeed familiar with the Lebesgue measure from Baby Rudin.)

% --------------------------------------------- %

\subsection*{Problem 18}

If we let an antiderivative of $f$ be $F$, then
\begin{align*}
	2\int_{a}^{b} \int_{x}^{b} f(x) f(y) \dd{y} \dd{x} &= 2\int_{a}^{b} f(x) (F(b) - F(x)) \dd{x} \\
	&= F(b) 2\int_{a}^{b} f(x) - 2\int_{a}^{b} f(x) F(x) \\
	&= 2F(b) (F(b) - F(a)) - \Big[ F(x)^{2} \Big]_{a}^{b} \\
	&= 2F(b)^{2} - 2 F(b) F(a) - F(b)^{2} + F(a)^{2} \\
	&= (F(b) - F(a))^{2} \\
	&= \left( \int_{b}^{a} f(x) \dd{x} \right)^{2},
\end{align*}
as desired.

% --------------------------------------------- %

\section{Secction 5.5}

% --------------------------------------------- %

\subsection*{Problem 3}

We have that
\begin{align*}
	\iiint_{B} x^{2} \dd{x} \dd{y} \dd{z} &= \int_{0}^{1} \int_{0}^{1} \int_{0}^{1} x^{2} \dd{x} \dd{y} \dd{z} \\
	&= \int_{0}^{1} \int_{0}^{1} \frac{1}{3} \dd{y} \dd{z} \\
	&= \int_{0}^{1} \frac{1}{3} \dd{z} \\
	&= \boxed{\frac{1}{3}}
\end{align*}

% --------------------------------------------- %

\subsection*{Problem 11}

We must compute the curve where the two regions intersect to find bounds of integration. We have that if
\[
	x^{2} + y^{2} = z = 10 - x^{2} - 2y^{2},
\]
then
\[
	10 = 2x^{2} + 3y^{2} \implies \frac{5}{3} = \left( \frac{x}{\sqrt{3}} \right)^{2} + \left( \frac{y}{\sqrt{2}} \right)^{2}.
\]
Thus, the boundary we seek is an ellipse, which we will denote $E$. The volume we seek is thus given by the integral
\begin{align*}
	\iint_{E} (10 - 2x^{2} - 3y^{2}) \dd{y} \dd{x} &= \int_{-\sqrt{5}}^{\sqrt{5}} \int_{-\frac{\sqrt{30 - 6x^{2}}}{3}}^{\frac{\sqrt{30 - 6x^{2}}}{3}} (10 - 2x^{2} - 3y^{2}) \dd{y} \dd{x} \\
	&= \int_{-\sqrt{5}}^{\sqrt{5}} \int_{-\frac{\sqrt{30 - 6x^{2}}}{3}}^{\frac{\sqrt{30 - 6x^{2}}}{3}} (10 - 2x^{2} - 3y^{2}) \dd{y} \dd{x} \\
	&= \int_{-\sqrt{5}}^{\sqrt{5}} \Big[ 10y - 2x^{2}y - y^{3} \Big]_{-\frac{\sqrt{30 - 6x^{2}}}{3}}^{\frac{\sqrt{30 - 6x^{2}}}{3}} \dd{x} \\
	&= \int_{-\sqrt{5}}^{\sqrt{5}} \Big[ y(10 - 2x^{2} - y^{2}) \Big]_{-\frac{\sqrt{30 - 6x^{2}}}{3}}^{\frac{\sqrt{30 - 6x^{2}}}{3}} \dd{x}.
\end{align*}
Observe that on the boundary, $10 - 2x^{2} = 3y^{2}$, so this evaluates to
\begin{align*}
	\int_{-\sqrt{5}}^{\sqrt{5}} \Big[ y(3y^{2} - y^{2}) \Big]_{-\frac{\sqrt{30 - 6x^{2}}}{3}}^{\frac{\sqrt{30 - 6x^{2}}}{3}} \dd{x} &= \int_{-\sqrt{5}}^{\sqrt{5}} \Big[ 2y^{3} \Big]_{-\frac{\sqrt{30 - 6x^{2}}}{3}}^{\frac{\sqrt{30 - 6x^{2}}}{3}} \dd{x} \\
	&= \int_{-\sqrt{5}}^{\sqrt{5}} 4 \left( \frac{\sqrt{30 - 6x^{2}}}{3} \right)^{3} \dd{x} \\
	&= \boxed{\frac{25\pi\sqrt{6}}{3}}
\end{align*}

% --------------------------------------------- %

\subsection*{Problem 16}

We have that
\begin{align*}
	\int_{0}^{1} \int_{0}^{x} \int_{0}^{y} (y + xz) \dd{z} \dd{y} \dd{x} &= \int_{0}^{1} \int_{0}^{x} \Big[ yz + \frac{xz^{2}}{2} \Big]_{0}^{y} \dd{y} \dd{x} \\
	&= \int_{0}^{1} \int_{0}^{x} \left( y^{2} + \frac{xy^{2}}{2} \right) \dd{y} \dd{x} \\
	&= \int_{0}^{1} \Big[ \frac{y^{3}}{3} + \frac{xy^{3}}{6} \Big]_{0}^{x} \dd{x} \\
	&= \int_{0}^{1} \left( \frac{x^{3}}{3} + \frac{x^{4}}{6} \right) \dd{x} \\
	&= \Big[ \frac{x^{4}}{12} + \frac{x^{5}}{24} \Big]_{0}^{1} \\
	&= \frac{1}{12} + \frac{1}{24} \\
	&= \boxed{\frac{1}{8}}.
\end{align*}

% --------------------------------------------- %

\subsection*{Problem 19}

We have that
\begin{align*}
	\iiint_{W} x^{2} \cos(z) \dd{x} \dd{y} \dd{z} &= \int_{0}^{\pi} \int_{0}^{1} \int_{0}^{1 - y} x^{2} \cos(z) \dd{x} \dd{y} \dd{z} \\
	&= \int_{0}^{\pi} \int_{0}^{1} \Big[ \frac{x^{3}\cos(z)}{3} \Big]_{0}^{1 - y} \dd{y} \dd{z} \\
	&= \int_{0}^{\pi} \int_{0}^{1} \left( \frac{(1 - y)^{3}\cos(z)}{3} \right) \dd{y} \dd{z} \\
	&= \int_{0}^{\pi} \Big[ - \frac{(1 - y)^{4}\cos(z)}{12} \Big]_{0}^{1} \dd{z} \\
	&= \int_{0}^{\pi} \frac{\cos(z)}{12} \dd{z} \\
	&= \Big[ \frac{\sin(z)}{12} \Big]_{0}^{\pi} \\
	&= \boxed{0}.
\end{align*}

% --------------------------------------------- %

\subsection*{Problem 21}

We have that
\begin{align*}
	\iiint_{W} (1 - z^{2}) \dd{x} \dd{y} \dd{z} &= \int_{0}^{1} \int_{0}^{1 - z} \int_{0}^{1 - z} (1 - z^{2}) \dd{x} \dd{y} \dd{z} \\
	&= \int_{0}^{1} (1 - z^{2}) \left( \int_{0}^{1 - z} \int_{0}^{1 - z} 1 \dd{x} \dd{y} \right) \dd{z} \\
	&= \int_{0}^{1} (1 - z^{2}) \left( \int_{0}^{1 - z} (1 - z) \dd{y} \right) \dd{z} \\
	&= \int_{0}^{1} (1 - z^{2}) (1 - z)^{2} \dd{z} \\
	&= \int_{0}^{1} 1 - 2z + 2z^{3} - z^{4} \dd{z} \\
	&= \Big[ z - z^{2} + \frac{z^{4}}{2} - \frac{z^{5}}{5} \Big]_{0}^{1} \\
	&= \boxed{\frac{3}{10}}.
\end{align*}

% --------------------------------------------- %

\subsection*{Problem 28}

The area of $W$ may be represented by a triple integral as follows:
\[
	\boxed{\int_{-1}^{1} \int_{-\sqrt{1 - z^{2}}}^{\sqrt{1 - z^{2}}} \int_{-\sqrt{1 - z^{2} - y^{2}}}^{\sqrt{1 - z^{2} - y^{2}}} 1 \dd{x} \dd{y} \dd{z}}
\]

% --------------------------------------------- %

\end{document}
