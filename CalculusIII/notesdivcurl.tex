\documentclass[11pt]{article}
\usepackage[T1]{fontenc}
\usepackage{geometry, changepage}
\usepackage{amsmath, amssymb, amsthm, bm}
\usepackage{physics}
\usepackage{hyperref}

\hypersetup{colorlinks=true, linkcolor=blue, urlcolor=cyan}
\setlength{\parindent}{0pt}
\setlength{\parskip}{5pt}

\newtheorem{theorem}{Theorem}
\newtheorem{lemma}{Lemma}
\newtheorem{claim}{Claim}
\newtheorem*{theorem*}{Theorem}
\newtheorem*{lemma*}{Lemma}
\newtheorem*{claim*}{Claim}

\renewcommand{\vec}[1]{\mathbf{#1}}
\newcommand{\uvec}[1]{\mathop{} \!\hat{\textbf{#1}}}
\newcommand{\mat}[1]{\mathbf{#1}}
\newcommand{\tensor}[1]{\mathsf{#1}}

\renewcommand{\div}{\nabla \cdot}
\renewcommand{\curl}{\nabla \cross}
\renewcommand{\grad}{\nabla}
\renewcommand{\laplacian}{\nabla^{2}}

\title{MATH-UA 129: Vector-Valued Functions}
\author{Vector-Valued Functions}
\date{James Pagan}

% --------------------------------------------- %

\begin{document}

\maketitle
\tableofcontents

\abstract

The previous document examined mappings $f : \mathbb{R}^{n} \to \mathbb{R}$. Now, we examine how the tools of calculus may be translated to a broader class of functions --- vector-valued functions.

\newpage

% --------------------------------------------- %

\section{Divergence and Curl}

% --------------------------------------------- %

\subsection{Divergence}

For the following lemmas, let $f : \mathbb{R}^{n} \to \mathbb{R}$ and $\mathbf{F}: \mathbb{R}^{n} \to \mathbb{R}^{n}$, where $\mathbf{F} = (f_{1}, \ldots, f_{n})$; define $\mathbf{G}$ similarly. The \textbf{divergence} of $f$ is defined as follows:
\[
	\div \mathbf{F} = \pdv{x_{1}} f_{1} + \cdots + \pdv{x_{n}} f_{n}.
\]

% --------------------------------------------- %

\begin{lemma*}
	$\div (\mathbf{F} + \mathbf{G}) = \div \mathbf{F} + \div \mathbf{G}$.
\end{lemma*}
\begin{adjustwidth}{0.7cm}{}
    \begin{proof}\renewcommand{\qedsymbol}{}
		This is a sum rule for the divergence. We have that
		\begin{align*}
			\div (\mathbf{F} + \mathbf{G}) &= \pdv{x_{1}} (f_{1} + g_{1}) + \cdots + \pdv{x_{n}} (f_{n} + g_{n}) \\
			&= \left( \pdv{x_{1}} f_{1} + \cdots + \pdv{x_{n}} f_{n} \right) + \left( \pdv{x_{1}} g_{1} + \cdots + \pdv{x_{n}} g_{n} \right) \\
			&= \div \mathbf{F} + \div \mathbf{G}.
		\end{align*}
	\end{proof}
\end{adjustwidth}

% --------------------------------------------- %

\begin{lemma*}
	$\div (f \mathbf{F}) = f (\div \mathbf{F}) + \grad f \cdot \mathbf{F}$.
\end{lemma*}
\begin{adjustwidth}{0.7cm}{}
    \begin{proof}\renewcommand{\qedsymbol}{}
		This is a product rule for the divergence. We have that
		\begin{align*}
			\div (f \mathbf{F}) &= \pdv{x_{1}} (f_{1}f) + \cdots + \pdv{x_{n}} (f_{n}f) \\
			&= \left( f \pdv{x_{1}}f_{1} + f_{1} \pdv{x_{1}} f \right) + \cdots + \left( f \pdv{x_{n}} f_{n} + f_{n} \pdv{x_{n}} f \right) \\
			&= f \left( \pdv{x_{1}} f_{1} + \cdots + \pdv{x^{n}} f_{n} \right) + \left( f_{1} \pdv{x_{1}} f + \cdots + f_{n} \pdv{x^{n}} f \right) \\
			&= f (\div \mathbf{F}) + \grad f \cdot \mathbf{F}.
		\end{align*}
	\end{proof}
\end{adjustwidth}

\begin{lemma*}
	$\div (\mathbf{F} \times \mathbf{G}) = \mathbf{G} \cdot (\curl \mathbf{F}) - \mathbf{F} \cdot (\curl \mathbf{G})$
\end{lemma*}
\begin{adjustwidth}{0.7cm}{}
    \begin{proof}\renewcommand{\qedsymbol}{}
		Of course, this presumes that $\mathbf{F}$ and $\mathbf{G}$ are three-dimensional. We have that
		\begin{align*}
			\div (\mathbf{F} \times \mathbf{G}) &= \div \big( (f_{2}g_{3} - f_{3}g_{2}) \uvec{\i} + (f_{3}g_{1} - f_{1}g_{3}) \uvec{\j} + (f_{1}g_{2} - f_{2}g_{1}) \uvec{k} \big) \\
			&= \pdv{x} (f_{2}g_{3} - f_{3}g_{2}) + \pdv{y} (f_{3}g_{1} - f_{1}g_{3}) + \pdv{z} (f_{1}g_{2} - f_{2}g_{1}) \\
			&= (f_{2} g_{3}' + f_{2}'g_{3} - f_{3}g_{2}' - f_{3}'g_{2}) + (f_{3}g_{1}' + f_{3}'g_{1} - f_{1}g_{3}' - f_{1}'g_{3}) \\
			& \quad + (f_{1}g_{2}' + f_{1}'g_{2} - f_{2}g_{1}' - f_{2}'g_{1}) \\
			&= (f_{2}g_{3}' - f_{3}g_{2}') + (f_{3}g_{1}' - f_{1}g_{3}') + (f_{1}g_{2}' - f_{2}g_{1}') \\
			&= \mathbf{G} \cdot (\curl \mathbf{F}) - \mathbf{F} \cdot (\curl \mathbf{G})
		\end{align*}
	\end{proof}
\end{adjustwidth}

% --------------------------------------------- %

\subsection{Curl}


For the following lemmas, let $f : \mathbb{R}^{3} \to \mathbb{R}$ and $\mathbf{F}: \mathbb{R}^{3} \to \mathbb{R}^{3}$, where $\mathbf{F} = (f_{1}, f_{2}, f_{3})$; define $\mathbf{G}$ similarly. The \textbf{curl} of $f$ is defined as follows:
\[
	\curl \mathbf{F} = \left( \pdv{f_{3}}{y} - \pdv{f_{2}}{z} \right) \uvec{\i} + \left( \pdv{f_{1}}{z} - \pdv{f_{3}}{x} \right) \uvec{\j} + \left( \pdv{f_{2}}{x} - \pdv{f_{1}}{y} \right) \uvec{k}
\]

% --------------------------------------------- %

\begin{lemma*}
	$\curl (\mathbf{F} + \mathbf{G}) = \curl \mathbf{F} + \curl \mathbf{G}$ 
\end{lemma*}
\begin{adjustwidth}{0.7cm}{}
    \begin{proof}\renewcommand{\qedsymbol}{}
		This is a sum rule for the curl. We have that
		\begin{align*}
			\curl (\mathbf{F} + \mathbf{G}) =& \left( \pdv{y} (f_{3} + g_{3}) - \pdv{z} (f_{2} + g_{2}) \right) \uvec{\i} \\
			&+ \left( \pdv{z} (f_{1} + g_{1}) + \pdv{x} (f_{3} + g_{3}) \right) \uvec{\j} \\
			&+ \left( \pdv{x} (f_{2} + g_{2}) + \pdv{y} (f_{1} + g_{1}) \right) \\
			=& \left( \pdv{f_{3}}{y} - \pdv{f_{2}}{z} \right) \uvec{\i} + \left( \pdv{f_{1}}{z} - \pdv{f_{3}}{x} \right) \uvec{\j} + \left( \pdv{f_{2}}{x} - \pdv{f_{1}}{y} \right) \\
			&+ \left( \pdv{g_{3}}{y} - \pdv{g_{2}}{z} \right) \uvec{\i} + \left( \pdv{g_{1}}{z} - \pdv{g_{3}}{x} \right) \uvec{\j} + \left( \pdv{g_{2}}{x} - \pdv{g_{1}}{y} \right) \\
			&= \curl \mathbf{F} + \curl \mathbf{G}.
		\end{align*}
	\end{proof} 
\end{adjustwidth}

% --------------------------------------------- %

\begin{lemma*}
	$\curl (f \mathbf{F}) = f \left( \curl \mathbf{F} \right) + \grad f \times \mathbf{F}$
\end{lemma*}
\begin{adjustwidth}{0.7cm}{}
    \begin{proof}\renewcommand{\qedsymbol}{}
		This is a product rule for the curl. We have that
		\begin{align*}
			\curl (f \mathbf{F}) =& \left( \pdv{y} (f_{1} f) - \pdv{z} (f_{2} f) \right) \uvec{\i} \\
			&+ \left( \pdv{z} (f_{1}f) - \pdv{x} (f_{3}f) \right) \uvec{\j} \\
			&+ \left( \pdv{x} (f_{2}f) + \pdv{y} (f_{1}f) \right) \uvec{k} \\
			=& \left( f \pdv{y} f_{3} + f_{3} \pdv{y} f - f \pdv{z} f_{2} - f_{2} \pdv{z} f) \right) \uvec{\i} \\
			&+ \left( f \pdv{z} f_{1} + f_{1} \pdv{z} f - f \pdv{x} f_{3} - f_{3} \pdv{x} f \right) \uvec{\j} \\
			&+ \left( f \pdv{x} f_{2} + f_{2} \pdv{x} f + f \pdv{y} f_{1} - f_{1} \pdv{y} f \right). \\
			=& f \left( \pdv{f_{3}}{y} - \pdv{f_{2}}{z} \right) \uvec{\i} + f \left( \pdv{f_{1}}{z} - \pdv{f_{3}}{x} \right) \uvec{\j} + f \left( \pdv{f_{2}}{x} - \pdv{f_{1}}{y} \right) \\
			&+ \left( f_{3} \pdv{y} f - f_{2} \pdv{z} f \right) \uvec{\i} + \left( f_{1} \pdv{z} f - f_{3} \pdv{x} f \right) \uvec{\j} + \left( f_{2} \pdv{x} f - f_{1} \pdv{y} f \right) \\
			=& f (\curl \mathbf{F}) + \grad f \times \mathbf{F}.
		\end{align*}
	\end{proof} 
\end{adjustwidth}

% --------------------------------------------- %

\section{Relationships between Operators}

\begin{lemma*}
	$\div (\curl \mathbf{F}) = 0$.
\end{lemma*}
\begin{adjustwidth}{0.7cm}{}
    \begin{proof}\renewcommand{\qedsymbol}{}
		We have that		
		\begin{align*}
			\div (\curl \mathbf{F}) &= \div \left(\left( \pdv{f_{3}}{y} - \pdv{f_{2}}{z} \right) \uvec{\i} + \left( \pdv{f_{1}}{z} - \pdv{f_{3}}{x} \right) \uvec{\j} + \left( \pdv{f_{2}}{x} - \pdv{f_{1}}{y} \right) \uvec{k} \right) \\
			&= \pdv{x} \left( \pdv{f_{3}}{y} - \pdv{f_{2}}{z} \right) + \pdv{y} \left( \pdv{f_{1}}{z} - \pdv{f_{3}}{x} \right) \uvec{\j} + \pdv{z} \left( \pdv{f_{2}}{x} - \pdv{f_{1}}{y} \right) \\
			&= \pdv[2]{f_{3}}{x}{y} - \pdv[2]{f_{2}}{z}{x} + \pdv[2]{f^{1}}{y}{z} - \pdv[2]{f_{3}}{x}{y} + \pdv[2]{f_{2}}{z}{x} - \pdv[2]{f_{1}}{y}{z} \\
			&= 0.
		\end{align*}
		Thus, the curl is divergence-free. The converse is true --- if a vector field is divergence-free, it is a curl of some other field.
	\end{proof}
\end{adjustwidth}

\begin{lemma*}
	$\curl (\grad f) = \vec{0}$.
\end{lemma*}
\begin{adjustwidth}{0.7cm}{}
    \begin{proof}\renewcommand{\qedsymbol}{}
		This is a compact way of expressing the equality of mixed partial derivatives. We have that		
		\begin{align*}
			\curl (\grad f) &= \curl \left( \pdv{f_{1}}{x} \uvec{\i} + \pdv{f_{2}}{y} \uvec{\j} + \pdv{f_{3}}{z} \uvec{k} \right) \\
			&= \left( \pdv[2]{f_{3}}{y}{z} - \pdv[2]{f_{3}}{z}{y} \right) \vec{i} + \left( \pdv[2]{f_{1}}{z}{x} - \pdv[2]{f_{1}}{x}{z} \right) \uvec{\j} + \left( \pdv[2]{f_{2}}{x}{y} - \pdv[2]{f_{2}}{y}{x} \right) \uvec{k} \\
			&= 0 \uvec{i} + 0 \uvec{j} + 0 \uvec{k} \\
			&= \vec{0}.
		\end{align*}
		Thus, the gradiant is curl-free. The converse is true --- if a vector field is curl-free, it is the gradiant of some other field.
	\end{proof}
\end{adjustwidth}


\begin{lemma*}
	$\div (\grad f \times \grad g) = 0$.
\end{lemma*}
\begin{adjustwidth}{0.7cm}{}
    \begin{proof}\renewcommand{\qedsymbol}{}
		By the formulas established above,
		\[
			\div (\grad f \times \grad g) = \grad g \cdot (\curl \grad f) - \grad f \cdot (\curl \grad g) = \grad f \cdot \vec{0} - \grad f \cdot \vec{0} = \vec{0}.
		\]
	\end{proof}
\end{adjustwidth}

% --------------------------------------------- %

\end{document}
