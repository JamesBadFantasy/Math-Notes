\documentclass[11pt]{article}
\usepackage[T1]{fontenc}
\usepackage{geometry, changepage}
\usepackage{amsmath, amssymb, amsthm, bm}
\usepackage{physics}
\usepackage{hyperref}

\hypersetup{colorlinks=true, linkcolor=blue, urlcolor=cyan}
\setlength{\parindent}{0pt}
\setlength{\parskip}{5pt}

\newtheorem{theorem}{Theorem}
\newtheorem{lemma}{Lemma}
\newtheorem{claim}{Claim}
\newtheorem*{theorem*}{Theorem}
\newtheorem*{lemma*}{Lemma}
\newtheorem*{claim*}{Claim}

\renewcommand{\vec}[1]{\mathbf{#1}}
\newcommand{\uvec}[1]{\mathop{} \!\hat{\mathbf{#1}}}
\newcommand{\mat}[1]{\mathbf{#1}}
\newcommand{\tensor}[1]{\mathsf{#1}}

\renewcommand{\div}{\nabla \cdot}
\renewcommand{\curl}{\nabla \cross}
\renewcommand{\grad}{\nabla}
\renewcommand{\laplacian}{\nabla^{2}}

\title{MATH-UA 140: Assignment 10}
\author{James Pagan, December 2023}
\date{Professor Raquépas}

% --------------------------------------------- %

\begin{document}

\maketitle
\tableofcontents
\newpage

% --------------------------------------------- %

\section{Problem 1}

% --------------------------------------------- %

\subsection{Part (a)}

We have that
\[
	A^{\top}A = \begin{bmatrix} 1 & -1 \\ 1 & -1 \end{bmatrix} \begin{bmatrix} 1 & 1 \\ -1 & -1 \end{bmatrix} = \begin{bmatrix} 2 & 2 \\ 2 & 2 \end{bmatrix}.
\]
Therefore, all eigenvalues $\lambda$ of $A^{\top}A$ satisfy
\[
	0 = \begin{bmatrix} 2 - \lambda & 2 \\ 2 & 2 - \lambda \end{bmatrix} = (2 - \lambda)^{2} - 4 = \lambda^{2} - 4\lambda = \lambda(\lambda - 4).
\]
Hence, $\lambda_{1} = 0$ and $\lambda_{2} = 4$. It is trivial to verify that
\[
	\boxed{ \begin{bmatrix} \tfrac{\sqrt{2}}{2} \\ -\tfrac{\sqrt{2}}{2} \end{bmatrix} \qquad \text{and} \qquad \begin{bmatrix} \tfrac{\sqrt{2}}{2} \\ \tfrac{\sqrt{2}}{2} \end{bmatrix}}
\]
have norm $1$ and posess these eigenvalues respectively.

% --------------------------------------------- %

\subsection{Part (b)}

Realize that
\begin{align*}
	\begin{bmatrix} -\tfrac{\sqrt{2}}{2} & -\tfrac{\sqrt{2}}{2} \\ \tfrac{\sqrt{2}}{2} & 0 \end{bmatrix} \begin{bmatrix} 4 & 0 \\ 0 & 0 \end{bmatrix} \begin{bmatrix} -\tfrac{\sqrt{2}}{2} & \tfrac{\sqrt{2}}{2} \\ -\tfrac{\sqrt{2}}{2} & 0 \end{bmatrix} &= \begin{bmatrix} -2 \sqrt{2} & 0 \\ 2 \sqrt{2} & 0 \end{bmatrix} \begin{bmatrix} -\tfrac{\sqrt{2}}{2} & \tfrac{\sqrt{2}}{2} \\ -\tfrac{\sqrt{2}}{2} & 0 \end{bmatrix} \\
	&= \begin{bmatrix} 2 & 2 \\ 2 & 2 \end{bmatrix} \\
	&= A^{\top}A.
\end{align*}
Furtheremore, see that
\[
	\begin{bmatrix} -\tfrac{\sqrt{2}}{2} & -\tfrac{\sqrt{2}}{2} \\ \tfrac{\sqrt{2}}{2} & 0 \end{bmatrix} \begin{bmatrix} -\tfrac{\sqrt{2}}{2} & \tfrac{\sqrt{2}}{2} \\ -\tfrac{\sqrt{2}}{2} & 0 \end{bmatrix} = \begin{bmatrix} 1 & 0 \\ 0 & 1 \end{bmatrix} = I,
\]
so 
\[
	\boxed{U \Lambda U^{\top} = \begin{bmatrix} -\tfrac{\sqrt{2}}{2} & -\tfrac{\sqrt{2}}{2} \\ \tfrac{\sqrt{2}}{2} & 0 \end{bmatrix} \begin{bmatrix} 4 & 0 \\ 0 & 0 \end{bmatrix} \begin{bmatrix} -\tfrac{\sqrt{2}}{2} & \tfrac{\sqrt{2}}{2} \\ -\tfrac{\sqrt{2}}{2} & 0 \end{bmatrix}},
\]
where $UU^{\top} = I$ and the entires along the diagonal of $\Lambda$ are the eigenvalues of $A^{\top}A$.

% --------------------------------------------- %

\subsection{Part (c)}

Since $\vec{v}_{1}$ and $\vec{v}_{2}$ are the eigenvalues discussed in Part (a), we have that $(A^{\top}A) \vec{v}_{1} = \vec{0}$ and $(A^{\top}A) \vec{v}_{2} = 4 \vec{v}_{2}$. Thus,
\begin{align*}
	(AA^{\top})(A \vec{v}_{1}) &= A (A^{\top}A \vec{v}_{1}) = A(\vec{0}) = 0(A \vec{v}_{1}) \\
	(AA^{\top})(A \vec{v}_{2}) &= A (A^{\top}A \vec{v}_{2}) = A(4 \vec{v}_{2}) = 4(A \vec{v}_{2}). \\
\end{align*}

% --------------------------------------------- %

\subsection{Part (d)}

The catch is that $A \vec{v}_{1}$ is the zero vector! It cannot be an eigenvector of $AA^{\top}$ and cannot be normalized.

% --------------------------------------------- %

\subsection{Part (e)}

Recalling that $\lambda_{1} = 0$ and $\lambda_{2} = 4$, we have that
\begin{align*}
	A \vec{v}_{1} &= \begin{bmatrix} 1 & 1 \\ -1 & -1 \end{bmatrix} \begin{bmatrix} \tfrac{\sqrt{2}}{2} \\ -\tfrac{\sqrt{2}}{2} \end{bmatrix} = \vec{0} = \sqrt{0} \begin{bmatrix} \tfrac{\sqrt{2}}{2} \\ \tfrac{\sqrt{2}}{2} \end{bmatrix} = \sqrt{\lambda_{1}} \vec{v}_{2}. \\
	A \vec{v}_{2} &= \begin{bmatrix} 1 & 1 \\ -1 & -1 \end{bmatrix} \begin{bmatrix} \tfrac{\sqrt{2}}{2} \\ \tfrac{\sqrt{2}}{2} \end{bmatrix} = \begin{bmatrix} \sqrt{2} \\ -\sqrt{2} \end{bmatrix} = \sqrt{4} \begin{bmatrix} \tfrac{\sqrt{2}}{2} \\ -\tfrac{\sqrt{2}}{2} \end{bmatrix} = \sqrt{\lambda_{2}} \vec{v}_{1}.
\end{align*}
Therefore, two such $\vec{u}_{1}$ and $\vec{u}_{2}$ are $\boxed{\text{$\vec{v}_{2}$ and $\vec{v}_{1}$ respectively}}$.

% --------------------------------------------- %

\subsection{Part (f)}

Realize that
\begin{align*}
	\begin{bmatrix} \tfrac{\sqrt{2}}{2} & \tfrac{\sqrt{2}}{2} \\ \tfrac{\sqrt{2}}{2} & -\tfrac{\sqrt{2}}{2} \end{bmatrix} \begin{bmatrix} 0 & 0 \\ 0 & 2 \end{bmatrix} \begin{bmatrix} -\tfrac{\sqrt{2}}{2} & \tfrac{\sqrt{2}}{2} \\ \tfrac{\sqrt{2}}{2} & \tfrac{\sqrt{2}}{2} \end{bmatrix} &= \begin{bmatrix} 0 & \sqrt{2} \\ 0 & -\sqrt{2} \end{bmatrix} \begin{bmatrix} - \tfrac{\sqrt{2}}{2} & \tfrac{\sqrt{2}}{2} \\ \tfrac{\sqrt{2}}{2} & \tfrac{\sqrt{2}}{2} \end{bmatrix} \\
	&= \begin{bmatrix} 1 & 1 \\ -1 & -1 \end{bmatrix} \\
	&= A.
\end{align*}
It is trivial to verify that the first and third matricies above are orthonormal. Therefore,
\[
	\boxed{U \Sigma V^{\top} = \begin{bmatrix} \tfrac{\sqrt{2}}{2} & \tfrac{\sqrt{2}}{2} \\ \tfrac{\sqrt{2}}{2} & -\tfrac{\sqrt{2}}{2} \end{bmatrix} \begin{bmatrix} 0 & 0 \\ 0 & 2 \end{bmatrix} \begin{bmatrix} -\tfrac{\sqrt{2}}{2} & \tfrac{\sqrt{2}}{2} \\ \tfrac{\sqrt{2}}{2} & \tfrac{\sqrt{2}}{2} \end{bmatrix}}
\]
% --------------------------------------------- %

\end{document}
