\documentclass[11pt]{article}
\usepackage[T1]{fontenc}
\usepackage{geometry, changepage}
\usepackage{amsmath, amssymb, amsthm, bm}
\usepackage{physics}
\usepackage{hyperref}

\hypersetup{colorlinks=true, linkcolor=blue, urlcolor=cyan}
\setlength{\parindent}{0pt}
\setlength{\parskip}{5pt}

\newtheorem{theorem}{Theorem}
\newtheorem{lemma}{Lemma}
\newtheorem{claim}{Claim}
\newtheorem*{theorem*}{Theorem}
\newtheorem*{lemma*}{Lemma}
\newtheorem*{claim*}{Claim}

\renewcommand{\vec}[1]{\mathbf{#1}}
\newcommand{\uvec}[1]{\mathop{} \!\hat{\mathbf{#1}}}
\newcommand{\mat}[1]{\mathbf{#1}}
\newcommand{\tensor}[1]{\mathsf{#1}}

\renewcommand{\div}{\nabla \cdot}
\renewcommand{\curl}{\nabla \cross}
\renewcommand{\grad}{\nabla}
\renewcommand{\laplacian}{\nabla^{2}}

\title{MATH-UA 140: Assignment 4}
\author{James Pagan, October 2023}
\date{Professor Raquépas}

% --------------------------------------------- %

\begin{document}

\maketitle
\tableofcontents

% --------------------------------------------- %

\section{Problem 1}

\textbf{Part (a)}: We have that 
\begin{align*}
	\begin{bmatrix} \cos(\theta) & -\sin(\theta) & 0 \\ \sin(\theta) & \cos(\theta) & 0 \\ 0 & 0 & -1 \end{bmatrix} &\implies \begin{bmatrix} \cos(\theta) & -\sin(\theta) & 0 \\ -\cos(\theta) & -\frac{\cos^{2}(\theta)}{\sin(\theta)} & 0 \\ 0 & 0 & -1 \end{bmatrix} \\
	&\implies  \begin{bmatrix} \cos(\theta) & -\sin(\theta) & 0 \\ 0 & -\frac{\cos^{2}(\theta)}{\sin(\theta)} - \sin(\theta) & 0 \\ 0 & 0 & -1 \end{bmatrix} \\
	&= \begin{bmatrix} \cos(\theta) & -\sin(\theta) & 0 \\ 0 & -\frac{1}{\sin(\theta)} & 0 \\ 0 & 0 & -1 \end{bmatrix}.
\end{align*}
\textbf{Part (b)}: Performing the same actions on $I_{3}$ yields
\[
	\begin{bmatrix} 1 & 0 & 0 \\ 1 & -\frac{\cos(\theta)}{\sin(\theta)} & 0 \\ 0 & 0 & 1 \end{bmatrix}.
\]
\textbf{Part (c)}: Following from our work in Part (a),
\begin{align*}
	\begin{bmatrix} \cos(\theta) & -\sin(\theta) &  0 \\ 0 & -\frac{1}{\sin(\theta)} & 0 \\ 0 & 0 & -1 \end{bmatrix} &\implies \begin{bmatrix} 1 & -\frac{\sin(\theta)}{\cos(\theta)} & 0 \\ 0 & 1 & 0 \\ 0 & 0 & 1 \end{bmatrix} \implies \begin{bmatrix} 1 & 0 & 0 \\ 0 & 1 & 0 \\ 0 & 0 & 1 \end{bmatrix}.
\end{align*}

\textbf{Part (d)}: Performing the same actions on our matrix in Part (a) yields
\begin{align*}
	\begin{bmatrix} 1 & 0 & 0 \\ 1 & -\frac{\cos(\theta)}{\sin(\theta)} & 0 \\ 0 & 0 & 1 \end{bmatrix} &\implies \begin{bmatrix} \frac{1}{\cos(\theta)} & 0 & 0 \\ -\sin(\theta) & \cos(\theta) & 0 \\ 0 & 0 & -1 \end{bmatrix} \\
	&\implies \begin{bmatrix} \frac{1}{\cos(\theta)} - \frac{\sin^{2}(\theta)}{\cos(\theta)} & \sin(\theta) & 0 \\ -\sin(\theta) & \cos(\theta) & 0 \\ 0 & 0 & -1 \end{bmatrix} = \begin{bmatrix} \cos(\theta) & \sin(\theta) & 0 \\ -\sin(\theta) & \cos(\theta) & 0 \\ 0 & 0 & -1 \end{bmatrix}
\end{align*}
\textbf{Part (e)}: Because $R_{0}$ is the identity matrix,
\[
	BA = (E^{-1}EI)A = E^{-1}EA = R_{0} = I.
\]
It is well-known that all left-inverses of matricies are also right-inverses, so $AB = 1$; we conclude that $\boxed{A^{-1} = B}$.

\textbf{Part (f)}: We invoked the fact that when $\theta$ is not an integer multiple of $\tfrac{1}{2} \pi$, then $\cos(\theta)$ and $\sin(\theta)$ are both nonzero, so we may divide by trigonometric functions --- in short, when $\boxed{\text{division by $\cos(\theta)$ or $\sin(\theta)$ occured}}$.

% --------------------------------------------- %

\section{Problem 2}

\textbf{Part (a)}: We have that
\begin{align*}
	\begin{bmatrix} 2 & 0 & 2 \\ -2 & 1 & 4 \\ 4 & -1 & -2 \\ 6 & 0 & 6 \end{bmatrix} \implies \begin{bmatrix}  2 & 0 & 2 \\ 0 & 1 & 6 \\ 0 & -1 & -6 \\ 0 & 0 & 0 \end{bmatrix} &\implies \begin{bmatrix} 2 & 0 & 2 \\ 0 & 1 & 6 \\ 0 & 0 & 0 \\ 0 & 0 & 0 \end{bmatrix} \implies \begin{bmatrix} 1 & 0 & 1 \\ 0 & 1 & 6 \\ 0 & 0 & 0 \\ 0 & 0 & 0 \end{bmatrix}.
\end{align*}
\textbf{Part (b)}: As the first two columns are pivots and the third is free, we deduce that $\boxed{\text{the rank is $2$ and the nullity is $1$}}$.

\textbf{Part (c)}: As noted above, the third column is a $\boxed{\text{free column}}$.

\textbf{Part (d)}: Because the nullity of the matrix is $1$, the null space is the span of only a single vector --- one such vector is trivially
\[
	\begin{bmatrix} 1 \\ 6 \\ -1 \end{bmatrix}.
\]
\textbf{Part (e)}: One such vector is $\boxed{(0, -\sqrt{7}, 0)}$, as may be verified by a trivial calculation.

\textbf{Part (f)}: The general solution to the equation is clearly all vectors of the form
\[
	\begin{bmatrix} 0 \\ -\sqrt{7} \\ 0 \end{bmatrix} + \lambda \begin{bmatrix} 1 \\ 6 \\ -1 \end{bmatrix} 
\]
for $\lambda \in \mathbb{R}$.

% --------------------------------------------- %

\end{document}
