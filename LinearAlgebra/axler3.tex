\documentclass[11pt]{article}
\usepackage[T1]{fontenc}
\usepackage{geometry, changepage, hyperref}
\usepackage{amsmath, amssymb, amsthm, bm}
\usepackage{physics, esint}
\usepackage{tgpagella, eulervm}

\hypersetup{colorlinks=true, linkcolor=blue, urlcolor=cyan}
\setlength{\parindent}{0pt}
\setlength{\parskip}{5pt}

\newtheorem{theorem}{Theorem}
\newtheorem{lemma}{Lemma}
\newtheorem{proposition}{Proposition}
\newtheorem{corollary}{Corollary}
\newtheorem{claim}{Claim}

\renewcommand{\vec}[1]{\mathbf{#1}}
\newcommand{\uvec}[1]{\mathop{} \!\hat{\textbf{#1}}}
\newcommand{\mat}[1]{\mathbf{#1}}
\newcommand{\tensor}[1]{\mathsf{#1}}
\newcommand{\spn}{\operatorname{span}}
\newcommand{\nll}{\operatorname{null}}
\newcommand{\range}{\operatorname{range}}

\renewcommand{\grad}{\nabla}
\renewcommand{\div}{\nabla \cdot}
\renewcommand{\curl}{\nabla \cross}

\title{Axler: Linear Maps}
\author{James Pagán}
\date{March 2024}

% --------------------------------------------- %

\begin{document}

\maketitle
\tableofcontents
\newpage

% --------------------------------------------- %

\section{Vector Space of Linear Maps}

% --------------------------------------------- %

\subsection{Definition}

Let $V$ and $W$ be $F$-vector spaces. An $F$-module homomorphism is an \textbf{F-linear map} --- a function $\mat{T} : V \to W$ such that for all $\vec{u}, \vec{v} \in V$ and $\lambda \in F$,
\begin{align*}
  \mat{T}(\vec{u} + \vec{v}) \, &= \, \mat{T} \vec{u} + \mat{T} \vec{v} \\
    \mat{T}(\lambda \vec{v}) \, &= \, \lambda \mat{T} \vec{v}.
\end{align*}
By the properties of module homomorphisms, $\mat{T}$ maps $\vec{0}$ to $\vec{0}$. Note that the indices of the following theorem depend on the Axiom of Choice:

\begin{adjustwidth}{0.5cm}{}
  \begin{lemma}[Linear Map Lemma]
    Suppose $(\vec{v}_{\alpha})$ is a basis of $V$ and $(\vec{w}_{\alpha})$ is a basis of $W$. Then there exists a unique linear map $\mat{T} : V \to W$ such that $\vec{v}_{\alpha} \leadsto \vec{w}_{\alpha}$.
  \end{lemma}
  \begin{proof}
    Define $\mat{T}$ as such a linear map: for all $\vec{u} \in V$, express $\vec{u} = \lambda_{1} \vec{v}_{\alpha_{1}} + \cdots + \lambda_{n} \vec{v}_{\alpha_{n}}$, and define
    \[
      \mat{T} \vec{u} \quad \stackrel{\text{def}}{=} \quad \lambda_{1} \vec{w}_{\alpha_{1}} + \cdots + \lambda_{n} \vec{w}_{\alpha_{n}}.
    \]
    $\mat{T}$ is well-defined, since representation via a basis is unique. A rather tedious argument demonstrates that it satisfies the additive and multiplicative conditions. Unicity follows from the fact that the valuation of every point is predetermined since $(\vec{v}_{\alpha})$ and $(\vec{w}_{\alpha})$ is a basis. In fact, $\mat{T}$ is an isomorphism!
  \end{proof}
\end{adjustwidth}

% --------------------------------------------- %

\subsection{Algebraic Operations on \texorpdfstring{$\mathcal{L}(V, W)$}{L(V, W)}}

The notation $\mathcal{L}(V, W)$ denotes the set of all linear maps from $V$ to $W$. Like with modules, we can define three operations on this set: if $\mat{T}, \mat{S} \in \mathcal{L}(V, W)$, then for all $\vec{v} \in V$ and $\lambda \in F$:
\begin{enumerate}
  \item \textbf{Addition}: $\mat{T} + \mat{S}$ is the unique linear map such that $(\mat{T} + \mat{S}) \vec{v} = \mat{T} \vec{v} + \mat{S} \vec{v}$.
  \item \textbf{Scalar Multiplication}: $\lambda \mat{T}$ is the unique linear map such that $(\lambda \mat{T}) \vec{v} = \lambda (\mat{T} \vec{v})$.
\end{enumerate}
Equipped with these operations, $\mathcal{L}(V, W)$ is a vector space over $F$. There is a third operation: if $\mat{T} \in \mathcal{L}(V, W)$ and $\mat{S} \in \mathcal{L}(W, U)$, then for all $\vec{v} \in V$ and $\lambda \in F$:
\begin{enumerate}
  \item \textbf{Multiplication}: $\mat{ST}$ is the unique linear map in $\mathcal{L}(V, U)$ such that $(\mat{ST}) \vec{v} \, = \, \mat{S} (\mat{T} \vec{v})$.
\end{enumerate}
Hence multiplication is simply a composition of mappings.

% --------------------------------------------- %

\section{Null Spaces and Ranges}

The kernel and image of an $F$-module homomorphism $\mat{T} \in \mathcal{L}(V, W)$ are given special names when $F$ is a field: the null space and range. For all such $\mat{T}$, we define:
\begin{enumerate}
  \item \textbf{Null Space}: $\nll \mat{T} = \{ \vec{v} \in V \, \mid \, \mat{T} \vec{v} = \vec{0} \}$.
  \item \textbf{Range}: $\range \mat{T} = \{ \mat{T} \vec{v} \, \mid \, \vec{v} \in V \}$.
\end{enumerate}
As with modules, $\nll \mat{T}$ is a subspace of $V$ and $\range \mat{R}$ is a subspace of $W$. Furthermore, $\mat{T}$ is injective if and only if $\nll \mat{T} = \vec{0}$ and surjective if and only if $\range \mat{T} = W$. A bijective linear map is called an \textbf{invertible linear map} or \textbf{isomorphism}.

\begin{adjustwidth}{0.5cm}{}
  \begin{theorem}[Fundamental Theorem of Linear Maps]
    Let $V$ be finite-dimensional and suppose $\mat{T} \in \mathcal{L}(V, W)$. Then $\range \mat{T}$ is a finite-dimensional subspace, and
    \[
      \dim V \, = \, \dim \nll \mat{T} + \dim \range \mat{T}.
    \]
  \end{theorem}
  \begin{proof}
    Let $\vec{u}_{1}, \ldots, \vec{u}_{m}$ be a basis of $\nll \mat{T}$; extend it to a basis of $V$, namely
    \[
      \vec{u}_{1}, \ldots, \vec{u}_{m}, \vec{v}_{1}, \ldots, \vec{v}_{n}.
    \]
    The proof is finished if we prove that $\mat{T} \vec{v}_{1}, \ldots, \mat{T} \vec{v}_{n}$ is a basis of $\range \mat{T}$.
    \begin{enumerate}
      \item \textbf{Spanning}: Select $\vec{w} \in V$ arbitrarily; there exist constants $\lambda_{i}$ and $\mu_{i}$ such that
      \[
        \vec{w} \, = \, \lambda_{1} \vec{u}_{1} + \cdots + \lambda_{m} \vec{u}_{n} + \mu_{1} \vec{v}_{1} + \cdots + \mu_{n} \vec{u}_{n}.
      \]
      Applying $\mat{T}$ to this equation, we find $\mat{T} \vec{w} = \mu_{1} \mat{T} \vec{v}_{1} + \cdots + \mu_{n} \mat{T} \vec{v}_{n}$. We conclude that $\range \mat{T} = \spn(\mat{T} \vec{v}_{1}, \ldots, \mat{T} \vec{v}_{n})$.
      \item \textbf{Independence}: Follows from the fact that $\vec{u}_{1}, \ldots, \vec{u}_{m}, \vec{v}_{1}, \ldots, \vec{v}_{n}$ is a basis of $V$.
    \end{enumerate}
    This establishes the desired basis: hence $\range \mat{T}$ is finite-dimensional, and $\dim V = m + n = \dim \nll \mat{T} + \dim \range \mat{T}$.
  \end{proof}
\end{adjustwidth}

The cardinals $\dim \nll \mat{T}$ and $\dim \range \mat{T}$ are called the \textbf{nullity} and \textbf{rank} of $\mat{T}$ respectively.

\begin{adjustwidth}{0.5cm}{}
  \begin{proposition}
    Let $V$ and $W$ be a finite-dimensional vector spaces.
    \begin{enumerate}
      \item If $\dim V < \dim W$, then no linear map $\mat{T} \in \mathcal{L}(V, W)$ can be surjective.
      \item If $\dim V > \dim W$, then no linear map $\mat{T} \in \mathcal{L}(V, W)$ can be injective.
    \end{enumerate}
  \end{proposition}
  \begin{proof}
    For (1), we have that
    \[
      \dim \range \mat{T} \, \le \, \dim \range \mat{T} + \dim \nll \mat{T} \, = \, \dim V \, < \, \dim W,
    \]
    so $\range \mat{T} \ne W$; hence $\mat{T}$ is not surjective. Similarly for (2), we have
    \[
      \dim \nll \mat{T} \, = \, \dim V - \dim \range \mat{T} \, \ge \, \dim V - \dim W \, > \, 0,
    \]
    so $\nll \mat{T} \ne 0$ and $\mat{T}$ is not injective.
  \end{proof}
\end{adjustwidth}

I do not know whether this result generalizes to infinite-dimensional vector spaces. It requires delicate cardinal arithmetic. However, it does imply this: if $V$ and $W$ are finite dimensional, the existence of invertible $\mat{T} \in \mathcal{L}(V, W)$ implies that $\dim V = \dim W$

% --------------------------------------------- %

\section{Matrices}

% --------------------------------------------- %

\end{document}
