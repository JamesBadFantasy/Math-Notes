\documentclass[11pt]{article}
\usepackage[T1]{fontenc}
\usepackage{geometry, changepage}
\usepackage{amsmath, amssymb, amsthm, bm}
\usepackage{physics}
\usepackage{hyperref}

\hypersetup{colorlinks=true, linkcolor=blue, urlcolor=cyan}
\setlength{\parindent}{0pt}
\setlength{\parskip}{5pt}

\newtheorem{theorem}{Theorem}
\newtheorem{lemma}{Lemma}
\newtheorem{claim}{Claim}
\newtheorem*{theorem*}{Theorem}
\newtheorem*{lemma*}{Lemma}
\newtheorem*{claim*}{Claim}

\renewcommand{\vec}[1]{\mathbf{#1}}
\newcommand{\uvec}[1]{\mathop{} \!\hat{\mathbf{#1}}}
\newcommand{\mat}[1]{\mathbf{#1}}
\newcommand{\tensor}[1]{\mathsf{#1}}

\renewcommand{\div}{\nabla \cdot}
\renewcommand{\curl}{\nabla \cross}
\renewcommand{\grad}{\nabla}
\renewcommand{\laplacian}{\nabla^{2}}

\title{Math-UA 148: Homework 2}
\author{James Pagan, Sep 2023}
\date{Professor Weare}

% --------------------------------------------- %

\begin{document}

\maketitle
\tableofcontents

% --------------------------------------------- %

\section{2A Problems}

% --------------------------------------------- %

\subsection{Problem 9}

The given result is \textbf{false}. In $\mathbb{R}^{2}$, see that the two lists $(1, 0), (0, 1)$ and $(1, 0), (0, -1)$ are both independent. Yet their sum of $(1, 0) + (1, 0), (0, 1) + (0, -1)$ or $(2, 0), (0, 0)$ is not an independent list, as $0(2, 0) + 5(0, 0) = \vec{0}$. 

% --------------------------------------------- %

\subsection{Problem 10}

As $\vec{v}_{1} + \vec{w}, \ldots, \vec{v}_{m} + \vec{w}$ is linearly dependent, there exist $\lambda_{1}, \ldots, \lambda_{m} \in \mathbb{F}$, not all zero, such that
\[
	\lambda_{1} (\vec{v}_{1} + \vec{w}) + \cdots + \lambda_{m} (\vec{v}_{m} + \vec{w}) = \vec{0}.
\]
This can be rearranged to 
\[
	\lambda_{1} \vec{v}_{1} + \cdots + \lambda_{m} \vec{v}_{m} = -(\lambda_{1} + \cdots + \lambda_{m}) \vec{w}.
\]
Now, suppose that $\lambda_{1} + \cdots + \lambda_{m} = 0$. Then the above equation rearranges to
\[
	\lambda_{1} \vec{v}_{1} + \cdots + \lambda_{m} \vec{v}_{m} = -(\lambda_{1} + \cdots + \lambda_{m}) \vec{w} = 0 \vec{w} = \vec{0},
\]
which implies that the list $\vec{v}_{1}, \ldots, \vec{v}_{m}$ is linearly dependent --- a contradiction. Then $\lambda_{1} + \cdots + \lambda_{m}$ must be nonzero. This allows us to divide both sides of the above equation by $-(\lambda_{1} + \cdots + \lambda_{m})$, which yields that
\[
	- \frac{\lambda_{1} \vec{v}_{1} + \cdots + \lambda_{m} \vec{v}_{m}}{\lambda_{1} + \cdots + \lambda_{m}} = \vec{w}.
\]
Hence, $\vec{w} \in \operatorname{span} (\vec{v}_{1}, \ldots, \vec{v}_{m})$.

% --------------------------------------------- %

\subsection{Problem 12}

If a list of polynomials in $\mathcal{P}_{4}(\mathbb{F})$ is linearly independent, then the length of the list is less than or equal to $5$ --- the dimension of $\mathcal{P}_{4}(\mathbb{F})$. By contraposition, a list of $6$ polynomials in $\mathcal{P}_{4}(\mathbb{F})$ cannot be linearly independent. 

% --------------------------------------------- %

\section{2B Problems}

% --------------------------------------------- %

\subsection{Problem 7}

Consider the four polynomials $1, x + 1, x^{2}, x^{3} \in \mathcal{P}_{3}(\mathbb{R})$, and define 
\[ 
	W = \{ {ax^{3} + bx^{2} + cx \mid a, b, c \in \mathbb{R}} \}.
\]
Clearly the four polynomials are a basis of $\mathcal{P}_{3}(\mathbb{R})$ and $W$ is a subspace of $\mathcal{P}_{3}(\mathbb{R})$. Observe that $x^{2}, x^{3} \in W$ and $1, x + 1 \notin W$ --- however, $x^{2}$ and $x^{3}$ do not constitute a basis of $W$, as no linear combination of the two generates the one-degree polynomials of $W$.


% --------------------------------------------- %

\subsection{Problem 8}

Observe that for all $\vec{v} \in V$, there exist $\vec{u} \in U$ and $\vec{w} \in W$ such that $\vec{v} = \vec{u} + \vec{w}$. Now, define $\lambda_{1}, \ldots, \lambda_{n + m} \in \mathbb{F}$ such that
\begin{align*}
	\vec{u} &= \lambda_{1} \vec{u}_{1} + \cdots + \lambda_{n} \vec{u}_{j} \\
	\vec{w} &= \lambda_{n + 1} \vec{w}_{1} + \cdots + \lambda_{n + m} \vec{w}_{m}.
\end{align*}
We find that
\[
	\vec{v} = \vec{u} + \vec{w} = \lambda_{1} \vec{u}_{1} + \cdots + \lambda_{n} \vec{u}_{n} + \lambda_{n + 1} \vec{w}_{1} + \cdots + \lambda_{n + m} \vec{w}_{m}.
\]
Therefore, $\vec{u}_{1}, \ldots, \vec{u}_{n}, \vec{w}_{1}, \ldots, \vec{w}_{m}$ spans $V$. Now, observe that if there is a nontrivial solution to
\[
	\lambda_{1} \vec{u}_{1} + \cdots + \lambda_{n} \vec{u}_{n} + \lambda_{n + 1} \vec{w}_{1} + \cdots + \lambda_{n + m} \vec{w}_{m} = \vec{0},
\]
we may rearrange this to 
\[
	\lambda_{1} \vec{u}_{1} + \cdots + \lambda_{n} \vec{u}_{n} = - \lambda_{n + 1} \vec{w}_{1}	- \cdots - \lambda_{n + m} \vec{w}_{m}.
\]
As $\vec{u}_{1}, \ldots, \vec{u}_{n}$ is a basis of $U$ --- and as $\lambda_{1}, \ldots, \lambda_{n}$ are not all equal to zero --- both sides of this equation are nonzero. Note that the left-hand side is in $U$ and the right-hand side is in $W$; thus, their sum is a nonzero vector in $U$ and $W$, so $U \cap W \ne \{ \vec{0} \}$. This contradicts the fact $U + W$ is a direct sum. We conclude that there is no nontrivial solution to 
\[
	\lambda_{1} \vec{u}_{1} + \cdots + \lambda_{n} \vec{u}_{n} + \lambda_{n + 1} \vec{w}_{1} + \cdots + \lambda_{n + m} \vec{w}_{m} = \vec{0},
\]
so $\lambda_{1} \vec{u}_{1}, \ldots, \lambda_{n} \vec{u}_{n}, \lambda_{n+1} \vec{w}_{1}, \ldots, \lambda_{n + m} \vec{w}_{m}$ is a linearly independent list. Therefore, the list is a basis of $V$.

% --------------------------------------------- %

\section{2C Problems}

% --------------------------------------------- %

\subsection{Problem 6}


Consider the four polynomials: $1$, $(x-2)(x-5)$, $(x-2)(x-5)(x)$, and $(x-2)(x-5)(x^{2})$. All four polynomials have different degrees, so they are linearly independent in $U$. 

Consider an arbitrary $p \in U$; $p$ has degree of four or less. Define $\lambda$ such that $p(2) = p(5) = \lambda$; then $p(2) - \lambda = p(5) - \lambda = 0$. The Factor Theorem thus guarantees that $p(x) - \lambda = (x-2)(x-5)(\alpha x^{2} + \beta x + \gamma)$ for some $\alpha, \beta, \gamma \in \mathbb{F}$. Then
\begin{align*}
	p(x) &= \lambda + (x-2)(x-5)(\alpha x^{2} + \beta x + \gamma) \\ 
	&= \lambda + \alpha(x-2)(x-5)(x^{2}) + \beta(x-2)(x-5)(x) + \gamma(x-2)(x-5)
\end{align*}
We conclude that these four polynomials span $U$, and are thus a basis of $U$.

\textbf{(b)} Extend the basis of $U$ with the polynomial $x$. Because all five polynomials have different degrees, they are linearly independent --- and because $\mathcal{P}_{4}(\mathbb{F})$ has dimension five, our five polynomials must be a basis of $\mathcal{P}_{4}(\mathbb{F})$.

\textbf{(c)} Consider the subspace $W = \{ \lambda x \mid \lambda \in \mathbb{F} \}$. The polynomial $x$ trivially spans $W$. Now, consider if $p \in W \cap U$; then $p = \lambda x$ for some $\lambda \in \mathbb{F}$ and $\lambda 2 = \lambda 5$. We deduce that $\lambda = 0$. Then $U \cap W = \{ \vec{0} \}$, and $U + V$ is a direct sum.

Observe that list $x$ spans $W$ and $1$, $(x-2)(x-5)$, $(x-2)(x-5)(x)$, $(x-2)(x-5)(x^{2})$ spam $U$; by the result of Section 2B Problem 8, their union is a basis of $U \oplus W$. This is the same basis of $\mathcal{P}_{4}(\mathbb{F})$ --- then $U \oplus W = \mathcal{P}_{4}(\mathbb{F})$, as desired.

% --------------------------------------------- %

\subsection{Problem 12}

Suppose for contradiction that $U \cap W = \{ \vec{0} \}$. Clearly, $U + W$ is thus a direct sum, and $U \oplus W$ is a subspace of $V$.

Let $\vec{u}_{1}, \ldots, \vec{u}_{5}$ and $\vec{w}_{1}, \ldots, \vec{w}_{5}$ be a basis of $W$. Via the result of Section 2B Problem 8, $\vec{u}_{1}, \ldots, \vec{u}_{5}, \vec{w}_{1}, \ldots, \vec{w}_{5}$ is a basis of $U \oplus W$. Then $U \oplus W$ has dimension $10$.

This contradicts the fact that no subspace of $V$ has a larger dimension than $V$. We conclude that $U \cap W \ne \{ \vec{0} \}$.

% --------------------------------------------- %

\subsection{Problem 16}

We proceed via induction.

\textbf{Base case}: Let $U_{1}$ and $U_{2}$ be subspaces of $V$ such that $U_{1} + U_{2}$ is a direct sum. We define the dimensions of $U_{1}$ and $U_{2}$ as $n$ and $m$ respectively and the bases of $U_{1}$ and $U_{2}$ as $\vec{v}_{1}, \ldots, \vec{v}_{n}$ and $\vec{w}_{1}, \ldots, \vec{w}_{m}$.

Via the result of Section 2B Problem 8, $\vec{v}_{1}, \ldots, \vec{v}_{n}, \vec{w}_{1}, \ldots, \vec{w}_{m}$ is a basis of $U \oplus W$. We conclude that $\dim U_{1} \oplus U_{2} = n + m = \dim U_{1} + \dim U_{2}$.

\textbf{Induction step}: Asssume that for all sets of $m$ subspaces $U_{1}, \ldots, U_{m}$ of $V$ such that $U_{1} + \cdots + U_{k}$ is a direct sum, we have that $\dim U_{1} \oplus \cdots \oplus U_{m} = \dim U_{1} + \cdots + \dim U_{m}$.

Let $U_{m + 1}$ be a subspace of $W$ such that $U_{1} + \cdots + U_{m + 1}$ is a direct sum. Then by our base case,
\[
	\dim U_{1} \oplus \cdots \oplus U_{m + 1} = \dim (U_{1} \oplus \cdots \oplus U_{m}) + \dim U_{m + 1} = \dim U_{1} \oplus \cdots \oplus \dim U_{m + 1}.
\]
This completes the induction.

% --------------------------------------------- %

\end{document}
