\documentclass[11pt]{article}
\usepackage[T1]{fontenc}
\usepackage{geometry, changepage}
\usepackage{amsmath, amssymb, amsthm, bm}
\usepackage{physics}
\usepackage{hyperref}

\hypersetup{colorlinks=true, linkcolor=blue, urlcolor=cyan}
\setlength{\parindent}{0pt}
\setlength{\parskip}{5pt}

\newtheorem{theorem}{Theorem}
\newtheorem{lemma}{Lemma}
\newtheorem{claim}{Claim}
\newtheorem*{theorem*}{Theorem}
\newtheorem*{lemma*}{Lemma}
\newtheorem*{claim*}{Claim}

\renewcommand{\vec}[1]{\mathbf{#1}}
\newcommand{\uvec}[1]{\mathop{} \!\hat{\mathbf{#1}}}
\newcommand{\mat}[1]{\mathbf{#1}}
\newcommand{\tensor}[1]{\mathsf{#1}}

\renewcommand{\div}{\nabla \cdot}
\renewcommand{\curl}{\nabla \cross}
\renewcommand{\grad}{\nabla}
\renewcommand{\laplacian}{\nabla^{2}}

\title{MATH-UA 148: Homework 3}
\author{James Pagan, October 2023}
\date{Professor Weare}

% --------------------------------------------- %

\begin{document}

\maketitle
\tableofcontents
\newpage

% --------------------------------------------- %

\section{Section 3A}

% --------------------------------------------- %

\subsection{Problem 4}

We prove the contrapositive --- suppose that $\vec{v}_{1}, \ldots, \vec{v}_{n}$ is linearly dependent. Then there exist scalars $\lambda_{1}, \ldots, \lambda_{n} \in \mathbb{F}$ such that 
\[
	\lambda_{1} \vec{v}_{1} + \cdots + \lambda_{n} vec{v}_{n} = \vec{0}.
\]
Therefore,
\[
	\lambda_{1} (T \vec{v}_{1})	+ \cdots + \lambda_{n} (T \vec{v}_{n}) = T (\lambda_{1} \vec{v}_{1}) + \cdots + T (\lambda_{n} \vec{v}_{n}) = T (\lambda_{1} \vec{v}_{1} + \cdots + \lambda_{n} \vec{v}_{n}) = T (\vec{0}) = \vec{0},
\]
so $T \vec{v}_{1}, \ldots, T \vec{v}_{n}$ is linearly dependent. Taking the contrapositive yields the desired result.

% --------------------------------------------- %

\subsection{Problem 10}

We define two vectors: $\vec{v}$, such that $\vec{v} \in V$ and $\vec{v} \notin U$ and $\vec{u}$, such that $\vec{u} \in U$ and $S \vec{u} \ne \vec{0}$. Clearly $T \vec{v} = \vec{0}$ --- and as $\vec{v} + \vec{u} \notin U$, we have that $T(\vec{v} + \vec{u}) = \vec{0}$, so
\[
	T(\vec{v} + \vec{u}) = 0 \ne T \vec{u} = T \vec{u} + T \vec{v}.
\]
We conclude that $T$ is not a linear map on $V$.

% --------------------------------------------- %

\subsection{Problem 14}

Let the basis of $V$ be $\vec{v}_{1}$ and $\vec{v}_{2}$. We thus define two linear maps:
\begin{itemize}
	\item $T$, such that $T \vec{v}_{1} = \vec{v}_{2}$ and $T \vec{v}_{2} = \vec{v}_{1}$.
	\item $S$, such that $T \vec{v}_{1} = \vec{v}_{1}$ and $T \vec{v}_{2} = - \vec{v}_{2}$.
\end{itemize}
Clearly $\vec{v}_{1}, \vec{v}_{2}$ are nonzero, and $T$ and $S$ are linear maps; thus,
\begin{align*}
	TS \vec{v}_{2} = T(-\vec{v}_{2}) = -T(\vec{v}_{2}) = - \vec{v}_{1} \ne \vec{v}_{1} = S \vec{v}_{1} = ST \vec{v}_{2}.
\end{align*}
Thus, $TS \ne ST$.

% --------------------------------------------- %

\section{Section 3B}

% --------------------------------------------- %

\subsection{Problem 18}

\textbf{Surjectivity implies Dimension}: Suppose that there exists a surjective linear map $T : V \to W$. Then $\operatorname{null} T = \{  \vec{0} \}$, so $\dim \operatorname{null} T = 0$; therefore,
\[
	\dim V = \dim \operatorname{null} T + \dim \operatorname{range} T = \dim \operatorname{range} T.
\]
As $\operatorname{range} T$ is a subspace of $W$, we have that $\dim V = \dim \operatorname{range T} \le \dim W$.

\textbf{Dimension implies Surjectivity}: Suppose that $\dim V \le \dim W$; let $\vec{v}_{1}, \ldots, \vec{v}_{n}$ and $\vec{w}_{1}, \ldots, \vec{w}_{n}$ be bases of $V$ and $W$, where $n \le m$.

Define $T \in \mathcal{L}(\mathbb{R}^{n}, \mathbb{R}^{m})$ as the unique linear map such that $T \vec{v}_{j} = \vec{w}_{j}$ for each $j \in \{ 1, \ldots, n \}$. Suppose $\vec{v}$ and $\vec{u}$ are vectors in $V$ such that $T \vec{v} = T \vec{u}$; let 
\begin{align*}
	\vec{v} = \lambda_{1} \vec{v}_{1} + \cdots + \lambda_{n} \vec{v}_{n} \\
	\vec{w} = \mu_{1} \vec{v}_{1} + \cdots + \mu_{n} \vec{v}_{n}.
\end{align*}
We thus dedice that
\begin{align*}
	(\lambda_{1} - \nu_{1}) \vec{w}_{1} + \cdots + (\lambda_{n} - \nu_{n}) \vec{w}_{n} &= (\lambda_{1} \vec{w}_{1} + \cdots + \lambda_{n} \vec{w}_{n}) - (\mu_{1} \vec{w}_{1} + \mu_{n} \vec{w}_{n}) \\
	&= (T \lambda_{1} \vec{v}_{1} + \cdots + T \lambda_{n} \vec{v}_{n}) - (T \mu_{1} \vec{v}_{1} + \cdots + T \mu_{n} \vec{v}_{n}) \\
	&= T(\lambda_{1} \vec{v}_{1} + \cdots + \lambda_{n} \vec{v}_{n}) - T(\mu_{1} \vec{v}_{1} + \cdots + \mu_{n} \vec{v}_{n}) \\
	&= T\vec{v} - T \vec{u} \\
	&= T(\vec{v} - \vec{u}) \\
	&= T \vec{0} \\
	&= \vec{0}
\end{align*}
As $\vec{w}_{1}, \ldots, \vec{w}_{n}$ are linearly independent, $\lambda_{j} = \mu_{j}$ for all $j \in \{ 1, \ldots, n \}$. This implies that $\vec{v} = \vec{u}$, so $T$ is surjective.

% --------------------------------------------- %

\subsection{Problem 22}

By the Fundamental Theorem of Linear Maps,
\begin{align*}
	\dim \operatorname{null} ST &= \dim U - \dim \operatorname{range} ST \\
	&= \dim \operatorname{null} T + \dim \operatorname{range} T - \dim \operatorname{range} ST \\
	&\le \dim \operatorname{null} T + \dim V - \dim \operatorname{range} ST \\
	&= \dim \operatorname{null} T + \dim \operatorname{null} S + \dim \operatorname{range} S - \dim \operatorname{range} ST \\
	&\le \dim \operatorname{null} S + \dim \operatorname{null} T
\end{align*}

% --------------------------------------------- %

\subsection{Problem 28}



% --------------------------------------------- %

\section{Section 3C}

% --------------------------------------------- %

\subsection{Problem 6}

% --------------------------------------------- %

\subsection{Problem 14}

% --------------------------------------------- %

\end{document}
