\documentclass[11pt]{article}
\usepackage[T1]{fontenc}
\usepackage{geometry, changepage, hyperref}
\usepackage{amsmath, amssymb, amsthm, bm}
\usepackage{physics, esint}

\hypersetup{colorlinks=true, linkcolor=blue, urlcolor=cyan}
\setlength{\parindent}{0pt}
\setlength{\parskip}{5pt}

\renewcommand{\vec}[1]{\mathbf{#1}}
\newcommand{\uvec}[1]{\mathop{} \!\hat{\textbf{#1}}}
\newcommand{\mat}[1]{\mathbf{#1}}
\newcommand{\nll}{\operatorname{null}}
\newcommand{\range}{\operatorname{range}}

\newcommand{\conjugate}[1]{\overline{#1}}

\newtheorem{theorem}{Theorem}
\newtheorem{lemma}{Lemma}
\newtheorem{claim}{Claim}
\newtheorem*{theorem*}{Theorem}
\newtheorem*{lemma*}{Lemma}
\newtheorem*{claim*}{Claim}

\title{MATH-UA 148: Homework 6}
\author{James Pagan, January 2024}
\date{Professor Weare}

% --------------------------------------------- %

\begin{document}

\maketitle
\tableofcontents
\newpage

% --------------------------------------------- %

\section{Section 6A}

% --------------------------------------------- %

\subsection{Problem 4}

\textbf{Part (a)}: As $V$ is a real vector space, $\ev{\vec{u}, \vec{v}} = \ev{\vec{v}, \vec{u}}$. Then
\begin{align*}
	\ev{\vec{u} + \vec{v}, \vec{u} - \vec{v}} &= \ev{\vec{u}, \vec{u} - \vec{v}} + \ev{\vec{v}, \vec{u} - \vec{v}} \\
	&= \ev{\vec{u}, \vec{u}} - \ev{\vec{u}, \vec{v}} + \ev{\vec{v}, \vec{u}} - \ev{\vec{v}, \vec{v}} \\
	&= \ev{\vec{u}, \vec{u}} - \ev{\vec{u}, \vec{v}} + \ev{\vec{u}, \vec{v}} - \ev{\vec{v}, \vec{v}} \\
	&= \ev{\vec{u}, \vec{u}} - \ev{\vec{v}, \vec{v}} \\
	&= \norm{\vec{u}}^{2} - \norm{\vec{v}}^{2}.
\end{align*}

\textbf{Part (b)}: Suppose that $\norm{\vec{v}} = \norm{\vec{u}}$. Then by Part (a),
\[
	\ev{\vec{u} + \vec{v}, \vec{u} - \vec{v}} = \norm{\vec{u}}^{2} - \norm{\vec{v}}^{2} = 0.
\]
We conclude that $\vec{u} + \vec{v}$ and $\vec{u} - \vec{v}$ are orthogonal.

\textbf{Part (c)}: If we translate a rhombus such that one of its verticies is zero, we may represent all four of its verticies in order as $\vec{0}$, $\vec{u}$, $\vec{u} + \vec{v}$, and $\vec{v}$ for some $\vec{u}, \vec{v} \in \mathbb{R}^{n}$.

Then its diagonals are $\vec{v} - \vec{u}$ (or $\vec{u} - \vec{v}$) and $\vec{u} + \vec{v}$; by the result of Part (b), these diagonals are orthogonal --- and thus, perpendicular under the Euclidean norm.

% --------------------------------------------- %

\subsection{Problem 11}

Let $\vec{v} = \left( \sqrt{a}, \sqrt{b}, \sqrt{c}, \sqrt{d} \right)$ and $\vec{w} = \left( \tfrac{1}{\sqrt{a}}, \tfrac{1}{\sqrt{b}}, \tfrac{1}{\sqrt{c}}, \tfrac{1}{\sqrt{d}} \right)$. Then by Cauchy-Schwarz under the dot product,
\begin{align*}
	(a + b + c + d) \left( \frac{1}{a} + \frac{1}{b} + \frac{1}{c} + \frac{1}{d} \right) &= \norm{\vec{v}}^{2} \norm{\vec{w}}^{2} \\
	&\ge \abs{\vec{v} \cdot \vec{w}}^{2} \\
	&= \abs{a \left( \frac{1}{a} \right) + b \left( \frac{1}{b} \right) + c \left( \frac{1}{c} \right) + d \left( \frac{1}{d} \right)}^{2} \\ 
	&= \abs{1 + 1 + 1 + 1}^{2} \\
	&= 16,
\end{align*}
as desired.

% --------------------------------------------- %

\subsection{Problem 20}

\begin{lemma}
	For all $\vec{a}, \vec{b} \in V$,
	\[
		\norm{\vec{a} + \vec{b}}^{2} - \norm{\vec{a} - \vec{b}}^{2} = \ev{2 \vec{a}, \vec{b}} + \ev{2 \vec{b}, \vec{a}}
	\]
\end{lemma}
\begin{adjustwidth}{1cm}{}
	\begin{proof}\renewcommand{\qedsymbol}{}
		We have that 
		\begin{align*}
			\norm{\vec{a} + \vec{b}}^{2} - \norm{\vec{a} - \vec{b}}^{2} &= \ev{\vec{a} + \vec{b}, \vec{a} + \vec{b}} - \ev{\vec{a} - \vec{b}, \vec{a} - \vec{b}} \\
			&= \ev{\vec{a} + \vec{b}, \vec{a}} + \ev{\vec{a} + \vec{b}, \vec{b}} - \ev{\vec{a} - \vec{b}, \vec{a}} + \ev{\vec{a} - \vec{b}, \vec{b}} \\
			&= \ev{(\vec{a} + \vec{b}) - (\vec{a} - \vec{b}), \vec{a}} + \ev{(\vec{a} + \vec{b}) + \ev{\vec{a} - \vec{b}}, \vec{b}} \\
			&= \ev{2 \vec{b}, \vec{a}} + \ev{2 \vec{a}, \vec{b}},
		\end{align*}
		as required.
	\end{proof}
\end{adjustwidth}

Using our lemma, we deduce that
\begin{align*}
	& \quad \frac{\norm{\vec{u} + \vec{v}}^{2} - \norm{\vec{u} - \vec{v}}^{2} + \norm{\vec{u} + i \vec{v}}^{2} i - \norm{\vec{u} - i \vec{v}}^{2} i}{4} \\
		  &= \frac{\ev{2 \vec{u}, \vec{v}} + \ev{2 \vec{v}, \vec{u}} + i \ev{2 \vec{u}, i \vec{v}} + i \ev{2i \vec{v}, \vec{u}}}{4} \\
		  &= \frac{2 \ev{\vec{u}, \vec{v}} + 2 \ev{\vec{v}, \vec{u}} + (-2 i^{2}) \ev{\vec{u}, \vec{v}} + (2i^{2}) \ev{\vec{v}, \vec{u}}}{4} \\
		  &= \frac{2 \ev{\vec{u}, \vec{v}} + 2 \ev{\vec{v}, \vec{u}} + 2\ev{\vec{u}, \vec{v}} - 2 \ev{\vec{v}, \vec{u}}}{4} \\
		  &= \frac{4 \ev{\vec{u}, \vec{v}}}{4} \\
		  &= \ev{\vec{u}, \vec{v}},
\end{align*}
as desired.

% --------------------------------------------- %

\subsection{Problem 24}

We must demonstrate that $\ev{\cdot, \cdot}_{1}$ satisfies the four criteria to be an inner product:
\begin{enumerate}
	\item \textbf{Conjugate Symmetry}: For all $\vec{u}, \vec{v} \in V$, $\ev{\vec{u}, \vec{v}}_{1} = \ev{S \vec{u}, S \vec{v}} = \conjugate{\ev{S \vec{v}, S \vec{u}}} = \conjugate{\ev{\vec{v}, \vec{u}}_{1}}$.
	\item \textbf{Positive-Definiteness}: We have for all $\vec{v} \in V$ that $\ev{\vec{v}, \vec{v}}_{1} = \ev{S \vec{v}, S \vec{v}} \ge 0$. Equality occurs if and only if $S \vec{v} = \vec{0}$, which occurs exclusively when $\vec{v} = \vec{0}$ by the injectivity of $S$.
	\item \textbf{Additivity in First Argument}: We have for all $\vec{u}, \vec{v}, \vec{w} \in V$ that $\ev{\vec{u} + \vec{v}, \vec{w}}_{1} = \ev{S (\vec{u} + \vec{v})}, S(\vec{w}) = \ev{S \vec{u} + S \vec{v}, S \vec{w}} = \ev{S \vec{u}, S \vec{w}} + \ev{S \vec{v}, S \vec{w}} = \ev{\vec{u}, \vec{w}}_{1} + \ev{\vec{v}, \vec{w}}_{1}$.
	\item \textbf{Homogenity in First Argument}: We have for all $\vec{u}, \vec{v} \in V$ and $\lambda \in \mathbb{F}$ that $\ev{\lambda \vec{u}, \vec{v}}_{1} = \ev{S (\lambda \vec{u}), S \vec{v}} = \ev{\lambda (S \vec{u}), S \vec{v}} = \lambda \ev{S \vec{u}, S \vec{v}} = \lambda \ev{\vec{u}, \vec{v}}_{1}$.
\end{enumerate}
Therefore, $\ev{\cdot, \cdot}_{1}$ is an inner product over $V$.

% --------------------------------------------- %

\section{Section 6B}

% --------------------------------------------- %

\subsection{Problem 2}

% --------------------------------------------- %

\subsection{Problem 3}

% --------------------------------------------- %

\subsection{Problem 12}

% --------------------------------------------- %

\subsection{Problem 16}

% --------------------------------------------- %

\end{document}
