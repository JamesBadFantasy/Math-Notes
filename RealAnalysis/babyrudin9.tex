\documentclass[11pt]{article}
\usepackage[T1]{fontenc}
\usepackage{geometry, changepage, hyperref}
\usepackage{amsmath, amssymb, amsthm, bm}
\usepackage{physics, esint}

\hypersetup{colorlinks=true, linkcolor=blue, urlcolor=cyan}
\setlength{\parindent}{0pt}
\setlength{\parskip}{5pt}

\renewcommand{\vec}[1]{\mathbf{#1}}
\newcommand{\uvec}[1]{\mathop{} \!\hat{\textbf{#1}}}
\newcommand{\mat}[1]{\mathbf{#1}}
\newcommand{\tensor}[1]{\mathsf{#1}}
\newcommand{\nll}{\operatorname{null}}
\newcommand{\range}{\operatorname{range}}

\renewcommand{\div}{\nabla \cdot}
\renewcommand{\curl}{\nabla \cross}
\renewcommand{\grad}{\nabla}

\newtheorem{theorem}{Theorem}
\newtheorem{lemma}{Lemma}
\newtheorem{claim}{Claim}
\newtheorem{corollary}{Corollary}
\newtheorem*{theorem*}{Theorem}
\newtheorem*{lemma*}{Lemma}
\newtheorem*{claim*}{Claim}

\title{Rudin: Functions of Several Variables}
\author{James Pagan}
\date{October 2023}

% --------------------------------------------- %

\begin{document}

\maketitle
\tableofcontents
\newpage

% --------------------------------------------- %

\section{Normed Vector Spaces}

% --------------------------------------------- %

\subsection{Definition}

% --------------------------------------------- %

\subsection{Matrix Norm}

Let $X$ and $Y$ be normed vector spaces. If $\mat{T}$ is bounded, the \textbf{norm} of $\mat{T}$ is as follows:
\[
	\norm{\mat{T}} = \sup \{ \norm{\mat{T} \vec{x}} \, \mid \, \vec{x} \in X, \norm{\vec{x}} \le 1\}.
\]
The \textbf{critical vector} of $\mat{T}$ is the vector $\vec{x} \in X$ such that $\norm{\vec{x}} \le 1$ and $\norm{\mat{T} \vec{x}} = \norm{\mat{T}}$; the critical vector always has norm $1$. Naturally, $\norm{\mat{T} \vec{x}} \le \norm{\mat{T}} \cdot \norm{\vec{x}}$; since equality is attained, $\norm{\mat{T} \vec{x}} \le \lambda \vec{x}$ implies $\norm{\vec{T}} \le \lambda$.

\begin{theorem}
	If $\mat{T}, \mat{S} \in \mathcal{B}(X, Y)$, then $\norm{\mat{T} + \mat{S}} \le \norm{\mat{T}} + \norm{\mat{S}}$. $X = Y$ entails $\norm{\mat{TS}} \le \norm{\mat{T}} \norm{\mat{S}}$.
\end{theorem}
\begin{adjustwidth}{0.5cm}{}
	\begin{proof}
		Let $\vec{x}$ be the critical vector of $\mat{T} + \mat{S}$. Then
		\[
			\norm{\mat{T} + \mat{S}} = \norm{(\mat{T} + \mat{S})\vec{x}} \le \norm{\mat{T} \vec{x}} + \norm{\mat{S} \vec{x}} \le \norm{\mat{T}} \norm{\vec{x}} + \norm{\mat{S}} \norm{\vec{x}} = \norm{\mat{T}} + \norm{\mat{S}}.
		\]
		Now suppose $X = Y$ and let $\vec{w}$ be the critical vector of $\mat{TS}$. Then
		\[
			\norm{\mat{TS}} = \norm{\mat{TS} \vec{w}} \le \norm{\mat{T}} \norm{\mat{S} \vec{w}} \le \norm{\mat{T}} \norm{\mat{S}} \norm{\vec{w}} = \norm{\mat{T}} \norm{\mat{S}}.
		\]
		This completes the proof.
	\end{proof}
\end{adjustwidth}

\begin{theorem}
  The matrix norm is a metric of all bounded linear maps in $\mathcal{B}(X, Y)$.
\end{theorem}
\begin{adjustwidth}{0.5cm}{}
	\begin{proof}
		Suppose $\mat{T}, \mat{S} \in \mathcal{B}(X, Y)$ are bounded. We must perform four rather routine calculations:
		\begin{enumerate}
			\item \textbf{Positivity}: The matrix norm is nonnegative. If $\norm{\mat{T} - \mat{S}} = 0$, then $\norm{\vec{x}} = 1$ implies $(\mat{T} - \mat{S}) \vec{x} = \vec{0}$; hence for \textit{all} $\vec{x} \in X$,
			\[
				(\mat{T} - \mat{S}) \vec{x} = \norm{\vec{x}} \left( (\mat{T} - \mat{S}) \left( \frac{\vec{\vec{x}}}{\norm{\vec{x}}} \right) \right) = \norm{\vec{x}} (0) = 0;
			\]	
			thus $\mat{T} - \mat{S} = \mat{0}$ and $\mat{T} = \mat{T}$.
			\item \textbf{Symmetry}: Notice that $(\mat{T} - \mat{S}) \vec{x} = - (\mat{S} - \mat{T}) \vec{x}$ for all $\vec{x} \in X$. Naturally if $\vec{w}$ is the critical vector of $\mat{T} - \mat{S}$, then $- \vec{w}$ is the critical vector of $\mat{S} - \mat{T}$; thus
			\[
				\norm{\mat{T} - \mat{S}} = \norm{\vec{w}} = \norm{-\vec{w}} = \norm{\mat{S} - \mat{T}}.
			\]	
			\item \textbf{Triangle Inequality}: For all bounded $\mat{R} \in \mathcal{L}(X, Y)$,
			\[
				\norm{\mat{T - S}} = \norm{\mat{(T - R) + (R - S)}} \le \norm{\mat{T - R}} + \norm{\mat{R - S}}.
			\]
		\end{enumerate}
		We conclude that the matrix norm is a metric of the bounded matricies of $\mathcal{L}(X, Y)$.
	\end{proof}
\end{adjustwidth}

It is straightforward that $\norm{\lambda \mat{T}} = \abs{\lambda} \norm{\mat{T}}$ for all $\lambda \in \mathbb{C}$ as well.

\newpage

% --------------------------------------------- %

\subsection{Properties of Linear Maps}

\begin{adjustwidth}{0.5cm}{}
  \begin{theorem}
    Let $X$ be a finite-dimensional normed vector space, and let $Y$ be a normed vector space. Then all $\mat{T} \in \mathcal{L}(X, Y)$ are bounded.
  \end{theorem}  
  \begin{proof}
    Let $\dim X = n$ and let $\vec{e}_{1}, \ldots, \vec{e}_{n}$ be a basis of $X$. Then for all $\vec{z} = z_{1} \vec{e}_{1} + \cdots + z_{n} \vec{e}_{n}$, we have
    \[
      \norm{\mat{T} \vec{z}} \, \le \,  \abs{z_{1}} \norm{\mat{T} \vec{e}_{1}} + \cdots + \abs{z_{n}} \norm{\mat{T} \vec{e}_{n}} \, \le \, C (\abs{z_{1}} + \cdots + \abs{z_{n}}),
    \]
    where $C = \max \{ \norm{\mat{T} \vec{e}_{i}} \}$. Realize that $\norm{\vec{z}}_{1} = \abs{z_{1}} + \cdots + \abs{z_{n}}$ defines a norm on $X$; since all norms finite-dimensional vector spaces are equivalent, there exists another constant $M$ such that $\abs{z_{1}} + \cdots + \abs{z_{n}} = \norm{\vec{z}}_{1} \le M \norm{\vec{z}}$. Therefore
    \[
      \norm{\mat{T} \vec{z}} \le CM \norm{\vec{z}},
    \]
    so $\mat{T}$ is bounded. This completes the proof.
  \end{proof}
\end{adjustwidth}

\begin{theorem}
	If $\mat{T} \in \mathcal{L}(X, Y)$ is bounded, then $\mat{T}$ is uniformly continuous.
\end{theorem}
\begin{adjustwidth}{0.5cm}{}
    \begin{proof}
		Let $\mat{T}$ be bounded. For all $\epsilon > 0$, then $0 \le \norm{\vec{x} - \vec{y}} < \tfrac{\epsilon}{\norm{\mat{T}}}$ implies
		\begin{align*}
			\norm{\mat{T} \vec{x} - \mat{T} \vec{y}} &\le \norm{\mat{T}} \cdot \norm{\vec{x} - \vec{y}} < \norm{\mat{T}} \left( \frac{\epsilon}{\norm{\mat{T}}} \right) = \epsilon.
		\end{align*}
		Thus, $\mat{T}$ is uniformly continuous.
	\end{proof}
\end{adjustwidth}

Let $\Omega$ be the set of all bounded, invertible linear operators on $X$. Recall from Linear Algebra that an operator $\mat{T} \in \mathcal{L}(\mathbb{C}^{n})$ is invertible if and only if $\range \mat{T} = \mathbb{C}^{n}$.

\begin{theorem}
	Let $X$ be a finite dimensional vector space. If $\mat{T} \in \Omega$ and $\mat{S} \in \mathcal{L}(X)$ are both bounded, then
	\[
		\norm{\mat{T}^{-1}} \cdot \norm{\mat{S} - \mat{T}} < 1
	\]
	implies $\mat{S} \in \Omega$.
\end{theorem}
\begin{adjustwidth}{0.5cm}{}
	\begin{proof}
		Suppose $\mat{S} \notin \Omega$. Then there exists $\vec{x} \in X$ of norm $1$ such that $\mat{S} \vec{x} = \vec{0}$, so
		\begin{align*}
			1 &= \norm{\vec{x}} \\
			&= \norm{\mat{T}^{-1}\mat{T} \vec{x}} \\
			&\le \norm{\mat{T}^{-1}} \cdot \norm{\mat{T} \vec{x}} \\
			&= \norm{\mat{T}^{-1}} \cdot \norm{\mat{S} \vec{x} - \mat{T} \vec{x}} \\
			&\le \norm{\mat{T}^{-1}} \cdot \norm{\mat{S} - \mat{T}} \cdot \norm{\vec{x}} \\
			&= \norm{\mat{T}^{-1}} \cdot \norm{\mat{S} - \mat{T}}.
		\end{align*}
		Taking the contrapositive yields the desired result.
	\end{proof}
\end{adjustwidth}

\newpage

\begin{theorem}
	Let $X$ be a finite-dimensional normed vector space. Then $\Omega$ is an open subset of $\mathcal{L}(X)$, and the bijection $f : \mat{T} \to \mat{T}^{-1}$ is continuous on $\Omega$.
\end{theorem}
\begin{adjustwidth}{0.5cm}{}
	\begin{proof}
		Let $\mat{T} \in \Omega$. Since $\norm{\mat{T}^{-1}}$ is nonzero, we may consider the open ball at $\mat{T}$ of radius $\tfrac{1}{\norm{\mat{T}^{-1}}}$; more specifically, all $\mat{S} \in \mathcal{L}(\mathbb{C}^{n})$ such that
		\[
			\norm{\mat{T} - \mat{S}} < \tfrac{1}{\norm{\mat{T}^{-1}}}.
		\]
		Since $\norm{\mat{T}}$ is nonzero, $\norm{\mat{T} - \mat{S}} \cdot \norm{\mat{T}^{-1}} < 1$; hence $\mat{S} \in \Omega$. The open ball is contained within $\Omega$, so the set $\Omega$ is open. As per continuity, realize that
		\[
			\lim\limits_{S \to T^{-1}} \norm{\mat{T} - \mat{S}}
		\]
		As per continuity: \textbf{Rudin's proof is nonrigorous}, and I don't know how to rectify it. The basic idea is that
		\[
			\norm{\mat{T}^{-1} - \mat{S}^{-1}} = \norm{\mat{T}^{-1}\mat{SS}^{-1} - \mat{T}^{-1}\mat{TS}^{-1}} \le \norm{\mat{T}^{-1}} \cdot \norm{\mat{S} - \mat{T}} \cdot \norm{\mat{S}^{-1}},
		\]
		\textit{but} using this to bound $\epsilon$ depends on $\mat{T}^{-1}$ and $\mat{S}^{-1}$. \textbf{I will leave this unfinished until instructor clarificiation}.
	\end{proof}
\end{adjustwidth}

% --------------------------------------------- %

\subsection{Completeness of \texorpdfstring{$\mathcal{L}(\mathbb{C}^{n}, \mathbb{C}^{m})$}{Matricies}}

Before we discuss the completeness of $\mathcal{L}(\mathbb{C}^{n}, \mathbb{C}^{m})$, we must uncover an important inequality. Let $\vec{z} = (z_{1}, \ldots, z_{n}) \in \mathbb{C}^{n}$; for any $\mat{T} \in \mathcal{L}(\mathbb{C}^{n}, \mathbb{C}^{m})$, define the components of $\mat{T}$ as $t_{ij}$ for $i \in \{ 1, \ldots, n \}$ and $j \in \{ 1, \ldots, m \}$. Then
\[
	\mat{T} \vec{z} = \left( \sum\limits_{j = 1}^{n} t_{1j}z_{j} \right) \vec{e}_{1} + \cdots + \left( \sum\limits_{j = 1}^{n} t_{mj} z_{j} \right) \vec{e}_{m}.
\]
Then via the Cauchy-Schwarz Inequality,
\begin{align*}
	\norm{\mat{T} \vec{z}}^{2} &= \left( \sum_{j = 1}^{n} t_{1j} z_{j} \right)^{2} + \cdots + \left( \sum_{j = 1}^{n} t_{mj} z_{j} \right)^{2} \\
	& \le \left( \sum_{j = 1}^{n} t_{1j}^{2} \right) \left( \sum_{j = 1}^{n} z_{j}^{2} \right) + \cdots + \left( \sum_{j = 1}^{n} t_{mj}^{2} \right) \left( \sum_{j = 1}^{n} z_{j}^{2} \right) \\
	&= \left( \sum_{j = 1}^{n} z_{j}^{2} \right) \left( \sum_{i = 1}^{m} \sum_{j = 1}^{n} t_{ij}^{2} \right) = \norm{\vec{z}}^{2} \left( \sum_{i = 1}^{m} \sum_{j = 1}^{n} t_{ij}^{2} \right).
\end{align*}
Let $\vec{w}$ be the critical vector of $\mat{T}$. Then
\[
	\norm{\mat{T}} = \norm{\mat{T} \vec{w}} \le \norm{\vec{z}} \sqrt{\sum\limits_{i = 1}^{m} \sum\limits_{j = 1}^{n} t_{ij}^{2}} = \sqrt{\sum\limits_{i = 1}^{m} \sum\limits_{j = 1}^{n} t_{ij}^{2}}.
\]

While $\mathcal{L}(\mathbb{C}^{n}, \mathbb{C}^{m})$ and $\mathbb{C}^{n + m}$ may be isomorphic, the relevant bijection is \textit{not} an isometry. 


\begin{theorem}
	$\mathcal{L}(\mathbb{C}^{n}, \mathbb{C}^{m})$ equipped with the matrix norm is a Banach space.
\end{theorem}
\begin{adjustwidth}{0.5cm}{}
    \begin{proof}
		Let the sequence $\mat{T}_{1}, \mat{T}_{2}, \ldots$ be a Cauchy sequence in $\mathcal{L}(\mathbb{C}^{n}, \mathbb{C}^{m})$; declare that for all $\epsilon > 0$, there exists an integer $N$ such that
		\[
			N < i, j \implies \norm{\mat{T}_{i} - \mat{T}_{j}} < \epsilon.
		\]
		Observe that if $k \in \{ 1, \ldots, n \}$, then $N \le i, j$ implies
		\[
			\norm{\mat{T}_{i} \vec{e}_{k} - \mat{T}_{j} \vec{e}_{k}} \le \norm{\vec{e}_{k}} \cdot \norm{\mat{T}_{i} - \mat{T}_{j}} = \norm{\mat{T}_{i} - \mat{T}_{j}} < \epsilon.
		\]
		Thus $\mat{T}_{1} \vec{e}_{k}, \mat{T}_{2} \vec{e}_{k}, \cdots$ is a Cauchy Sequence in $\mathbb{C}^{n}$. We may thus define $\mat{T}$ as the unique matrix that maps $\vec{e}_{k}$ to the limit of $\mat{T}_{1} \vec{e}_{k}, \mat{T}_{2} \vec{e}_{k}, \ldots$ in $\mathbb{C}^{m}$.
		
		For $i \in \mathbb{Z}_{> 0}$, let $_{i} t_{jk}$ be the components of $\mat{T}_{i}$ and let $t_{jk}$ be the components of $\mat{T}$. It is straightforward that $\lim\limits_{i \to \infty} \null_{i} t_{jk} = t_{jk}$; then there exist $N_{jk}$ such that
		\[
			N_{jk} < i \implies \abs{_{i} t_{jk} - t_{jk}} < \frac{\epsilon}{\sqrt{mn}}.
		\]
		Set $N = \max \{ N_{11}, \ldots, N_{jk} \}$. Then $N < i$ implies that
		\[
			\norm{\mat{T}_{i} - \mat{T}} \le \sqrt{\sum\limits_{j = 1}^{m} \sum\limits_{k = 1}^{n} \abs{_{i} t_{jk} - t_{jk}}^{2}} < \sqrt{\sum\limits_{j = 1}^{m} \sum\limits_{k = 1}^{n} \frac{\epsilon^{2}}{mn}} = \epsilon.
		\]
		We conclude that $\mat{T}_{1}, \mat{T}_{2}, \ldots$ converges to $\mat{T}$, so $\mathcal{L}(\mathbb{C}^{n}, \mathbb{C}^{m})$ is a Banach space.
	\end{proof}
\end{adjustwidth}

\begin{corollary}
	If the components of $\mat{T} \in \mathcal{L}(\mathbb{C}^{n}, \mathbb{C}^{m})$ are continuous functions from a metric space $X$ to $\mathbb{R}$, then the mapping $X \to \mat{T}$ is continuous.
\end{corollary}
\begin{adjustwidth}{0.5cm}{}
	\begin{proof}
		Let the continuous components be $f_{ij}$. For all $\epsilon > 0$, there are $N_{ij}$ such that
		\[
			0 < d(x, y) < \delta_{ij} \implies \abs{f_{ij}(x) - f_{ij}(y)} < \frac{\epsilon}{\sqrt{mn}}.
		\]
		Then identical means as Theorem 6 demonstrate that the mapping $X \to \mat{T}$ is continuous.
	\end{proof}
\end{adjustwidth}

% --------------------------------------------- %

% --------------------------------------------- %

\section{Differentiation}

% --------------------------------------------- %

\subsection{The Derivative}

Let $f : E \to \mathbb{R}^{m}$ for an open set $E \subset \mathbb{R}^{n}$. Then $f$ is \textbf{differentiable} at $\vec{x} \in E$ if there exists a linear map $D \in \mathcal{L}(\mathbb{R}^{n}, \mathbb{R}^{m})$ such that
\[
	\lim\limits_{\vec{h} \to \vec{0}} \frac{\norm{f(\vec{x} + \vec{h}) - f(\vec{x}) - \mat{J}\vec{h}}}{\norm{\vec{h}}} = 0,
\] 
we say that $f$ is \textbf{differentiable} at $\vec{x}$ and write $f'(\vec{x}) = \mat{J}$, where $J$ is the \textbf{total derivative} of $f$ at $\vec{x}$ --- also called the matrix of partial derivatives, the differential, or the total derivative. If $f$ is differentiable at \textit{all} $\vec{x} \in U$, we say that $f$ itself is differentiable over $U$.

\begin{adjustwidth}{0.5cm}{}
	\begin{lemma}
		The total derivative is unique.
	\end{lemma}
    \begin{proof}\renewcommand{\qedsymbol}{}
		Define $f$ like above. Suppose that for contradiction that there exist two matricies $\mat{J} \ne K$ such that
		\[
			\lim\limits_{\vec{h} \to \vec{0}} \frac{\norm{f(\vec{x} + \vec{h}) - f(\vec{x}) - \mat{J}\vec{h}}}{\norm{\vec{h}}}  = \lim\limits_{\vec{h} \to \vec{0}} \frac{\norm{f(\vec{x} + \vec{h}) - f(\vec{x}) - K\vec{h}}}{\norm{\vec{h}}} = 0.
		\]
		See that $\mat{J} - K \ne 0$, so $\norm{J - K} > 0$. Then there exist $d_{1}$ and $d_{2}$ such that
		\begin{align*}
			0 < \norm{\vec{h}} < \delta_{1} &\implies \frac{ \norm{-f(\vec{x} + \vec{h}) + f(\vec{x}) + \mat{J}\vec{h}}}{\norm{\vec{h}}} < \frac{\norm{J - K}}{2} \\
			0 < \norm{\vec{h}} < \delta_{2} &\implies \frac{\norm{f(\vec{x} + \vec{h}) - f(\vec{x}) - K\vec{h}}}{\norm{\vec{h}}} < \frac{\norm{\mat{J} - K}}{2} \\
		\end{align*}
		For $0 < \norm{\vec{h}} < \min \{ \delta_{1}, \delta_{2} \}$, we have that
		\begin{align*}	
			\norm{\mat{J} - K} &= \frac{\norm{J - K}}{2} + \frac{\norm{J - K}}{2} \\
			&> \frac{ \norm{-f(\vec{x} + \vec{h}) + f(\vec{x}) + \mat{J}\vec{h}}}{\norm{\vec{h}}} +\frac{\norm{f(\vec{x} + \vec{h}) - f(\vec{x}) - K\vec{h}}}{\norm{\vec{h}}} \\
			&\ge \frac{\norm{(-f(\vec{x} + \vec{h}) + f(\vec{x}) + \mat{J} \vec{h}) + (f(\vec{x} + \vec{h}) - f(\vec{x}) - K \vec{h})}}{\vec{h}} \\
			&= \frac{\norm{(\mat{J} - K) \vec{h}}}{\vec{h}},
		\end{align*}
		so $\norm{\mat{J} - K} \norm{\vec{h}} > \norm{(J - K) \vec{h}}$, which is our desired contradiction.
	\end{proof}
\end{adjustwidth}

As an example, if $T \in \mathcal{L}(\mathbb{R}^{n}, \mathbb{R}^{m})$ and $\vec{x} \in \mathbb{R}^{n}$, then the derivative of $T$ at $\vec{x}$ is $T$, as
\[
	\lim\limits_{\vec{h} \to \vec{0}} \frac{\norm{T(\vec{x} + \vec{h}) - T \vec{x} - T \vec{h}}}{\norm{\vec{h}}} = \lim\limits_{\vec{h} \to \vec{0}} \frac{\norm{\vec{0}}}{\norm{\vec{h}}} = 0.
\]

It is very intuitive to think of $\mat{J}$ as an approimation of $f$ at $\vec{x}_{0}$ --- namely, that there exists $r(\vec{h})$ such that $f(\vec{x}_{0} + \vec{h}) - f(\vec{x}_{0}) = J \vec{h} - r(\vec{h})$ and $\lim\limits_{\vec{h} \to \vec{0}} \tfrac{r(\vec{h})}{\vec{h}} = 0$. This strategy will be exhibited in the following proof:

\subsection{Chain Rule}

\begin{adjustwidth}{0.5cm}{}
	\begin{theorem*}
		Let $f : \mathbb{R}^{n} \to \mathbb{R}^{m}$ and $g : \mathbb{R}^{m} \to \mathbb{R}^{k}$. If $f$ is differentiable at $\vec{x}_{0}$ and $g$ is differentiable at $f(\vec{x}_{0})$ --- and if $\vec{x}_{0}$ and $f(\vec{x}_{0})$ are contained within open sets in the domains of $f$ and $g$ respectively --- then $g \circ f$ is differentiable at $\vec{x}_{0}$, and
		\[
			(g \circ f)' = g'(f(\vec{x}_{0})) f'(\vec{x}_{0}).
		\]
	\end{theorem*}
    \begin{proof}\renewcommand{\qedsymbol}{}
		Let $f'(\vec{x}_{0}) = \mat{J}$ and $g'(f(\vec{x}_{0})) = K$. We have that
		\[
		\lim\limits_{\vec{h} \to \vec{0}} \frac{\norm{f(\vec{x}_{0} + \vec{h}) - f(\vec{x}_{0}) - \mat{J} \vec{h}}}{\norm{\vec{h}}} = 0 = \lim\limits_{\vec{h} \to \vec{0}} \frac{\norm{g(f(\vec{x}_{0}) + \vec{h}) - g(f(\vec{x}_{0})) - K \vec{h}}}{\norm{h}}.
		\]
		Define the function $\vec{k} = f(\vec{x}_{0} + \vec{h}) - f(\vec{x}_{0})$; clearly, $\lim\limits_{\vec{h} \to \vec{0}} \vec{k} = 0$. We have that
		\begin{align*}
			& g(f(\vec{x}_{0} + \vec{h})) - g(f(\vec{x}_{0})) - (K\mat{J}) \vec{h} \\
			=& g(f(\vec{x}_{0}) + \vec{k}) - g(f(\vec{x}_{0})) - K(\mat{J} \vec{h} - f(\vec{x_{0} + \vec{h}}) - f(\vec{x}_{0}) + \vec{k}) \\
			=& g(f(\vec{x}_{0}) + \vec{k}) - g(f(\vec{x}_{0})) - K \vec{k} + K(f(\vec{x}_{0} + \vec{h}) - f(\vec{x}_{0}) - \mat{J} \vec{h}).
		\end{align*}
		We now establish bounds for $\norm{\vec{k}}$: 
		\[
			\norm{\vec{k}} = \norm{f(\vec{x}_{0} + \vec{h}) - f(\vec{x}_{0}) - \mat{J} \vec{h} + J \vec{h}} \le \norm{\vec{h}} \left(\norm{J} + \frac{\norm{f(\vec{x}_{0} + \vec{h}) - f(\vec{x}_{0}) - J \vec{h}}}{\norm{\vec{h}}}\right)
		\]
		Then using the composition rule for limits,
		\begin{align*}
			0 &\le \lim\limits_{\vec{h} \to \vec{0}} \frac{\norm{g(f(\vec{x}_{0} - \vec{h})) - g(f(\vec{x}_{0})) - (K\mat{J}) \vec{h}}}{\norm{\vec{h}}} \\
			&\le \lim\limits_{\vec{h} \to \vec{0}} \frac{\norm{g(f(\vec{x}_{0}) + \vec{k}) - g(f(\vec{x}_{0})) - K \vec{k}}}{\norm{\vec{h}}} + \lim\limits_{\vec{h} \to \vec{0}} \frac{\norm{K}\norm{f(\vec{x}_{0} + \vec{h}) - f(\vec{x}_{0}) - \mat{J} \vec{h}}}{\norm{\vec{h}}} \\
			&\le \lim\limits_{\vec{h} \to \vec{0}} \frac{\norm{g(f(\vec{x}_{0}) + \vec{k}) - g(f(\vec{x}_{0})) - K \vec{k}}}{\norm{\vec{k}}} \lim\limits_{\vec{h} \to \vec{0}} \left( \norm{\mat{J}} + \frac{\norm{f(\vec{x}_{0} + \vec{h}) - f(\vec{x}_{0}) - J \vec{h}}}{\vec{h}}\right) \\
			&= (0)(\norm{\mat{J}} + 0) = 0.	
		\end{align*}
		so $(g \circ f)'(\vec{x}_{0}) = K\mat{J} = g'(f(\vec{x}_{0})) f'(\vec{x}_{0})$ as required. Yes, that last step of Rudin's is nonrigorous --- define it as a piecewise expression equal to $0$ whenever $\vec{k} = \vec{0}$, etc.
	\end{proof}
\end{adjustwidth}

% --------------------------------------------- %

\subsection{The Partial Derivative}

Consider $f : U \subset \mathbb{R}^{n} \to \mathbb{R}^{m}$, where $U$ is an open subset of $\mathbb{R}^{n}$; let $\vec{e}_{1}, \ldots, \vec{e}_{n}$ and $\vec{e}_{1}, \ldots, \vec{e}_{m}$ be the standard bases of $\mathbb{R}^{n}$ and $\mathbb{R}^{m}$. The \textbf{components} of $f$ are the real functions $f_{1}, \ldots, f_{m}$ defined by
\[
	f(\vec{x}) = f_{1}(\vec{x}) \vec{e}_{1} + \cdots + f_{m}(\vec{x}) \vec{e}_{m},
\]
or equivalently by $f_{i}(\vec{x}) = f(\vec{x}) \cdot \vec{e}_{i}$ for each $i \in \{ 1, \ldots, m \}$. Then for $x \in U$, $i \in \{ 1, \ldots, m \}$, and $j \in \{ 1, \ldots, n \}$, we define the \textbf{partial derivative} of $f_{i}$ with respect to $x_{j}$ as
\[
	\pdv{f_{i}}{x_{j}} = \lim\limits_{t \to 0} \frac{f_{i}(\vec{x} + t \vec{e}_{j}) - f_{i}(\vec{x})}{t}.
\]
The fact this is a one-variable limit explains why the computation of partial derivatives aligns so closely with differentiation of univarite functions.

\begin{adjustwidth}{0.5cm}{}
	\begin{lemma*}
		The entries of the total derivative are the partial derivatives: namely, if $f: U \subset \mathbb{R}^{n} \to \mathbb{R}^{m}$ (where $U$ is open) and $f$ is differentiable at $x_{0}$, then the partial derivatives exist and 
		\[
			f'(\vec{x}_{0}) \vec{e}_{j} = \sum_{i = 1}^{m}\left(\pdv{f_{i}}{x_{j}}\right) (\vec{x})\vec{e}_{i} 
		\]
	\end{lemma*}
    \begin{proof}\renewcommand{\qedsymbol}{}
		Let $j$ be any integer in the set $\{ 1, \ldots, n \}$. Since $f$ is differentiable at $\vec{x}$,
		\[
			\lim\limits_{t \to 0} \frac{f(\vec{x}_{0} + t \vec{e}_{j}) - f(\vec{x}_{0})}{t} - f'(\vec{x}) \vec{e}_{j} = \lim\limits_{t \to 0} \frac{\norm{f(\vec{x}_{0} + t \vec{e}_{j}) - f(\vec{x}_{0}) - f'(\vec{x}_{0})(t \vec{e}_{j})}}{\norm{t \vec{e}_{j}}} = 0.
		\]
		Therefore,
		\begin{align*}
			f'(\vec{x}) \vec{e}_{j} &= \lim\limits_{t \to 0} \frac{f(\vec{x}_{0} + t \vec{e}_{j}) -   f(\vec{x}_{0})}{t} \\
			&= \lim\limits_{t \to 0} \frac{\sum_{i = 1}^{m} (f_{1}(\vec{x}_{0} + t \vec{e}_{j})\vec{e}_{i}) - \sum_{i = 1}^{m} (f_{i}(\vec{x}_{0})\vec{e}_{i})}{t} \\
			&= \sum_{i = 1}^{m} \left(\lim\limits_{t \to 0} \frac{f(\vec{x}_{0} + t \vec{\vec{e}}_{j}) - f(\vec{x}_{0})}{t} \vec{e}_{i}\right) \\
			&= \sum_{i = 1}^{m} \left(\pdv{f_{i}}{x_{j}}\right) (\vec{x}_{0}) \vec{e}_{i},
		\end{align*}
		as desired.
	\end{proof}
\end{adjustwidth}

% --------------------------------------------- %

We denote $D_{21}$ as the double partial derivative of $f$ with respect to the first, \textit{then the second}, variable.

\subsection{Mixed Partial Derivatives}

\begin{adjustwidth}{0.5cm}{}
	\begin{lemma*}
		Suppose $f$ is defined in an open set $U \subset R^{2}$ and $D_{1}$ and $D_{21}$ exist at every point of $U$. Let $Q \subset U$ be a closed rectangle with sides parallel to the coordinate exes with opposite verticies $(a, b)$ and $(a + h, b + k)$ for $h, k \ne 0$, and define
		\[
			\triangle (f, Q) = f(a + h, b - k) - f(a + h, b) - f(a, b + k) + f(a, b).
		\]
		Then there is a point $(x, y)$ in the interior of $Q$ such that
		\[
			\triangle(f, Q) = hk(D_{21}f)(x, y).
		\]
	\end{lemma*}
    \begin{proof}\renewcommand{\qedsymbol}{}
		Define $u(t) = f(t, b + k) - f(t, b)$. Then by the Mean Value Theorem, there exists a $x$ between $a$ and $a + h$ and $y$ between $b$ and $b + K$ such that
		\begin{align*}
			\triangle (f, Q) &= u(a + h) - u(a) \\
 							 &= h u'(x) \\
							 &= h(D_{1}f(x, b + k) - D_{1}f(x, b)) \\
							 &= hk D_{21}f(x, y).
		\end{align*}
	\end{proof}
\end{adjustwidth}
\begin{theorem}
	Suppose $f$ is defined in an open set $U \in \mathbb{R}^{2}$, that $D_{1}$ and $D_{2}$ exist at all points of $U$, and that $D_{21}$ is continuous at some point $(a, b) \in U$. Then $D_{12}$ exists at $(a, b)$,and
	\[
		D_{12} f(a, b) = D_{21} f(a, b).
	\]
\end{theorem}
\begin{adjustwidth}{0.5cm}{}
    \begin{proof}\renewcommand{\qedsymbol}{}
		
	\end{proof}
\end{adjustwidth}
% --------------------------------------------- %

\subsection{Real-Valued Functions} 

Let $f : \mathbb{R}^{n} \to \mathbb{R}$ be a differentiable real-valued function. Then $f'$ is a $1$-by-$n$ matrix:
\[
	f' = \begin{bmatrix} \pdv{f}{x_{1}} & \cdots & \pdv{f}{x^{n}} \end{bmatrix}.
\]
The transpose of this matrix is called the \textbf{gradient} of $f$;
\[
	\grad f = f'^{\top} = \begin{bmatrix} \pdv{f}{x_{1}} \\ \vdots \\ \pdv{f}{x_{n}} \end{bmatrix} = \pdv{f}{x_{1}} \vec{e}_{1} + \cdots + \pdv{f}{x_{n}} \vec{e}_{n},
\]
where $\vec{e}_{1}, \ldots, \vec{e}_{n}$ is the standard basis of $\mathbb{R}^{n}$. Observe that $f' \vec{v} = \grad f \cdot \vec{v}$ for all $\vec{v} \in \mathbb{R}^{n}$ --- so the gradient and the derivative of a real-valued function are dual concepts. We can define the derivative of $f$ as a vector $\grad f$ such that 
\[
	\lim\limits_{\vec{h} \to 0} \frac{\norm{f(\vec{x} + \vec{h}) - f(\vec{x}) - \grad f \cdot \vec{h}}}{\norm{\vec{h}}} = 0.
\]

The \textbf{directional derivative} of $f$ at $\vec{x}$ along a unit vector $\vec{v}$ is defined as
\[
	\grad_{\vec{v}} f = \lim\limits_{t \to 0} \frac{f(\vec{x} + t \vec{v}) - f(\vec{x})}{t}
\]
Observe that $\grad_{\vec{e}_{i}} f = \pdv{f}{x_{i}} = \grad f \cdot \vec{e}_{i}$ for all $i \in \{ 1, \ldots, n \}$. This might lead us to conclude the following lemma:
\begin{adjustwidth}{0.5cm}{}
	\begin{lemma*}
		If $f: \mathbb{R}^{n} \to \mathbb{R}$ is differentiable at $\vec{x}_{0}$, then $\grad_{\vec{v}} f = \vec{v} \cdot (\grad f)$ for all unit vectors $\vec{v}$.
	\end{lemma*}
    \begin{proof}\renewcommand{\qedsymbol}{}
		Observe that $f' \vec{v} = \grad f \cdot \vec{v}$, so we may express the definition of the total derivative in terms of the gradient of $f$, and that $\norm{t \vec{v}}$ = $\abs{t}$:
		\begin{align*}
			\grad_{\vec{v}} f &= \lim\limits_{t \to 0} \frac{f(\vec{x} + t \vec{v}) - f(\vec{x})}{t} \\ 
			&= \lim\limits_{t \to 0} \frac{f(\vec{x} + t \vec{v}) - f(\vec{x}) - \grad f \cdot (t \vec{v})}{t} + \lim\limits_{t \to 0} \frac{\grad f \cdot (t \vec{v})}{t} \\
			&= 0 + \lim\limits_{t \to 0} \grad f \cdot \vec{v} \\
			&= \grad f \cdot \vec{v},
		\end{align*}
		as required.
	\end{proof}
\end{adjustwidth}

\begin{adjustwidth}{0.5cm}{}
	\begin{lemma*}
		If $f: \mathbb{R}^{n} \to \mathbb{R}$ is differentiable at $\vec{x}_{0}$, then the maximum of $\grad_{\vec{v}} f(\vec{x}_{0})$ across all unit vectors $\vec{v}$ occurs when $\vec{v}$ points in the direction of $\grad f(\vec{x}_{0})$.
	\end{lemma*}
    \begin{proof}\renewcommand{\qedsymbol}{}
		If $\vec{v}$ is a unit vector, then the Cauchy-Schwarz Inequality guarantees that
		\[
			\grad_{\vec{v}} f(\vec{x}_{0}) = \vec{v} \cdot \grad f(\vec{x}_{0}) \le \norm{\vec{v}} \norm{\grad f (\vec{x}_{0})} = \norm{\grad f(\vec{x}_{0})} = \left( \frac{\grad f(\vec{x}_{0})}{\norm{\vec{x}_{0}}} \right) \cdot \grad f (\vec{x}_{0})
		\]
		so the maximum of $\grad_{\vec{v}} f(\vec{x}_{0})$ occurs when $\vec{v}$ is the normalization of the gradient vector and points in the direction of $\grad f (\vec{x}_{0})$.
	\end{proof}
\end{adjustwidth}

More generally, we have that if $\theta$ is the angle between the unit vector $\vec{v}$ and $\grad f$, then 
\[
	\grad_{\vec{v}} f = \vec{v} \cdot \grad f = \norm{\vec{v}} \norm{\grad f} \cos(\theta) = \norm{\grad f} \cos(\theta).
\]
The directional derivative oscilates like a sine wave as $\vec{v}$ walks around the unit hypersphere.

% --------------------------------------------- %

\section{The Inverse Function Theorem}

% --------------------------------------------- %

\subsection{The Contraction Principle}

Let $X$ be a metric space with metric $d$. If $\varphi : X \to X$ and there exists a real $c < 1$ such that
\[
	d(\varphi(x), \varphi(y)) \le c \cdot d(x, y)
\]
for all $x, y \in X$, then $\varphi$ is a \textbf{contraction} of $X$ into $X$.

\begin{theorem*}
	If $X$ is a complete metric space and if $\varphi$ is a contraction of $X$ into $X$, then there exists a unique element $x \in X$ such that $\varphi(x) = x$
\end{theorem*}
\begin{adjustwidth}{0.5cm}{}
    \begin{proof}\renewcommand{\qedsymbol}{}
		Let $c$ be a constant such that $d(\varphi(x), \varphi(y)) \le c \cdot d(x, y)$ for all $x, y \in X$, and select from $X$ some element $x_{0}$. We define the sequence $x_{0}, x_{1}, \ldots$ recursively by setting
		\[
			x_{n + 1} = \varphi(x_{n}) 
		\]
		for all $n \in \mathbb{Z}_{\ge 0}$. We have via induction that
		\[
			d(x_{n + 1}, x_{n}) = d(\varphi(x_{n}), \varphi(x_{n - 1})) = c \cdot d(x_{n}, x_{n - 1}) = \cdots = c^{n} \cdot d(x_{1}, x_{0}).
		\]
		We seek to invoke the completeness of $X$. Observe that for all $N < n < m$, 
		\begin{align*}
			d(x_{m}, x_{n}) &\le d(x_{n+1}, x_{n}) + d(x_{n + 2}, x_{n + 1}) + \cdots + d(x_{m}, x_{m - 1}) \\
			&\le (c^{n} + c^{n+1} + \cdots + c^{m-1}) d(x_{1}, x_{0}) \\
			&= c^{n} (1 + \cdots + c^{m - n - 1}) d(x_{1}, x_{0}) \\
			&= \left( \frac{c^{m} - c^{n}}{c - 1} \right) d(x_{1}, x_{0})
		\end{align*}
		As the right-hand side of this equation gets arbitrarily small (select $N = \log_{c}(\epsilon)$ and let magic happen), we find that $x_{0}, x_{1}, \ldots$ is a Cauchy sequence. By completeness, it converges to some $x \in X$. Therefore,
		\[
			x = \lim\limits_{n \to \infty} x_{n} = \lim\limits_{n \to \infty} \varphi(x_{n}) = \varphi(f).
		\]
		To prove that $f$ is unique, note that if $\varphi(x) = x$ and $\varphi(y) = y$ for $x, y \in X$, then
		\[
			d(x, y) = d(\varphi(x), \varphi(y)) \le c \cdot  d(x, y).
		\]
		As $c < 1$, we must have that $d(x, y) = 0$, so $x = y$.
	\end{proof}
\end{adjustwidth}

\subsection{The Inverse Function Theorem}

\begin{theorem*}
	Suppose that $C$ is a $C^{1}$ mapping of an open set $E \subset \mathbb{R}^{n}$ to $\mathbb{R}^{n}$, that the matrix $f'(\vec{a})$ is invertible for $\vec{a} \in E$, and define $\vec{b} = f(\vec{a})$ --- then there exist open sets $U, V \in \mathbb{R}^{n}$ such that $\vec{a} \in U$, $\vec{b} \in V$, and $f$ is a bijective mapping from $U$ to $V$ --- and the inverse $g: V \to U$ of $f$ defined by $g(f(\vec{x})) = \vec{x}$ for all $\vec{x} \in U$ is $C^{1}$.
\end{theorem*}
\begin{adjustwidth}{0.5cm}{}
    \begin{proof}\renewcommand{\qedsymbol}{}
		Let $f'(\vec{a}) = \mat{J}$, and let $\lambda = \tfrac{1}{2 \norm{J^{-1}}}$, and let $U$ be the open ball defined by all vectors $\vec{x}$ such that
		\[
			\norm{f'(\vec{x}) - \mat{J}} < \lambda.
		\]
		Further define $V = f(U)$ (or more formally, $\vec{y} \in \mathbb{R}^{m} \mid \exists \vec{x} \in U, \vec{x} = \vec{y}$). We must prove that $f$ is invertible, that $V$ is open, and that $g$ is $C^{1}$.

		\textbf{Invertability of $f$}: We now associate to each $y \in \mathbb{R}^{n}$ a function $\varphi_{\vec{y}}$ defined by
		\[
			\varphi_{\vec{y}}(\vec{x}) = \vec{x} + \mat{J}^{-1} (\vec{y} - f(\vec{x})).
		\]
		Clearly, $f(\vec{x}) = \vec{y}$ if and only if $\varphi_{\vec{y}}(\vec{x}) = \vec{x}$. Since $\varphi_{\vec{y}}'(\vec{x}) = I - \mat{J}^{-1} f'(\vec{x}) = J^{-1}(J - f'(\vec{x})$ for all $\vec{x}$), we find that
		\[
			\norm{\varphi_{\vec{y}}'(\vec{x})} = \norm{\mat{J}^{-1}(J - f'(\vec{x}))} \le \norm{J^{-1}} \norm{J - f'(\vec{x})} < \norm{J^{-1}} \lambda = \frac{1}{2}.
		\]
		We use an above theorem to conclude that for all $x_{1}, x_{2} \in U$,
		\[
			\norm{\varphi_{\vec{y}}(\vec{x}_{1}) - \varphi_{\vec{y}}(\vec{x}_{2})} < \tfrac{1}{2} \norm{\vec{x}_{1} - \vec{x}_{2}}.
		\]
		We conclude that the Contraction Principle guarantees that $\varphi_{\vec{y}}(\vec{x}) = \vec{x}$ has exactly one solution --- so $f(\vec{x}) = \vec{y}$ has exactly one solution. We conclude that $f$ is bijective (and thus invertible) over $U$.

		\textbf{Openness of $V$}: For all $\vec{y}_{0} \in V$, select $\vec{x}_{0}$ such that $f(\vec{x}_{0}) = \vec{y}_{0}$, and let $r$ be the radius of an open ball $B_{\vec{x}_{0}}$ centered at $\vec{x}_{0}$ contained within $U$. We claim that if $\norm{\vec{y} - \vec{y}_{0}} < \lambda r$, then $\vec{y} \in V$. 

		We must construct $\vec{x} \in U$ such that $f(\vec{x}) = \vec{y}$, which we do by proving that $\varphi_{\vec{y}}$ is a contraction of $B_{\vec{x}_{0}}$ into $U$. If $\norm{\vec{y} - \vec{y}_{0}} < \lambda r$, observe that
		\[
			\norm{\varphi_{\vec{y}}(\vec{x}_{0}) - \vec{x}_{0}} = \norm{\mat{J}^{-1}(\vec{y} - \vec{y}_{0})} < \norm{J^{-1}} \lambda r < \frac{r}{2}.
		\]
		Then if $\vec{x} \in B$, $\norm{\vec{x} - \vec{x}_{0}} < r$, so
		\[
			\norm{\varphi_{\vec{y}}(\vec{x}) - \vec{x}_{0}} \le \norm{\varphi_{\vec{y}}(\vec{x}) - \varphi_{\vec{y}}(\vec{x}_{0})} + \norm{\varphi_{\vec{y}}(\vec{x}_{0}) - \vec{x}_{0}} < \tfrac{1}{2} \norm{\vec{x} - \vec{x}_{0}} + \frac{r}{2} \le r,
		\]
		so $\varphi_{\vec{y}}(\vec{x}) \in B_{\vec{x}_{0}}$. We conclude that $\varphi_{\vec{y}}$ is a contraction of the complete metric space $B_{\vec{x}_{0}}$ into itself, so it must have some fixed point $\vec{x} \in B_{\vec{x}_{0}}$ such that $\varphi_{\vec{y}}(\vec{x}) = \vec{x}$. Then $f(\vec{x}) = \vec{y}$, so $\vec{y} \in f(B_{\vec{x}_{0}}) \subset f(U) = V$. Thus $V$ is an open set.

		\textbf{Smoothness of Inverse}: For all $\vec{y} \in V$ and $\vec{y + k} \in V$, there exists $\vec{x}$ and $\vec{x} + \vec{h} \in U$ such that $\vec{y} = f(\vec{x})$ and $\vec{y} + \vec{k} = f(\vec{x} + \vec{h})$. Then
		\begin{align*}
			\norm{\vec{h} - \mat{J}^{-1} \vec{k}} &= \norm{\vec{h} + J^{-1} (f(\vec{x} + \vec{h}) - f(\vec{x}))} \\
			&= \norm{\varphi_{\vec{y}}(\vec{x} + \vec{h}) - \varphi_{\vec{y}}(\vec{x})} \\
			&\le \tfrac{1}{2} \norm{\vec{x} + \vec{h} - \vec{x}} \\
			&= \tfrac{1}{2} \norm{\vec{h}}.
		\end{align*}
		Then $\norm{\mat{J}^{-1} \vec{k}} \ge \tfrac{1}{2} \norm{\vec{h}}$, so $\vec{h} \le 2 \norm{J^{-1}} \vec{k} = \tfrac{1}{\lambda} \vec{k}$. We begin to now investigate the derivative: see that as
		\[
			\norm{f'(\vec{a}) - \mat{J}} \norm{J^{-1}} < \lambda \norm{J^{-1}} = \frac{1}{2} < 1
		\]
		$f'(\vec{a})$ is invertible. Since 
		\[
			g(\vec{y} + \vec{k}) - g(\vec{y}) - f'(\vec{x})^{-1} \vec{k} = \vec{h} - f'(\vec{x}) \vec{k} = -f'(\vec{x})^{-1} (f(\vec{x} + \vec{h}) - f(\vec{x}) - f'(\vec{x}) \vec{h}),
		\]
		we have that
		\[
			\frac{\norm{g(\vec{y} + \vec{k}) - g(\vec{y}) - f'(\vec{x})^{-1} \vec{k}}}{\norm{\vec{k}}} \le \frac{\norm{f'(\vec{x})^{-1}}}{\lambda} \frac{\norm{f(\vec{x} + \vec{h}) - f(\vec{x}) - f'(\vec{x}) \vec{h}}}{\vec{h}}.
		\]
		As $\vec{k} \to \vec{0}$, we have that $\vec{h} \to \vec{0}$ (this is nonrigorous; we'd need to define a piecewise function in the instance that $\vec{k} = 0$). Then as the right-hand side of this inequality tends to $0$, the left-hand side does by the Squeeze Theorem. Thus,
		\[
			g'(\vec{y}) = f'(\vec{x})^{-1} = f(g(\vec{y}))^{-1}
		\]
		Finally, note that as $g$ is a continuous mapping of $V$ onto $U$, that $f'$ is a continuous mapping of $U$ into $\Omega$, then $(f')^{-1}$ is a continuous mapping of $U$ into $\Omega$, so $g'(\vec{y})$ is a continuous mapping of $V$ into $\Omega$. This completes the proof of the most complex (and beautiful) theorem I've ever studied.
	\end{proof}
\end{adjustwidth}

If we lessen the restriction that $f$ need be $C^{1}$, the only part of the Inverse Function Theorem that fails is that $g$ is $C^{1}$; if $f$ is merely differentiable, we may derive that $g$ is differentiable too.

% --------------------------------------------- %

\subsection{The Implicit Function Theorem}

% --------------------------------------------- %

\end{document}
