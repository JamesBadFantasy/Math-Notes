\documentclass[11pt]{article}
\usepackage[T1]{fontenc}
\usepackage{geometry, changepage, hyperref}
\usepackage{amsmath, amssymb, amsthm, bm}
\usepackage{physics, esint}

\hypersetup{colorlinks=true, linkcolor=blue, urlcolor=cyan}
\setlength{\parindent}{0pt}
\setlength{\parskip}{5pt}

\newtheorem{theorem}{Theorem}
\newtheorem{lemma}{Lemma}
\newtheorem{proposition}{Proposition}
\newtheorem{corollary}{Corollary}
\newtheorem{claim}{Claim}
\newtheorem{definition}{Definition}

\renewcommand{\vec}[1]{\mathbf{#1}}
\newcommand{\uvec}[1]{\mathop{} \!\hat{\textbf{#1}}}
\newcommand{\mat}[1]{\mathbf{#1}}
\newcommand{\tensor}[1]{\mathsf{#1}}
\newcommand{\nll}{\operatorname{null}}
\newcommand{\range}{\operatorname{range}}

\renewcommand{\grad}{\nabla}
\renewcommand{\div}{\nabla \cdot}
\renewcommand{\curl}{\nabla \cross}

\title{MATH-UA 329: Homework 3A}
\author{James Pagan, March 2024}
\date{Professor Güntürk}

% --------------------------------------------- %

\begin{document}

\maketitle
\tableofcontents
\newpage

% --------------------------------------------- %

\section{Problem 1}
 
Let $\vec{x}$ be the vector in $X$ such that $\norm{\vec{x}}_{X} = 1$ and and $\norm{\mat{ST} \vec{x}}_{Z} = \norm{\mat{ST}}_{X \to Z}$. The existence of this vector is ensured by Extreme Value Theorem, since $\norm{\mat{ST}}_{X \to Z}$ is a supremum of the image of a compact set. Observing that $\norm{\vec{x}}_{X} = 1$, we have that
\begin{align*}
  \norm{\mat{ST}}_{X \to Z} \, & = \, \norm{\mat{ST} \vec{x}}_{Z} \\
                               & \le \, \norm{\mat{S}}_{Y \to Z} \norm{\mat{Tx}}_{Y} \\
                               & \le \, \norm{\mat{S}}_{Y \to Z} \norm{\mat{T}}_{X \to Y} \norm{\vec{x}}_{X} \\
                               & = \, \norm{\mat{S}}_{Y \to Z} \norm{\mat{T}}_{X \to Y}.
\end{align*}
This completes the proof.

% --------------------------------------------- %

\end{document}
