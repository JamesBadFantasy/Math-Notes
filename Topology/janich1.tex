\documentclass[11pt]{article}
\usepackage[T1]{fontenc}
\usepackage{geometry, changepage, hyperref}
\usepackage{amsmath, amssymb, amsthm, bm}
\usepackage{physics, esint, accents}

\hypersetup{colorlinks=true, linkcolor=blue, urlcolor=cyan}
\setlength{\parindent}{0pt}
\setlength{\parskip}{5pt}

\newtheorem{theorem}{Theorem}
\newtheorem{lemma}{Lemma}
\newtheorem{proposition}{Proposition}
\newtheorem{corollary}{Corollary}
\newtheorem{claim}{Claim}

\renewcommand{\O}{\mathcal{O}}

\title{Jänich: Fundamental Concepts}
\author{February 2024}

% --------------------------------------------- %

\begin{document}

\maketitle
\tableofcontents
\newpage

% --------------------------------------------- %

\section{The Concept of a Topological Space}

We proceed assuming baisc familiarity with sets, as seen in RealAnalysis/babyrudin2.tex.

% --------------------------------------------- %

\subsection{Definition}

A \textbf{topological space} $(X, \O)$ is a set $X$ and a set $\O$ of subsets of $X$, called \textbf{open sets}, such that the following three axioms hold:
\begin{enumerate}
  \item Any arbitrary union of open sets in $\O$ lies in $\O$.
  \item Any finite intersection of open sets in $\O$ lies in $\O$.
  \item $X$ itself and $\varnothing$ are open sets in $\O$.
\end{enumerate}
The set $\O$ is a \textbf{topology} in $X$. Henceforth, we will speak of a topological space $X$ instead of the pair $(X, \O)$. The following concepts are critical to Point-Set Topology:
\begin{enumerate}
  \item A set $F \subseteq X$ is \textbf{closed} if its complement is open; that is, if $F^{\complement} \subset \O$.
  \item A set $N \subset X$ is a \textbf{neighborhood} of $x$ if $N$ contains an open set $U$ which contains $x$.
  \item A point $x$ is an \textbf{interior}, \textbf{exterior}, or \textbf{boundary point} of a set $S$ according to whether $S$, $S^{\complement}$, or neither is a neighborhood of $x$.
  \item The set $\mathring{S}$ of all the interior points of $S$ is the \textbf{interior} of $S$.
  \item The set $\overline{S}$ of all the interior and boundary points of $S$ is the \textbf{closure} of $S$.
\end{enumerate}

Naturally, exterior points of $S$ are interior points of $\overline{S}$ and the complement of $\mathring{S}$ is $\overline{S^{\complement}}$.

% --------------------------------------------- %

\subsection{Basic Consequences}

The following theorem allows for an alternative definition of topological spaces $(X, \O)$ by \textit{closed sets}:

\begin{theorem}
  Let $X$ be a topological space. Then
  \begin{enumerate}
    \item Any arbitrary intersection of closed sets is closed.
    \item Any finite union of closed sets is closed.
    \item $X$ itself and $\varnothing$ are closed.
  \end{enumerate}
\end{theorem}
\begin{adjustwidth}{0.5cm}{}
  \begin{proof}
    Let $F_{\alpha}$ be a collection of closed sets. Since their compliments are open, we have
    \[
      F_{\alpha}^{\complement} \text{ are open} \implies \bigcup\limits_{\alpha} F_{\alpha}^{\complement} \text{ is open} \implies \left( \bigcup\limits_{\alpha} F_{\alpha}^{\complement} \right)^{\complement} \text{ is closed} \implies \bigcap\limits_{\alpha} F_{\alpha} \text{ is closed}.
    \]
    If we let $F_{n}$ be closed for $n \in \{ 1, \ldots, k \}$, a similar argument follows:
    \[
      F_{n}^{\complement} \text{ are open} \implies \bigcap\limits_{n = 1}^{k} F_{n}^{\complement} \text{ is open} \implies \left( \bigcap\limits_{n = 1}^{k} F_{n}^{\complement} \right)^{\complement} \text{ is closed} \implies \bigcup\limits_{n = 1}^{k} F_{n} \text{ is closed}.
    \]
    The sets $X$ and $\varnothing$ are clearly closed, which completes the proof. 
  \end{proof}
\end{adjustwidth}

\begin{theorem}
  A set is open if and only if all of its points are interior.
\end{theorem}
\begin{adjustwidth}{0.5cm}{}
  \begin{proof}
    If each point of $S$ is interior, then $x \in S$ implies the existence of an open set $U_{x}$ such that $x \in U_{x} \subseteq S$. Thus we define:
    \[
      U = \bigcap\limits_{x \in \mathring{S}} U_{x}.
    \]
    Observe that $U$ is open. We claim that $S = U$ by a two-part argument:
    \begin{enumerate}
      \item Suppose $x \in S$. Then clearly $x \in U_{x} \subseteq U$, so $S \subseteq U$.
      \item Suppose $x \in U$. Then $x \in U_{y}$ for some $y$; since $U_{y} \subseteq S$, we have $x \in S$. Thus $U \subseteq S$.
    \end{enumerate}
    Hence $S$ is open. The reverse direction follows naturally: if $S$ is open and $x \in S$, then $S$ is a neighborhood of all $x$. Hence all $x$ is interior.
  \end{proof}
\end{adjustwidth}

\begin{theorem}
  The interior of $S$ is the union of all open sets contained in $S$.
\end{theorem}
\begin{adjustwidth}{0.5cm}{}
  \begin{proof}
    Let $U$ be the union of all open sets contained in $S$. We claim that $\mathring{S} = U$ by a two-part argument:
    \begin{enumerate}
      \item Suppose $x \in \mathring{S}$. Then there exists an open set $U_{x}$ such that $x \subseteq U_{x} \subseteq S$ --- hence $x \in U_{x} \in U$.
      \item Suppose $x \in U$. Then $x$ lies in an open set contained in $S$, so $x \in S$.
    \end{enumerate}
    Hence $\mathring{S} = U$
  \end{proof}
\end{adjustwidth}

\begin{corollary}
  The interior of a set is open.
\end{corollary}

\newpage

By taking complements of these results about open sets and interiors, we find:
\begin{enumerate}
  \item A set is closed if and only if all of its points are interior or boundary points.
  \item The closure of $S$ is the intersection of all closed sets containing $S$.
  \item The closure of a set is closed.
\end{enumerate}

% --------------------------------------------- %

\section{Metric Spaces}


% --------------------------------------------- %

\end{document}
